%ICMP: The Good, the Bad, and the Ugly
%https://www.researchgate.net/publication/224349250_Implementation_of_an_ICMP-based_covert_channel_for_file_and_message_transfer 
%
In uno Storage Covert Channel, la risorsa condivisa sarà il pacchetto ICMP inviato e i dati saranno 
scritti nei suoi campi. 
Il destinatario, una volta ricevuto, potrà poi leggere i valori codificati al loro interno. 
\vspace{2ex} \newline
Nelle tabelle [Tabella \ref{table:icmpv4:tipologie} ][Tabella \ref{table:icmpv6:tipologie} ] sono 
indicati quali tipologie di messaggi verranno sfruttati oltre a quali campi sono stati utilizzati 
e quanti byte pesa un singolo pacchetto. 
%Inoltre è stato indicato anche quant'è il peso di un singolo pacchetto di quella tipologia. 
Nel caso di tipologie come i messaggi Echo Request/Reply, si avranno delle varianti sia 
nella quantità di dati esfiltrabili sia sulla quantità di byte che vengono trasmessi per singolo datagram.  
Ciò è dovuto alla presenza del campo \textit{data} che permette di inserire una quantità "illimitata" di dati. 
%Questo perchè il messaggio può trasportare un payload al suo interno. 
%Le varianti implicheranno quindi l'utliizzazione o meno di questo campo. 
%Nel caso lo si utilizzi il payload o conterrà 32 byte di informazione oppure 56 byte; 
%il primo caso verrà usato perchè è il valore di defualt usato nei sistemi Windows il secondo è invece il sistema usato nei sistemi Linux. 
%Anche il TTL varierà, il suo valore sarà 128 nei sistemi windows mentre 64 nei sistemi Linux. 
%\vspace{2ex} \newline 
%Altre varianti potranno essere presenti per qeulle tipologie che richiedono il pacchetto che ha causato l'errore (Redirect, Destination Unreachable). 
%In questi casi si potranno usare diverse tipologie di pacchetti purchè la loro lunghezza non superi i 28 bit. 
%Possibili alternative trovate sono con IP/ICMP e con TCP/IP con entrambe che permettono l'invio di due byte. 
\begin{longtable}{|c|c|c|}
    \hline 
    \textbf{Tipologia} & \textbf{Byte per pacchetto} & \textbf{Campi Utilizzati}  \\ %messaggi di errore
    \hline
    Destination Unreachable  & 20+8+21 & unused(4 bytes) \\ 
    &$\rightarrow$min 49 byte& header+64 bits (2bytes)\\ %messaggi di errore
    \hline 
    Time Exceeded  & 20+8+21 & unused(4 byte)\\ 
    &$\rightarrow$min 49 byte&header+64 bits(2 byte) \\ %messaggi di errore 
    \hline
    Parameter Problem  & 20+8+21 & pointer(1byte)\\ 
    &$\rightarrow$min 49 byte&unused(3byte)\\ 
    &&header+64 bits(2byte) 4-8  \\ %messaggi di errore
    \hline 
    Source Quench   & 20+8+21 & unused(4byte)\\ 
    &$\rightarrow$min 49 byte&header+64 bits(2byte) 3-8   \\ %messaggi di errore
    \hline     
    Redirect Message & 20+8+21=min 49 byte & header+64 bits(2byte) 3-4\\ %messaggi di errore
    \hline 
    Echo Request/Reply & 20+8+32 & identifier(2byte)\\ 
    &$\rightarrow$min 60 byte&data($\geq32$) 2\\ %messaggi di informazione
    \hline 
    Timestamp Request/Reply & 20+8+12+32& identifier(2byte)\\ 
    &$\rightarrow$min 72 byte&timestamp(4byte*3)\\ 
    &&data($\geq32$) 5 \\ %messaggi di informazione
    \hline 
    Information Request/Reply & 20+8=28 byte & identifier(2byte) 2\\ %messaggi di informazione
    \hline 
    %9 & 0 & CC  & Router Advertisement & Routers announce themselves to hosts. \\
    %\hline
    %10 & 0 & CC  & Router Solicitation & Hosts request router advertisements. \\
    %\hline 
\end{longtable}
\captionof{table}{Tipologie di messaggi ICMPv4} 
\label{table:icmpv4:tipologie} 
\vspace{4ex}
\begin{longtable}{|c|c|c|} 
    \hline 
    \textbf{Tipologia} & \textbf{Byte per pacchetto} & \textbf{Campi Sfruttatti}  \\
    \hline
    Destination Unreachable & 40+8+41 & unused(4 bytes)\\ 
    &$\rightarrow$min 89 byte&invoking packet(2bytes) 4-8  \\ %messaggi di errore
    \hline 
    Time Exceeded & 40+8+41 & unused(4 bytes)\\ 
    &$\rightarrow$min 89 byte&invoking packet(2bytes) 4-8 \\ %messaggi di errore
    \hline
    Parameter Problem & 40+8+41 & pointer(4byte)\\ 
    &$\rightarrow$min 89 byte&invoking packet(2byte) 8 \\  %messaggi di errore
    \hline 
    Echo Request/Reply & 40+8+32 & identifier(2byte)\\ 
    &$\rightarrow$min 80 byte&data($\geq32$) 2\\ %messaggi di informazione
    \hline 
    Packet Too Big & 40+8+41 & mtu(4byte)\\ 
    &$\rightarrow$min 89 byte&invoking packet(2byte) 8 \\  %messaggi di errore
    \hline 
\end{longtable}
\captionof{table}{Tipologie di messaggi ICMPv6} 
\label{table:icmpv6:tipologie} 
%\import{./icmp}{messaggi_errore}  
%\import{./icmp}{messaggi_informativi} 


\subsubsection{Come i dati vengono inseriti nei campi utilizzati}
\subsubsection*{Campo \textbf{unused}}
Dalle specifiche RFC 792\cite{icmpv4_message_type} (e RFC 4434 \cite{icmpv6_message_type}) deve 
essere 0 quindi Scapy, quando manderà il pacchetto, azzererà qualsiasi valore inserito al suo 
interno [Figura \ref{fig:covertStorage:sourcequench:errScapy} ]. 
\begin{minipage}{\textwidth}
    \centering
    \includegraphics[width=\textwidth]{./img/ICMP_storageCC/destination_CC_errscapy.png}
    \captionof{figure}{Messaggio Destination Unreachable con il campo \textit{unused} azzerato} 
    \label{fig:covertStorage:sourcequench:errScapy} 
\end{minipage}  
\vspace{2ex} \newline 
Tramite degli accorgimenti è possibile inserire dei dati all'interno del campo [Figura \ref{fig:covertStorage:sourcequench:wUnused}].  
Tramite la libreria \textit{struct} \cite{struct}, si può riscrivere, il pacchetto evitando di 
utilizzare il livello ICMP di Scapy; così che al suo invio i dati siano presenti. 
\begin{minipage}{\textwidth}
    \centering
    \includegraphics[width=\textwidth]{./img/ICMP_storageCC/destination_unused_CC_1.png}
    \captionof{figure}{Messaggio Destination Unreachable che usa il campo \textit{unused}} 
    \label{fig:covertStorage:sourcequench:wUnused} 
\end{minipage} 
\vspace{2ex} \newline 
Siccome il suo utilizzo rende il messaggio non conforme allo standard RFC \cite{icmpv4_message_type}\cite{icmpv6_message_type}; 
un IDS che implementi la Deep Packet Inspection rileverà l'anomali. 
Si potrà quindi scegliere se inserire dati nel campo o no [Figuara \ref{fig:covertStorage:sourcequench:noUnused}]. 
%Tuttavia nel nostro caso è stato utilizzato per testare la presenza della \textit{Deep Packet Inspection}. 
\begin{minipage}{\textwidth}
    \centering
    \includegraphics[width=\textwidth]{./img/ICMP_storageCC/destination_CC.png}
    \captionof{figure}{Messaggio Destination Unreachable che non usa il campo \textit{unused}} 
    \label{fig:covertStorage:sourcequench:noUnused} 
\end{minipage}  

\subsubsection*{Campo \textbf{Header+64 bits}} 
Nel campo si dovranno inserire l'header IP più 8 byte. 
%si potranno usare varie coppie nelle quali inserire i dati. 
Nel nostro caso si è usato IP/ICMP e siccome il campo rappresenta un datagram già spedito, si sfruttano un 
maggior numero di campi (in particolare quelli del protocollo IP [Tabella \ref{table:icmpv4:header64bit}]).  
\begin{longtable}{|p{0.3\textwidth}|p{0.3\textwidth}|p{0.3\textwidth}|} 
    \multicolumn{3}{c}{Header IPv4} \\  
    \hline 
    \textbf{Campo} & \textbf{Byte} & \textbf{Utilizzo} \\
    \hline 
    Time to live & 1 byte & Tempo di vita del pacchetto\\  
    \hline 
    Total length & 2 byte & Dimensione dell'intero pacchetto\\  
    \hline
    Identification & 2 byte & Identifica i frammenti di un pacchetto IP \\ 
    \hline 
    \noalign{\vskip 2ex}
    \multicolumn{3}{c}{Header ICMPv4} \\ 
    \hline
    \textbf{Campo} & \textbf{Byte} & \textbf{Utilizzo} \\
    \hline
    %Aiuta nell'accoppiare le richieste e le risposte
    Identifier & 2 byte & Identificativo del messaggio di richiesta invocato \\  
    \hline 
    Sequence & 2 byte & Numero di sequenza del messaggio di richiesta invocato \\  
    \hline
\end{longtable}
\captionof{table}{Struttura del campo Header+64 bits} 
\label{table:icmpv4:header64bit} 
%Ma può essere utilizzato anche TCP/IP o altri protocolli purchè lunghezza passata, 
%non sfori i 28 bytes.-
%si userà il campo \textbf{len} del protocollo \textit{IP} e il campo \textbf{id} del protocollo \textit{ICMP}. 
%Tuttavia si dovrebbe inserire solo il primo byte del datagram originale. 
%Ciò cambierà la visibilità del pacchetto siccome non conforme allo standard. 
\begin{minipage}{\textwidth}
    \centering
    \includegraphics[width=\textwidth]{./img/ICMP_storageCC/ttl_CC_header.png}
    \captionof{figure}{Messaggio Time Exceeded e il campo \textit{Header+64 bit}} 
    \label{fig:covertStorage:timexceeded:noUnused} 
\end{minipage}  

\subsubsection*{Campo \textbf{Invoking Packet}} 
Nel campo, il pacchetto usato per indicare l'errore, sarà quello con il protocollo IPV6 e ICMPv6. 
Tuttavia, in questo caso si dovrà stare attenti a non superare la \textit{IPv6 MTU}, 
ma ciò non succederà siccome l'intestazione IPv6 sarà di 40 byte mentre l'intestazione ICMPv6 sarà di 8 byte. 
%Nel campo \textbf{Invoking Packet} si userà il campo \textbf{plen} del protocollo \textit{IPv6} e il campo \textbf{id} del protocollo \textit{ICMPv6}. 
\begin{longtable}{|p{0.3\textwidth}|p{0.3\textwidth}|p{0.3\textwidth}|} 
    \multicolumn{3}{c}{Header IPv6} \\ 
    \hline 
    \textbf{Campo} & \textbf{Byte} & \textbf{Utilizzo} \\
    \hline 
    Payload Length & 2 byte & Tempo di vita del pacchetto\\  
    \hline 
    Hop Limit & 1 byte & Decrementato di 1 per ogni nodo che inoltra il pacchetto\\  
    \hline 
    \multicolumn{3}{c}{Header ICMPv6} \\
    \hline
    \textbf{Campo} & \textbf{Byte} & \textbf{Utilizzo} \\
    \hline
    Identifier & 2 byte & Identificativo del messaggio di richiesta invocato\\  
    \hline 
    Sequence & 2 byte & Numero di sequenza del messaggio di richiesta invocato \\  
    \hline
\end{longtable}
\captionof{table}{Struttura del campo Invoking Packet} 
\label{table:icmpv6:invokinppacket} 

\subsubsection*{Campo \textbf{Pointer}} 
Indica l'ottetto del messaggio in cui è presente l'errore. 
Tuttavia può contenere dei dati non correlati ad esso; 
in questo caso, verranno inseriti tanti dati quanto la sua dimensione in byte. 
%può contenere i dati e, siccome rappresenta l'ottetto in cui è stato rilevato l'errore, potrà essere variabile.
    
\subsubsection*{Campo \textbf{Identifier} e \textbf{Sequenza}}
Il campo \textit{Identifier} definisce l'identificativo delle richieste e può assumere un qualsiasi valore. 
Il campo \textit{Sequenza}, dalle specifiche RFC 792, viene incrementato ad ogni richiesta inviata con lo stesso Identifier. 
Se il valore cambiasse troppo spesso, a differenza del campo \textit{Identifier}, 
la cosa potrebbe risultare sospetta. %in maniera estremamente variabile (e non fosse strettamente crescente)
\vspace{1ex} \newline 
Quindi sebbene può essere utilizzato per inserire le informazioni, si userà quasi esclusivamente il campo Identifier. 

\subsubsection*{Campo \textbf{Data}} 
In questo campo il mittente può inserire quanti dati preferisce. 
Ma per sicurezza la dimensione sarà fra i 32 ed i 64 byte. 
%dovrà inferiore a 1400 bytes se non 
%strettamente uguale o inferiore a 32 bytes. %(o 56 bytes nel caso li si mandi da Linux).
\vspace{2ex} \newline  
\begin{minipage}{\textwidth}
    \centering
    \includegraphics[width=\textwidth]{./img/ICMP_storageCC/echo_CC_payload.png}
    \captionof{figure}{Messaggio Echo Reply e il campo \textit{Data}} 
    \label{fig:covertStorage:echo:payload} 
\end{minipage}  
    
\subsubsection*{Campo \textbf{Timestamp}}
Contiene i millisecondi dalla mezzanotte UT e ciascun campo ha un ordine temporale specifico; 
che dovrà rimanere congruente quando si inseriscono i dati. 
%Siccome i campi hanno un ordine temporale L'ordine dei tempi presenti in ciascuno dei campi dovrà essere congruente. 
%Si potrebbero inserire un numero magigore di dati sfruttando l'intera lunghezza dei campi ma 
%poi i tempi non risulterebbero congruenti fra di loro
%\footnote{Si potrebbe avere il caso che il destinatario ha modificato il messagio prima di poterlo ricevere}
In ciascun campo timestamp verrà inserito un byte nella parte dei millisecondi [Figura \ref{fig:covertStorage:timestamp:output}]. 
\newline
\begin{lstlisting} [language=python]
Dati nasocsti:  b'r'     114
Dati nasocsti:  b'c'     99
Dati nasocsti:  b'o'     111

Timestamp origin prima:  2025-11-07 23:13:36.146254+00:00
Timestamp origin dopo:  2025-11-07 23:13:36.114000+00:00
Tempo campo origin:  83616114 
\end{lstlisting}
\captionof{lstlisting}{Output per la creazione di un messaggio Timestamp}
\label{fig:covertStorage:timestamp:output} 
\vspace{2ex} 
Non è possibile inserire informazioni in ulteriori parti del timestamp siccome hanno dei valori ben definiti, pocihè 
rappresentando le ore, i minuti ed i secondi [Figura \ref{fig:covertStorage:timestamp:time}]. 
Si potrebbero inserire i dati sottoforma di dati temporali; ma questo ci espone al rischio di avere tempi errati (e.g ricevo il messaggio prima che arrivi). 
\vspace{2ex} \newline  
\begin{minipage}{\textwidth}
    \centering
    \includegraphics[width=0.8\textwidth]{./img/ICMP_storageCC/timestamp_CC_originitime_crop.png}
    \captionof{figure}{Messaggio Echo Reply e il campo \textit{Data}} 
    \label{fig:covertStorage:timestamp:time} 
\end{minipage}  

\subsubsection*{Campo \textbf{MTU}}
Nel campo viene indicata la capacità massima del collegamento; siccome il suo valore è variabile, 
e quindi non c'è un valore prestabilito, potrà essere usato per inserire dei dati. 

