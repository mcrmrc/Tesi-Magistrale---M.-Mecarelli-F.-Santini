\documentclass[a4paper,12 pt]{article}
\usepackage{latexsym}
\usepackage[utf8]{inputenc}
\usepackage{hyperref}

%\usepackage[english]{babel}
%\usepackage[italian]{babel}

\usepackage{pgfplots}
\pgfplotsset{compat=1.18}

%\usepackage[english]{babel}
\usepackage[italian]{babel}

%pacchetti matematica
\usepackage{amsmath}
\usepackage{amsthm}
\usepackage{amsfonts}
\usepackage{amssymb}
\usepackage{mathrsfs}
\usepackage{stackengine} 
\usepackage{algpseudocode}
\usepackage{caption} 
\usepackage{minted}

\usepackage{graphicx} 

\usepackage[table,xcdraw, dvipsnames]{xcolor}
\usepackage{longtable}
\usepackage{setspace} 

\usepackage{listings}
\usepackage{import}
\usepackage[normalem]{ulem}

\usepackage{bytefield} 

\usepackage[table,xcdraw]{xcolor} 

\usepackage{url}

\usepackage{xcolor}
\usepackage[T1]{fontenc}
\newcommand\myworries[1]{\textcolor{red}{#1}}

\usepackage{enumitem}
\usepackage{float} 
\usepackage{tikz}
\usepackage{multirow}
\usepackage{todonotes}
\usepackage{amsmath, amssymb}
\usepackage{placeins} 
 
\usepackage{array}
\usepackage{booktabs} 
\usepackage{pifont}
\newcommand{\xmark}{\ding{55}}
\newcommand{\cmark}{\ding{51}} 

\usepackage[square,numbers]{natbib}

\usepackage{changepage} 

\newtheorem{definizione}{Definizione}[section]

\newtheorem{teorema}{Theorem}[section] 
\newtheorem{lemma}{Lemma}[teorema] %[section]  
\newtheorem{corollario}{Corollary}[teorema] %[section]

\newtheorem{lemm}{Lemma}[section]
\newtheorem{teorem}{Teorema}[lemm] 
\newtheorem{proposizione}{Proposizione}[lemm] %[section] 
\newtheorem{corollari}{Corollario} [lemm] 
\newtheorem{esempi}{Esempio} [lemm]

\newtheorem{dimostrazione}{Dimostrazione}[section]

\newtheorem{esempio}{Esempio}[section]

\newtheorem{dimostrazioneT}{Dimostrazione}[teorema]
\newtheorem{dimostrazioneL}{Dimostrazione}[lemma]
\newtheorem{dimostrazioneP}{Dimostrazione}[proposizione]
\newtheorem{dimostrazioneC}{Dimostrazione}[corollario]

\newtheorem{esempioT}{Esempio}[teorema]
\newtheorem{esempioL}{Esempio}[lemma]
\newtheorem{esempioP}{Esempio}[proposizione]
\newtheorem{esempioC}{Esempio}[corollario]   

\newcommand{\cvd}{\begin{flushright}$\Box$\end{flushright}}
\newcommand{\tr}{{\rm Tr}\;}
\newcommand{\eq}{\begin{equation}}
\newcommand{\feq}{\end{equation}}
\theoremstyle{definition}
\newtheorem{definition}{Definition}[section]
 
\theoremstyle{remark}
\newtheorem*{remark}{Remark}

% package italiano
%
% Opzionale
%
%\renewcommand{\contentsname}{Sommario}
%\renewcommand{\listfigurename}{List of Figures}
%\renewcommand{\listtablename}{List of Tables}
%\renewcommand{\bibname}{Bibliografia}
%\renewcommand{\indexname}{Indice}
%\renewcommand{\figurename}{Figura}
%\renewcommand{\tablename}{Tavola}
%\renewcommand{\partname}{Parte}
%\renewcommand{\chaptername}{Capitolo}
%\renewcommand{\appendixname}{Appendice}
%\renewcommand{\abstractname}{Abstract}
%\renewcommand{\footnotesize}{\scriptsize}
%\renewcommand{\today}{\ifcase\month\or
%  Gennaio\or Febbraio\or Marzo\or Aprile\or Maggio\or Giugno\or
%  Luglio\or Agosto\or Settembre\or Ottobre\or Novembre\or Dicembre\fi
%  \space\number\day, \number\year}

% package formato
\pagestyle{plain}
%\setlength{\topmargin}{0.0in}
%\setlength{\headheight}{0.2in}
%\setlength{\headsep}{0.0in}
%\setlength{\footskip}{0.5in}
%\setlength{\textheight}{8.3in}
\setlength{\textwidth}{6in}
\setlength{\oddsidemargin}{0.2in}
\setlength{\evensidemargin}{0.2in}
\setlength{\parindent}{0.2 in}
\onehalfspacing


\def\cent{\centerline}
\def\vs{\vskip 10 pt plus 1 pt}
\def\bs{\bf}
\def\grad{\vec{\nabla}}
\def\gradx{\vec{\nabla}_x}
\def\epsilon{\varepsilon} 

\definecolor{blu_dmi}{HTML}{002e62}

\usepackage{showframe}
\begin{document}
    % Thesis frontmatter --------------------------------------------

\thispagestyle{empty} %suppress page number

	\noindent % just to prevent indentation narrowing the line width for this line
	%\includegraphics[width=0.3\textwidth]{img/logoDMI.png}
	\begin{minipage}[b]{0.7\textwidth}
		\flushleft
		{\Large \textcolor{blu_dmi}{\textsc{Universit{\`a} degli Studi di Perugia}}}\\
		\vspace{0.25 em}
		{\large \textcolor{blu_dmi}{Dipartimento di Matematica e Informatica}}
		\vspace{0.25 em}
	\end{minipage}%
	%\hspace{0.2cm}
	\includegraphics[width=0.28\textwidth]{img/logoDMI.png}
	
	\vspace{5 em}

	\begin{center}
		
		{\large \textcolor{blu_dmi}{\textsc{Tesi triennale/magistrale in ...}}}
		\vspace{8 em}
		
		{\Huge \textcolor{blu_dmi}{Titolo Tesi}}
		\vspace{10 em}
		
		\makebox[380pt][c]{\textcolor{blu_dmi}{\textit{Relatore} \hfill \textit{Laureando}}}
		\makebox[380pt][c]{\textcolor{blu_dmi}{\textbf{Prof. Tizio \hfill  Caio}}}
%		\makebox[380pt][c]{\textcolor{blu_dmi}{\textit{Advisor} \hfill \textit{}}}
%		\makebox[380pt][c]{\textcolor{blu_dmi}{\textbf{Dott. Francesco Santini \hfill}}}
		
		\vspace{6 em}
		\vfill
		
		\textcolor{blu_dmi}{\rule{380pt}{.4pt}}\\
		\vspace{1.2 em}
		\large{\textcolor{blu_dmi}{Anno Accademico  2024-2025}}
		
		
		
		
	\end{center}

% ------------------------------------------------------------------
    %  \newpage 
  
  {%\clearpage           % we want a new page          %% I commented this
   \thispagestyle{empty}% no header and footer
   \vspace*{\stretch{1}}% some space at the top
   \itshape             % the text is in italics
   \raggedleft          % flush to the right margin
  }
  \begin{flushright}
  
   Quì la dedica...
  
  
  \end{flushright}
  {\par % end the paragraph
   \vspace{\stretch{3}} % space at bottom is three times that at the top
   \clearpage           % finish off the page
  }

\newpage



    
    % Opzionale
    % \thispagestyle{empty} %suppress page number

\centerline{\emph{Sommario}}


\newpage

\tableofcontents \newpage
\lstlistoflistings \newpage
\listoffigures \newpage

\section{Parte 1 - Introduzione} \addcontentsline{toc}{section}{\protect\numberline{}Introduzione}  
\import{./file}{introduzione} %\newpage 
%
\subsection{Strumenti Utilizzati}
\import{./file}{tools} \newpage  
%
\section{Parte 2 - Covert Channel implementati} \addcontentsline{toc}{section}{\protect\numberline{}Covert Channel implementati}  
\import{./file}{covert_channel_implementati} \newpage 
%Per ognuno di essi, dove possibile, si effettuerà una richiesta alla vittima; prima partendo da un comando leggero (e.g \textbf{pwd}) per poi passare a comandi sempre più difficili e quindi maggiormente rilevabili (e.g \textbf{cd /path; ls -l} oppure \textbf{cat ./file\_grande}, \dots)

%
\section*{Parte 3 - Test e Risultati} \addcontentsline{toc}{section}{\protect\numberline{}Test e Risultati}  
\import{./file}{RITA} \newpage 
%
%\section*{Parte 4 - Mitigazioni} \addcontentsline{toc}{section}{\protect\numberline{}Mitigazioni}  
%\import{./file}{mitigazione} \newpage 
%\import{./mitigazione/file}{pratiche} 
%\import{./mitigazione/file}{strategie} 
%\import{./mitigazione/file}{covertchannel_strategie} 

\appendix
%\import{./file/appendix}{ } \newpage  


\end{document}

%https://github.com/io22m007/ICMP-Tunnel
%https://thecybersecurityman.com/2018/03/01/covert-channels-how-insiders-abuse-tcp-ip-to-create-covert-channels/
%https://www.pentesting.org/hidden-data-transfer/
%https://www.sciencedirect.com/org/science/article/pii/S1546221821001703
%https://www.ictsecuritymagazine.com/articoli/covert-channel-canali-segreti-per-comunicare-in-maniera-non-convenzionale/
%https://www.trustwave.com/en-us/resources/blogs/spiderlabs-blog/backdoor-at-the-end-of-the-icmp-tunnel/
%https://cted.cybbh.io/tech-college/cttsb/cctc/net/Net/Modules/08_SSH_Tunneling_and_Covert_Channels/fg.html#832-network-analysis


%https://www.cynet.com/attack-techniques-hands-on/how-hackers-use-icmp-tunneling-to-own-your-network/

%https://www.ionos.it/digitalguide/server/know-how/che-cose-il-protocollo-icmp-e-come-funziona/
%https://networkengineering.stackexchange.com/questions/58137/how-to-use-icmp-to-send-message
%https://info.support.huawei.com/hedex/api/pages/EDOC1100277644/AEM10221/03/resources/command/yunshan/ICMPSECURITY_SEND(PP4OM).html

%https://github.com/tdeebswihart/scapy-tools/blob/master/icmpsend


%https://github.com/maddie/go-ping/blob/master/README.md
%https://github.com/maddie/go-ping


%https://github.com/io22m007/ICMP-Tunnel

%https://www.cynet.com/attack-techniques-hands-on/how-hackers-use-icmp-tunneling-to-own-your-network/

%https://www.extrahop.com/blog/detect-and-stop-icmp-tunneling
%https://latest.gost.run/en/tutorials/icmp/

%https://www.hackingarticles.in/command-and-control-tunnelling-via-icmp/
%https://www.bordergate.co.uk/icmp-tunneling/
%https://www.extrahop.com/blog/detect-and-stop-icmp-tunneling
%https://hackaday.com/2009/08/21/tunneling-ip-traffic-over-icmp/
%https://www.trustwave.com/en-us/resources/blogs/spiderlabs-blog/backdoor-at-the-end-of-the-icmp-tunnel/
%https://noxenius.medium.com/icmpdoor-how-i-used-an-icmp-reverse-shell-12814ca1b0e7
%https://medium.com/@nganbarle/icmpdoor-building-an-icmp-reverse-shell-in-python-3-c9b38a12013d
%https://www.ids-sax2.com/understanding-and-implementing-an-icmp-backdoor-using-python-a-step-by-step-guide/
%https://www.trustwave.com/en-us/resources/blogs/spiderlabs-blog/backdoor-at-the-end-of-the-icmp-tunnel/
%https://zeltser.com/reverse-icmp-shell/
%https://medium.com/@s12deff/bypass-firewall-with-icmp-reverse-shell-2018437eeb37
%https://www.infosecinstitute.com/resources/hacking/icmp-reverse-shell/
%https://pentestlab.blog/tag/icmpsh/
%https://www.cynet.com/attack-techniques-hands-on/how-hackers-use-icmp-tunneling-to-own-your-network/
%


%https://www.cs.uit.no/~daniels/PingTunnel/

%https://github.com/secdev/awesome-scapy
%https://github.com/medTrigui/Scapy-Scripts
%https://github.com/0xbharath/scapy-scripts
%https://scapy.readthedocs.io/en/latest/usage.html
%https://python.plainenglish.io/python-basics-packet-crafting-with-scapy-b3e4ea5c8111
%https://thepythoncode.com/topic/scapy
%https://innocentmichael.org/blog/build-your-own-custom-dynamic-firewall-with-python-and-scapy/
%https://www.endpointdev.com/blog/2015/04/raw-packet-manipulation-with-scapy/
%https://github.com/daemoneye/scapy-scripts


%https://www.bing.com/search?q=Icmpsend%2FIcmpsendd&gs_lcrp=EgRlZGdlKgYIABBFGDkyBggAEEUYOTIGCAEQRRg6qAIAsAIA&FORM=ANCMS9&PC=U531


