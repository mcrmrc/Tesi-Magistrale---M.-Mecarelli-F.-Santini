\subsubsection*{Virtual Box \cite{virtualbox}} 
Il codice sviluppato è stato testato in un ambiente Linux. 
Per poter far ciò sono state create, per ciascun entità necessaria, una macchina virtuale contenente Ubuntu. 
Alla fine si sono ottenute quattro macchine virtuali: una per l'attaccante e la vittima, mentre le altre due per i proxy. 
Si poteva usare anche un singolo proxy ma si voleva testare anche come i dati ricavati dalla vittima, e da inoltrare all'attaccante, 
venissero distribuiti ai proxy connessi a essa.  

\subsubsection*{Scapy \cite{scapy}} 
%https://scapy.readthedocs.io/en/latest/index.html
%https://github.com/secdev/scapy 

%#iface: Specify the network interface to sniff on.
%#count: The number of packets to capture. If omitted, sniffing will continue until stopped.
%#filter: Apply a BPF (Berkeley Packet Filter) to capture only certain packets.
%#prn: Define a callback function to execute with each captured packet.
%#store: Whether to store sniffed packets or discard them.
%#Scapy can also store sniffed packets in a .pcap file, which can be analyzed later with tools like Wireshark. To save packets to a file, use the wrpcap() function:
%#   Save captured packets to a file
%#   wrpcap('captured.pcap', packets)
%#Scapy can read packets from a .pcap file using the rdpcap() function or by setting the offline parameter in the sniff() function:
%#   Read packets from a file
%#   packets = rdpcap('captured.pcap')
%
%#Try disabling the firewall temporarily on the VM to test:
%#   On Windows: 
%#   netsh advfirewall set allprofiles state off
%#On Linux: 
%#   sudo ufw disable (if ufw is used)
Scapy è un framework per la manipolazione dei pacchetti scritto in Python che consente di falsificare molti tipi di pacchetti (http, tcp, ip, udp, icmp, ecc.)
È in grado di creare o decodificare pacchetti di vari protocolli. 
Inoltre può inviarli in rete, catturarli, memorizzarli o leggerne i dati. 
%read them using pcap files, match requests and replies, and much more. 
%It is designed to allow fast packet prototyping by using default values that work.
\vspace{2ex} \newline 
Svolge principalmente due funzioni: invia i pacchetti e riceve le risposte, consentendo all'utente di
inviare, intercettare, analizzare e falsificare pacchetti di rete.
Questa capacità consente la creazione di strumenti in grado di sondare, scansionare o attaccare le reti. 
%You define a set of packets, it sends them, receives answers, matches requests with answers and 
%returns a list of packet couples (request, answer) and a list of unmatched packets. 
%We can easily capture some packets or even clone tcpdump or tshark. 
%Either one interface or a list of interfaces to sniff on can be provided. 
%If no interface is given, sniffing will happen on conf.iface
\vspace{1ex} \newline  
Nella libreria il meotdo \textbf{send} (o similari) permetteranno di inviare un definito pacchetto. 
Per definire un pacchetto basterà concatenare i livelli che dovranno essere presenti, e opportunatamente inizializzati; 
mentre per poter ascoltare il traffico di rete basterà una variabile di tipo \textit{AsyncSniffer}. 

%The send() function will send packets at layer 3. That is to say, it will handle routing and layer 2 for you.
%The sendp() function will work at layer 2. It’s up to you to choose the right interface and the right link layer protocol.


\subsubsection*{RITA \cite{rita}} %https://github.com/activecm/rita
RITA (Real Intelligence Threat Analytics) è uno strumento open source per la ricerca delle minacce di rete, 
progettato per identificare attività di comando e controllo (C2) dannose.
Acquisisce i log di Zeek e utilizza l'analisi comportamentale per identificare sistemi potenzialmente compromessi.
%There is often a massive disconnect between what attackers are doing and what we, as defenders, are doing to detect them. 
%There is currently a huge push to develop better and better Indicators of Compromise (IOC) or better threat intelligence.
%A newer development in information security is hunt teaming.  This is where an organization has a team of individuals who actively go looking for evil on a network. 
%There are several different frameworks for Pentesting, such as Metasploit, SET, and Recon-ng. 
%The idea of a framework is that it is effective, extensible, and allows people in the InfoSec community to add additional modules to it continuously is our goal.
%What's the purpose of Threat Hunting
%Perimeter defensive tools like firewalls and IDS protect your valuable systems. 
%A solid incident response plan allows your team to contain and eliminate a discovered threat. 
Per l'installazione si è seguita la seguente \href{https://www.activecountermeasures.com/free-tools/rita/}{guida}. 
Siccome si utilizza un computer Windows, i comandi sono stati esguiti tramite WSL (Windows Subsystem for Linux). 
\vspace{2ex} \newline 
Le funzionalità principali sono: 
il rilevamento dei Beacon, %Beacon Detection: Search for signs of beaconing behavior in and out of your network 
rilevamento del tunneling DNS, %DNS Tunneling Detection: Identify signs of DNS-based covert channels
rilevameno di connessioni che hanno comunicato per tempi lunghi, % Long Connection Detection: Easily see connections that have communicated for long periods of time
controllo dei feed per le Threat Intel (domini e host sospetti), %Threat Intel Feed Checking: Query threat intel feeds to search for suspicious domains and hosts
valutazione per gravità delle connessioni, %Connections Scored by Severity: Critical, High, Medium, Low
quanti host hanno comunicato con un determinato host, %Prevalence: Displays how many of your internal hosts are communicating with a particular external host
il primo incontro di un host, %First Seen: Displays when the external host was first seen on the network 
\vspace{2ex} \newline 
Una costante assoluta su cui RITA fà affidamento per rilevare i malware, è che dovranno chiamare "casa". 
Da questo presupposto; analizzando il traffico di rete, rilevare le chiamate C2 indipendentemente dalla piattaforma. 
%e senza la necessità di agenti endpoint.
%By analyzing network traffic, we can detect C2 calling home regardless of platform and without the need of endpoint agents. 
%\vspace{2ex} \newline 
%L'analisi manuale dei log, e la lettura delle singole voci di connessione, difficilmente porterà alla 
%rilevazione di potenziali attività C2. %vi allerteranno
%Per una maggiore fedeltà, l'analisi del comportamento e dei pattern presenti, deve analizzare la comunicazione 
%che avviene nel tempo fra un host interno ed esterno.  
%For better fidelity, behavior and pattern analysis needs to look at the communication between an 
%internal and external host over time. 
\vspace{2ex} \newline 
La persistenza è l'attributo chiave che si ricerca quando si analizza sistemi compromessi. 
RITA cerca gli indicatori principali relativi a questa persistenza. 
%consentendovi di concentrarvi sull'analisi delle attività segnalate dai suoi risultati.
%RITA looks for the primary indicators of this persistency, 
%allowing you to focus on vetting the activities flagged by its findings.
Viene fatto acquisendo i log di connessione di Zeek e tramite l'utilizzo dell'analisi comportamentale 
identifica i sistemi potenzialmente compromessi. 
%It ingests Zeek connection logs and uses behavioral analytics to identify potentially compromised systems.




%\subsubsection{Progetti che hanno già affrontato il problema}  
%Inoltre, prima di implementare il Covert Channel, ricerchiamo informazioni su software o documenti che abbiano già affrontato l'argomento. 
%Il processo servirà a identificare le caratterisitche principali che un Covert Channel dovrà avere, e nel caso dei programmi a capire possibili aree di miglioramento (dovendoli utilizzare). 

\subsubsection*{ICMP Door \cite{icmp-door}} %https://github.com/krabelize/icmpdoor
\uppercase{è} stato studiato per compredere come potesse effettuare il tunneling dei dati. 
Oltre alla struttura delle entità, che successivamente verrà ridefinita, risulta interessante come il programma richiede degli argomenti dall'utente. 
Tramite la libreria \textit{argparse} richiede all'utente l'interfaccia su cui ascoltare i dati e l'indirizzo di destinazione dei pacchetti. 
Inoltre usa il metodo \textit{sniff} del framework Scapy per ascoltare il flusso dei dati mentre tramite \textit{sr} invia i paccheti. 
%ma nel nostro caso verrà usato direttamente l'AsyncSniffer. 
\vspace{2ex} \newline 
Sebbe il programma ci introduce a una possibile struttura del Covert Channel; gli si sono trovati dei difetti. 
Gli svantaggi sono che i dati vengono trasemssi non solo nel campo data, del messaggio di tipologia ICMP Echo Reply, ma anche in chiaro. 
Inoltre il valore del campo identifier rimano invariato per tutta la sessione. 
Un sistema di sicurezza, se vedesse le molteplici risposte (che non combaciano con il numero di richieste) e leggesse i testi in chiaro, potrebbe identificare il canale nascosto. 
Di solito per ogni Echo Request corrisponde una singola Echo Reply in cui la risposta rimanda i dati ricevuti e il campo data di solito contiene frasi gia preimpostate e sempre costanti (e.g '\textit{helloworld}'). 
%Il campo per i dati in ICMP è opzionale e viene normalmente utilizzato per la messaggistica di errore. 
%Tuttavia, in questo caso verrà usato per il payload della reverse shell (campo Raw). 
%Al suo interno quindi si nasconderanno i comandi dell'attaccante e le risposte della vittima. 
%La sua dimensione massima può essere di 576 byte.
%Quindi se la dimensione totale superasse questo limite, dovremo frammentare il payload.
%Da notare inoltre che i dati non sono cifrati (e.g non usano la cifratura a base 64). 
%Sebbene questo fatto sia semplice da notare se si guarda attentamente, molte reti non loggano o ispezionano in dettaglio i payload ICMP. 
%\begin{center} 
%    \includegraphics[width=0.5\textwidth]{../img/icmpdoor/icmp-reverse-shell.png} 
%    \captionof{figure}{}
%    \label{fig:icmpdoor:struttura} 
%\end{center}  
%\begin{center} 
%    \includegraphics[width=0.5\textwidth]{../img/icmpdoor/wireshark_ipaddr_numReply.jpg} 
%    \captionof{figure}{Traffico relatico al comando \textit{ls -s}}
%    \label{fig:icmpdoor:ls} 
%\end{center}
%\import{./tools}{icmpdoor} \newpage  


\subsubsection*{ICMP Exfil \cite{icmp-exfill}} 
Analizzato per vedere come un Covert Channel temporalizzato postesse funzionare. 
Riceve un dato, lo converte in binario e dopodichè avrà una lista di numeri binari. 
Per mandare il dato, invia un ping con un timeout pari a \textbf{binary\_number+leway}. 
Il valore leway viene utilizzato per rallentare il numero di pacchetti inviati ed avere una connesisone maggiormente silenziosa. 
L'autore infine indica come miglioria, la crittografia dei dati per aggiungere del rumore, dell'entropia. 
\vspace{1ex} \newline 
Ciò su cui si affida è che un osservatore, vedendo i pacchetti ICMP, li veda come vailidi; 
Siccome non riuscirebbero a trovare alcun dato, a meno che non sappiano della tecnica utilizzata. 
\vspace{2ex} \newline 
%Ciò su cui si è preso spunto è la teoria su come i pacchetti vengono mandati ma non l'applicazione. 
%Alcuni svantaggi notati sono, oltre al tempo maggiore che si bilancia con la furtività, è il modo in cui i dati vengono mandati. 
%Sebbene l'autore dica che il programma funzioni, io non sono mai riuscito a vedere i dati. 

%\import{./tools}{ICMPExfil} \newpage 


\subsubsection*{ICMP Tunnel \cite{icmp-tunnel}}  
Strumento che permette il tunneling del traffico IP. 
Tramite delle richieste e risposte ICMP Echo, incapsula il traffico e lo invia al server proxy. 
Quest'ultimo lo decapsulano e lo inoltra. 
I pacchetti in entrata, che sarebbero diretti alla macchina vittima, sono poi incapsulati dal proxy e inviati. 
L'approccio è possibile siccome RFC-792, che indica le linee guida del protocollo ICMP, permette una quantità arbitraria di dati nei pacchetti ICMP Echo (sia richiesta che risposta). 

%\begin{itemize}
%    \item[] \textbf{PRO}: Tramite la delle Echo Reply crea un canale di comunicazione tra attaccante e vittima. 
%    Inoltre un proxy comunica con la vittima (al posto dell'attaccante) e i dati vengono cifrati.
%    \item[] \textbf{CONS}: Le Echo Reply, se numerose, destano sospetti. 
    %Quindi se mai la comunicazione venisse scoperta la vuittima potrebbe filtrare i pacchetti con quello specifico ID
%\end{itemize} 
%\import{./tools}{icmptunnel} \newpage 


