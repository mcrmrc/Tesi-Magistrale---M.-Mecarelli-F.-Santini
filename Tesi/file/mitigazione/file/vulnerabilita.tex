%\subsection{Vulnerabilità Utilizzate}
%I Covert Channel sfruttano le vulnerabilità nel design del sistema, nelle politiche di sicurezza e nei protocolli di comunicazione per trasferire informazioni segretamente. 
%Sfruttando queste vulnerabilità, gli attaccanti possono stabilire Covert Channel che evitano il controllo degli standard  sicurezza, 
%permettendo esfiltrazione non autorizzata di dati o comunicazione fra processi interni (inter-process communication). 
%\vspace{2ex}\newline
%La loro mitigazione richiede controllo degli accessi, randomizzazione dei tempi, iniezione di rumore e una sicurezza hardware migliore. 
\subsection{Principali vulnerabilità sfruttate}
\subsubsection*{Sfruttamento delle risorse condivise} 
\begin{itemize}
    \item \textbf{Scheduling della CPU}: l'attaccante può modulare l'uso della CPU per diffondere informazioni. 
    \item \textbf{Memoria Cache}: gli attacchi side-channel alla cache sfruttano le differenze nei tempi di accesso per dedurre i dati. 
    \item \textbf{Accesso al File System}: i processi possono dedurre informazioni in base ai lock dei file, timestamp o sull'attività del disco
\end{itemize}
\subsubsection*{Vulnerabilità basate sulla temporizzazione} 
\begin{itemize}
    \item \textbf{Variabilità del tempo di risposta}: l'attacante misura i tempi di risposta del sistema per estrarre segreti. 
    \item \textbf{Ritardi nell'esecuzione delle istruzioni}: le differenze del tempo di esecuzione tra le operazioni privilegiate e non possono causare la fuoriuscita di dati. 
    \item \textbf{Tempistica dei pacchetti}: le informazioni possono essere codificate negli intervalli durante la trasmissione dei pacchetti 
    \item \textbf{Manipolazione delle intestazioni}: campi come TTL, sequenza dei numeri o bit non utilizzati possono essere utilizzati per codificare i dati 
    \item \textbf{Pattern del traffico}: le variazioni nel flusso del traffico (es burst size) si possono comportare come un Covert Channel. 
\end{itemize}
\subsubsection*{Manipolazione della Memoria e dello Stato della CPU} 
\begin{itemize}
    \item \textbf{Previsione delle ramificazioni ed esecuzione speculativa}: sfruttato in attacchi come Spectre e Meltdown 
    \item \textbf{Analisi del consumo energeticos}: i canali secondari possono rilevare chiavi crittografiche 
\end{itemize}
\subsubsection*{Falle nel sistema operativo e nella Virtualizzazione} 
\begin{itemize}
    \item \textbf{Abuso della comunicazione fra processi (Inter-Process Communication IPC)}: i processi posono ricavare i dati tramite la memoria condivisa o il passaggio di messaggi
    \item \textbf{Debolezze dell'hypervisor}: le macchine virtuali posso far trapelare informazioni tra le guest instances
\end{itemize}
\subsubsection*{Vulnerabilità Hardware} 
\begin{itemize}
    \item \textbf{Emissioni elettromagnetiche}: dati sensibili possono essere divulgati tramite dei segnali EM (attacco TEMPEST) 
    \item \textbf{Canali laterali acustici}: è possibile analizzare i suoni/rumori della tastiera, le variazioni della velocità della ventola o il rumore dell'alimentatore.
\end{itemize}