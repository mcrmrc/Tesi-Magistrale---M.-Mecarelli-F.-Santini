\subsection{Possibili attacchi tramite ICMP}
\subsubsection{Attacchi Denial-of-Service (DoS/DDoS)}
\subsubsection*{Icmp Flood (Ping Flood)}
Usato per sopraffare un bersaglio con richieste Echo
\begin{itemize}
    \item \textbf{Attacco}: \newline 
    L'attaccante invia un gran numero di richieste di ICMP Echo (richieste di ping) a un sistema bersaglio.
    Se il sistema risponde con risposte ICMP Echo, consuma potenza di elaborazione e larghezza di banda. 
    Se più macchine attaccano contemporaneamente, si parla di un attacco DDoS (Distributed DoS) ICMP Flood.
    \item \textbf{Mitigazione}: \newline 
    Limitare la velocità del traffico ICMP su firewall e router.
    Disattivare le richieste di eco ICMP dalle reti esterne se non necessarie. 
    Utilizzare sistemi di rilevamento delle intrusioni (IDS) per monitorare le richieste di ping eccessive.
\end{itemize} 
\subsubsection*{Attacco Smurf} 
Richieste ICMP contraffatte amplificano il traffico verso una vittima.
\begin{itemize}
    \item \textbf{Attacco}: \newline 
    L'aggressore invia richieste ICMP Echo con un IP sorgente falsificato (l'IP della vittima). 
    Le richieste vengono inviate a un indirizzo broadcast, provocando la risposta di tutti gli host della rete.
    La vittima viene sommersa da risposte ICMP Echo, che portano a una condizione DoS.
    \item \textbf{Mitigazione}: \newline 
    Disabilitare le richieste di broadcast ICMP sui router (nessuna trasmissione diretta IP)
    Implementare filtri in ingresso per bloccare i pacchetti con indirizzi di origine falsificati. 
    Utilizzare le regole del firewall per bloccare il traffico ICMP non necessario.
\end{itemize} 
\subsubsection*{Ping della morte (attacco storico)} 
Invio di pacchetti ICMP di grandi dimensioni per mandare in crash i sistemi
\begin{itemize}
    \item \textbf{Attacco}: 
    L'attaccante invia un pacchetto ICMP sovradimensionato ($>$ 65.535 byte) 
    causano crash da buffer overflow nei sistemi vulnerabili.
    I sistemi operativi più vecchi potrebbero crashare, bloccarsi o riavviarsi quando gestiscono tali pacchetti.
    \item \textbf{Mitigazione}: 
    I sistemi moderni rifiutano i pacchetti di dimensioni eccessive.
    Applicare aggiornamenti e patch di sistema per prevenire questa vulnerabilità.
\end{itemize} 
\subsubsection*{ICMP Unreachable Flood}
\begin{itemize}
    \item \textbf{Attacco}: \newline 
    L'attaccante invia un numero massiccio di messaggi ICMP Destination Unreachable.
    Può sovraccaricare i dispositivi di rete e causare un denial of service. 
    \item \textbf{Mitigazione}: \newline 
    Configurare limiti di velocità per i messaggi di errore ICMP.
    Implementare regole firewall per eliminare il traffico ICMP eccessivo
\end{itemize}
%
\subsubsection{Attacchi di ricognizione}
\subsubsection*{ICMP Ping Sweep}
\begin{itemize}
    \item \textbf{Attacco}: \newline 
    L'aggressore invia richieste ICMP Echo a più host su una rete.
 Sulla base delle risposte, l'attaccante identifica gli host attivi per ulteriori attacchi.
    \item \textbf{Mitigazione}: \newline 
    Blocca le richieste ICMP Echo da fonti esterne. 
    Utilizzare sistemi di prevenzione delle intrusioni (IPS) per rilevare e bloccare attività di scansione sospette.
\end{itemize} 
\subsubsection*{Attacco Timestamp ICMP}
\begin{itemize}
    \item \textbf{Attacco}: \newline 
    Le richieste ICMP Timestamp (tipo 13) consentono agli aggressori di determinare il tempo di attività del sistema.
    Queste informazioni aiutano gli aggressori a individuare i sistemi vulnerabili o riavviati di recente.
    \item \textbf{Mitigazione}: \newline 
    Disattivare le richieste di timestamp ICMP su firewall e router.
 Utilizzare protocolli di sincronizzazione temporale (NTP) con autenticazione anziché query orarie basate su ICMP.
\end{itemize} 
\subsubsection*{Attacco ICMP che maschera l'indirizzo}
\begin{itemize}
    \item \textbf{Attacco}: \newline 
    L'aggressore invia una richiesta di mascheramento dell'indirizzo ICMP (tipo 17) a un bersaglio.
    Se l'obiettivo risponde con la sua maschera di sottorete (subnet mask), rivela i dettagli della rete all'attaccante.
    \item \textbf{Mitigazione}: \newline 
    Disattivare le risposte ICMP Address Mask a meno che non siano necessarie per le operazioni di rete.
    Utilizzare i firewall per filtrare il traffico ICMP proveniente da fonti non attendibili.
\end{itemize} 
%
\subsubsection{Attacchi ICMP Tunneling e Covert Channel}
\subsubsection*{ICMP Tunneling} 
Covert Channel che utilizzano pacchetti ICMP per aggirare i firewall.
\begin{itemize}
    \item \textbf{Attacco}: \newline 
    Gli attaccanti incapsulano dati dannosi all'interno delle richieste e delle risposte ICMP Echo.
    I dati sono incorporati nei pacchetti ICMP per poter aggirare i firewall che consentono il traffico ICMP 
    (ma bloccano le connessioni TCP/UDP) ed esfiltrare così le informazioni.
    Spesso utilizzato per comunicazioni segrete in malware e canali C2 (comando e controllo). 
    \item Esempi di strumenti: 
    \begin{itemize}
        \item Icmpsh - Crea una reverse shell utilizzando ICMP. 
        \item PingTunnel - Incanala il traffico TCP attraverso pacchetti ICMP.
    \end{itemize} 
    \item \textbf{Mitigazione}: \newline 
    Ispezione approfondita dei pacchetti (DPI) per rilevare ICMP Tunneling. 
    Blocca le richieste/risposte di ICMP Echo da reti non attendibili.
    Monitorare il traffico di rete per individuare modelli ICMP insoliti.
\end{itemize} 
\subsubsection*{Esfiltrazione ICMP (furto di dati tramite ICMP)}
\begin{itemize}
    \item \textbf{Attacco}: \newline 
    Gli attaccanti inseriscono dati sensibili (password, file, comandi) all'interno dei pacchetti ICMP.
    I dati vengono inviati a un server esterno controllato dall'attaccante.
    \item \textbf{Mitigazione}: \newline 
    Monitorare e registrare il traffico ICMP per rilevare attività anomale.
    Utilizzare i firewall per limitare il traffico ICMP solo ai dispositivi necessari.
    Utilizzare soluzioni DLP (Data Loss Prevention) per rilevare i tentativi di esfiltrazione.
\end{itemize}
\subsubsection*{ICMP Covert Channels}
\begin{itemize}
    \item \textbf{Attacco}: \newline
    Malware e attaccanti utilizzano pacchetti ICMP per stabilire un canale di comunicazione nascosto.
    Spesso utilizzato nella comunicazione C2 per botnet o operazioni di malware furtive.
    \item \textbf{Mitigazione}: \newline
    Monitorare il traffico ICMP per individuare modelli di utilizzo insoliti.
    Utilizzare i sistemi di rilevamento delle intrusioni di rete (NIDS) per rilevare Covert Channel.
    Limitare la comunicazione ICMP tra reti interne ed esterne.
\end{itemize}
\subsubsection*{Attacco di reindirizzamento ICMP}
\begin{itemize}
    \item Messaggi di reindirizzamento ICMP non autorizzati reindirizzano il traffico 
    verso un gateway dannoso.
    \item Mitigazione: disabilitare il reindirizzamento ICMP.
\end{itemize}