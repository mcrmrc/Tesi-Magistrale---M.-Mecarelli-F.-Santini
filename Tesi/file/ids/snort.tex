\subsection{Installazione}
Prima dell'installazione sono state necessarie l'installazione di alcune dipendenze. 
Lista dei paccheti richiesti: 
cmake to build from source
The Snort 3 libdaq for packet IO
dnet for network utility functions
flex >= 2.6.0 for JavaScript syntax parsing
g++ >= 5 or other C++14 compiler
hwloc for CPU affinity management
LuaJIT for configuration and scripting
OpenSSL for SHA and MD5 file signatures, the protected_content rule option, and SSL service detection
pcap for tcpdump style logging
pcre for regular expression pattern matching
pkgconfig to locate build dependencies
zlib for decompression 

sudo apt update
sudo apt install -y cmake build-essential libpcap-dev libpcre3-dev \
    libdumbnet-dev bison flex zlib1g-dev liblzma-dev openssl libssl-dev \
    pkg-config 
sudo apt install -y \
    build-essential \
    cmake \
    libpcap-dev \
    libpcre3-dev \
    libdnet-dev \
    zlib1g-dev \
    libdumbnet-dev \
    bison \
    flex \
    liblzma-dev \
    pkg-config \
    libssl-dev \
    libnghttp2-dev \
    libevent-dev
%libdnet-dev was renamed → it’s now called libdumbnet-dev
sudo apt install -y \
    build-essential \
    cmake \
    libpcap-dev \
    libpcre3-dev \
    libdumbnet-dev \
    zlib1g-dev \
    bison \
    flex \
    liblzma-dev \
    pkg-config \
    libssl-dev \
    libnghttp2-dev \
    libevent-dev
sudo apt install cmake flex bison libpcap-dev libpcre3-dev libdnet-dev libhwloc-dev liblua5.3-dev libssl-dev
sudo apt install -y libhwloc-dev 
sudo apt install -y luajit libluajit-5.1-dev
sudo apt install -y build-essential git wget curl cmake pkg-config autoconf automake libtool flex bison g++
sudo apt install -y libpcap-dev libdumbnet-dev
sudo apt install -y libpcre3-dev zlib1g-dev libssl-dev
sudo apt install -y libhwloc-dev
sudo apt install -y luajit libluajit-5.1-dev
sudo apt install -y libpcre2-dev



Una volta intallate quelle non presenti installiamo SnLibDAQort cosi:
cd ~
git clone https://github.com/snort3/libdaq.git
cd libdaq
./bootstrap
./configure --prefix=/usr/local/lib/daq\_s3 
sudo make
sudo make install

# Create the config file pointing to your libdaq libraries 
#Create the ld.so.conf.d file (if not already)
echo "/usr/local/lib/daq_s3/lib/" | sudo tee /etc/ld.so.conf.d/libdaq3.conf

Dopo averlo installato eseguiamo \textbf{ldconfig} per configurare il linker dinamico del runtime building del sistema (ystem's dynamic linker run-time bindings). 
sudo ldconfig

You should now see entries like libdaq.so.
ldconfig -p | grep daq


Se si è installato DAQ  in una posizione non atandard. 
Si deve prima dirlo al sistema dove troavare la libreria. 
cat /etc/ld.so.conf.d/libdaq3.conf
    /usr/local/lib/daq_s3/lib/


Installazione di Snort: 
git clone https://github.com/snort3/snort3.git 

Si può scegliere di instalalrtlo nella cartella di defautl del sistema o si può specificare di installarlo da un altra parte. 
Si è scelta la prima opzione. 
Se la libreria LibDAQ  è stata installata in una posizione non standard, si deve includere gli argomenti 
--with-daq-libraries e --with-daq-includes e impostarli in modo prorpio. 

%export my\_path=/path/to/snorty
%mkdir -p $my\_path 
%./configure\_cmake.sh --prefix=$my\_path 

cd snort3 
./configure_cmake.sh \backslash
    --with-daq-includes=/usr/local/lib/daq_s3/include/ \backslash
    --with-daq-libraries=/usr/local/lib/daq_s3/lib/ 
cd build
sudo make -j \$(nproc)
sudo make install

Per vedere se tutto và bene si esegue 'snort -V' per verificare l'installazione. 
/usr/local/snort/bin/snort -V



Verifichiamo che l'installaizone ha i DAQ appropriati
\$my_path/bin/snort --daq-list
    Available DAQ modules:
    afpacket(v7): live inline multi unpriv
     Variables:
      buffer_size_mb <arg> - Packet buffer space to allocate in megabytes
      debug - Enable debugging output to stdout
      fanout_type <arg> - Fanout loadbalancing method
      fanout_flag <arg> - Fanout loadbalancing option
      use_tx_ring - Use memory-mapped TX ring

Se si riceve No available DAQ modules (provare ad aggiugnere le cartelle con --daq-dir). 
Specificare poi --daq-dir. 
\$my_path/bin/snort --daq-dir /usr/local/lib/daq_s3/lib/daq --daq-list

Se questo è il caso si può creare un alias per il comando snort così da non specificare --daq-dir ogni volta: 
alias snort='/path/to/bin/snort --daq-dir /usr/local/lib/daq\_s3/lib/daq'




ls -l /usr/local/snort/bin/snort %You should see the executable there. 
/usr/local/snort/bin/snort -V %to run Snort 3 
%source ~/.bashrc















