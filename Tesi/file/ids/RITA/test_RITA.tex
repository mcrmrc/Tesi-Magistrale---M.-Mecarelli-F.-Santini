%Zeek sites
%https://kifarunix.com/analyze-network-traffic-using-zeek/ 
%https://medium.com/@Mohamed-Medhat/zeek-commands-sheet-cheat-de4358bd277a 
%https://docs.zeek.org/en/master/logs/conn.html
%   %https://gist.github.com/donovanrodriguez/a6cb54c1e1a525a2810f3eb2b8b7ffaa
%   %https://community.zeek.org/t/icmp-not-in-conn-log/4990 
%   %https://github.com/zeek/zeek/issues/3915
%https://docs.zeek.org/en/current/install.html

In questi test a delle saranno presenti delle variazioni che sono: 
\begin{itemize}
    \item La quantità di dati inviati. In questo caso saranno 1KB, 10KB, 100KB, 1MB. 
    \item I campi usati dal Covert Channel. In quest caso si usa il CC che sfrutta il protocollo ICMP Echo; 
    le variazioni si baseranno sull'utilizzo o meno del campo Identifier o del payload. 
    Nel caso si utilizzasse il payload, quest'ultimo potrà contenere una quantià di dati fissa o variabile per ogni pacchetto. 
    \item Il delay fra i blocchi di dati. I dati verranno suddivisi in blocchi aventi una certa dimensione (e.g 100KB); 
    si può decidere se avere un delay fra l'invio di un blocco e il successivo oppure un invio sequenziale. 
    \item Tempo di attesa prima di un nuovo invio. Una volta che si sono inviati tutti i dati ci potrà essere un tempo di riposo 
    prima di poter inviare nuovamente altri dati. Questo tempo sarà variabile (e.g 1 ora, 30 minuti) e 
    dipenderà dalla quantità di dati esfiltrati (e.g 100KB, 1MB) 
    \item Il mittente del messaggio. Può essere un singolo mittente che invierà tutti i pacchetti oppure si potrà 
    falsificare il campo ed inserire valori diversi; così da far sembrare che il pacco sia stato spedito da un altro host. 
\end{itemize} 
L'obbiettivo è determinare come RITA risponde alle comunicazioni con queste caratteristiche. 
Il Covert Channel utilizzato durante la fase di test è quello basato sul protocollo ICMP Echo. 

\subsubsection{Singolo invio di dati} 
In questa tipologia di test; i dati sono stati mandati una singola volta. 
I dati sono divisi in blocchi da 1024 bytes ($\approx\text{1KB}$). 
Invece il tempo di delay, fra un blocco e il successivo, sarà uniforme fra 1 secondo e 15 secondi. 
\vspace{1ex} \newline 
Nella tabella [Tabella \ref{tab:testRita:icmpEcho:soloCampi} ] vengono riportate le etichette assegnate da RITA ad ogni comunicazione. 
In particolare si è utilizzato un Covert Channel che invia dati tramite ICMP Echo Request; 
in cui può utilizzare determinati campi del messaggio. 
\begin{itemize}
    \item[] Un valore \textbf{None} equivale a dire che RITA ha inserito nel database la 
    comunicazione ma che non gli ha assegnato alcun'etichetta. 
    \item[] Invece \textbf{Non presente}  (o \textbf{NP}) vuol dire che la comunicazione non è presente nel 
    database creato da RITA. 
    \item[] Infine \textbf{Low}, \textbf{Medium} e \textbf{High} sono le etichette assegnate da 
    RITA sulla gravità della comunicazione
\end{itemize}
\begin{center} 
\begin{longtable}{|p{0.16\textwidth}|p{0.16\textwidth}|p{0.16\textwidth}|p{0.16\textwidth}|p{0.16\textwidth}|p{0.16\textwidth}|} 
    \hline
    \textbf{Covert Channel} & \multicolumn{4}{c|}{\textbf{Quantità dei dati}} \\
    \hline 
    \noalign{\vskip 2ex}
    \cline{2-5}
    \multicolumn{1}{c|}{} & 1 KB & 10 KB & 100 KB & 1 MB \\ \hline 
    \hline 
    Campo ID & Non presente & Non presente & Non presente & Non presente \\ \hline 
    Campo ID + delay & Non presente & Non presente & Non presente & Low \\ \hline 
    \hline
    Campo ID + payload fisso & Non presente & Non presente & Non presente & Non presente \\ \hline 
    Campo ID + payload fisso + delay & Non presente & Non presente & Non presente & Low \\ \hline 
    \hline
    Payload fisso & Non presente & Non presente & Non presente & Non presente \\ \hline  
    Payload fisso + delay & Non presente & Non presente & Non presente & Low \\ \hline  
    \hline
    Payload randomico & Non presente & Non presente & Non presente & None \\ \hline 
    Payload randomico + delay & Non presente & Non presente & Non presente & Low \\ \hline 
\caption{Test tramite un Covert Channel ICMP Echo} 
\label{tab:testRita:icmpEcho} 
\end{longtable}  
\end{center} 
I risultati non mostrano una presenza di un Covert Channel siccome i dati sono stati mandati una singola volta nella giornata. 
Un successivo step è l'invio degli stessi dati molteplici volte. 

\subsubsection{Molteplici invii dei dati} 
In questi test si invierà per tre volte lo stesso dato variando alcune condizioni di invio. 
Le condizioni chepotranno variare fra i vari test possono essere: 
\begin{itemize}
    \item[] La \textbf{dimensione del dato inviato}. In questo i valori possibili sono 1MB e 100KB. 
    \item[] Il \textbf{tempo di attesa} dopo il quale si può di ritornare ade sfilatrare i dati. I possibili tempi possono essere 1 ora, 30 minuti, 15 minuti. 
    \item[] Se usare un \textbf{delay fra un bloccho di dati e il successivo}. Nel caso affermativo si invierà un blocco di 1KB e poi si aspettà 
    fra i 2 ed i 15 secondi; dopodichè si manda il successivo blocco da 1KB. Nel caso negativo si invieranno i dati sequenzialmente, senza pause.
    \item[] Se \textbf{falsificare il mittente o no}. In questo caso i pacchetti non avranno l'IP della machcina vittima ma un indirizzo preso da 
    un pool di indirizzi IP. Questa pool può contenere o gli indirizzi IP degli host attivi in quel momento sulla rete locale oppure gli 
    indirizzi IP non utilizzati nella rete; e quindi non associati ad alcun host fisico. 
\end{itemize} 
Siccome nei testi il Covert Channel esfiltrerà i dati tramite messaggi ICMP Echo; i dati verranno inseriti nel payload. 
Quest'ultimo, rispetto al campo Identifier, riesce a contenere una quantità magigore di dati e questo porterebbe a meno pacchetti inviati. 
Infatti, se il canale venisse rilevato tramite l'utilizzo del payload, sicuramente verrà rilevato utilizzando solo il campo Identifier 
(siccome quest'ultimo ha una capacità ridotta rispetto al campo payload). 
%Infatti, sebbene il campo Identifier risulti maggiormente discreto (siccome i dati non sono immediatamente visibili), la scarsa capacità 
%risulterebbe in un maggior numero di pacchetti inviati. 
%E siccome si cercheranno di inviare grandi quantità di dati (e.g 1MB); l'uso del campo Identifier risulterebbe inefficace; 
%al contrario se invece si volessero esfiltrare piccole quantità di dati. 
\begin{center} 
\begin{longtable}{|p{0.25\textwidth}|c|c|c|} 
    \multicolumn{4}{c}{\textbf{payload FISSO 1MB DELAY}} \\ \noalign{\vskip 2ex}
    %\cline{2-3} 
    \hline
    \textbf{Covert Channel} & \textbf{Connessione più lunga} & \textbf{Etichetta} & \textbf{Beacon Score} \\
    \hline
    \noalign{\vskip 2ex}
    \hline  
    Pausa di un ora e mittente reale & 10h 3m 6s & Medium & 14.7\% \\ \hline 
    Pausa di 30 minuti e mittente reale & 10h 4m 51s & Medium & 37.9\% \\ \hline 
    Pausa di 15 minuti e mittente reale & 10h 9m 20s  & Medium & 0\% \\ \hline  
    \noalign{\vskip 5ex} 
    \hline  
    %\cline{2-3} 
    Pausa di un ora e mittenti falsificati & 6h 52m 26s & Low & 0.0\% \\ \hline 
    \multicolumn{1}{c|}{} & 6h 52m 11s & Low & 0.0\% \\ \cline{2-4}
    \multicolumn{1}{c|}{} & 6h 52m 8s & Low & 0.0\% \\ \cline{2-4}
    \multicolumn{1}{c|}{} & 6h 52m 6s & Low & 0.0\% \\ \cline{2-4}
    \multicolumn{1}{c|}{} & 6h 50m 48s & Low & 0.0\% \\ \cline{2-4}
    \multicolumn{1}{c|}{} & 6h 50m 32s & Low & 0.0\% \\ \cline{2-4}
    \multicolumn{1}{c|}{} & 6h 50m 11s & Low & 0.0\% \\ \cline{2-4}
    \multicolumn{1}{c|}{} & 6h 50m 8s & Low & 0.0\% \\ \cline{2-4}
    \multicolumn{1}{c|}{} & 6h 49m 21s & Low & 0.0\% \\ \cline{2-4}
    \multicolumn{1}{c|}{} & 6h 49m 12s & Low & 0.0\% \\ \cline{2-4}
    \multicolumn{1}{c|}{} & 6h 45m 56s & Low & 0.0\% \\ \cline{2-4}
    \multicolumn{1}{c|}{} & 4h 34m 58s & Low & 0.0\% \\ \cline{2-4}
    \multicolumn{1}{c|}{} & 6h 32m 57s & Low & 0.0\% \\ \cline{2-4}
    \multicolumn{1}{c|}{} & 2h 17m 12s & None & 0.0\% \\ \cline{2-4}
    \multicolumn{1}{c|}{} & 2h 15m 51s & None & 0.0\% \\ \hline 
    %
    Pausa di 30 minuti e mittenti falsificati & 6h 50m 55s & Medium & 13.5\% \\ \hline 
    \multicolumn{1}{c|}{} & 6h 50m 14s & Low & 0.0\% \\ \cline{2-4}
    \multicolumn{1}{c|}{} & 6h 50m 4s & Low & 0.0\% \\ \cline{2-4} 
    \multicolumn{1}{c|}{} & 6h 50m 1s & Low & 0.0\% \\ \cline{2-4}
    \multicolumn{1}{c|}{} & 6h 49m 58s & Low & 0.0\% \\ \cline{2-4}
    \multicolumn{1}{c|}{} & 6h 49m 49s & Low & 0.0\% \\ \cline{2-4}
    \multicolumn{1}{c|}{} & 6h 49m 26s & Low & 0.0\% \\ \cline{2-4}
    \multicolumn{1}{c|}{} & 6h 49m 14s & Low & 30.7\% \\ \cline{2-4}
    \multicolumn{1}{c|}{} & 6h 48m 51s & Low & 35.1\% \\ \cline{2-4}
    \multicolumn{1}{c|}{} & 4h 30m 22s & Low & 0.0\% \\ \cline{2-4}
    \multicolumn{1}{c|}{} & 0h 0m 57s & None & 65.1\% \\ \cline{2-4}
    \multicolumn{1}{c|}{} & 2h 19m 41s & None & 0.0\% \\ \cline{2-4}
    \multicolumn{1}{c|}{} & 2h 18m 48s & None & 0.0\% \\ \cline{2-4}
    \multicolumn{1}{c|}{} & 2h 18m 48s & None & 0.0\% \\ \cline{2-4}
    \multicolumn{1}{c|}{} & 2h 11m 39s & None & 0.0\% \\ \cline{2-4}
    \multicolumn{1}{c|}{} & 2h 11m 36s & None & 0.0\% \\ \hline 
    %
    Pausa di 15 minuti e mittenti falsificati & 4h 47m 22s  & Low & 15.8\% \\ \hline 
    \multicolumn{1}{c|}{} & 4h 47m 14s & Low & 0.0\% \\ \cline{2-4}
    \multicolumn{1}{c|}{} & 4h 47m 14s & Low & 0.0\% \\ \cline{2-4}
    \multicolumn{1}{c|}{} & 4h 47m 10s & Low & 0.0\% \\ \cline{2-4}
    \multicolumn{1}{c|}{} & 4h 46m 59s & Low & 0.0\% \\ \cline{2-4}
    \multicolumn{1}{c|}{} & 4h 46m 50s & Low & 0.0\% \\ \cline{2-4}
    \multicolumn{1}{c|}{} & 4h 46m 5s & Low & 26.9\% \\ \cline{2-4}
    \multicolumn{1}{c|}{} & 4h 45m 54s & Low & 30.7\% \\ \cline{2-4}
    \multicolumn{1}{c|}{} & 4h 32m 10s & Low & 0.0\% \\ \cline{2-4}
    \multicolumn{1}{c|}{} & 4h 31m 10s & Low & 0.0\% \\ \cline{2-4}
    \multicolumn{1}{c|}{} & 2h 31m 59s & Low & 0.0\% \\ \cline{2-4} 
    \multicolumn{1}{c|}{} & 2h 31m 56s & None & 0.0\% \\ \cline{2-4}
    \multicolumn{1}{c|}{} & 2h 15m 11s & None & 0.0\% \\ \cline{2-4}
    \multicolumn{1}{c|}{} & 0h 0m 35s & None & 52.8\% \\ \cline{2-4} 
\caption{Test con payload fisso, 1MB di dati e un delay durante l'invio} 
\label{tab:testRita:icmpEcho:payloadFISSO:1MB:DELAY} 
\end{longtable} 
\end{center}  
%
\begin{center} 
\begin{longtable}{|p{0.25\textwidth}|c|c|c|} 
    \multicolumn{4}{c}{\textbf{payload FISSO 1MB NO DELAY}} \\ \noalign{\vskip 2ex}
    %\cline{2-3} 
    \hline 
    \textbf{Covert Channel} & \textbf{Connessione più lunga} & \textbf{Etichetta} & \textbf{Beacon Score} \\
    \hline
    \noalign{\vskip 2ex}
    \hline  
    Pausa di un ora e mittente reale & NP & NP & NP \\ \hline 
    Pausa di 30 minuti e mittente reale & NP & NP & NP \\ \hline 
    Pausa di 15 minuti e mittente reale & NP & NP & NP \\ \hline 
    \noalign{\vskip 5ex} 
    \hline  
    %\cline{2-3} 
    Pausa di un ora e mittenti falsificati \footnote{il dato si riferisce ad una connessione TCP} & 0h 1m 5s & None & 33\% \\ \hline 
    Pausa di 30 minuti e mittenti falsificati & NP & NP & NP \\ \hline 
    Pausa di 15 minuti e mittenti falsificati & NP & NP & NP \\ \hline 
    \caption{Test con payload fisso, 1MB di dati e nessun delay durante l'invio} 
\label{tab:testRita:icmpEcho:payloadFISSO:1MB:NODELAY} 
\end{longtable} 
\end{center}  
% 
\begin{center} 
\begin{longtable}{|p{0.25\textwidth}|c|c|c|} 
    \multicolumn{4}{c}{\textbf{payload FISSO 100KB DELAY}} \\
    \noalign{\vskip 2ex}
    %\cline{2-3}
    \hline
    \textbf{Covert Channel} & \textbf{Connessione più lunga} & \textbf{Etichetta} & \textbf{Beacon Score} \\
    \hline 
    \noalign{\vskip 2ex}
    \hline  
    Pausa di un ora e mittente reale & NP & NP & NP \\ \hline 
    Pausa di 30 minuti e mittente reale & NP & NP & NP \\ \hline 
    Pausa di 15 minuti e mittente reale \footnote{il dato si riferisce ad una connessione TCP} & 0h 0m 3s & High & 100\% \\ \hline 
    \noalign{\vskip 5ex} 
    \hline  
    %\cline{2-3} 
    Pausa di un ora e mittenti falsificati & 0h 39m 42s & None & 30.4\% \\ \hline    
    Pausa di 30 minuti e mittenti falsificati & NP & NP & NP \\ \hline  
    Pausa di 15 minuti e mittenti falsificati & NP  & NP & NP \\ \hline 
    \caption{Test con payload fisso, 100KB di dati e un delay durante l'invio} 
\label{tab:testRita:icmpEcho:payloadFISSO:100KB:DELAY} 
\end{longtable} 
\end{center}  
%
\begin{center} 
\begin{longtable}{|p{0.25\textwidth}|c|c|c|} 
    \multicolumn{4}{c}{\textbf{payload FISSO 100KB NO DELAY}} \\ 
    \noalign{\vskip 2ex}
    %\cline{2-3}
    \hline
    \textbf{Covert Channel} & \textbf{Connessione più lunga} & \textbf{Etichetta} & \textbf{Beacon Score} \\
    \hline 
    \noalign{\vskip 2ex}
    %\cline{2-3}
    \hline 
    Pausa di un ora e mittente reale \footnote{il dato si riferisce ad una connessione TCP} & 0h 0m 1s & High & 100\% \\ \hline 
    Pausa di 30 minuti e mittente reale & NP & NP & NP \\ \hline
    Pausa di 15 minuti e mittente reale & NP & NP & NP \\ \hline %0:30:19 
    \noalign{\vskip 5ex} 
    \hline  
    %\cline{2-3} 
    Pausa di un ora e mittenti falsificati & NP & NP & NP \\ \hline    
    Pausa di 30 minuti e mittenti falsificati & NP & NP & NP \\ \hline  
    Pausa di 15 minuti e mittenti falsificati & NP  & NP & NP \\ \hline  
    \caption{Test con payload fisso, 100KB di dati e nessun delay durante l'invio}  
\label{tab:testRita:icmpEcho:payloadFISSO:100KB:NODELAY} 
\end{longtable} 
\end{center}  
%
\begin{center} 
\begin{longtable}{|p{0.25\textwidth}|c|c|c|} 
    \multicolumn{4}{c}{\textbf{payload RANDOM 1MB DELAY}} \\ 
    \noalign{\vskip 2ex}
    %\cline{2-3}
    \hline
    \textbf{Covert Channel} & \textbf{Connessione più lunga} & \textbf{Etichetta} & \textbf{Beacon Score} \\
    \hline 
    \noalign{\vskip 2ex}
    %\cline{2-3}
    \hline  
    Pausa di un ora e mittente reale & 10h 4m 6s & Medium & 0\% \\ \hline 
    Pausa di 30 minuti e mittente reale & Low & 6h 49m 12s & 25\% \\ \hline  
    %Pausa di 15 minuti e mittente reale & aaa  & aaa & aaa \\ \hline  
    \noalign{\vskip 5ex} 
    \multicolumn{4}{c}{I mittenti falsificati provengono da un pool di indirizzi IP attivi nella rete locale} \\
    \noalign{\vskip 2ex} 
    \hline  
    %\cline{2-3} 
    Pausa di un ora e mittenti falsificati & 6h 52m 13s & Low & \% \\ \hline  
        \multicolumn{1}{c|}{} & 6h 52m 13s & Low & \% \\ \cline{2-4}
        \multicolumn{1}{c|}{} & 6h 52m 4s & Low & \% \\ \cline{2-4}
        \multicolumn{1}{c|}{} & 6h 51m 58s & Low & \% \\ \cline{2-4}
        \multicolumn{1}{c|}{} & 6h 51m 53s & Low & \% \\ \cline{2-4}
        \multicolumn{1}{c|}{} & 6h 51m 46s & Low & \% \\ \cline{2-4} 
        \multicolumn{1}{c|}{} & 6h 51m 40s & Low & \% \\ \cline{2-4}
        \multicolumn{1}{c|}{} & 6h 51m 21s & Low & 38.5\% \\ \cline{2-4}
        \multicolumn{1}{c|}{} & 6h 51m 12s & Low & 32.8\% \\ \cline{2-4}
        \multicolumn{1}{c|}{} & 6h 51m 0s & Low & 38.1\% \\ \cline{2-4}
        \multicolumn{1}{c|}{} & 6h 50m 48s & Low & 28.8\% \\ \cline{2-4}
        \multicolumn{1}{c|}{} & 6h 50m 41s & Low & 27.6\% \\ \cline{2-4}
        \multicolumn{1}{c|}{} & 6h 50m 33s & Low & 0\% \\ \cline{2-4} 
        \multicolumn{1}{c|}{} & 4h 34m 37s & Low & 0\% \\ \cline{2-4} 
    Pausa di 30 minuti e mittenti falsificati & 6h 43m 16s & Low & 0\% \\ \hline 
        \multicolumn{1}{c|}{} & 6h 43m 16s & Low & 0\% \\ \cline{2-4}
        \multicolumn{1}{c|}{} & 6h 43m 16s & Low & 0\% \\ \cline{2-4}
        \multicolumn{1}{c|}{} & 6h 43m 14s & Low & 0\% \\ \cline{2-4}
        \multicolumn{1}{c|}{} & 6h 43m 11s & Low & 0\% \\ \cline{2-4}
        \multicolumn{1}{c|}{} & 6h 43m 9s & Low & 0\% \\ \cline{2-4} 
        \multicolumn{1}{c|}{} & 6h 43m 2s & Low & 0\% \\ \cline{2-4}
        \multicolumn{1}{c|}{} & 6h 42m 56s & Low & 0\% \\ \cline{2-4}
        \multicolumn{1}{c|}{} & 4h 31m 14s & Low & 0\% \\ \cline{2-4}
        \multicolumn{1}{c|}{} & 4h 31m 6s & Low & 0\% \\ \cline{2-4}
        \multicolumn{1}{c|}{} & 4h 25m 1s & Low & 0\% \\ \cline{2-4}  
    Pausa di 15 minuti e mittenti falsificati & 6h 47m 1s  & Low & 0.4\% \\ \hline  
        \multicolumn{1}{c|}{} & 6h 47m 1s & Low & 0.395\% \\ \cline{2-4}
        \multicolumn{1}{c|}{} & 6h 47m 1s & Low & 0.384\% \\ \cline{2-4}
        \multicolumn{1}{c|}{} & 6h 46m 58s & Low & 0.395\% \\ \cline{2-4}
        \multicolumn{1}{c|}{} & 6h 46m 52s & Low & 0.399\% \\ \cline{2-4} 
        \multicolumn{1}{c|}{} & 6h 46m 49s & Low & 0.394\% \\ \cline{2-4}
        \multicolumn{1}{c|}{} & 6h 46m 48s & Low & 0.4\% \\ \cline{2-4}
        \multicolumn{1}{c|}{} & 6h 46m 47s & Low & 0.402\% \\ \cline{2-4}
        \multicolumn{1}{c|}{} & 6h 46m 42s & Low & 0.399\% \\ \cline{2-4}
        \multicolumn{1}{c|}{} & 4h 30m 53s & Low & 0\% \\ \cline{2-4}
        \multicolumn{1}{c|}{} & 4h 40m 16s & Low & 0\% \\ \cline{2-4} 
        \multicolumn{1}{c|}{} & 4h 29m 4s & Low & 0\% \\ \cline{2-4} 
    \noalign{\vskip 5ex} 
    \multicolumn{4}{c}{I mittenti falsificati provengono da un pool di indirizzi IP non assegnati nella rete locale} \\ 
    \noalign{\vskip 2ex} 
    \hline  
    %\cline{2-3} 
    Pausa di un ora e mittenti falsificati & 1h 3m 15s & Low & 79\% \\ \hline  
        \multicolumn{1}{c|}{} & 1h 12m 6s & Low & 79\% \\ \cline{2-4}
        \multicolumn{1}{c|}{} & 1h 15m 53s & Low & 78.5\% \\ \cline{2-4}
        \multicolumn{1}{c|}{} & 1h 8m 0s & Low & 78.5\% \\ \cline{2-4}
        \multicolumn{1}{c|}{} & 0h 59m 58s & Low & 78.2\% \\ \cline{2-4}
        \multicolumn{1}{c|}{} & 1h 4m 20s & Low & 77.9\% \\ \cline{2-4} 
        \multicolumn{1}{c|}{} & 1h 6m 33s & Low & 77.8\% \\ \cline{2-4}
        \multicolumn{1}{c|}{} & 1h 15m 39s & Low & 77.6\% \\ \cline{2-4}
        \multicolumn{1}{c|}{} & 1h 10m 51s & Low & 77.5\% \\ \cline{2-4}
        \multicolumn{1}{c|}{} & 1h 1m 21s & Low & 77.4\% \\ \cline{2-4}  
        \multicolumn{1}{c|}{} & 1h 7m 51s & Low & 77.3 \% \\ \cline{2-4} 
        \multicolumn{1}{c|}{} & 0h 55m 52s & Low & 77.2\% \\ \cline{2-4}
        \multicolumn{1}{c|}{} & 0h 0m 39s & Low & 77.2\% \\ \cline{2-4}
        \multicolumn{1}{c|}{} & 0h 56m 11s & Low & 77.1\% \\ \cline{2-4} 
        \multicolumn{1}{c|}{} & 0h 58m 13s & Low & 76.8\% \\ \cline{2-4} 
        \multicolumn{1}{c|}{} & 1h 11m 32s & Low & 76.3\% \\ \cline{2-4}
        \multicolumn{1}{c|}{} & 1h 1m 24s & Low & 76.3\% \\ \cline{2-4}
        \multicolumn{1}{c|}{} & 0h 58m 12s & Low & 76.2\% \\ \cline{2-4}
        \multicolumn{1}{c|}{} & 0h 55m 57s & Low & 76.1\% \\ \cline{2-4}
        \multicolumn{1}{c|}{} & 0h 59m 11s & Low & 76\% \\ \cline{2-4} 
        \multicolumn{1}{c|}{} & 1h 7m 16s & Low & 76\% \\ \cline{2-4} 
        \multicolumn{1}{c|}{} & 0h 59m 10s & Low & 76\% \\ \cline{2-4}
        \multicolumn{1}{c|}{} & 1h 1m 25s & Low & 76\% \\ \cline{2-4} 
        \multicolumn{1}{c|}{} & 1h 0m 28s & Low & 75.9\% \\ \cline{2-4} 
        \multicolumn{1}{c|}{} & 1h 0m 7s & Low & 75.8\% \\ \cline{2-4}
        \multicolumn{1}{c|}{} & 0h 59m 36s & Low & 75.6\% \\ \cline{2-4} 
        \multicolumn{1}{c|}{} & 1h 6m 20s & Low & 75.5\% \\ \cline{2-4} 
        \multicolumn{1}{c|}{} & 0h 56m 23s & Low & 75.5\% \\ \cline{2-4}
        \multicolumn{1}{c|}{} & 1h 2m 11s & Low & 75.5\% \\ \cline{2-4} 
        \multicolumn{1}{c|}{} & 1h 3m 24s & Low & 75.5\% \\ \cline{2-4} 
        \multicolumn{1}{c|}{} & 0h 55m 5s & Low & 75.4\% \\ \cline{2-4}
        \multicolumn{1}{c|}{} & 0h 55m 45s & Low & 75.4\% \\ \cline{2-4} 
        \multicolumn{1}{c|}{} & 1h 0m 47s & Low &  75.3\% \\ \cline{2-4} 
        \multicolumn{1}{c|}{} & 1h 4m 9s & Low & 75.2\% \\ \cline{2-4}
        \multicolumn{1}{c|}{} & 0h 58m 10s & Low & 74.9\% \\ \cline{2-4} 
        \multicolumn{1}{c|}{} & 1h 3m 13s & Low & 74.9\% \\ \cline{2-4} 
        \multicolumn{1}{c|}{} & 0h 56m 28s & Low & 74.8\% \\ \cline{2-4}
        \multicolumn{1}{c|}{} & 0h 58m 22s & Low & 74.6\% \\ \cline{2-4} 
        \multicolumn{1}{c|}{} & 1h 22m 13s & Low & 74.6\% \\ \cline{2-4} 
        \multicolumn{1}{c|}{} & 0h 58m 28s & Low & 74.4\% \\ \cline{2-4}
        \multicolumn{1}{c|}{} & 1h 14m 30s & Low & 74.3\% \\ \cline{2-4} 
        \multicolumn{1}{c|}{} & 1h 10m 42s & Low & 74.2\% \\ \cline{2-4} 
        \multicolumn{1}{c|}{} & 1h 11m 28s & Low & 73.9\% \\ \cline{2-4}
        \multicolumn{1}{c|}{} & 1h 8m 6s & Low & 73.4\% \\ \cline{2-4} 
        \multicolumn{1}{c|}{} & 1h 1m 45s & Low & 73.4\% \\ \cline{2-4} 
        \multicolumn{1}{c|}{} & 0h 56m 26s & Low & 73.4\% \\ \cline{2-4}
        \multicolumn{1}{c|}{} & 1h 5m 41s & Low & 73.3\% \\ \cline{2-4} 
        \multicolumn{1}{c|}{} & 1h 7m 25s & Low &  73.2\% \\ \cline{2-4} 
        \multicolumn{1}{c|}{} & 1h 4m 26s & Low & 73.2\% \\ \cline{2-4}
        \multicolumn{1}{c|}{} & 1h 9m 4s & Low & 72.9\% \\ \cline{2-4} 
\caption{Test con payload randomico, 1MB di dati e un delay durante l'invio} 
\label{tab:testRita:icmpEcho:payloadRANDOM:1MB:DELAY} 
\end{longtable} 
\end{center}  
%
\begin{center} 
\begin{longtable}{|p{0.25\textwidth}|c|c|c|} 
    \multicolumn{4}{c}{\textbf{payload RANDOM 1MB NO DELAY}} \\
    \noalign{\vskip 2ex}
    %\cline{2-3}
    \hline
    \textbf{Covert Channel} & \textbf{Connessione più lunga} & \textbf{Etichetta} & \textbf{Beacon Score} \\
    \hline 
    \noalign{\vskip 2ex}
    %\cline{2-3}
    \hline  
    Pausa di un ora e mittente reale \footnote{il dato si riferisce ad una connessione TCP} & 0h 0m 3s & High & 100\% \\ \hline 
    Pausa di 30 minuti e mittente reale & NP & NP & NP \\ \hline
    Pausa di 15 minuti e mittente reale & NP & NP & NP \\ \hline
    \noalign{\vskip 5ex} 
    \hline  
    %\cline{2-3} 
    Pausa di un ora e mittenti falsificati & 0h 0m 3s & High & 100\% \\ \hline    
    Pausa di 30 minuti e mittenti falsificati & 0h 0m 1s & High & 100\% \\ \hline  
    Pausa di 15 minuti e mittenti falsificati & NP  & NP & NP \\ \hline  
    \caption{Test con payload randomico, 1MB di dati e nessun delay durante l'invio} 
\label{tab:testRita:icmpEcho:payloadRANDOM:1MB:NO DELAY} 
\end{longtable} 
\end{center}  
%
\begin{center} 
\begin{longtable}{|p{0.25\textwidth}|c|c|c|} 
    \multicolumn{4}{c}{\textbf{payloadRANDOM-100KB-DELAY}} \\
    \noalign{\vskip 2ex}
    %\cline{2-3}
    \hline
    \textbf{Covert Channel} & \textbf{Connessione più lunga} & \textbf{Etichetta} & \textbf{Beacon Score} \\
    \hline 
    \noalign{\vskip 2ex}
    %\cline{2-3}
    \hline  
    Pausa di un ora e mittente reale & NP & NP & NP \\ \hline 
    Pausa di 30 minuti e mittente reale & NP & NP & NP \\ \hline
    Pausa di 15 minuti e mittente reale & NP & NP & NP \\ \hline
    \noalign{\vskip 5ex} 
    \hline  
    %\cline{2-3} 
    Pausa di un ora e mittenti falsificati & NP & NP & NP \\ \hline    
    Pausa di 30 minuti e mittenti falsificati & NP & NP & NP \\ \hline  
    Pausa di 15 minuti e mittenti falsificati & NP  & NP & NP \\ \hline  
    \caption{Test con payload randomico, 100KB di dati e un delay durante l'invio} 
\label{tab:testRita:icmpEcho:payloadRANDOM-100KB-DELAY} 
\end{longtable} 
\end{center}  
%
\begin{center} 
\begin{longtable}{|p{0.25\textwidth}|c|c|c|} 
    \multicolumn{4}{c}{\textbf{payloadRANDOM-100KB-NODELAY}} \\
    \noalign{\vskip 2ex}
    %\cline{2-3}
    \hline
    \textbf{Covert Channel} & \textbf{Connessione più lunga} & \textbf{Etichetta} & \textbf{Beacon Score} \\
    \hline 
    \noalign{\vskip 2ex}
    %\cline{2-3}
    \hline  
    Pausa di un ora e mittente reale & NP & NP & NP \\ \hline 
    Pausa di 30 minuti e mittente reale & NP & NP & NP \\ \hline
    Pausa di 15 minuti e mittente reale & NP & NP & NP \\ \hline
    \noalign{\vskip 5ex} 
    \hline  
    %\cline{2-3} 
    Pausa di un ora e mittenti falsificati & NP & NP & NP \\ \hline    
    Pausa di 30 minuti e mittenti falsificati & 0h 0m 14s & None & 28.3\% \\ \hline  
    Pausa di 15 minuti e mittenti falsificati & NP  & NP & NP \\ \hline  
    \caption{Test con payload randomico, 100KB di dati e nessun delay durante l'invio} 
\label{tab:testRita:icmpEcho:payloadRANDOM-100KB-NODELAY} 
\end{longtable} 
\end{center}  
%
Dai risultati ricaviamo che l'utilizzo di un singolo mittente per l'esfiltrazione dei dati comporta un'elevata probabilità di essere 
rilevati da parte di strumenti come RITA. 
Nella tabella [Tabella.\ref{tab:testRita:icmpEcho:payloadFISSO:1MB:DELAY}] si può notare che, con qualunque periodo di pausa, l'etichetta 
assegnata risulta in una gravitamedia con un tasso di beaconing che si abbassa man mano che il tempo di pausa tra un nuovo invio cresce 
\footnote{Per il test che aspetta solo 15 minuti RITA riporta un tasso di beaconing dello 0\%. Dato ritenuto non plausibile siccome negli altri test associa un tasso del 37.9\% e del 14.7\%}. 
\vspace{2ex} \newline 
Invece tramite l'utilizzo di mittenti falsificati [Tabella.\ref{tab:testRita:icmpEcho:payloadFISSO:1MB:DELAY}], si nota come l'etichetta 
assegnata risulta sempre "Low" con tassi di beaconing bassi. Ciò indica che l'uso di indirizzi IP falsificati aiuta a mitigare il rischio 
di rilevamento. 
%specialmente quando si utilizzano pause lunghe tra un invio e l'altro. Infatti, in tutti i test condotti con 
%mittente reale, le connessioni più lunghe sono state etichettate come "High" o "Medium", con punteggi di Beacon 
%Score molto elevati, indicando una forte probabilità di attività sospetta.
\vspace{1ex} \newline
Inoltre si nota che l'implementazione dei mittenti falsificati non è ottimale, siccome per ogni pacchetto si sceglie un 
mittente diverso presente nel pool. Un implementazione in cui per ogni blocco si utilizza un singolo mittente scelto dal pool, risulterebbe 
maggiormente efficiente e meno rilevabile.  
\vspace{4ex} \newline 
Per quanto riguarda la presenza o meno di un delay fra l'invio di un blocco di porzione di dati e l'altro, non si nota che nessun test 
viene rilevato da RITA. Questo significa che RITA, analizzando i log creati da Zeek, non ha inserito nel suo database la comunicazione. 
Ciò può essere dovuto al fatto che il traffico generato risulti veloce e che quindi la soglia di rilevamento risulti bassa. 
Tuttavia, l'assenza di un delay, comporta un maggior utilizzo della banda di rete che porta alla creazione di rumore da parte del Covert Channel. 
Siccome il traffico potrebbe ostruire il normale flusso di rete. 





