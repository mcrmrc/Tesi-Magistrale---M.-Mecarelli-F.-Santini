%Prima di iniizare a testare RITA, è necessario configurarlo. 
%Questo passaggio è fondamentale per il corretto funzionamento del sistema. %di rilevamento delle intrusioni. 
%Senza una configurazione appropriata, il sitema non avrebbe rilevato l'attaccante in modo efficace.

Sebbene per i test verrà utilizzato RITA con le configurazioni di default, si studierà la struttura del file 
di configurazione per compèrendere al meglio come funziona RITA e strutturate dei test migliori. 
%in modo da poter eventualmente modificare i parametri in futuro per migliorare le capacità di rilevamento di RITA.
\vspace{1ex} \newline
La configurazione di RITA avviene tramite il file di configurazione \textbf{config.hjson} che si trova nella 
cartella \textbf{/etc/rita/}. In questo file si possono settare diversi parametri che influenzano il funzionamento di RITA.

Alcuni campi presenti sono:
\begin{itemize}
    \item \textbf{filtering}: 
    Il campo imposta i filtri che RITA utilizzerà durante l'analisi dei log di Zeek [Tabella \ref{tabella:RITA:config:filtering}]. 
    %# These are filters that affect the import of connection logs. They
    %# currently do not apply to dns logs.
    %# A good reference for networks you may wish to consider is RFC 5735.
    %# https://tools.ietf.org/html/rfc5735#section-4
    \item \textbf{scoring}: 
    In questo campo si può impostare come RITA assegnerà i punteggi ad ogni tipologia di comunicazione 
    sospetta [Tabella \ref{tabella:RITA:config:scoring}]. 
    %In particolare definirà in base a quali parametri verràdefinito un beacon, 
    %il punteggio che verrà assegnato in base alla durata della connessione ed altri. 
    %\item 
\end{itemize} 
%
\begin{center} 
\begin{longtable}{|p{0.35\textwidth}|p{0.6\textwidth}|} 
    \hline
    \textbf{Sottocampo} & \textbf{Descrizione} \\
    \hline
    internal\_subnets &  
    RITA analizza il traffico di rete in base alla distinzione tra host interni ed esterni.
    Il campo specifica gli indirizzi IP che si considereranno interni alla rete. 
    %In questo caso è stato lasciato il valore di default ma si poteva specifcare o direttamente l'indirizzo IP 
    %della macchina vittima o la sottorete a cui appartiene. 
    %Si può specificare l'indirizzo di una sottorete (e.g 192.168.1.0/24) ma 
    %Siccome l'attaccante e la vittima sono all'interno della stesse rete; nel campo è stato inserito solamente l'indirizzo IP della vititma. 
    %I valori di defualt sono stati rimossi. 
    \\ 
    \hline 
    always\_included\_subnets &  
    In esso vengono specificate le reti che RITA dovrà sempre includere nell'analisi. 
    %Siccome come rete interna si è specificato solo la vittima, il campo verrà lasciato al suo valore di default. 
    %Altrimenti si sarebbe potuta specificare la sottorete della macchina vittima (ovvero 192.168.1.0/24).
    %Il valore di default è un array vuoto, ma nel nostro caso si è aggiunta la sottorete della macchina vittima. 
    %Questo perchè sia l'attaccante che la vittima si trovano nella stessa sottorete: 192.168.1.0/24.
    \\
    \hline 
    filter\_external\_to\_internal &  
    Se impostato a true, indica di ingnorare tutte le comunicazioni che avvengono tra un host esterno ad 
    uno interno. Se è impostato a false, RITA analizzerà anche queste comunicazioni.
    %Ciò è importante perchè l'attaccante si trova all'interno della rete e la vittima anche.
    \\
    \hline 
\caption{Sottocampi in \textit{filtering}} 
\label{tabella:RITA:config:filtering} 
\end{longtable} 
\end{center} 
%
\begin{center} 
\begin{longtable}{|p{0.35\textwidth}|p{0.6\textwidth}|} 
    \hline
    \textbf{Ccampo} & \textbf{Descrizione} \\
    \hline
    beacon &  
    RITA computes a final beacon score (0-100) for each host-pair, based on a weighted average of four 
    independent subscores [Tabella \ref{tabella:RITA:config:beacon:sottocampi}]. 
    Each subscore individually ranges from 0 to 100. 
    Default weights all equal 0.25, so each category contributes equally.
    \\ 
    \hline 
    long\_conn\_score\_thresholds &  
    Detect C2-like long-lived TCP connections. 
    base    = 1 hour
    low     = 4 hours
    medium  = 8 hours
    high    = 12 hours
    \\
    \hline 
    c2\_score\_thresholds &  
    Used in DNS analysis.
    \\
    \hline 
    strobe\_impact &  
    Any strobe hit or threat-intel match is automatically placed in category = high.
    \\
    \hline 
    %threat\_intel\_impact &  
    %
    %\\
    %\hline 
\caption{Campi in \textit{scoring}} 
\label{tabella:RITA:config:scoring} 
\end{longtable} 
\end{center} 
%
\begin{center} 
\begin{longtable}{|p{0.35\textwidth}|p{0.6\textwidth}|} 
    \hline
    \textbf{Ccampo} & \textbf{Descrizione} \\
    \hline
    Timestamp Score &  
    Focus on the regularity of intervals between connections (the essence of beaconing).
    Looks at inter-arrival times.
    Low variance means more beacon-like and so higher score.
    Has no special config fields beyond the weight.
    \\ 
    \hline 
    Data Size Score &  
    Focus on the consistency in packet sizes. 
    If the data transferred each time is very similar it will have an higher score. 
    If sizes fluctuate heavily it will have lower score. 
    Weight configurable via datasize\_score\_weight. 
    \\
    \hline 
    Duration Score &  
    This score is only calculated if: hours\_seen > duration\_min\_hours\_seen 
    Default: duration\_min\_hours\_seen = 6 
    So if the client and the server communicate for less than 6 hours total (across the dataset), 
    duration scoring is skipped or heavily penalized.
    \vspace{1ex} \newline
    Consistency ideal:  duration\_consistency\_ideal\_hours\_seen = 12
    If beacon appears over $\geq$12 hours, the consistency subscore can be 100.
    \\
    \hline 
    Histogram Score &  
    The histogram score tries to detect: 
    Mode flatness (how uniform the traffic per time bucket is) 
    Bimodality (are there two “peaks”? Often malware behavior) 
    Subsettings: 
    a) histogram\_mode\_sensitivity (default 0.05) 
    Defines bucket width as:  bucket\_size = max\_connection\_count * histogram\_mode\_sensitivity 
    Higher sensitivity means the algorithm more tolerant to variation.
    b) histogram\_bimodal\_outlier\_removal (default 1) 
    How many “outlier buckets” can be removed to avoid false flags. 
    c) histogram\_bimodal\_min\_hours\_seen (default 11) 
    If the conversation spans fewer than 11 hours, the bimodal subscore is skipped (avoiding noisy small samples).
    \\
    \hline 
\caption{Sottocampi in \textit{beacon}} 
\label{tabella:RITA:config:beacon:sottocampi} 
\end{longtable} 
\end{center} 
After computing the final 0–100 score, RITA assigns the connection to a category:
Final Score	Category	Meaning
< 50	insignificant	probably not a beacon
50–69	base	weak suspicion
70–89	low	suspicious
90–99	medium	strong suspicion
100	high	extremely strong, very regular beacon
\vspace{1ex} \newline
A score of 100 generally means:
perfect timing regularity
very stable packet sizes
long and consistent duration
histogram strongly suggests periodic behavior



