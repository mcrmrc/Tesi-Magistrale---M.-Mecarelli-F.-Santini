\subsection{Covert Channel} 
Un \textbf{Covert Channel} è un attacco che permette (in ambienti ritenuti sicuri) la capacità di comunicare e/o trasferire dati in maniera non autorizzata e non voluta.  
%Ciò avverrà fra processi e/o entità comunicanti senza che vengano rivelati ed evitando (se non violando) le normali politiche di sicurezza. 
Solitamente operano al di fuori degli usuali meccanismi di comunicazioni sfruttando vulnerabilità o comportamenti non previsti nei sistemi. 
%Non usando i normali protocolli e/o canali di comunicazione (es network sockets, emails) 
%ciò gli permette di non generare segnali di un uso improprio del sistema. 
Ciò gli permette di non generare segnali di un uso improprio del sistema ed inoltre, 
nascondendosi all'interno dei normali processi del sistema, sono difficili da rilevare e/o identificare. 
%Questi attacchi sono un problema significativo in tutti quegli ambienti dove una fuoriuscita 
%di informazioni può avere conseguenze gravi (es ambienti militari, governativi,\dots).  
%Siccome l'suo imporprio che si fà di queste risorse: porta alla fuoriuscita (o scambio) dei dati. 
%
%In generale il canale è incorporato in operazioni di sistema legittime per poter mascherare la trasmissione dei dati. 
%(e.g carico della CPU, accesso alla memoria, traffico della rete, metadati del file systema). 
%sere usate (in maniere non previste) per la comunicazione. %(e.g memoria condivisa, uso della CPU, attributi dei file)
%Questo perchè, permetterebbero l
%    Di solilto si sfruttano servizi e/o risorse già presenti e quindi non sospette; 
%    uno dei maggiori problemi nell'implementazione di un canale nascosto è il “rumore” che potrebbe attirare l'attenzione 
%    da parte degli amministratori (es. se si sfruttano eccessivamente le risorse). 
%    La necessità è quella di riuscire a trasmette attraverso un canale nascosto mantenendo conforme e inalterato il 
%    funzionamento della risorsa utilizzata così da rendersi “indistinguibili” dalla risorsa autorizzata e di 
%    conseguenza invisibili ai sistemi di monitoraggio.
%    Per evitare la rilevazione, il canale è incorporato in operazioni di sistema legittime per poter mascherare la trasmissione dei dati. 
%    (e.g carico della CPU, accesso alla memoria, traffico della rete, metadati del file systema). 
%\vspace{2ex} \newline
%Per evitare la presenza dei Covert Channel, bisogna quindi prestare attenzione all'\textbf{uso involontario delle risorse} 
%ed evitare che esse possano eso scambio non non autorizzato di informazioni; 
%potenzialmente violando le politiche di sicurezza; 
%nelle quali i principali requisiti sono: la confidenzialità dei dati, l'integrità di quest'ultimi o la disponibilità delle risorse e/o servizi. 
%Di conseguenza un metodo per la \textbf{rilevazione delle violazioni} delle politiche di sicurezza, risulta necessaria. 
\vspace{2ex} \newline 
Qualsiasi risorsa condivisa può essere utilizzata per la creazione di un canale nascosto. 
%E per questo la loro esistenza rappresenta un problema che spesso rimane non notato. %dagli amministratori o dai tipici strumenti di monitoraggio. 
%Questo permetterà ai Covert Channel di esistere in qualsiasi sistema. 
E per poter essere efficace, un Covert Channel deve possedere determinate caratterisitche [Tabella \ref{tabella:caratteristiche:covert:channel} ]. 
L'\textbf{indistinguibilità} è la principale; è estremamente importante riuscire a trasmettere informazioni mantenendo conforme lo stato del sistema. 
L'obbiettivo è rendere il canale indistinguibile rispetto alle altre risorse presenti nel sistema così da risultare invisibili ai sistemi di monitoraggio. 
\begin{center} 
\begin{longtable}{|p{0.4\textwidth}|p{0.4\textwidth}|} 
    \hline
    \textbf{Caratteristica} & \textbf{Descrizione} \\
    \hline
    \textbf{Furtività} & 
        Evitare di attirare le attenzioni sia degli amministratori che degli strumenti utilizzati per il rilevamento degli attacchi. 
        %Si devono poter aggirare i controlli in maniera nascosta.
    \\
    \hline 
    \textbf{Capacità di trasmissione} &
        %Espressa in termini di \textbf{throughput} ($\frac{dati}{tempo}$). 
        Più dati il canale trasmette per un determinato intervallo di tempo, 
        maggiore sarà il rischio che venga scoperto.  
        %siccome potrebbe rendere anomalo il funzionamento delle altre risorse.
        %e un valore maggiore permetterà di scambiare un maggior numero di informazioni. 
        %Evitare trasmissione risultasse anomala o eccessiva, un eccessivo carico di informazioni, 
        %potrebbe rendere anomalo il funzionamento delle risorse.
        %Quindi se la lunghezza di banda (Bandwith) risultasse limitata;  
        %Quindi il throughput è inversamente correlato alla segretezza di un canale: 
        %più dati un canale trasmette in un determinato periodo di tempo, maggiore è il rischio che il canale venga scoperto
    \\
    \hline
    \textbf{Uso delle risorse} &
        Un uso delle risorse improprio o sproporzionato %aumenta il rischio di essere individuati. 
        da parte del canale potrebbe andare in conflitto con le risorse legittime presenti nel sistema. 
        %I Covert channels sfruttano le risorse del sistema (e.g memoria cxondivisa, uso della CPU, attributi dei file) 
        %in maniere non previste per la comunicazione. 
    \\
    \hline
    \textbf{Rumore} &
        L'uso dei servizi e/o delle risorse nel sistema potrebbe alterare il loro funzionamento. 
        %L'alterazione del comportamento della risorsa o del servizio sfruttato 
        Ciò potrebbe attirare l'attenzione da parte degli amministratori. 
    \\
    \hline
    \textbf{Indistinguibilità} & 
        %Si ha la necessità di riuscire a 
        La trasmissione dei dati mantiene conforme e inalterato il funzionamento delle risorse utilizzate. 
        L'obbietivo è quello di rendersi indistinguibili dalla risorsa autorizzata. 
        %e di conseguenza invisibili ai sistemi di monitoraggio.
    \\ 
    \hline
    %Violazione delle politiche di sicurezza &
    %    Permettono lo scambio non autorizzato di informazioni, potenzialmente violando 
    %    i requisiti di confidenzialità, di integrità o quelli di disponibilità.  
    %\\
    %\hline 
\caption{Caratterisitche di un Covert Channel} 
\label{tabella:caratteristiche:covert:channel} 
\end{longtable} 
\end{center} 


\subsubsection{Tipologie di Covert Channel \cite{covert_channel_tipologie1} \cite{covert_channel_tipologie2}}

\subsubsection*{Timing Covert Channel}
I covert channel di temporizzazione sono metodi di comunicazione che permettono ad un osservatore (un umano o un processo) 
di acquisire informazioni attraverso il cambiamento del tempo di risposta di una risorsa. 
%Sfruttando gli intervalli di tempo o l'ordine degli eventi per codificare informazioni (e.g. ritardi fra i pacchetti di rete,\dots). 
Qualsiasi metodo che utilizza un orologio (o una misurazione del tempo) per segnalare il valore può implementarlo.
%\begin{esempio}
%        Modificare i permessi dei file o i metadati per codificare informazioni oppure modificare variabili condivise o buffer.
%    \end{esempio}
%\href{https://www.youtube.com/watch?v=QIvsmQQ6vu8}{Video Esempio}  

\subsubsection*{Storage Covert Channel}
Il canale viene creato scrivendo dei dati su un'area di memoria condivisa accessibile da tutte entità presenti. %entrambi i processi. 
%(e.g attributi del file, i bit di memoria, gli stati della cache,\dots) 
%Coinvolgono quindi la scrittura di dati su un'area di memoria condivisa accedibile da entrambi i processi  
I possibili veicoli saranno tutte quelle risorse che consentiranno la scrittura (diretta o indiretta) da parte di un processo e 
la lettura (diretta o indiretta) da parte di un altro. 
%Quindi in uno Storage Covert channel, un processo scrive su una risorsa condivisa mentre un altro processo legge da essa. 
%Nei covert channel di archiviazione un processo scrive su una risorsa condivisa, mentre un altro processo legge da essa. 
%Possono essere quindi utilizzati da processi all'interno di un singolo computer o tra più computer in una rete.
%\vspace{2ex} \newline
%\begin{esempio}
%    Un esempio di un canale di archiviazione è la condivisione di un file. 
%    Supponendo che l'utente A (con privilegi sufficenti) voglia trasmettere segretamente dei dati riservati all'utente B (con un livello di sicurezza inferiore). 
%    Utilizzando un file di testo in cui apparentemente verranno scritte informazioni non classificate, mentre in verità occulterà l'informazione riservata. 
%\end{esempio}
%\vspace{2ex} 
%\begin{esempio}
%    Variare deliberatamente il tempo fra delle azioni (es trasmissione di network packet, patter di uso della CPU) 
%    oppure codificando dati nella temporalizzazione dell'esecuzione dei processi o delay di risposta. 
%\end{esempio} 

\subsubsection*{Behavioural Covert Channel} 
%I canali nascosti comportamentali operano estraendo 
Le informazioni vengono codificate modificando il comportamento del sitema. 
Quindi i dati vengono ricavati osservando il comportamento del sistema 
%durante l'esecuzione di operazioni, 
piuttosto che attraverso l'accesso diretto ai dati.
%La trasmissione dei dati avviene quindi in base ai diversi eventi che avvengono e 
%generalmente suddividendo e trasmettendo i dati in pacchetti più piccoli. 
%Operano quindi trasmettendo dati in base all'avvenimento di diversi eventi. 
%di processi, sistemi e applicazioni, generalmente suddividendo e trasmettendo i dati in pacchetti più piccoli. 


\subsubsection*{Hybrid Covert Channel} 
Un Hybrid Covert Channel combina più tecniche per aumentare la capacità di trasmissione dei 
dati nascosti e rendere più difficile la loro rilevazione. 

