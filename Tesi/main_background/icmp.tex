\subsection{ICMP \cite{icmp}}
ICMP (Internet Control Message Protocol) è un protocollo che opera al livello di rete (livello 3 nel modello ISO/OSI). 
e permette la \textbf{segnalazione errori}, la \textbf{diagnostica di rete} e la \textbf{messaggistica di controllo}. 
%utilizzato per la diagnostica, per la segnalazione di errori, per ottenere informazioni di controllo
Proprio per questo viene utilizzato per il monitoraggio dello stato di una rete e per la risoluzione dei problemi che avvengono in essa. 
%A differenza di altri protocolli (eg TCP/IP, UDP,\dots) non è utilizzato per la trasmissione di dati e di conseguenza non stabilirà una connessione e non presenterà un numero di porta specifico. 
%Si avrà quindi una comunicazione Stateless e Connectionless che può avvenire senza specificare alcun tipo di porta. 
%
%I principali strumenti di rete che sfruttano questo protocollo sono: \textbf{Ping} e \textbf{Traceroute}
\vspace{4ex} \newline 
In un \textbf{Covert Channel ICMP} (Internet Control Message Protocol) verranno utilizzati i messaggi ICMP %(di solito richieste e risposte Eco) 
per nascondere i dati. 
%Principalmente all'interno dei campi che normalmente vengono ignorati o non monitorati. 
%Il canale sarà possibile siccome il protocollo ICMP consente agli attaccanti di trasferire dati 
%aggirando le politiche di sicurezza ed eludendo il rilevamento.
L'implementazione del canale è possibile siccome è un protocollo che, dati i suoi utilizzi [Tabella \ref{tabella:utilizzi:ICMP} ], 
non può essere del tutto disabilitato. 
Molti firewall e dispositivi di sicurezza consentono il traffico ICMP. %per la diagnostica della rete. 
%Siccome ICMP è spesso consentito nei firewall, gli aggressori lo utilizzano per aggirare le politiche di sicurezza. 
%In sinergia con il protocollo IP, permette di informare il mittente 
%sui problemi di rete, di aiutare nella risoluzione dei problemi di rete e di gestire la congestione della rete. %oltre gli aggiornamenti di routing (in alcuni casi)
%I pacchetti ICMP possono trasportare dati (payload) nascosti senza destare sospetti. 
Tuttavia, sebbene sia essenziale per la diagnostica di rete, %e la segnalazione degli errori, 
può essere comunque utilizzato in modo improprio per degli attacchi o per la ricognizione della rete. 

%Data la sua necessita [Tabella \ref{tabella:utilizzi:ICMP} ]  per la diagnostica di rete e la segnalazione degli errori, 
%Può quindi essere utilizzato in modo improprio per mettere a segno degli attacchi o per studiare la rete (ricognizione della rete). %network reconnaissance
\begin{center} 
\begin{longtable}{|p{0.4\textwidth}|p{0.4\textwidth}|} 
    \hline
    \textbf{Utilizzi} & \textbf{Descrizione} \\
    \hline
    Diagnostica della rete & 
    %Utilizzando strumenti come ping e traceroute 
    %Crea l'infrastruttura per strumenti diagnostici di rete di base come ping e traceroute 
    Il protocollo fornisce metriche critiche per il monitoraggio continuo delle prestazioni della rete 
    %(es tempo di risposta, pacchettiu persi, banda utilizzata, \dots)
    %Response times: Measurement of network latencies
    %Packet Losses Detection of transmission failures
    %Bandwidth Utilisation: Analysing network traffic
    %Availability Status: Control of the operating status of the systems 
    %
    %Traceroute: Used to determine the path packets take across routers to reach the destination. 
    %Ping: Sends echo-request and echo-reply messages to measure round-trip time and test connectivity.
    \\
    \hline 
    Segnalazione di errori & 
    %ad esempio, destinazione non raggiungibile, perdita di pacchetti 
    %Rileva e segnala errori di comunicazione tra dispositivi sulla rete 
    Il protocollo rileva e segnala i problemi riscontrati durante la trasmissione dei dati tra i dispositivi sulla rete
    %(Destination Unreachable Notifications, TTL Overrun Messages, Parameter Problems, Parameter Problems)
    %Destination Unreachable Notifications: When packets fail to reach their destination
    %TTL Overrun Messages: When the lifetime of the packages expires
    %Parameter Problems: When an error is detected in the IP header
    %Source Suppression: In cases of network congestion
    %
    %Error reporting: When data packets cannot reach their destination due to issues such as unreachable hosts, timeouts or fragmentation errors. 
    %If a message cannot be delivered, ICMP informs the source about the failure. 
    %Example: If a packet is too large and cannot be forwarded, the receiver drops the packet and sends an ICMP error message to the sender.
    \\
    \hline
    Risoluzione dei problemi & 
    Gli amministratori di rete possono ricevere avvisi in tempo reale e rispondere rapidamente ai problemi di rete grazie agli strumenti di monitoraggio basati su ICMP.
    %Questi strumenti consentono una gestione proattiva della rete fornendo notifiche istantanee sullo stato delle applicazioni e dei servizi web critici. 
    \\
    \hline
    Ottenimento di informazioni & 
    Può inviare messaggi senza la necessità di una connessione preventiva permettendo così di ottenere informazioni. %tramite la messagistica di controllo 
    \\
    \hline
    %detect and report errors on the network &  The protocol checks whether the data reaches the destination on time and transmits information to the sending device in case of any problems \\
    %
\caption{Utilizzi del protocollo ICMP} 
\label{tabella:utilizzi:ICMP} 
\end{longtable} 
\end{center} 
%Nei seguenti capitoli verrà illustrato come può essere sfruttato per la creazione di un Covert Channel; 
%un attacco che permette (in ambienti ritenuti sicuri) la capacità di comunicare e/o trasferire dati in maniera non autorizzata e non voluta. 
%L'attacco opera al di fuori degli usuali meccanismi di comunicazioni %Inoltre sfruttando vulnerabilità o comportamenti non previsti nei sistemi,  
%e per questo risulta difficile da rilevare e/o identificare. 
%Sia dagli amministratori che dai tipici strumenti di monitoraggio. 
%Infine, siccome qualsiasi risorsa condivisa può essere utilizzata per la sua creazione, può esistere in qualunque sistema. 
%\vspace{4ex} %\newline
%\subsection{Struttura di un pacchetto ICMP} 
I messaggi ICMP vengono inviati utilizzando l'intestazione IP di base. 
In essi i primi venti byte indicano l'intestazione IP mentre il primo ottetto, della porzione dati del datagramma, riguarda l'intestazione ICMP. 
%I campi relativa al protocollo ICMP sono i seguenti:
Nell'intestazione, nel caso del protocolllo ICMP, il campo \textit{protocol} avrà valore 1
%Qualsiasi campo etichettato come "non utilizzato" è riservato per future estensioni e deve essere zero quando 
%inviato, ma i destinatari non dovrebbero utilizzare questi campi (eccetto per includerli nel checksum). 
\newline
\begin{minipage}{\textwidth}
    \centering
    \includegraphics[width=0.7\textwidth]{./img/ICMP-packet-structure.png}
    \captionof{figure}{Struttura pacchetto ICMPv4/IPv4 \cite{img_icmp_packet_structure}} 
\end{minipage} 
\begin{itemize} 
    \item \textbf{Type}: 
    Identifica la tipologia di messaggio. 
    %(ad esempio, Echo Request, Destinazione irraggiungibile). 
    In base a questo campo verrà determinato il formato dei rimanenti dati. 
    \item \textbf{Code}: Fornisce dettagli aggiuntivi sulla tipologia di messaggio.
    \item \textbf{Checksum}: 
    %Complemento a 16 bit relativo alla somma del messaggio ICMP. 
    Garantisce l'integrità dei dati. 
    \item \textbf{Data}: Campo opzionale, può contenere ulteriori dati.
    %può contenere la parte del pacchetto IP originale che ha causato il messaggio o altre tipologie di dati. 
\end{itemize} 


\subsubsection{Tipologie di Messaggi ICMP \cite{icmpv4_message_type} \cite{icmpv6_message_type}}
I messaggi presenti in ICMP sono classificati o come messaggi di errore o come messaggi informativi. 
I primi segnalano problemi nella comunicazione di rete 
%(Destination Unreachable, source Quench, Redirect Message,Time Excedeed, Parameter Problem, Packet too Big) 
mentre i secondi vengono utilizzati per scopi diagnostici e di controllo. 
%(Echo Request, Timestamp Request, information Request) 
%Di seguito la struttura delle varie tipolgie. 

\subsubsection*{Destination Unreachable} 
%Un messaggio \textit{Destination Unreachable}, viene inviato quando la rete specificata 
%(nel campo di destinazione) è irraggiungibile. E quindi 
Viene inviato quando il pacchetto non può essere recapitato alla destinazione specificata nell'header IP. 
Di solito perchè il percorso definito non può essere seguito. 
%(es la rete irraggiungibile, l'host è irraggiungibile, il protocollo o la porta di destinazione non attivi). 
%risposta a un pacchetto che non può essere recapitato (alla sua destinazione) per motivi diversi dalla congestione. 
%Di conseguenza il gateway potrà inviare questa tipologia di messaggio all'host mittente del pacchetto. 
%Altri possibili casi per cui verrà inviato è se l'host è irraggiungibile o se il protocollo indicato o 
%la porta di destinazione specificati non sono attivi.  
%il pachcetto deve venrire frammetnato per poterlo inoltrare ad un gateway, \dots 
Il messaggio non verrà (e non dovrà essere) generato se un pacchetto viene scartato a causa della congestione del traffico. 
%Inoltre un nodo che riceve un messaggio \textit{Destination Unreachable} deve notificare l'evento al 
%processo di livello superiore (se il processo in questione può essere identificato). 
Il codice impostato nel messaggio indicherà il motivo [Tabella \ref{tabella:caso:DestUnreach}]
\begin{center} 
\begin{longtable}{|p{0.25\textwidth}|p{0.55\textwidth}|} 
    \hline
    \textbf{Codice} & \textbf{Descrizione} \\
    \hline
    Rete irraggiungibile &  
    %Il percorso specificato nell'header IP non può essere seguito. 
    La rete di destinazione non è raggiungibile. %(es. routing mancante). 
    %Il router non ha un percorso verso la rete di destinazione. 
    %La rete di destinazione non è conosciuta dal router, 
    %la comunicazione è bloccata da un firewall %(in generale da ACL/firewall).
    %oppure non esiste un percorso verso di essa.
    \\
    \hline 
    Host irraggiungibile &  
    Il router può raggiungere la rete ma non l'host specifico.
    siccome non risponde o non è raggiungibile. %(è isolato). 
    %L'host di destinazione non è conosciuto. \newline %(es. mancanza di entry ARP). 
    %La comunicazione con l'host di destinazione è proibita per motivi amministrativi.
    %L'host non è raggiungibile per il servizio scelto. %Type of Service specificato.
    \\
    \hline
    Protocollo non attivo & 
    %Il protocollo richiesto non è supportato. %sull'host. \newline
    %Il protocollo IP richiesto (es. TCP, UDP) non è supportato sull'host. 
    %La rete o l'host non supporta il servizio richiesto (ToS).  
    Il protocollo specificato nel pacchetto IP non è attivo. %o non è supportato dall'host di destinazione.
    %La rete non è raggiungibile per il Type of Service specificato.
    \\
    \hline
    Porta non attiva & 
    La porta di destinazione non è attiva o nessun servizio è in ascolto.  
    %La porta di destinazione non è attiva o non è in ascolto. Nessuna applicazione sta ascoltando su quella porta UDP.
    %La comunicazione è proibita per policy/filtri di sicurezza.
    \\
    \hline 
    Necessaria frammentazione & 
    %Il pacchetto richiede frammentazione, ma il flag Don’t Fragment è impostato.
    È necessaria la frammentazione del pacchetto %per attraversare un link 
    ma il flag "Don't Fragment" (DF) è impostato. %nel pacchetto IP.
    \\
    \hline 
\caption{Destination Unreachable possibili codici} 
\label{tabella:caso:DestUnreach} 
\end{longtable} 
\end{center} 
Nel protocollo \textbf{ICMPv4} il pacchetto è strutturato in questo modo: 
\vspace{2ex} \newline 
\begin{bytefield}[bitwidth=1.1em]{32} 
    %\bitbox{8}{0} & \bitbox{8}{1} & \bitbox{8}{2} & \bitbox{8}{3} \\
    \bitheader{0-31} \\
    \bitbox{8}{Type=3 (1 byte)} & \bitbox{8}{Code=0-5 (1 byte)} & \bitbox{16}{Checksum (2 byte)} \\
    \bitbox{32}{Unused (4 byte)} \\
    \bitbox{32}{Internet Header + 64 bits of Original Datagram ($\geq$ 21 byte)} 
\end{bytefield}  
\newline
Il campo \textit{Internet Header}: viene utilizzato dall'host per accoppiare il messaggio di errore 
al processo appropriato. 
%Se un protocollo di livello superiore utilizza numeri di porta, si presume che siano nei primi 64 bit dei dati del datagramma originale. 
%\footnote{L'\textit{intestazione IP} può variare dai 20 byte ai 40 byte} 
\vspace{4ex}  \newline
Nel protocollo \textbf{ICMPv6} il pacchetto è strutturato in questo modo: 
\vspace{1ex} \newline
\begin{bytefield}[bitwidth=1.1em]{32} 
    %\bitbox{8}{0} & \bitbox{8}{1} & \bitbox{8}{2} & \bitbox{8}{3} \\
    \bitheader{0-31} \\
    \bitbox{8} {Type=1 (1 byte)} & \bitbox{8}{Code=0-6 (1 byte)} & \bitbox{16}{Checksum (2 byte)} \\
    \bitbox{32} {Unused (4 byte)} \\
    \bitbox{32} {As much of invoking packet as possible without} \\
    \bitbox{32} {the ICMPv6 packet exceeding the minimum IPv6 MTU ($\geq$ 0 byte)} 
\end{bytefield} 
\newline  
Il campo \textit{Invoking Packet} indica quanta parte del pacchetto (che ha attivato l'errore ICMPv6) debba essere inclusa. 
Il tutto senza eccedere il \textit{IPv6 MTU} il cui valore di default equivale a 1280 bytes. 

\subsubsection{Time Exceeded} 
Questa tipologia di messaggio viene usata quando il gateway che elabora un pacchetto trova che il suo TTL (tempo di vita) è zero. 
%generato se un router riceve un pacchetto con un limite di hop pari a zero,  
%o se un router decrementa il limite di hop di un pacchetto a zero. 
In questi casi il gateway dovrà scartare il datagramma e notificare l'host sorgente della cosa. 
%Ciò indica un loop nel routing o un valore iniziale della quantità di hop possibili troppo basso.
%Altri casi possibili in cui questo messaggio può avvenire è quando un host che riassembla un datagramma frammentato, 
%non riesce a completare il riassemblaggio a causa di frammenti mancanti entro il proprio limite di tempo.  
%codice 1 invece viene utilizzato per segnalare il timeout nel riassemblaggio dei frammenti. 
Il codice impostato nel messaggio indicherà il motivo [Tabella \ref{tabella:caso:TimeExceeded}]
\begin{center} 
\begin{longtable}{|p{0.25\textwidth}|p{0.55\textwidth}|} 
    \hline
    \textbf{Evento} & \textbf{Esempio} \\
    \hline
    TTL scaduto &  
    Il TTL specificato inizialmente è troppo basso o è presente un loop nel routing. 
    %Il pacchetto ha superato il numero massimo di hop (TTL=0)
    \\
    \hline 
    Tempo per il riassemblaggio scaduto & 
    Il tempo per riassemblare i frammenti scaduto. 
    %Errore nel riassemblaggio a causa di frammenti mancanti. 
    \\
    \hline 
\caption{Time Exceeded possibili codici} 
\label{tabella:caso:TimeExceeded} 
\end{longtable} 
\end{center}  
Nel protocollo \textbf{ICMPv4} il pacchetto è strutturato in questo modo: 
\vspace{2ex} \newline
\begin{bytefield}[bitwidth=1.1em]{32} 
    %\bitbox{8}{0} & \bitbox{8}{1} & \bitbox{8}{2} & \bitbox{8}{3} \\
    \bitheader{0-31} \\
    \bitbox{8}{Type=11 (1 byte)} & \bitbox{8}{Code=0-1 (1 byte)} & \bitbox{16}{Checksum (2 byte)} \\
    \bitbox{32}{Unused (4 byte)} \\
    \bitbox{32}{Internet Header + 64 bits of Original Datagram ($\geq$ 21 byte)} 
\end{bytefield} 
\newline
In ICMPv4 il campo \textit{Internet Header}: viene utilizzato dall'host per accoppiare il messaggio di errore 
al processo appropriato. 
%Se un protocollo di livello superiore utilizza numeri di porta, si presume che siano nei primi 64 bit dei dati del datagramma originale. 
%\footnote{L'\textit{intestazione IP} può variare dai 20 byte ai 40 byte} 
\vspace{4ex}  \newline
Nel protocollo \textbf{ICMPv6} il pacchetto è strutturato in questo modo: 
\vspace{1ex} \newline 
\begin{bytefield}[bitwidth=1.1em]{32} 
    %\bitbox{8}{0} & \bitbox{8}{1} & \bitbox{8}{2} & \bitbox{8}{3} \\
    \bitheader{0-31} \\
    \bitbox{8} {Type=3 (1 byte)} & \bitbox{8}{Code=0-1 (1 byte)} & \bitbox{16}{Checksum (2 byte)} \\
    \bitbox{32} {Unused (4 byte)} \\
    \bitbox{32} {As much of invoking packet as possible without} \\
    \bitbox{32} {the ICMPv6 packet exceeding the minimum IPv6 MTU ($\geq$ 0 byte)} 
\end{bytefield}  
\newline  
In ICMPv6 il campo \textit{Invoking Packet} indica quanta parte del pacchetto (che ha attivato l'errore ICMPv6) debba essere inclusa. 
Il tutto senza eccedere il \textit{IPv6 MTU} il cui valore di default equivale a 1280 bytes. 

\subsubsection{Parameter Problem} 
Viene usata quando il gateway che elabora un pacchetto  trova un problema con i parametri dell'intestazione 
in modo tale da non poter completare l'elaborazione del datagramma. 
In questo caso dovrà scartare il datagramma e notificare la cosa all'host indicando il tipo e la posizione del problema. 
%generato se un nodo che elabora un pacchetto, rileva un problema con un campo nell'intestazione IPv6 o 
%nelle intestazioni di estensione tale da impedirgli di completare l'elaborazione di esso. 
%Una potenziale sorgente di tale problema è rappresentata da argomenti non corretti in un'opzione. 
Il messaggio viene inviato solo se l'errore ha causato lo scarto del pacchetto. 
%Il codice impostato nel messaggio indicherà il motivo [Tabella \ref{tabella:caso:ParamProblem}]
%\begin{center} 
%\begin{longtable}{|p{0.25\textwidth}|p{0.55\textwidth}|} 
%    \hline
%    \textbf{Evento} & \textbf{Esempio} \\
%    \hline
%    Problema nell'header &  
    %Argomenti non corretti in un opzione. 
    %Un campo specifico dell'header IP contiene un valore non valido. 
    %Nell'header IP manca un'opzione obbligatoria. 
    %La lunghezza dell'header IP non è corretta. \newline
    %Errori nei campi dell'header come versione IP errata, checksum non valido, lunghezza header incorretta
%    Valori non validi in campi critici dell'header IP. 
%    \\
%    \hline 
%    Next Header non riconosciuto & 
%    Tipologia di Next Header non riconosciuta. 
%    \\
%    \hline 
%    Opzione non riconosciuta & %Opzione IP non riconosciuta
    %Opzione IP non supportata o non riconosciuta
%    Opzione IP obbligatoria non presenti, malformata o con lunghezza errata. 
%    \\
%    \hline 
%\caption{Parameter Problem possibili codici} 
%\label{tabella:caso:ParamProblem} 
%\end{longtable} 
%\end{center} 
Nel protocollo \textbf{ICMPv4} il pacchetto è strutturato in questo modo: 
\vspace{2ex} \newline
\begin{bytefield}[bitwidth=1.1em]{32} 
    %\bitbox{8}{0} & \bitbox{8}{1} & \bitbox{8}{2} & \bitbox{8}{3} \\
    \bitheader{0-31} \\
    \bitbox{8}{Type=12 (1 byte)} & \bitbox{8}{Code=0 (1 byte)} & \bitbox{16}{Checksum (2 byte)} \\
    \bitbox{8}{Pointer (1 byte)} & \bitbox{24}{Unused (3 byte)} \\
    \bitbox{32}{Internet Header + 64 bits of Original Datagram ($\geq$ 21 byte)} 
\end{bytefield}  
\newline
In ICMPv4 il campo \textit{Internet Header}: viene utilizzato dall'host per accoppiare il messaggio di errore 
al processo appropriato. 
%Se un protocollo di livello superiore utilizza numeri di porta, si presume che siano nei primi 64 bit dei dati del datagramma originale. 
%\footnote{L'\textit{intestazione IP} può variare dai 20 byte ai 40 byte} 
\vspace{1ex} \newline  
Il puntatore identifica l'ottetto nell'intestazione del pacchetto originale in cui è 
stato rilevato l'errore. %(può trovarsi nel mezzo di un'opzione). 
%Ad esempio, 1 indica che c'è qualcosa di sbagliato con il Tipo di Servizio, e (se sono presenti opzioni) 20 indica che 
%c'è qualcosa di sbagliato con il codice di tipo della prima opzione. 
\vspace{4ex}  \newline
Nel protocollo \textbf{ICMPv6} il pacchetto è strutturato in questo modo: 
\vspace{2ex} \newline 
\begin{bytefield}[bitwidth=1.1em]{32} 
    \bitheader{0-31} \\
    \bitbox{8} {Type=4 (1 byte)} & \bitbox{8}{Code=0-2 (1 byte)} & \bitbox{16}{Checksum (2 byte)} \\
    \bitbox{32} {Pointer (4 byte)} \\
    \bitbox{32} {As much of invoking packet as possible without} \\
    \bitbox{32} {the ICMPv6 packet exceeding the minimum IPv6 MTU ($\geq$ 0 byte)} 
\end{bytefield}
\newline 
In ICMPv6 il campo \textit{Invoking Packet} indica quanta parte del pacchetto (che ha attivato l'errore ICMPv6) debba essere inclusa. 
Il tutto senza eccedere il \textit{IPv6 MTU} il cui valore di default equivale a 1280 bytes. 
\vspace{1ex} \newline  
Il puntatore identifica l'ottetto nell'intestazione del pacchetto originale in cui è 
stato rilevato l'errore.
%Ad esempio, 1 indica che c'è qualcosa di sbagliato con il Tipo di Servizio, e (se sono presenti opzioni) 20 indica che 
%c'è qualcosa di sbagliato con il codice di tipo della prima opzione. 

\subsubsection{Source Quench} 
Questa tipologia viene usata quando il gateway vuole richiedere di ridurre la velocità di invio dei pacchetti. 
%Ciò indicherà una richiesta da parte dell'host destinatario all'host mittente, di ridurre la velocità di invio dei pacchetti nel traffico. 
Questo perchè il gateway ha scartato un pacchetto ma non a causa di un errore.  
%siccome i datagrammi arrivano troppo velocemente per poter essere elaborati. 
%In questo caso invierà un messaggio di Source Quench all'host mittente.  
%Il gateway può inviare un messaggio per ogni messaggio che scarta.  
%Inoltre un host di destinazione può inviare un messaggio di questo tipo, anche nel caso in cui i datagrammi arrivino troppo velocemente per poter essere elaborati.
Al suo ricevimento, l'host sorgente dovrà ridurre la velocità sino a quando non 
riceverà più messaggi del tipo Source Quench dal gateway.
%L'host mittente potrà successivamente aumentare gradualmente la velocità con 
%cui sta inviando i pacchetti fino a quando non riceverà nuovamente questi messaggi. 
%Il codice impostato nel messaggio indicherà il motivo [Tabella \ref{tabella:caso:SourceQuench}]
%\begin{center} 
%\begin{longtable}{|p{0.25\textwidth}|p{0.55\textwidth}|} 
 %   \hline
 %   \textbf{Evento} & \textbf{Esempio} \\
 %   \hline
 %   Buffer pieno &  
    %Una motivazione per cui un gateway può scartare un pacchetto è se non ha lo spazio necessario nel buffer 
    %per mantenere in coda i pacchetti; che dovranno essere inoltrati alla rete successiva, 
    %la quale farà parte della rotta per la rete di destinazione. 
    %per l'uscita verso la rete successiva nella rotta per la rete di destinazione. 
 %   Il gateway ha esaurito lo spazio nel buffer o è vicino alla capacità massima.  
    %Il gateway non riesce a mantenere in coda i pacchetti da inoltrare alla destinazione.  
    %Di solito il gateway o l'host invia il messaggio di Source Quench quando si avvicina al limite di capacità; 
    %piuttosto che aspettare e lasciare che questa capacità venga superata. 
    %Questo porterà al vantaggio che il pacchetto che ha attivato il messaggio potrebbe essere consegnato; 
    %mentre nel caso precedente non vi era abbastanza spazio per poterlo memorizzare (siccome la coda era piena).  
    %Il gateway si avvicina alla sua capacità massima. 
 %   \\
 %   \hline 
 %   Velocità di trasmissione elevata &  
    %Inoltre un host di destinazione può inviare un messaggio di questo tipo, 
    %anche nel caso in cui i datagrammi arrivino troppo velocemente per poter essere elaborati. 
 %   Si trasmettono i datagram troppo velocemente.
    %e l'host di destinazione non riesce ad elaborarli. %così velocemente.  
    %Squilibrio tra la velocità di trasmissione e la capacità di elaborazione
%    \\
%    \hline 
%\caption{Source Quench possibili codici} 
%\label{tabella:caso:SourceQuench} 
%\end{longtable} 
%\end{center} 
Nel protocollo \textbf{ICMPv4} il pacchetto è strutturato in questo modo: 
\vspace{2ex} \newline
\begin{bytefield}[bitwidth=1.1em]{32} 
    %\bitbox{8}{0} & \bitbox{8}{1} & \bitbox{8}{2} & \bitbox{8}{3} \\
    \bitheader{0-31} \\
    \bitbox{8}{Type=4 (1 byte)} & \bitbox{8}{Code=0 (1 byte)} & \bitbox{16}{Checksum (2 byte)} \\
    \bitbox{32}{Unused (4 byte)} \\
    \bitbox{32}{Internet Header + 64 bits of Original Datagram ($\geq$ 21 byte)} 
\end{bytefield} 
\newline 
Il campo \textit{Internet Header}: viene utilizzato dall'host per accoppiare il messaggio di errore 
al processo appropriato. Se un protocollo di livello superiore utilizza numeri di porta, si presume che 
siano nei primi 64 bit dei dati del datagramma originale. 
%\footnote{L'\textit{intestazione IP} può variare dai 20 byte ai 40 byte} 

\subsubsection{Redirect} 
Indica un messaggio di reindirizzamento a un host.  
Il gateway manda questo tipo di messaggio se, dopo aver controllato la sua tabella di routing, 
trova che esiste un gateway migliore che si trova sulla sua stessa rete. 
Questo secondo gateway rappresenterà un percorso migliore per la destinazione.  
Il codice impostato nel messaggio indicherà il motivo [Tabella \ref{tabella:caso:Redirect}]
\begin{center} 
\begin{longtable}{|p{0.25\textwidth}|p{0.55\textwidth}|} 
    \hline
    \textbf{Evento} & \textbf{Esempio} \\
    \hline
    Route migliore &  
    Il gateway riceve il pacchetto e dalla tabella di routing ottiene che il secondo gateway si trova sulla stessa rete. 
    \\
    \hline 
\caption{Redirect possibili codici} 
\label{tabella:caso:Redirect} 
\end{longtable} 
\end{center} 
Nel protocollo \textbf{ICMPv4} il pacchetto è strutturato in questo modo: 
\vspace{2ex} \newline 
\begin{bytefield}[bitwidth=1.1em]{32} 
    %\bitbox{8}{0} & \bitbox{8}{1} & \bitbox{8}{2} & \bitbox{8}{3} \\
    \bitheader{0-31} \\
    \bitbox{8}{Type=5 (1 byte)} & \bitbox{8}{Code=0-3 (1 byte)} & \bitbox{16}{Checksum (2 byte)} \\
    \bitbox{32}{Gateway Internet Address (4 byte)} \\
    \bitbox{32}{Internet Header + 64 bits of Original Datagram ($\geq$ 21B)} 
\end{bytefield}
\newline 
Nel campo \textit{Gateway Internet Address} verrà indicato l'indirizzo del nuovo gateway a cui dovrà essere 
inviato il traffico per la rete di destinazione (specificata nel campo di destinazione del datagram originale). 
Si potrebbe pensare di utilizzare il campo ma un gateway o la vittima per necessità potrebbero leggere i dati e scoprire che non sono conformi. 
\vspace{1ex} \newline 
Il campo \textit{Internet Header} viene utilizzato dall'host per accoppiare il messaggio di errore 
al processo appropriato. Se un protocollo di livello superiore utilizza numeri di porta, si presume che 
siano nei primi 64 bit dei dati del datagramma originale. 
%\footnote{L'\textit{intestazione IP} può variare dai 20 byte ai 40 byte} 
\vspace{1ex} \newline 
Se nell'itestazione IP è presente l'opzione \textit{IP Source Route}, 
il messaggio di reindirzzamento non verrà inviato 
anche se è presente un percorso migliore. %per raggiungere la destinazione. 

\subsubsection{Echo Request / Echo Reply} 
Un messaggio \textit{Echo}, viene usato per ricevere indietro una risposta da un host. 
Si inviano dei dati tramite una Echo Request, e questi'ultimi dovranno essere restituiti in un messaggio di risposta integralmente e senza modifiche. 
%Il codice impostato nel messaggio indicherà il motivo [Tabella \ref{tabella:caso:Echo}]
%\begin{center} 
%\begin{longtable}{|p{0.25\textwidth}|p{0.55\textwidth}|} 
 %   \hline
 %   \textbf{Evento} & \textbf{Esempio} \\
 %   \hline
 %   Echo Request &  
 %   Un host testa se un host è raggiungibile e misurare il tempo di risposta. \\ 
 %   \hline 
    %Velocità di trasmissione elevata &  
    %
    %\\
    %\hline 
%\caption{Quando Echo Reques/Reply viene generato} 
%\label{tabella:caso:Echo} 
%\end{longtable} 
%\end{center} 
Nel protocollo \textbf{ICMPv4} il pacchetto è strutturato in questo modo: 
\vspace{2ex} \newline 
\begin{bytefield}[bitwidth=1.1em]{32} 
    %\bitbox{8}{0} & \bitbox{8}{1} & \bitbox{8}{2} & \bitbox{8}{3} \\
    \bitheader{0-31} \\
    \bitbox{8}{Type=8 (1 byte)} & \bitbox{8}{Code=0 (1 byte)} & \bitbox{16}{Checksum (2 byte)} \\
    \bitbox{16}{Identifier (2 byte)} && \bitbox{16}{Sequence Number (2 byte)} \\
    \bitbox{32}{Data ... ($\geq$ 0 byte)} 
\end{bytefield}  
\vspace{2ex} \newline 
Nel protocollo \textbf{ICMPv6} il pacchetto è strutturato in questo modo: 
\vspace{2ex} \newline 
\begin{bytefield}[bitwidth=1.1em]{32} 
    %\bitbox{8}{0} & \bitbox{8}{1} & \bitbox{8}{2} & \bitbox{8}{3} \\
    \bitheader{0-31} \\
    \bitbox{8}{Type=128 (1 byte)} & \bitbox{8}{Code=0 (1 byte)} & \bitbox{16}{Checksum (2 byte)} \\
    \bitbox{16}{Identifier (2 byte)} && \bitbox{16}{Sequence Number (2 byte)} \\
    \bitbox{32}{Data ... ($\geq$ 0B)} 
\end{bytefield} 
\vspace{1ex} \newline 
%I campi \textit{Identifier} e \textit{Sequence Number} servono per facilitare la corrispondenza tra le richieste Echo e le Risposte Echo (possono essere zero). 
Nel messaggio, i campi identificatore e numero di sequenza possono essere utilizzati dal mittente per facilitare l'abbinamento delle risposte con le richieste.
%Ad esempio, l'identificatore potrebbe essere utilizzato come una porta in TCP o UDP per identificare una 
%sessione e il numero di sequenza potrebbe essere incrementato a ogni richiesta di eco inviata. 
%Per creare un messaggio di risposta, gli indirizzi di origine e di destinazione vengono semplicemente invertiti,
%il codice da 8 viene modificato in 0 e il checksum viene ricalcolato. 
%Per creare un messaggio di risposta, gli indirizzi di origine e di destinazione vengono semplicemente invertiti,
%il codice da 128 viene modificato in 129 e il checksum viene ricalcolato. 
%In aggiunta, ogni nodo deve implementare una funzione di risposta ai messaggi \textit{Echo ICMPv6} così che 
%quando riceve delle richieste Echo, generi le relative risposte. 
%A node SHOULD also implement an application-layer interface for originating Echo Requests and receiving Echo Replies, for diagnostic purposes.
%E inoltre, un nodo dovrebbe implementare un'interfaccia a livello applicazione per poter generare Richieste Echo e ricevere Risposte Echo, a fini diagnostici. 
\vspace{3ex} \newline 
%Inoltre il mittente può inserire nel campo del payload quanti dati desidera, a patto che risultino minori di 65.000 bytes 
%(Questo perchè nell'intestazione IP, il campo relativo alla lunghezza totale del pacchetto è composto da 16 bit).
Il mittente non ha alcuna limitazione sulla quantità di dati inseribili  nel campo del payload. 
%nei messaggi \textit{Echo} (finchè la dimensione risulti minore di 65 KB). 
Tuttavia una limitazione potrà essere data dalla massima capacità di trasporto dei collegamenti. 
%\vspace{1ex} \newline 
%\textbf{NB} Una limitazione inferiore sui dati potrà essere definità dalla massima capacità di trasporto dei collegamenti. 
Nel caso la dimensione del messaggio la superasse; il pacchetto dovrà essere frammentato per poter essere spedito. 
In media il valore si attesta sui 1400 bytes \cite{max_MTU_capacity}.  

\subsubsection{Timestamp Request / Timestamp Reply} 
Viene usato per ricevere indietro una risposta da un host. 
I dati ricevuti nel messaggio di richiesta, vengono restituiti in quello di risposta insieme a dei timestamp aggiuntivi. 
%Un utilizzo di questi timestamp è descritto da Mills [5].
Il timestamp è pari a 32 bit e indica i millisecondi che sono passati dalla mezzanotte UT.
%Se l'ora non è disponibile in millisecondi o non può essere fornita rispetto alla mezzanotte UT, 
%è possibile inserire qualsiasi ora in un timestamp, a condizione che anche il bit di ordine superiore del 
%timestamp sia impostato per indicare questo valore non standard. 
%Il codice impostato nel messaggio indicherà il motivo [Tabella \ref{tabella:caso:Timestamp}]
%\begin{center} 
%\begin{longtable}{|p{0.25\textwidth}|p{0.55\textwidth}|} 
%    \hline
%    \textbf{Evento} & \textbf{Esempio} \\
%    \hline
%    Timestamp Request &  
%    Un host invia un pacchetto per chiedere all'altro di fornire l'ora corrente. 
    %utile per calcolare il tempo di propagazione e 
    %sincronizzare gli orologi.
%    L'host ricevente risponderà con il proprio timestamp (ora di trasmissione e ricezione). 
    %permettendo al mittente di stimare la 
    %latenza e correggere eventuali differenze di clock. 
%    \\
%    \hline 
%\caption{Timestamp Reques/Reply possibili codici} 
%\label{tabella:caso:Timestamp} 
%\end{longtable} 
%\end{center} 
Nel protocollo \textbf{ICMPv4} il pacchetto è strutturato in questo modo: 
\vspace{2ex} \newline 
\begin{bytefield}[bitwidth=1.1em]{32} 
    %\bitbox{8}{0} & \bitbox{8}{1} & \bitbox{8}{2} & \bitbox{8}{3} \\
    \bitheader{0-31} \\
    \bitbox{8}{Type=13 (1 byte)} & \bitbox{8}{Code=0 (1 byte)} & \bitbox{16}{Checksum (2 byte)} \\
    \bitbox{16}{Identifier (2 byte)} && \bitbox{16}{Sequence Number (2 byte)} \\ 
    \bitbox{32}{Originate Timestamp (4 byte)} \\
    \bitbox{32}{Receive Timestamp (4 byte)} \\
    \bitbox{32}{Transmit Timestamp (4 byte)} \\
    \bitbox{32}{Data ... ($\geq$ 0 byte)} 
\end{bytefield}
\vspace{1ex} \newline 
%I campi \textit{Identifier} e \textit{Sequence Number} servono per facilitare la corrispondenza tra le richieste Echo e le Risposte Echo (possono essere zero). 
L'identificatore e il numero di sequenza possono essere utilizzati dal mittente del pacchetto per facilitare l'abbinamento delle risposte con le richieste.
%Ad esempio, l'identificatore potrebbe essere utilizzato come una porta in TCP o UDP per identificare una sessione e il numero di sequenza potrebbe essere incrementato a ogni richiesta inviata.
%La destinazione restituisce gli stessi valori nella risposta. 
Mentre i campi relativi ai timestamp indicheranno rispettivamente: 
il tempo in cui il mittente ha toccato il messaggio per l'ultima volta prima di inviarlo, 
il tempo in cui il destinatario ha toccato per la prima volta il messaggio (alla ricezione) e 
il tempo in cui il destinatario ha toccato il messaggio per l'ultima volta prima di inviarlo. 

\subsubsection{Information Request / Information Reply} 
La tipologia \textit{Information} viene usata per consentire di scoprire il numero della rete in cui un host si trova. 
Serve quindi per capire se si trova nella stesse rete dell'host che risponde. 
Sebbene il messaggio può essere inviato con 
%la rete sorgente nel campo mittente e 
la destinazione nell'intestazione IP pari a zero (ciò significa "questa" rete); 
l'intestazione IP presente nel messaggio di risposta dovrà essere inviata con gli indirizzi IP completamente specificati.
%Per creare un messaggio di risposta, gli indirizzi di origine e di destinazione vengono invertiti,
%il codice da 15 viene modificato in 16 e il checksum viene ricalcolato. 
%Il codice impostato nel messaggio indicherà il motivo [Tabella \ref{tabella:caso:Information}]
%\begin{center} 
%\begin{longtable}{|p{0.25\textwidth}|p{0.55\textwidth}|} 
 %   \hline
%    \textbf{Evento} & \textbf{Esempio} \\
%    \hline
%    Information Request &  
%    Un host invia questo messaggio per chiedere a un router o host informazioni sulla propria rete di appartenenza 
%    (es. identificazione di rete).
%    Il router/host risponde con le informazioni richieste (es. identificatore di rete), permettendo all’host mittente di configurarsi. \\
%    \hline 
    %Velocità di trasmissione elevata &  
    %
    %\\
    %\hline 
%\caption{Quando Information Reques/Reply viene generato} 
%\label{tabella:caso:Information} 
%\end{longtable} 
%\end{center} 
Nel protocollo \textbf{ICMPv4} il pacchetto è strutturato in questo modo: 
\vspace{2ex} \newline
\begin{bytefield}[bitwidth=1.1em]{32} 
    %\bitbox{8}{0} & \bitbox{8}{1} & \bitbox{8}{2} & \bitbox{8}{3} \\
    \bitheader{0-31} \\
    \bitbox{8}{Type=15 (1 byte)} & \bitbox{8}{Code=0 (1 byte)} & \bitbox{16}{Checksum (2 byte)} \\
    \bitbox{16}{Identifier (2 byte)} && \bitbox{16}{Sequence Number (2 byte)} 
\end{bytefield} 
\vspace{1ex} \newline 
%I campi \textit{Identifier} e \textit{Sequence Number} servono per facilitare la corrispondenza tra le richieste Echo e le Risposte Echo (possono essere zero). 
L'identificatore e il numero di sequenza possono essere utilizzati dal mittente del pacchetto per facilitare 
l'abbinamento delle risposte con le richieste. 

\subsubsection{Packet Too Big} 
Un messaggio \textit{Packet Too Big} viene generato da un router in risposta a un pacchetto che 
non può inoltrare perché è più grande dell'MTU del collegamento in uscita.
%Le informazioni contenute in questo messaggio vengono utilizzate come parte del processo di Path MTU Discovery
%\href{https://www.rfc-editor.org/rfc/rfc4443#ref-PMTU}{Path MTU Discovery}.
%Originating a Packet Too Big Message makes an exception to one of the rules as to when to originate an ICMPv6 error message.  
%Unlike other messages, it is sent in response to a packet received with an IPv6 multicast destination address, or with a link-layer multicast or link-layer broadcast address. 
Un nodo che riceve un messaggio \textit{ICMPv6 Packet Too Big} deve notificare la cosa al 
processo di livello superiore (se il processo in questione può essere identificato). 
%Il codice impostato nel messaggio indicherà il motivo [Tabella \ref{tabella:caso:Information}]
%\begin{center} 
%\begin{longtable}{|p{0.25\textwidth}|p{0.55\textwidth}|} 
%    \hline
%    \textbf{Evento} & \textbf{Esempio} \\
%    \hline
%    Packet Too Big &  
%    Un router ha ricevuto un pacchetto più grande della MTU del collegamento e non può frammentarlo. Invia un messaggio al mittente 
%    indicando la MTU massima supportata, così il mittente può ridurre la dimensione dei pacchetti successivi. \\
%    \hline 
%\caption{Packet Too Big possibili codici} 
%\label{tabella:caso:PacketTooBig} 
%\end{longtable} 
%\end{center}
Nel protocollo \textbf{ICMPv6} il pacchetto è strutturato in questo modo: 
\vspace{2ex} \newline
\begin{bytefield}[bitwidth=1.1em]{32} 
    %\bitbox{8}{0} & \bitbox{8}{1} & \bitbox{8}{2} & \bitbox{8}{3} \\
    \bitheader{0-31} \\
    \bitbox{8} {Type=2 (1 byte)} & \bitbox{8}{Code=0 (1 byte)} & \bitbox{16}{Checksum (2 byte)} \\
    \bitbox{32} {MTU (4 byte)} \\
    \bitbox{32} {As much of invoking packet as possible without} \\
    \bitbox{32} {the ICMPv6 packet exceeding the minimum IPv6 MTU ($\geq$ 0 byte)} 
\end{bytefield} 
\vspace{1ex} \newline 
Il campo \textit{MTU} indica la massima unità di trasmissione del collegamento nel salto successivo. 
Mentre il campo \textit{Invoking Packet} indica quanta parte del pacchetto (che ha attivato l'errore ICMPv6) 
debba essere inclusa. Il tutto senza eccedere il \textit{IPv6 MTU} il cui valore di default equivale a 1280 bytes.


%\subsubsection{TRASH} 
%L'unica differenza presente nella versione (del Timing Channel) che utilizza il protocollo IPV6, 
%è la struttura del pacchetto. 
%Si aggiungerà un livello \textit{Ether} in cui si specifica l'indirizzo MAC di destinazione e l'indirizzo MAC sorgente. 
%Inoltre nel livello \textit{IPv6} bisognerà definire lo ScopeId dell'indirizzo di destinazione; 
%ciò verrà fatto specificanto nel campo non solo l'indirizzo IP di destinazione, 
%ma anche l'interfaccia per poter raggiungerlo. 
%Un livello \textit{Raw} contenente solo il testo '\textit{Hello Neighbour}'. 