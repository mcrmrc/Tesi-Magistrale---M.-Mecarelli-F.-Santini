\subsection{Buona practice di sicurezza}
Per prevenire gli attacchi basati su ICMP; buone misure di sicurezza sono: 
\begin{itemize}
    \item limitare e filtrare l'utilizzo di ICMP tramite i firewall 
    \item la limitazione della velocità 
    \item il monitoraggio del traffico (tramite strumenti di sicurezza) per rilevare le anomalie
\end{itemize} 
\subsubsection*{Regole del firewall}
Limitare o bloccare il traffico ICMP non necessario. %sui firewall. 
%Consentire solo i messaggi ICMP strettamente necessari (e.g. Echo Reply, Destinazione non raggiungibile, ma non Redirect).
%Bloccare i tipi di ICMP non necessari sui firewall (ad esempio, Redirect, Timestamp, Source Quench). 
\begin{itemize}
    \item Bloccare le richieste Echo di ICMP da reti esterne, a meno che non siano necessarie.
    \item Disabilitare le risposte a ICMP Timestamp e Address Mask per impedire la ricognizione.
    \item Consentire solo i messaggi di errore ICMP necessari (ad esempio, Destinazione non raggiungibile).
    \item Eliminare i messaggi di reindirizzamento ICMP per impedire la manipolazione dell'instradamento (del routing).
\end{itemize}
\subsubsection*{Limitazione della velocità} 
Limitare la velocità delle richieste ICMP. %per evitare di essere sopraffatti
\begin{itemize}
    \item Limita il numero di pacchetti ICMP al secondo per prevenire la sovrastazione. 
    \item Configura i criteri di limitazione della velocità ICMP su router e firewall.
\end{itemize}
\subsubsection*{Monitoraggio della rete e Rilevamento}
%Utilizzare sistemi di rilevamento delle intrusioni (IDS) per monitorare attività ICMP sospette. 
\begin{itemize}
    \item Utilizzare i sistemi di rilevamento delle intrusioni (IDS/IPS) per rilevare abusi del protocollo ICMP.
    \item Analizza i registri di rete per attività ICMP insolite (ad esempio, pacchetti ICMP di grandi dimensioni, ping frequenti).
    \item Implementa l'ispezione approfondita dei pacchetti (DPI) per identificare il Tunneling ICMP.
\end{itemize}
\subsubsection*{Rafforzamento del sistema}
\begin{itemize}
    \item Mantieni aggiornati i sistemi e il firmware per correggere le vulnerabilità ICMP note
    \item Disattivare i servizi ICMP sui sistemi critici se non necessari.
    \item Utilizzare soluzioni di sicurezza degli endpoint per rilevare malware che utilizzano ICMP per la comunicazione
\end{itemize} 
