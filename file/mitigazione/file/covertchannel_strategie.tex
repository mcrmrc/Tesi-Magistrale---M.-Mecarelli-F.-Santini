\subsection{Strumenti di Mitigazione e Protezione} 
Siccome la completa eliminazione è difficle, strategie di rilevazione e minimizzazione sono essenziali 
(es randomizzazione, rigoroso controllo degli accessi delle risorse, rilevamento delle anomalie). 
\begin{itemize}
    \item \textbf{Difendersi} richiede una combinazione di rinforzo delle politiche, gestione delle risorse e tecniche di monitoraggio. 
    \item \textbf{Mitigarli}, richiede una sicurezza multi livello fra hardware, OS, applicazioni e reti. 
\end{itemize} 
Il rilevamento e la mitigazione dei covert Channel richiede quindi un rigoroso monitoraggio  delle anomalie, l'isolamento delle risorse e tecniche per introdurre rumore. 
\begin{esempio}
    Un file può essere aperto e chiuso da un progremma in modo specifico pattern temporale così che possa essere rilevato da un altro programma; 
    lo schema potrà essere poi interpretato come una stringa di bit formando così un Covert Channel. 
    Di conseguenza, siccome è improbabile che l'utente legittimo controlli i pattern relativi alla chiusura/apertura dei file; 
    questo tipo ti Covert Channel può rimanere non identificato per un lungo periodo. 
\end{esempio}
%
%\subsection*{Principali strategie di difesa per la mitigazione} 
\subsubsection*{Difese basate sul Sistema e sule Politiche(Policy)}  
\begin{enumerate}
    \item \textbf{Politiche di controllo degli accessi}: \newline 
    Applicare un forte controllo degli accessi (MAC, RBAC) per evitare interazioni non autorizzate con i processi. 
    Limitare i permessi implementando il minimo privilegio e il controllo obbligatorio dell'accesso (MAC) per limitare e/o prevenire la comunicazione non autorizzata tra i processi (e quindi lo scambio di informazioni non autorizzato). 
    Utilizzare sandbox e compartimentazione per isolare i processi.
    \item \textbf{Controllo del flusso di informazioni}: \newline 
    Utilizzare obbligatoriamente modelli di controllo del flusso di dati (Bell-LaPadula, 
    Biba) per evitare fughe di informazioni e impedire così che i processi ad alta 
    sicurezza perdano dati ai processi a bassa sicurezza.
    \item \textbf{Separazione e isolamento dei processi}: \newline 
    Disattivare le risorse condivise non necessarie (ad esempio, comunicazione tra processi, memoria condivisa).
    Utilizzare la virtualizzazione e la containerizzazione per separare i processi. 
    Applicare l'air-gapping per i sistemi altamente sensibili.
\end{enumerate}
\subsubsection*{Protezioni basate sulla gestione delle risorse e dei tempi} 
\begin{itemize}
    \item \textbf{Tecniche di Randomizzazione}: \newline 
    Introdurre rumore (Noise Injection) nelle risposte del sistema (ad esempio, randomizzando i tempi di esecuzione, aggiungendo 
    ritardi) per interrompere i Covert Channel basati sul tempo. 
    Utilizzare tecniche di randomizzazione o svuotamento della cache per prevenire attacchi side-channel basati sulla cache.
    \item \textbf{Limitazione della velocità e controllo della larghezza di banda}: \newline 
    Limitare la CPU, la memoria o la larghezza di banda della rete per limitare la capacità di un canale nascosto. 
    Implementare meccanismi di throttling (limitazione) per le risorse condivise. 
    E analizzare i comportamenti del sistema per rilevare anomalie. 
\end{itemize}
\subsubsection*{Protezioni basate sulla sicurezza della rete}  
\begin{itemize}
    \item \textbf{Ispezione e filtraggio dei pacchetti}: \newline 
    Monitoraggio del Traffico utilizzando la Deep Packet Inspection (DPI) per rilevare schemi anomali nel traffico di rete.
    Bloccare o sanificare i campi inutilizzati dei protocolli (ad esempio, le intestazioni TCP/IP). 
    \item \textbf{Analisi del traffico e rilevamento delle anomalie}: \newline 
    Applicare la segmentazione della rete per limitare i flussi di dati non autorizzati.
    Utilizza il monitoraggio basato sull'intelligenza artificiale per rilevare modelli di comunicazione insoliti.
    Utilizza sistemi di rilevamento delle intrusioni (IDS) e analisi dei log per identificare attività sospette.
\end{itemize} 
\subsubsection*{Miglioramenti della sicurezza hardware e software} 
\subsubsection*{Difese hardware e OS}
\begin{itemize}
    \item Randomizzare i tempi di esecuzione e iniettare rumore nelle risposte del sistema (per interrompere gli attacchi basati sulla temporizzazione). 
    \item Implementare operazioni crittografiche a tempo costante per prevenire i canali laterali di temporizzazione. 
    \item Svuotare e partizionare le cache della CPU per prevenire gli attacchi alla cache cross-process.
\end{itemize}
\begin{itemize}
    \item Progettazione hardware sicura: \newline 
    Implementare operazioni crittografiche a tempo costante per prevenire attacchi basati sulla temporizzazione. 
    Utilizzare enclave sicuri (ad esempio, Intel SGX, ARM TrustZone) per proteggere i calcoli sensibili. 
    \item Protezioni a livello di sistema operativo: \newline 
    Applicare l'isolamento della memoria e disabilitare la memoria condivisa quando non è necessaria. 
    Implementare algoritmi di pianificazione sicuri per prevenire fuoriusicte di dati tramite la temporizzazione basata sui processi. 
    Introdurre casualità nei pattern temporali o di accesso alla memoria per rendere il prelevamento dei dati difficile.  
\end{itemize} 
\subsubsection*{Verifica e test dei Covert Channel} 
\begin{itemize}
    \item Eseguire regolarmente analisi dei canali nascosti nei test di penetrazione.
    \item Utilizzare strumenti di rilevamento dei Covert Channel (ad esempio, analisi del flusso di rete, monitoraggio del comportamento del sistema).
\end{itemize}

