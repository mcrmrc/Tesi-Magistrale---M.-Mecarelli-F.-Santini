\subsection{Tipologie di attacchi Covert Channel} 
I Covert Channel sono spesso applicati per: 
\begin{itemize}
    \item \textbf{Malware e Spionaggio}: usati per esfiltrare dati sensibili. 
    \item \textbf{Test di sicurezza}: identificare e mitigare i Covert Channel. \uppercase{è} una parte fondamentale nel stabilire la sicurezza del sistema. 
    \item \textbf{Ricerca}: espolare i Covert Channel aiuta a capire potenziali vulnerabilità in sistemi complessi. 
\end{itemize} 
Gli attacchi tramite Covert Channel sfruttando le debolezze relative al tempo, alle risorse condivise, al design del sistema, ai protocolli di rete, \dots per trasmettere dati nascosti; 
pongono una seria minaccia nella comunicazione fra malware, esfiltrazione dei dati e il cyber-spionaggio. 
\vspace{2ex} \newline
Gli attachi tramite Covert Channel sfruttano vulnerabilita nel , nelle risorse condivise e nelle
politiche di sicurezza per trasmettere segretamente dati fra processi o sistemi 
Potendo aggire i tradizionali  controlli di sicurezza, questi attacchi sono spessi usati per l'esfiltrazione dei dati, privilege escaletion o comunicazioni silenzione tra componenti malware.  
\subsubsection*{Attacchi basati sulla memoria}
Questi attacchi manipolano le risorse di sistema condivise per memorizzare e recuperare informazioni nascoste.
    \begin{esempio}{\quad \newline}
        $\bullet$\underline{Manipolazione degli attributi dei file}: \newline
        il malware altera i metadati dei file (e.g. timestamp, permessi) per codificare i messaggi.
        \vspace{1ex} \newline 
        $\bullet$\underline{Sfruttamento della memoria condivisa}: \newline 
        i processi comunicano modificando le regioni di memoria condivise. 
        \vspace{1ex} \newline 
        $\bullet$\underline{Segnali tramite l'utilizzo del disco}: \newline 
        un processo scrive o elimina i dati mentre un altro processo rileva le modifiche.
        Disk Usage Signaling: One process writes or deletes data, and another process detects changes. 
        \vspace{1ex} \newline 
        $\bullet$\underline{Campi nell intestazione TCP/IP}: \newline 
        gli attaccani codificano i dati in campi inutilizzati o facoltativi dei pacchetti di rete 
        (e.g. ID IP, numeri di sequenza o valori TTL). 
    \end{esempio}
\subsubsection*{Attacchi basati sulla temporizzazione} 
Questi attacchi manipolano la tempistica o le prestazioni del sistema per trasmettere informazioni nascoste.
    \begin{esempio}{\quad\newline}
        $\bullet$\underline{Fluttuazione del carico della CPU}: 
        il malware altera gli schemi di utilizzo della CPU, che un altro processo misura per decodificare le informazioni. 
        \vspace{1ex} \newline 
        $\bullet$\underline{Temporizzazione dei pacchetti di rete}: 
        il mittente trasmette i pacchetti a intervalli di tempo specifici per codificare i dati binari. 
        \vspace{1ex} \newline 
        $\bullet$\underline{Attacchi basati sulla cache}: 
        gli aggressori utilizzano i tempi di accesso alla cache (e.g. Flush+Reload, Prime+Probe) per far trapelare segreti
        \vspace{1ex} \newline 
        $\bullet$\underline{Analisi del consumo energetico}: 
        i dati sensibili vengono estratti analizzando le variazioni del consumo energetico 
        (utilizzate negli attacchi crittografici side-channel). 
    \end{esempio} 
\subsubsection*{Esempi reali di attacchi Covert Channel} 
\begin{itemize}
    \item Attacchi basati sui Malware: \newline 
    Duqu 2.0 (2015) utilizzava canali TCP/IP occulti per esfiltrare i dati evitando il rilevamento
    \item Attacchi di tunneling DNS: \newline 
    il malware nasconde i dati all'interno delle query DNS (ad esempio, comunicazione C2 per le botnet). 
    %\item Covert Channels basati sul Cloud e sulla Virtualizatione: \newline 
    \item Hypervisor Covert Channels: 
    Le macchine virtuali (VM) sullo stesso host fisico perdono dati attraverso la cache o la memoria della CPU condivisa.
    %\vspace{1ex} \newline 
    %Cloud Timing Attacks: Cloud tenants use execution timing differences to infer co-resident VM activities. 
\end{itemize} 
\begin{minipage}{\linewidth}
    \begin{tabular}{|p{0.3\linewidth}|p{0.3\linewidth}|p{0.3\linewidth}|}
        \hline
        Nome Attacco & Tipo & Descrizione \\
        \hline 
        Spectre and Meltdown & Timing (Cache) & Sfrutta l'esecuzione speculativa per divulgare i contenuti della memoria \\ %Exploit speculative execution to leak memory contents
        \hline 
        Flush+Reload & Timing (Cache) & L'attaccante svuota la memoria condivisa e la ricarica per osservare i modelli di accesso. \\ %Attacker flushes shared memory and reloads it to observe access patterns.
        \hline 
        Prime+Probe & Timing (Cache) & L'attaccante riempie la cache e monitora i modelli di espulsione per inferire dati segreti. \\ %Attacker fills cache and monitors eviction patterns to infer secret data.
        \hline 
        Packet Timing Attack & Timing (Network) & Varia il timing di trasmissione dei pacchetti per inviare messaggi nascosti. \\ %Varies packet transmission timing to send hidden messages.
        \hline 
        Keystroke Timing Attack & Timing (Human Interaction) & Inferisce i tasti digitati in base alle variazioni temporali tra i colpi di tasto. \\ %Infers typed keys based on timing variations between keystrokes.
        \hline 
        TCP Covert Channel & Storage (Network) & Codifica i dati nei campi dei pacchetti TCP (ad esempio, numeri di sequenza, flag). \\ %Encodes data in TCP packet fields (e.g., sequence numbers, flags).
        \hline 
        File Lock Covert Channel & Storage (Filesystem) & Utilizza il blocco/sblocco dei file come meccanismo di segnalazione. \\ %Uses file locking/unlocking as a signaling mechanism.
        \hline 
    \end{tabular}
    \captionof{table}{Menzione a degli attacchi Covert Channel}
\end{minipage}