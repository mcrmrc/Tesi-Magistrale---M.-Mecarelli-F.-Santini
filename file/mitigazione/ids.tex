%https://github.com/Veda-Samhitha/Intrusion-Detection-System-

1. Packet Sniffing (Scapy)
    Captures live packets using scapy
    Analyzes for ICMP floods and TCP port scan behavior
2. ICMP Flood Detection
    Tracks the number of pings (ICMP packets) per IP within a short window
    Triggers alert if threshold is crossed
3. TCP Port Scan Detection
    Tracks how many unique ports an IP hits within 5 seconds
    If more than 2, it logs a port scan alert
4. Logging + Real-time Monitoring
    Alerts are written to alerts_only_log.txt
    You can monitor new alerts live using alerts_only.py

Cybersecurity Concepts:
    Intrusion Detection System (IDS)
    ICMP Flood Detection (Ping Flood)
    TCP Port Scan Detection
    Threshold-based alert logic
    Real-time monitoring & alert logging

%https://medium.com/@pasanglamatamang/network-security-assessment-configuring-and-testing-intrusion-detection-systems-ids-with-icmp-ef18a1922184

IDS sensor to monitor packets on a LAN router’s interface with the outside internetwork. 
I will be using the Security Onion Linux distribution bundled with Snort IDS as the sensor.

%https://nmap.org/book/subvert-ids.html

If an IDS is suspected or detected, there are many effective techniques for subverting it. 
They fall into three categories that vary by intrusiveness: 
avoiding the IDS as if the attacker is not there, confusing the IDS with misleading data, and exploiting the IDS to 
gain further network privilege or just to shut it down. Alternatively, attackers who are not concerned with stealth 
can ignore the IDS completely as they pound away at the target network.

\section{Intrusion Detection System Detection}
Early on in the never-ending battle between network administrators and malicious hackers, administrators defended 
their turf by hardening systems and even installing firewalls to act as a perimeter barrier. 
Hackers developed new tools to penetrate or sneak around the firewalls and exploit vulnerable hosts.

While intrusion detection systems are meant to be passive devices, many can be detected by attackers over the network.

The least conspicuous IDS is one that passively listens to network traffic without ever transmitting. 
Special network tap hardware devices are available to ensure that the IDS cannot transmit, even if it is 
compromised by attackers. Despite the security advantages of such a setup, it is not widely deployed due to 
practical considerations. Modern IDSs expect to be able send alerts to central management consoles and the like. 
If this was all the IDS transmitted, the risk would be minimal. 
But to provide more extensive data on the alert, they often initiate probes that may be seen by attackers.

\subsection{Reverse probes}
One probe commonly initiated by IDSs is reverse DNS query of the attacker's IP address. 
A domain name in an alert is more valuable than just an IP address, after all. 
Unfortunately, attackers who control their own rDNS (quite common) can watch the logs in real time and learn that they have been detected. 
This is a good time for attackers to feed misinformation, such as bogus names and cache entries to the requesting IDS.

Some IDSs go much further and send more intrusive probes to the apparent attackers. When an attacker sees his target scan him back, 
there is no question that he has set off alarms. 
Some IDSs send Windows NetBIOS information requests back to the attacker.

\subsection{Sudden firewall changes and suspicious packets}
Many intrusion detection systems have morphed into what marketing departments label intrusion prevention systems. 
Some can only sniff the network like a normal IDS and send triggered packet responses. 
The best IPS systems are inline on the network so that they can restrict packet flow when suspicious activity is detected.

For example, an IPS may block any further traffic from an IP address that they believe has port scanned them, 
or that has attempted a buffer overflow exploit. 

Attackers are likely to notice this if they port scan a system, then are unable to connect to the reported open ports. 
Attackers can confirm that they are blocked by trying to connect from another IP address.

Suspicious response packets can also be a tip-off that an attacker's actions have been flagged by an IDS. 
In particular, many IDSs that are not inline on the network will forge RST packets in an attempt to tear down connections.

\subsection{Naming conventions}
Naming conventions can be another giveaway of IDS presence. 
If an Nmap list scan returns host names such as realsecure, ids-monitor, or dragon-ids, you may have found an intrusion detection system. 
The administrators might have given away that information inadvertently, or they may think of it like the alarm 
stickers on house and car windows. 
Perhaps they think that the script kiddies will be scared away by IDS-related names. 
It could also be misinformation. 
You can never fully trust DNS names

\subsection{Unexplained TTL jumps}
One more way to detect certain IDSs is to watch for unexplained gaps (or suspicious machines) in traceroutes. 
While most operating systems include a traceroute command (it is abbreviated to tracert on Windows), Nmap offers a faster and more 
effective alternative with the --traceroute option. Unlike standard traceroute, Nmap sends its probes in parallel and is able to 
determine what sort of probe will be most effective based on scan results.

While traceroute is the best-known method for obtaining this information, it isn't the only one. 
IPv4 offers an obscure option called record route for gathering this information. 
Due to the maximum IP header size, a maximum of nine hops can be recorded. 
In addition, some hosts and routers drop packets with this option set.

\subsection{Avoiding Intrusion Detection Systems}
The most subtle way to defeat intrusion detection systems is to avoid their watchful gaze entirely. 
The reality is that rules governing IDSs are pretty brittle in that they can often be defeated by manipulating the attack slightly.

Attackers have dozens of techniques, from URL encoding to polymorphic shellcode generators for escaping IDS detection of their exploits. 
This section focuses on stealthy port scanning, which is even easier than stealthily exploiting vulnerabilities.

\textbf{Slow down}
When it comes to avoiding IDS alerts, patience is a virtue. Port scan detection is usually threshold based. 
The system watches for a given number of probes in a certain timeframe. 
This helps prevent false positives from innocent users.

Examining the handy open-source Snort IDS provides a lesson on sneaking under the radar. 
Snort has had several generations of port scan detectors. 
The Flow-Portscan module is quite formidable. 
A scan that slips by this is likely to escape detection by many other IDSs as well.

The simpler detection method in Flow-portscan is known as the fixed time scale. 
This simply watches for scanner-fixed-threshold probe packets in scanner-fixed-window seconds. 
Those two variables, which are set in snort.conf, each default to 15. Note that the counter includes any probes sent 
from a single machine to any host on the protected network. So quickly scanning a single port on each of 15 protected 
machines will generate an alert just as surely as scanning 15 ports on a single machine.

It has another detection method, known as sliding time scale. 
This method is similar to the fixed-window method just discussed, except that it increases the window whenever a new probe from a 
host is detected. An alarm is raised if scanner-sliding-threshold probes are detected during the window. The window starts at 
scanner-sliding-window seconds, and increases for each probe detected by the amount of time elapsed so far in the window times 
scanner-sliding-scale-factor. 
Those three variables default to 40 probes, 20 seconds, and a factor of 0.5 in snort.conf.

The sliding scale is rather insidious in the way it grows continually as new packets come in. 
The simplest (if slow) solution would be to send one probe every 20.1 seconds. 
This would evade both the default fixed and sliding scales.
You could speed this up by an order of magnitude by sending 14 packets really fast, waiting 20 seconds for the window to expire, 
then repeating with another 14 probes.

\subsection{Scatter probes across networks rather than scanning hosts consecutively}
IDSs are often programmed to alarm only after a threshold of suspicious activity has been reached. 
This threshold is often global, applying to the whole network protected by the IDS rather than just a single host. 
Occasionally they specifically watch for traffic from a given source address to consecutive hosts. 
If a host sends a SYN packet to port 139 of host 10.0.0.1, that isn't too suspicious by itself. 
But if that probe is followed by similar packets to 10.0.0.2, .3, .4, and .5, a port scan is clearly indicated.

One way to avoid triggering these alarms is to scatter probes among a large number of hosts rather than scanning them consecutively.
Sometimes you can avoid scanning very many hosts on the same network. If you are only conducting a research survey, consider scattering 
probes across the whole Internet with -iR rather than scanning one large network. 
The results are likely to be more representative anyway.

In most cases, you want to scan a particular network and Internet-wide sampling isn't enough. 
Avoiding the consecutive-host probe alarms is easy. Nmap offers the --randomize-hosts option which splits up the target networks into 
blocks of 16384 IPs, then randomizes the hosts in each block. 
If you are scanning a huge network, such as class B or larger, you may get better (more stealthy) results by randomizing larger blocks.

\subsection{Fragment packets}
IP fragments can be a major problem for intrusion detection systems, particularly because the handling of oddities such as overlapping 
fragments and fragmentation assembly timeouts are ambiguous and differ substantially between platforms. 
Because of this, the IDS often has to guess at how the remote system will interpret a packet. 
Fragment assembly can also be resource intensive. 
For these reasons, many intrusion detection systems still do not support fragmentation very well. 
Specify the -f to specify that a port scan use tiny (8 data bytes or fewer) IP fragments

\subsection{Evade specific rules}
Most IDS vendors brag about how many alerts they support, but many (if not most) are easy to bypass. 
The most popular IDS among Nmap users is the open-source Snort.

Advanced attackers install the IDS they are concerned with on their own network, then alter and test scans in advance 
to ensure that they do not trigger alarms.
Snort was only chosen for this example because its rules database is public and it is a fellow open-source network security tool. 
Commercial IDSs suffer from similar issues.

\subsection{Avoid easily detected Nmap features}
Some features of Nmap are more conspicuous than others. 
In particular, version detection connects to many different services, which will often leave logs on those machines 
and set off alarms on intrusion detection systems. 
OS detection is also easy to spot by intrusion detection systems, because a few of the tests use 
rather unusual packets and packet sequences.

One solution for pen-testers who wish to remain stealthy is to skip these conspicuous probes entirely. 
Service and OS detection are valuable, but not essential for a successful attack. 
They can also be used on a case-by-case basis against machines or ports that look interesting, rather than probing the 
whole target network with them.


\section{Misleading Intrusion Detection Systems}
The previous section discussed using subtlety to avoid the watchful eye of intrusion detection systems. 
An alternative approach is to actively mislead or confuse the IDS with packet forgery. 
Nmap offers numerous options for effecting this.

\subsection{Decoys}
Street criminals know that one effective means for avoiding authorities after a crime is to blend into any nearby crowds. 
The police may not be able to tell the purse snatcher from all of the innocent passersby. 
In the network realm, Nmap can construct a scan that appears to be coming from dozens of hosts across the world. 
The target will have trouble determining which host represents the attackers, and which ones are innocent decoys. 
While this can be defeated through router path tracing, response-dropping, and other active mechanisms, 
it is generally an effective technique for hiding the scan source.

Decoys are added with the -D option. 
The argument is a list of hosts, separated by commas. 
he string ME can be used as one of the decoys to represent where the true source host should appear in the scan order. 
Otherwise it will be a random position. Including ME in the 6th position or further in the list prevents some common 
port scan detectors from reporting the activity. 
For example, Solar Designer's excellent Scanlogd only reports the first five scan sources to avoid flooding its logs with decoys.

\subsection{Port scan spoofing}
While a huge group of decoys is quite effective at hiding the true source of a port scan, the IDS alerts will make it obvious 
that someone is using decoys. A more subtle, but limited, approach is to spoof a port scan from a single address. 
Specify -S followed by a source IP, and Nmap will launch the requested port scan from that given source. No useful 
Nmap results will be available since the target will respond to the spoofed IP, and Nmap will not see those responses. 
IDS alarms at the target will blame the spoofed source for the scan.

\subsection{DNS proxying}
Even the most carefully laid plans can be foiled by one little overlooked detail. 
If the plan involves ultra-stealth port scanning, that little detail can be DNS. 

Nmap performs reverse-DNS resolution by default against every responsive host. 
If the target network administrators are the paranoid log-everything type or they have an extremely sensitive IDS, 
these DNS lookup probes could be detected. 
Even something as unintrusive as a list scan (-sL) could be detected this way.

The probes will come from the DNS server configured for the machine running Nmap. 
This is usually a separate machine maintained by your ISP or organization, though it is sometimes your own system.

The most effective way to eliminate this risk is to specify -n to disable all reverse DNS resolution. 
The problem with this approach is that you lose the valuable information provided by DNS. 
Fortunately, Nmap offers a way to gather this information while concealing the source.

A substantial percentage of DNS servers on the Internet are open to recursive queries from anyone. 
Specify one or more of those name servers to the --dns-servers option of Nmap, and all rDNS queries will be proxied through them.
Keep in mind that forward DNS still uses your host's configured DNS server, so specify target IP addresses rather than domain 
names to prevent even that tiny potential information leak.


\section{DoS Attacks Against Reactive Systems}
Many vendors are pushing what they call intrusion prevention systems. 
These are basically IDSs that can actively block traffic and reset established connections that are deemed malicious. 
These are usually inline on the network or host-based, for greater control over network activity. 
Other (non-inline) systems listen promiscuously and try to deal with suspicious connections by forging TCP RST packets.

In addition to the traditional IPS vendors that try to block a wide range of suspicious activity, 
many popular small programs such as Port Sentry are designed specifically to block port scanners.

While blocking port scanners may at first seem like a good idea, there are many problems with this approach. 
The most obvious one is that port scans are usually quite easy to forge, as previous sections have demonstrated. 
It is also usually easy for attackers to tell when this sort of scan blocking software is in place, because they will not 
be able to connect to purportedly open ports after doing a port scan. 
They will try again from another system and successfully connect, confirming that the original IP was blocked.

Attackers can then use the host spoofing techniques discussed previously (-S option) to cause the target host 
to block any systems the attacker desires. 
This may include important DNS servers, major web sites, software update archives, mail servers, and the like. 
It probably would not take long to annoy the legitimate administrator enough to disable reactive blocking.

While most such products offer a whitelist option to prevent blocking certain important hosts, enumerating them 
all is extraordinarily difficult. 
Attackers can usually find a new commonly used host to block, annoying users until the administrator determines 
the problem and adjusts the whitelist accordingly.

\section{Exploiting Intrusion Detection Systems}
The most audacious way to subvert intrusion detection systems is to hack them. 
Many commercial and open source vendors have pitiful security records of product exploitability.

Internet Security System's flagship RealSecure and BlackICE IDS products had a vulnerability which allowed the 
Witty worm to compromise more than ten thousand installations, then disabled the IDSs by corrupting their filesystems. 
Other IDS and firewall vendors such as Cisco, Checkpoint, Netgear, and Symantec have suffered serious remotely 
exploitable vulnerabilities as well.

Open source sniffers have not done much better, with exploitable bugs found in Snort, Wireshark, tcpdump, FakeBO, and many others. 
Protocol parsing in a safe and efficient manner is extremely difficult, and most of the applications need to parse hundreds of protocols. 
Denial of service attacks that crash the IDS (often with a single packet) are even more common 
than these privilege escalation vulnerabilities. 
A crashed IDS will not detect any Nmap scans.

\section{Ignoring Intrusion Detection Systems}
While advanced attackers will often employ IDS subversion techniques described in this chapter, the 
much more common novice attackers (script kiddies) rarely concern themselves with IDSs. 
Many companies do not even deploy an IDS, and those that do often have them misconfigured or pay little attention to the alerts. 
An Internet-facing IDS will see so many attacks from script kiddies and worms that a few Nmap scans to locate 
a vulnerable service are unlikely to raise any flags.

Hackers want to compromise negligently administered and poorly monitored networks that will provide long-lasting nodes for criminal activity.
Being tracked down and prosecuted is rarely a concern of the IDS-ignoring set. 
They usually launch attacks from other compromised networks rr they may use anonymous connectivity such as provided by 
some Internet cafes, school computer labs, libraries, or the prevalent open wireless access points. 
Throwaway dialup accounts are also commonly used. 
Even if they get kicked off, signing up again with another (or the same) provider takes only minutes. 
Many attackers come from Romania, China, South Korea, and other countries where prosecution is highly unlikely.

Internet worms are another class of attack that rarely bothers with IDS evasion. 
Shameless scanning of millions of IP addresses is preferred by both worms and script kiddies as it 
leads to more compromises per hour than a careful, targeted approach that emphasizes stealth.

While most attacks make no effort at stealth, the fact that most intrusion detection systems are so easily subverted is a major concern. 
Skilled attackers are a small minority, but are often the greatest threat. 
Do not be lulled into complacency by the large number of alerts spewed from IDSs. 
They cannot detect everything, and often miss what is most important.
Even skilled hackers sometimes ignore IDS concerns for initial reconnaissance. 
They simply scan away from some untraceable IP address, hoping to blend in with all of the other 
attackers and probe traffic on the Internet. 
After analyzing the results, they may launch more careful, stealthy attacks from other systems.

%https://it.wikipedia.org/wiki/Sistema_di_rilevamento_delle_intrusioni
Un IDS è composto da quattro componenti:
\begin{itemize}
    \item uno o più sensori utilizzati per ricevere le informazioni dalla rete o dai computer
    \item un motore che analizza i dati prelevati dai sensori e provvede a individuare eventuali falle nella sicurezza informatica.
    \item una console utilizzata per monitorare lo stato della rete e dei computer
    \item un database cui si appoggia il motore di analisi e dove sono memorizzate una serie di regole utilizzate per identificare violazioni della sicurezza.
\end{itemize}
Esistono diversi tipi di IDS che si differenziano a seconda del loro compito specifico e dei metodi usati per individuare violazioni della sicurezza. 
Il più semplice IDS è un dispositivo che integra tutte le componenti in un solo apparato.

Un IDS consiste quindi in un insieme di tecniche e metodi realizzati ad-hoc per rilevare pacchetti dati sospetti a livello di rete, di trasporto o di applicazione.

Due sono le categorie base: sistemi basati sulle firme (signature) e sistemi basati sulle anomalie (anomaly).
\begin{itemize}
    \item La tecnica basata sulle firme è in qualche modo analoga a quella per il rilevamento dei virus, che permette di bloccare file infetti e si tratta della tecnica più utilizzata. 
    \item  sistemi basati sul rilevamento delle anomalie utilizzano un insieme di regole che permettono di distinguere ciò che è "normale" da ciò che è "anormale".
\end{itemize}
L'IDS non cerca di bloccare le eventuali intrusioni, cosa che spetta al firewall, ma cerca di rilevarle laddove si verifichino.


Le tecniche di rilevamento intrusione possono essere divise in misuse detection, 
che usano pattern di attacchi ben conosciuti o di punti deboli del sistema per identificare le intrusioni, 
ed in anomaly detection, che cercano di determinare una possibile deviazione dai pattern stabiliti di utilizzazione normale del sistema

Un misuse detection system, conosciuto anche come signature based intrusion detection system, 
identifica le intrusioni ricercando pattern nel traffico di rete o nei dati generati dalle applicazioni.
Questi sistemi codificano e confrontano una serie di segni caratteristici (signature action) dei vari tipi di scenari di intrusione conosciute.
\begin{itemize}
    \item I principali svantaggi di tali sistemi sono che i pattern di intrusione conosciuti richiedono normalmente di essere 
    inseriti manualmente nel sistema, ma il loro svantaggio è soprattutto di non essere in grado di rilevare qualsiasi futuro 
    (quindi sconosciuto) tipo di intrusione se esso non è presente nel sistema.
    \item Il grande beneficio che invece hanno è quello di generare un numero relativamente basso di falsi positivi e 
    di essere adeguatamente affidabili e veloci.
\end{itemize}

Per ovviare al problema delle mutazioni sono nati gli anomaly based intrusion detection system, 
che analizzano il funzionamento del sistema alla ricerca di anomalie. 
Questi sistemi fanno uso di profili (pattern) dell'utilizzo normale del sistema ricavati da misure statistiche ed euristiche 
sulle caratteristiche dello stesso, per esempio, la cpu utilizzata e le attività di i/o di un particolare utente o programma. 
Le anomalie vengono analizzate e il sistema cerca di definire se sono pericolose per l'integrità del sistema.

Gli IDS si possono suddividere anche a seconda di cosa analizzano: 
esistono gli IDS che analizzano le reti locali, quelli che analizzano gli Host e gli IDS ibridi che analizzano la rete e gli Host.


A differenza del firewall che, con una lista di controllo degli accessi, definisce un insieme di regole che i pacchetti devono 
rispettare per entrare o per uscire dalla rete locale, un IDS controlla lo stato dei pacchetti che 
girano all'interno della rete locale confrontandolo con situazioni pericolose già successe prima o 
con situazioni di anomalia definite dall'amministratore di sistema.
Un firewall può bloccare un pacchetto, mentre un IDS agisce in modo passivo, cioè quando rileva la 
presenza di un'anomalia genera un allarme senza però bloccarla.

%https://en.wikipedia.org/wiki/Intrusion_detection_system
An intrusion detection system (IDS) is a device or software application that monitors a network or 
systems for malicious activity or policy violations.
Any intrusion activity or violation is typically either reported to an administrator or collected centrally.

The most common classifications are network intrusion detection systems (NIDS) and host-based intrusion detection systems (HIDS). 
A system that monitors important operating system files is an example of an HIDS, while a system 
that analyzes incoming network traffic is an example of an NIDS.

It is also possible to classify IDS by detection approach. 
The most well-known variants are signature-based detection (recognizing bad patterns, such as exploitation attempts) and 
anomaly-based detection (detecting deviations from a model of "good" traffic, which often relies on machine learning).

Some IDS products have the ability to respond to detected intrusions. 
Systems with response capabilities are typically referred to as an intrusion prevention system (IPS).

Intrusion prevention systems (IPS), also known as intrusion detection and prevention systems (IDPS), are network security 
appliances that monitor network or system activities for malicious activity. 
The main functions of intrusion prevention systems are to identify malicious activity, 
log information about this activity, report it and attempt to block or stop it.


%https://www.geeksforgeeks.org/intrusion-detection-system-ids/
Intrusion is when an attacker gets unauthorized access to a device, network, or system. 
Cyber criminals use advanced techniques to sneak into organizations without being detected.

Intrusion Detection System (IDS) observes network traffic for malicious transactions and sends immediate alerts when it is observed. 
It is software that checks a network or system for malicious activities or policy violations.

Common Methods of Intrusion
\begin{itemize}
    \item Address Spoofing: Hiding the source of an attack by using fake or unsecured proxy servers making it hard to identify the attacker.
    \item Fragmentation: Sending data in small pieces to slip past detection systems.
    \item Pattern Evasion: Changing attack methods to avoid detection by IDS systems that look for specific patterns.
    \item Coordinated Attack: Using multiple attackers or ports to scan a network, confusing the IDS and making it hard to see what is happening.
\end{itemize}

An IDS (Intrusion Detection System) monitors the traffic on a computer network to detect any suspicious activity.
It analyzes the data flowing through the network to look for patterns and signs of abnormal behavior.
The IDS compares the network activity to a set of predefined rules and patterns to identify any activity that might indicate an attack or intrusion.
If the IDS detects something that matches one of these rules or patterns, it sends an alert to the system administrator.
The system administrator can then investigate the alert and take action to prevent any damage or further intrusion.

 Intrusion Detection System are classified into 5 types:
 \begin{itemize}
    \item Network Intrusion Detection System (NIDS): NIDS are set up at a planned point within the network to examine traffic from all devices on the network. 
    It performs an observation of passing traffic on the entire subnet and matches the traffic that is passed on the subnets to the collection of known attacks.
    Once an attack is identified or abnormal behavior is observed, the alert can be sent to the administrator
    \item Host Intrusion Detection System (HIDS): HIDS run on independent hosts or devices on the network.
    It monitors the incoming and outgoing packets from the device only and will alert the administrator if suspicious or malicious activity is detected.
    \item Hybrid Intrusion Detection System: HIDS is made by the combination of two or more approaches to the intrusion detection system. 
    In the hybrid intrusion detection system, the host agent or system data is combined with network information to develop a complete view of the network system.
    \item Application Protocol-Based Intrusion Detection System (APIDS): APIDS is a system or agent that generally resides within a group of servers. 
    It identifies the intrusions by monitoring and interpreting the communication on application-specific protocols. 
    For example, this would monitor the SQL protocol explicitly to the middleware as it transacts with the database in the web server.
    \item Protocol-Based Intrusion Detection System (PIDS):  It comprises a system or agent that would consistently reside at the front end of a server, controlling and interpreting the protocol between a user/device and the server.
    \item Signature-Based Detection: Signature-based detection checks network packets for known patterns linked to specific threats. A signature-based IDS compares packets to a database of attack signatures and raises an alert if a match is found.
 \end{itemize}

Intrusion Detection System Evasion Techniques
\begin{itemize}
    \item Fragmentation: Dividing the packet into smaller packet called fragment and the process is known as fragmentation.
    \item Packet Encoding: Encoding packets using methods like Base64 or hexadecimal can hide malicious content from signature-based IDS.
    \item Traffic Obfuscation: By making message more complicated to interpret, obfuscation can be utilised to hide an attack and avoid detection.
    \item Encryption: Several security features such as data integrity, confidentiality, and data privacy, are provided by encryption. Unfortunately, security features are used by malware developers to hide attacks and avoid detection.
\end{itemize}

Detection Method of IDS
$\bullet$
Signature-Based Method: Signature-based IDS detects the attacks on the basis of the specific patterns such as the number of bytes or 
a number of 1s or the number of 0s in the network traffic. 
It also detects on the basis of the already known malicious instruction sequence that is used by the malware. 
The detected patterns in the IDS are known as signatures. 
Signature-based IDS can easily detect the attacks whose pattern (signature) already exists in the system but it is 
quite difficult to detect new malware attacks as their pattern (signature) is not known.
$\bullet$
Anomaly-Based Method: Anomaly-based IDS was introduced to detect unknown malware attacks as new malware is developed rapidly. 
In anomaly-based IDS there is the use of machine learning to create a trustful activity model and anything coming is compared 
with that model and it is declared suspicious if it is not found in the model. 
The machine learning-based method has a better-generalized property in comparison to signature-based IDS as these models 
can be trained according to the applications and hardware configurations.


\subsection{Vantaggi}
\subsection{Svantaggi}


















