%https://github.com/Veda-Samhitha/Intrusion-Detection-System-

1. Packet Sniffing (Scapy)
    Captures live packets using scapy
    Analyzes for ICMP floods and TCP port scan behavior
2. ICMP Flood Detection
    Tracks the number of pings (ICMP packets) per IP within a short window
    Triggers alert if threshold is crossed
3. TCP Port Scan Detection
    Tracks how many unique ports an IP hits within 5 seconds
    If more than 2, it logs a port scan alert
4. Logging + Real-time Monitoring
    Alerts are written to alerts\_only\_log.txt
    You can monitor new alerts live using alerts\_only.py

Cybersecurity Concepts:
    Intrusion Detection System (IDS)
    ICMP Flood Detection (Ping Flood)
    TCP Port Scan Detection
    Threshold-based alert logic
    Real-time monitoring \& alert logging

%https://medium.com/@pasanglamatamang/network-security-assessment-configuring-and-testing-intrusion-detection-systems-ids-with-icmp-ef18a1922184

IDS sensor to monitor packets on a LAN router’s interface with the outside internetwork. 
I will be using the Security Onion Linux distribution bundled with Snort IDS as the sensor.

%https://nmap.org/book/subvert-ids.html

If an IDS is suspected or detected, there are many effective techniques for subverting it. 
They fall into three categories that vary by intrusiveness: 
avoiding the IDS as if the attacker is not there, confusing the IDS with misleading data, and exploiting the IDS to 
gain further network privilege or just to shut it down. Alternatively, attackers who are not concerned with stealth 
can ignore the IDS completely as they pound away at the target network.

\section{Intrusion Detection System Detection}
Early on in the never-ending battle between network administrators and malicious hackers, administrators defended 
their turf by hardening systems and even installing firewalls to act as a perimeter barrier. 
Hackers developed new tools to penetrate or sneak around the firewalls and exploit vulnerable hosts.

While intrusion detection systems are meant to be passive devices, many can be detected by attackers over the network.

The least conspicuous IDS is one that passively listens to network traffic without ever transmitting. 
Special network tap hardware devices are available to ensure that the IDS cannot transmit, even if it is 
compromised by attackers. Despite the security advantages of such a setup, it is not widely deployed due to 
practical considerations. Modern IDSs expect to be able send alerts to central management consoles and the like. 
If this was all the IDS transmitted, the risk would be minimal. 
But to provide more extensive data on the alert, they often initiate probes that may be seen by attackers.

\subsection{Reverse probes}
One probe commonly initiated by IDSs is reverse DNS query of the attacker's IP address. 
A domain name in an alert is more valuable than just an IP address, after all. 
Unfortunately, attackers who control their own rDNS (quite common) can watch the logs in real time and learn that they have been detected. 
This is a good time for attackers to feed misinformation, such as bogus names and cache entries to the requesting IDS.

Some IDSs go much further and send more intrusive probes to the apparent attackers. When an attacker sees his target scan him back, 
there is no question that he has set off alarms. 
Some IDSs send Windows NetBIOS information requests back to the attacker.

\subsection{Sudden firewall changes and suspicious packets}
Many intrusion detection systems have morphed into what marketing departments label intrusion prevention systems. 
Some can only sniff the network like a normal IDS and send triggered packet responses. 
The best IPS systems are inline on the network so that they can restrict packet flow when suspicious activity is detected.

For example, an IPS may block any further traffic from an IP address that they believe has port scanned them, 
or that has attempted a buffer overflow exploit. 

Attackers are likely to notice this if they port scan a system, then are unable to connect to the reported open ports. 
Attackers can confirm that they are blocked by trying to connect from another IP address.

Suspicious response packets can also be a tip-off that an attacker's actions have been flagged by an IDS. 
In particular, many IDSs that are not inline on the network will forge RST packets in an attempt to tear down connections.

\subsection{Naming conventions}
Naming conventions can be another giveaway of IDS presence. 
If an Nmap list scan returns host names such as realsecure, ids-monitor, or dragon-ids, you may have found an intrusion detection system. 
The administrators might have given away that information inadvertently, or they may think of it like the alarm 
stickers on house and car windows. 
Perhaps they think that the script kiddies will be scared away by IDS-related names. 
It could also be misinformation. 
You can never fully trust DNS names

\subsection{Unexplained TTL jumps}
One more way to detect certain IDSs is to watch for unexplained gaps (or suspicious machines) in traceroutes. 
While most operating systems include a traceroute command (it is abbreviated to tracert on Windows), Nmap offers a faster and more 
effective alternative with the --traceroute option. Unlike standard traceroute, Nmap sends its probes in parallel and is able to 
determine what sort of probe will be most effective based on scan results.

While traceroute is the best-known method for obtaining this information, it isn't the only one. 
IPv4 offers an obscure option called record route for gathering this information. 
Due to the maximum IP header size, a maximum of nine hops can be recorded. 
In addition, some hosts and routers drop packets with this option set.

\subsection{Avoiding Intrusion Detection Systems}
The most subtle way to defeat intrusion detection systems is to avoid their watchful gaze entirely. 
The reality is that rules governing IDSs are pretty brittle in that they can often be defeated by manipulating the attack slightly.

Attackers have dozens of techniques, from URL encoding to polymorphic shellcode generators for escaping IDS detection of their exploits. 
This section focuses on stealthy port scanning, which is even easier than stealthily exploiting vulnerabilities.

\textbf{Slow down}
When it comes to avoiding IDS alerts, patience is a virtue. Port scan detection is usually threshold based. 
The system watches for a given number of probes in a certain timeframe. 
This helps prevent false positives from innocent users.

Examining the handy open-source Snort IDS provides a lesson on sneaking under the radar. 
Snort has had several generations of port scan detectors. 
The Flow-Portscan module is quite formidable. 
A scan that slips by this is likely to escape detection by many other IDSs as well.

The simpler detection method in Flow-portscan is known as the fixed time scale. 
This simply watches for scanner-fixed-threshold probe packets in scanner-fixed-window seconds. 
Those two variables, which are set in snort.conf, each default to 15. Note that the counter includes any probes sent 
from a single machine to any host on the protected network. So quickly scanning a single port on each of 15 protected 
machines will generate an alert just as surely as scanning 15 ports on a single machine.

It has another detection method, known as sliding time scale. 
This method is similar to the fixed-window method just discussed, except that it increases the window whenever a new probe from a 
host is detected. An alarm is raised if scanner-sliding-threshold probes are detected during the window. The window starts at 
scanner-sliding-window seconds, and increases for each probe detected by the amount of time elapsed so far in the window times 
scanner-sliding-scale-factor. 
Those three variables default to 40 probes, 20 seconds, and a factor of 0.5 in snort.conf.

The sliding scale is rather insidious in the way it grows continually as new packets come in. 
The simplest (if slow) solution would be to send one probe every 20.1 seconds. 
This would evade both the default fixed and sliding scales.
You could speed this up by an order of magnitude by sending 14 packets really fast, waiting 20 seconds for the window to expire, 
then repeating with another 14 probes.

\subsection{Scatter probes across networks rather than scanning hosts consecutively}
IDSs are often programmed to alarm only after a threshold of suspicious activity has been reached. 
This threshold is often global, applying to the whole network protected by the IDS rather than just a single host. 
Occasionally they specifically watch for traffic from a given source address to consecutive hosts. 
If a host sends a SYN packet to port 139 of host 10.0.0.1, that isn't too suspicious by itself. 
But if that probe is followed by similar packets to 10.0.0.2, .3, .4, and .5, a port scan is clearly indicated.

One way to avoid triggering these alarms is to scatter probes among a large number of hosts rather than scanning them consecutively.
Sometimes you can avoid scanning very many hosts on the same network. If you are only conducting a research survey, consider scattering 
probes across the whole Internet with -iR rather than scanning one large network. 
The results are likely to be more representative anyway.

In most cases, you want to scan a particular network and Internet-wide sampling isn't enough. 
Avoiding the consecutive-host probe alarms is easy. Nmap offers the --randomize-hosts option which splits up the target networks into 
blocks of 16384 IPs, then randomizes the hosts in each block. 
If you are scanning a huge network, such as class B or larger, you may get better (more stealthy) results by randomizing larger blocks.

\subsection{Fragment packets}
IP fragments can be a major problem for intrusion detection systems, particularly because the handling of oddities such as overlapping 
fragments and fragmentation assembly timeouts are ambiguous and differ substantially between platforms. 
Because of this, the IDS often has to guess at how the remote system will interpret a packet. 
Fragment assembly can also be resource intensive. 
For these reasons, many intrusion detection systems still do not support fragmentation very well. 
Specify the -f to specify that a port scan use tiny (8 data bytes or fewer) IP fragments

\subsection{Evade specific rules}
Most IDS vendors brag about how many alerts they support, but many (if not most) are easy to bypass. 
The most popular IDS among Nmap users is the open-source Snort.

Advanced attackers install the IDS they are concerned with on their own network, then alter and test scans in advance 
to ensure that they do not trigger alarms.
Snort was only chosen for this example because its rules database is public and it is a fellow open-source network security tool. 
Commercial IDSs suffer from similar issues.

\subsection{Avoid easily detected Nmap features}
Some features of Nmap are more conspicuous than others. 
In particular, version detection connects to many different services, which will often leave logs on those machines 
and set off alarms on intrusion detection systems. 
OS detection is also easy to spot by intrusion detection systems, because a few of the tests use 
rather unusual packets and packet sequences.

One solution for pen-testers who wish to remain stealthy is to skip these conspicuous probes entirely. 
Service and OS detection are valuable, but not essential for a successful attack. 
They can also be used on a case-by-case basis against machines or ports that look interesting, rather than probing the 
whole target network with them.


\section{Misleading Intrusion Detection Systems}
The previous section discussed using subtlety to avoid the watchful eye of intrusion detection systems. 
An alternative approach is to actively mislead or confuse the IDS with packet forgery. 
Nmap offers numerous options for effecting this.

\subsection{Decoys}
Street criminals know that one effective means for avoiding authorities after a crime is to blend into any nearby crowds. 
The police may not be able to tell the purse snatcher from all of the innocent passersby. 
In the network realm, Nmap can construct a scan that appears to be coming from dozens of hosts across the world. 
The target will have trouble determining which host represents the attackers, and which ones are innocent decoys. 
While this can be defeated through router path tracing, response-dropping, and other active mechanisms, 
it is generally an effective technique for hiding the scan source.

Decoys are added with the -D option. 
The argument is a list of hosts, separated by commas. 
he string ME can be used as one of the decoys to represent where the true source host should appear in the scan order. 
Otherwise it will be a random position. Including ME in the 6th position or further in the list prevents some common 
port scan detectors from reporting the activity. 
For example, Solar Designer's excellent Scanlogd only reports the first five scan sources to avoid flooding its logs with decoys.

\subsection{Port scan spoofing}
While a huge group of decoys is quite effective at hiding the true source of a port scan, the IDS alerts will make it obvious 
that someone is using decoys. A more subtle, but limited, approach is to spoof a port scan from a single address. 
Specify -S followed by a source IP, and Nmap will launch the requested port scan from that given source. No useful 
Nmap results will be available since the target will respond to the spoofed IP, and Nmap will not see those responses. 
IDS alarms at the target will blame the spoofed source for the scan.

\subsection{DNS proxying}
Even the most carefully laid plans can be foiled by one little overlooked detail. 
If the plan involves ultra-stealth port scanning, that little detail can be DNS. 

Nmap performs reverse-DNS resolution by default against every responsive host. 
If the target network administrators are the paranoid log-everything type or they have an extremely sensitive IDS, 
these DNS lookup probes could be detected. 
Even something as unintrusive as a list scan (-sL) could be detected this way.

The probes will come from the DNS server configured for the machine running Nmap. 
This is usually a separate machine maintained by your ISP or organization, though it is sometimes your own system.

The most effective way to eliminate this risk is to specify -n to disable all reverse DNS resolution. 
The problem with this approach is that you lose the valuable information provided by DNS. 
Fortunately, Nmap offers a way to gather this information while concealing the source.

A substantial percentage of DNS servers on the Internet are open to recursive queries from anyone. 
Specify one or more of those name servers to the --dns-servers option of Nmap, and all rDNS queries will be proxied through them.
Keep in mind that forward DNS still uses your host's configured DNS server, so specify target IP addresses rather than domain 
names to prevent even that tiny potential information leak.


\section{DoS Attacks Against Reactive Systems}
Many vendors are pushing what they call intrusion prevention systems. 
These are basically IDSs that can actively block traffic and reset established connections that are deemed malicious. 
These are usually inline on the network or host-based, for greater control over network activity. 
Other (non-inline) systems listen promiscuously and try to deal with suspicious connections by forging TCP RST packets.

In addition to the traditional IPS vendors that try to block a wide range of suspicious activity, 
many popular small programs such as Port Sentry are designed specifically to block port scanners.

While blocking port scanners may at first seem like a good idea, there are many problems with this approach. 
The most obvious one is that port scans are usually quite easy to forge, as previous sections have demonstrated. 
It is also usually easy for attackers to tell when this sort of scan blocking software is in place, because they will not 
be able to connect to purportedly open ports after doing a port scan. 
They will try again from another system and successfully connect, confirming that the original IP was blocked.

Attackers can then use the host spoofing techniques discussed previously (-S option) to cause the target host 
to block any systems the attacker desires. 
This may include important DNS servers, major web sites, software update archives, mail servers, and the like. 
It probably would not take long to annoy the legitimate administrator enough to disable reactive blocking.

While most such products offer a whitelist option to prevent blocking certain important hosts, enumerating them 
all is extraordinarily difficult. 
Attackers can usually find a new commonly used host to block, annoying users until the administrator determines 
the problem and adjusts the whitelist accordingly.

\section{Exploiting Intrusion Detection Systems}
The most audacious way to subvert intrusion detection systems is to hack them. 
Many commercial and open source vendors have pitiful security records of product exploitability.

Internet Security System's flagship RealSecure and BlackICE IDS products had a vulnerability which allowed the 
Witty worm to compromise more than ten thousand installations, then disabled the IDSs by corrupting their filesystems. 
Other IDS and firewall vendors such as Cisco, Checkpoint, Netgear, and Symantec have suffered serious remotely 
exploitable vulnerabilities as well.

Open source sniffers have not done much better, with exploitable bugs found in Snort, Wireshark, tcpdump, FakeBO, and many others. 
Protocol parsing in a safe and efficient manner is extremely difficult, and most of the applications need to parse hundreds of protocols. 
Denial of service attacks that crash the IDS (often with a single packet) are even more common 
than these privilege escalation vulnerabilities. 
A crashed IDS will not detect any Nmap scans.

\section{Ignoring Intrusion Detection Systems}
While advanced attackers will often employ IDS subversion techniques described in this chapter, the 
much more common novice attackers (script kiddies) rarely concern themselves with IDSs. 
Many companies do not even deploy an IDS, and those that do often have them misconfigured or pay little attention to the alerts. 
An Internet-facing IDS will see so many attacks from script kiddies and worms that a few Nmap scans to locate 
a vulnerable service are unlikely to raise any flags.

Hackers want to compromise negligently administered and poorly monitored networks that will provide long-lasting nodes for criminal activity.
Being tracked down and prosecuted is rarely a concern of the IDS-ignoring set. 
They usually launch attacks from other compromised networks rr they may use anonymous connectivity such as provided by 
some Internet cafes, school computer labs, libraries, or the prevalent open wireless access points. 
Throwaway dialup accounts are also commonly used. 
Even if they get kicked off, signing up again with another (or the same) provider takes only minutes. 
Many attackers come from Romania, China, South Korea, and other countries where prosecution is highly unlikely.

Internet worms are another class of attack that rarely bothers with IDS evasion. 
Shameless scanning of millions of IP addresses is preferred by both worms and script kiddies as it 
leads to more compromises per hour than a careful, targeted approach that emphasizes stealth.

While most attacks make no effort at stealth, the fact that most intrusion detection systems are so easily subverted is a major concern. 
Skilled attackers are a small minority, but are often the greatest threat. 
Do not be lulled into complacency by the large number of alerts spewed from IDSs. 
They cannot detect everything, and often miss what is most important.
Even skilled hackers sometimes ignore IDS concerns for initial reconnaissance. 
They simply scan away from some untraceable IP address, hoping to blend in with all of the other 
attackers and probe traffic on the Internet. 
After analyzing the results, they may launch more careful, stealthy attacks from other systems.

%https://it.wikipedia.org/wiki/Sistema_di_rilevamento_delle_intrusioni
Un IDS è composto da quattro componenti:
\begin{itemize}
    \item uno o più sensori utilizzati per ricevere le informazioni dalla rete o dai computer
    \item un motore che analizza i dati prelevati dai sensori e provvede a individuare eventuali falle nella sicurezza informatica.
    \item una console utilizzata per monitorare lo stato della rete e dei computer
    \item un database cui si appoggia il motore di analisi e dove sono memorizzate una serie di regole utilizzate per identificare violazioni della sicurezza.
\end{itemize}
Esistono diversi tipi di IDS che si differenziano a seconda del loro compito specifico e dei metodi usati per individuare violazioni della sicurezza. 
Il più semplice IDS è un dispositivo che integra tutte le componenti in un solo apparato.

Un IDS consiste quindi in un insieme di tecniche e metodi realizzati ad-hoc per rilevare pacchetti dati sospetti a livello di rete, di trasporto o di applicazione.

Due sono le categorie base: sistemi basati sulle firme (signature) e sistemi basati sulle anomalie (anomaly).
\begin{itemize}
    \item La tecnica basata sulle firme è in qualche modo analoga a quella per il rilevamento dei virus, che permette di bloccare file infetti e si tratta della tecnica più utilizzata. 
    \item  sistemi basati sul rilevamento delle anomalie utilizzano un insieme di regole che permettono di distinguere ciò che è "normale" da ciò che è "anormale".
\end{itemize}
L'IDS non cerca di bloccare le eventuali intrusioni, cosa che spetta al firewall, ma cerca di rilevarle laddove si verifichino.


Le tecniche di rilevamento intrusione possono essere divise in misuse detection, 
che usano pattern di attacchi ben conosciuti o di punti deboli del sistema per identificare le intrusioni, 
ed in anomaly detection, che cercano di determinare una possibile deviazione dai pattern stabiliti di utilizzazione normale del sistema

Un misuse detection system, conosciuto anche come signature based intrusion detection system, 
identifica le intrusioni ricercando pattern nel traffico di rete o nei dati generati dalle applicazioni.
Questi sistemi codificano e confrontano una serie di segni caratteristici (signature action) dei vari tipi di scenari di intrusione conosciute.
\begin{itemize}
    \item I principali svantaggi di tali sistemi sono che i pattern di intrusione conosciuti richiedono normalmente di essere 
    inseriti manualmente nel sistema, ma il loro svantaggio è soprattutto di non essere in grado di rilevare qualsiasi futuro 
    (quindi sconosciuto) tipo di intrusione se esso non è presente nel sistema.
    \item Il grande beneficio che invece hanno è quello di generare un numero relativamente basso di falsi positivi e 
    di essere adeguatamente affidabili e veloci.
\end{itemize}

Per ovviare al problema delle mutazioni sono nati gli anomaly based intrusion detection system, 
che analizzano il funzionamento del sistema alla ricerca di anomalie. 
Questi sistemi fanno uso di profili (pattern) dell'utilizzo normale del sistema ricavati da misure statistiche ed euristiche 
sulle caratteristiche dello stesso, per esempio, la cpu utilizzata e le attività di i/o di un particolare utente o programma. 
Le anomalie vengono analizzate e il sistema cerca di definire se sono pericolose per l'integrità del sistema.

Gli IDS si possono suddividere anche a seconda di cosa analizzano: 
esistono gli IDS che analizzano le reti locali, quelli che analizzano gli Host e gli IDS ibridi che analizzano la rete e gli Host.


A differenza del firewall che, con una lista di controllo degli accessi, definisce un insieme di regole che i pacchetti devono 
rispettare per entrare o per uscire dalla rete locale, un IDS controlla lo stato dei pacchetti che 
girano all'interno della rete locale confrontandolo con situazioni pericolose già successe prima o 
con situazioni di anomalia definite dall'amministratore di sistema.
Un firewall può bloccare un pacchetto, mentre un IDS agisce in modo passivo, cioè quando rileva la 
presenza di un'anomalia genera un allarme senza però bloccarla.

%https://en.wikipedia.org/wiki/Intrusion_detection_system
An intrusion detection system (IDS) is a device or software application that monitors a network or 
systems for malicious activity or policy violations.
Any intrusion activity or violation is typically either reported to an administrator or collected centrally.

The most common classifications are network intrusion detection systems (NIDS) and host-based intrusion detection systems (HIDS). 
A system that monitors important operating system files is an example of an HIDS, while a system 
that analyzes incoming network traffic is an example of an NIDS.

It is also possible to classify IDS by detection approach. 
The most well-known variants are signature-based detection (recognizing bad patterns, such as exploitation attempts) and 
anomaly-based detection (detecting deviations from a model of "good" traffic, which often relies on machine learning).

Some IDS products have the ability to respond to detected intrusions. 
Systems with response capabilities are typically referred to as an intrusion prevention system (IPS).

Intrusion prevention systems (IPS), also known as intrusion detection and prevention systems (IDPS), are network security 
appliances that monitor network or system activities for malicious activity. 
The main functions of intrusion prevention systems are to identify malicious activity, 
log information about this activity, report it and attempt to block or stop it.


%https://www.geeksforgeeks.org/intrusion-detection-system-ids/
Intrusion is when an attacker gets unauthorized access to a device, network, or system. 
Cyber criminals use advanced techniques to sneak into organizations without being detected.

Intrusion Detection System (IDS) observes network traffic for malicious transactions and sends immediate alerts when it is observed. 
It is software that checks a network or system for malicious activities or policy violations.

Common Methods of Intrusion
\begin{itemize}
    \item Address Spoofing: Hiding the source of an attack by using fake or unsecured proxy servers making it hard to identify the attacker.
    \item Fragmentation: Sending data in small pieces to slip past detection systems.
    \item Pattern Evasion: Changing attack methods to avoid detection by IDS systems that look for specific patterns.
    \item Coordinated Attack: Using multiple attackers or ports to scan a network, confusing the IDS and making it hard to see what is happening.
\end{itemize}

An IDS (Intrusion Detection System) monitors the traffic on a computer network to detect any suspicious activity.
It analyzes the data flowing through the network to look for patterns and signs of abnormal behavior.
The IDS compares the network activity to a set of predefined rules and patterns to identify any activity that might indicate an attack or intrusion.
If the IDS detects something that matches one of these rules or patterns, it sends an alert to the system administrator.
The system administrator can then investigate the alert and take action to prevent any damage or further intrusion.

 Intrusion Detection System are classified into 5 types:
 \begin{itemize}
    \item Network Intrusion Detection System (NIDS): NIDS are set up at a planned point within the network to examine traffic from all devices on the network. 
    It performs an observation of passing traffic on the entire subnet and matches the traffic that is passed on the subnets to the collection of known attacks.
    Once an attack is identified or abnormal behavior is observed, the alert can be sent to the administrator
    \item Host Intrusion Detection System (HIDS): HIDS run on independent hosts or devices on the network.
    It monitors the incoming and outgoing packets from the device only and will alert the administrator if suspicious or malicious activity is detected.
    \item Hybrid Intrusion Detection System: HIDS is made by the combination of two or more approaches to the intrusion detection system. 
    In the hybrid intrusion detection system, the host agent or system data is combined with network information to develop a complete view of the network system.
    \item Application Protocol-Based Intrusion Detection System (APIDS): APIDS is a system or agent that generally resides within a group of servers. 
    It identifies the intrusions by monitoring and interpreting the communication on application-specific protocols. 
    For example, this would monitor the SQL protocol explicitly to the middleware as it transacts with the database in the web server.
    \item Protocol-Based Intrusion Detection System (PIDS):  It comprises a system or agent that would consistently reside at the front end of a server, controlling and interpreting the protocol between a user/device and the server.
    \item Signature-Based Detection: Signature-based detection checks network packets for known patterns linked to specific threats. A signature-based IDS compares packets to a database of attack signatures and raises an alert if a match is found.
 \end{itemize}

Intrusion Detection System Evasion Techniques
\begin{itemize}
    \item Fragmentation: Dividing the packet into smaller packet called fragment and the process is known as fragmentation.
    \item Packet Encoding: Encoding packets using methods like Base64 or hexadecimal can hide malicious content from signature-based IDS.
    \item Traffic Obfuscation: By making message more complicated to interpret, obfuscation can be utilised to hide an attack and avoid detection.
    \item Encryption: Several security features such as data integrity, confidentiality, and data privacy, are provided by encryption. Unfortunately, security features are used by malware developers to hide attacks and avoid detection.
\end{itemize}

Detection Method of IDS
$\bullet$
Signature-Based Method: Signature-based IDS detects the attacks on the basis of the specific patterns such as the number of bytes or 
a number of 1s or the number of 0s in the network traffic. 
It also detects on the basis of the already known malicious instruction sequence that is used by the malware. 
The detected patterns in the IDS are known as signatures. 
Signature-based IDS can easily detect the attacks whose pattern (signature) already exists in the system but it is 
quite difficult to detect new malware attacks as their pattern (signature) is not known.
$\bullet$
Anomaly-Based Method: Anomaly-based IDS was introduced to detect unknown malware attacks as new malware is developed rapidly. 
In anomaly-based IDS there is the use of machine learning to create a trustful activity model and anything coming is compared 
with that model and it is declared suspicious if it is not found in the model. 
The machine learning-based method has a better-generalized property in comparison to signature-based IDS as these models 
can be trained according to the applications and hardware configurations.


\subsection{Vantaggi}
\subsection{Svantaggi}


%https://www.ibm.com/it-it/topics/intrusion-detection-system
Un sistema di rilevamento delle intrusioni (IDS) è uno strumento di sicurezza della rete che monitora il traffico di 
rete e i dispositivi alla ricerca di attività dannose note, attività sospette o violazioni delle politiche di sicurezza.

Un IDS può aiutare ad accelerare e automatizzare il rilevamento delle minacce di rete avvisando gli amministratori della 
sicurezza di minacce note o potenziali o inviando avvisi a uno strumento di sicurezza centralizzato. 

Uno strumento di sicurezza centralizzato come un sistema SIEM (Security Information and Event Management) può 
combinare dati provenienti da altre fonti per aiutare i team addetti alla sicurezza a individuare e rispondere alle 
minacce informatiche che potrebbero sfuggire ad altre misure di sicurezza.

Un sistema IDS non può fermare da solo le minacce alla sicurezza. Le funzionalità IDS di oggi sono solitamente 
integrate (o incorporate) all'interno dei sistemi di prevenzione delle intrusioni (IPS), in grado di rilevare le 
minacce e di agire automaticamente per prevenirle.


\subsection{Come funzionano i sistemi di rilevamento delle intrusioni}

\textbf{Rilevamento basato sulla firma}
Il rilevamento basato sulle firme analizza i pacchetti di rete alla ricerca di firme di attacchi, caratteristiche o 
comportamenti unici associati a una specifica minaccia. 
Una sequenza di codice visualizzata in una particolare variante malware è un esempio di una firma di attacco.

Un IDS basato su firme mantiene un database di firme di intrusioni con cui confronta i pacchetti di rete. 
Se un pacchetto attiva una corrispondenza con una delle firme, l'IDS lo segnala. 
Per essere efficaci, i database delle firme devono essere aggiornati regolarmente con nuova threat intelligence man mano che 
emergono nuovi attacchi informatici e gli attacchi esistenti si evolvono. 
I nuovi attacchi non ancora analizzati per le firme possono eludere un IDS basato su firme.

\hyperlink{https://www.ibm.com/it-it/topics/threat-intelligence}{threat intelligence}


\textbf{Rilevamento basato sulle anomalie}
I metodi di rilevamento basati sulle anomalie utilizzano l'apprendimento automatico per creare e perfezionare 
costantemente un modello di riferimento per la normale attività di rete. 
Quindi confronta l'attività di rete con il modello e contrassegna le deviazioni

Gli IDS basati sulle anomalie spesso riescono a rilevare nuovi attacchi informatici che potrebbero eludere il rilevamento basato sulle firme
Ma gli IDS basati sulle anomalie sono anche più soggetti a falsi positivi. 
Anche l'attività non dannosa, ad esempio un utente autorizzato che accede a una risorsa di rete sensibile per la prima volta, può attivare un IDS basato sulle anomalie.


\subsection{Metodi di rilevamento meno comuni}
Il rilevamento basato sulla reputazione blocca il traffico proveniente da indirizzi IP e domini connessi a attività dannose o sospette. 
L'analisi del protocollo con stato si concentra sul comportamento del protocollo: ad esempio, potrebbe individuare un attacco 
denial-of-service (DoS) rilevando un singolo indirizzo IP, facendo numerose richieste di connessione TCP simultanee in un breve periodo.


Indipendentemente dai metodi utilizzati, quando un IDS rileva una potenziale minaccia o violazione delle politiche, 
avvisa il team di risposta agli incidenti affinché indaghi. 
Gli IDS registrano anche gli incidenti di sicurezza, sia nei propri log, sia registrandoli all'interno di uno strumento di 
gestione delle informazioni e degli eventi di sicurezza (SIEM) 
I log degli incidenti possono essere utilizzati per affinare i criteri degli IDS, ad esempio aggiungendo nuove 
firme di attacchi o aggiornando il modello di comportamento della rete.


\subsection{Tipi di sistemi di prevenzione delle intrusioni}
Gli IDS sono classificati in base alla posizione in cui si trovano all'interno di un sistema e al tipo di attività che monitorano. 

I \textbf{sistemi di rilevamento delle intrusioni di rete} (NIDS) monitorano il traffico in entrata e in uscita verso i dispositivi in tutta la rete. 
I sistemi NIDS sono posizionati in punti strategici della rete, spesso immediatamente dietro i firewall sul perimetro della rete, in modo 
da poter segnalare qualsiasi traffico dannoso che vi penetra.
I sistemi NIDS possono anche essere inseriti all'interno della rete per individuare le minacce interne o eventuali hacker che hanno violato gli account degli utenti.

Un NIDS analizza le copie dei pacchetti di rete piuttosto che i pacchetti stessi. 
In questo modo, il traffico autorizzato non deve attendere l'analisi e il sistema NIDS sarà comunque in grado di rilevare e contrassegnare il traffico dannoso.


I \textbf{sistemi di rilevamento delle intrusioni host} (HIDS) sono installati su un endpoint specifico, ad esempio un laptop, un router o 
un server. L'HIDS monitora solo l'attività di quel dispositivo, incluso il traffico in entrata e uscita. 
Un HIDS funziona in genere eseguendo snapshot periodici dei file critici del sistema operativo e confrontandoli nel tempo. 
Se l'HIDS nota un cambiamento, ad esempio la modifica dei file di log o l'alterazione delle configurazioni, avvisa il team di sicurezza.

Un \textbf{IDS basato sul protocollo} (PIDS) monitora i protocolli di connessione tra server e dispositivi. I PIDS vengono spesso collocati sui server web per monitorare le connessioni HTTP o HTTPS.

Un \textbf{IDS basato sul protocollo applicativo} (APIDS) funziona a livello di applicazione, monitorando i protocolli specifici delle applicazioni. Un APIDS viene spesso distribuito tra un server web e un database SQL per rilevare l'SQL injection.


\subsection{Tattiche di evasione agli IDS}
Alcune tattiche comuni di evasione degli IDS includono:

Attacchi DDoS (Distributed Denial of Service): mettono offline gli IDS inondandoli di traffico ovviamente dannoso proveniente da più fonti. 
Quando le risorse dell'IDS sono sopraffatte dalle finte minacce, gli hacker si intrufolano.

Spoofing : falsificazione di indirizzi IP e record DNS per far sembrare che il traffico provenga da un'origine affidabile.

Frammentazione: la suddivisione di malware o altri payload dannosi in 
pacchetti più piccoli, che oscura la firma e impedisce il rilevamento. 
Ritardando strategicamente l'invio dei pacchetti o inviandoli in ordine sparso, gli hacker impediscono che 
gli IDS possano riassemblarli e notare l'attacco.

Crittografia: l'utilizzo di protocolli crittografati per aggirare un IDS se quest'ultimo non dispone della chiave di decodifica corrispondente.

Sfinimento dell'operatore: generare un gran numero di avvisi IDS per distrarre il team di risposta agli incidenti dalle attività reali.


\subsection{IDS e altre soluzioni di protezione}
Gli IDS non sono strumenti autonomi. 
Sono progettati come parte di un sistema di sicurezza informatica olistico e spesso sono strettamente integrati con una o più delle seguenti soluzioni di sicurezza:

\subsubsection{IDS e SIEM (informazioni sulla sicurezza e gestione degli eventi)}
Gli avvisi IDS vengono spesso incanalati nel SIEM di un'organizzazione, dove possono essere combinati con 
avvisi e informazioni provenienti da altri strumenti di sicurezza in un'unica dashboard centralizzato.

L'integrazione degli IDS con i SIEM consente ai team addetti alla sicurezza di arricchire gli avvisi IDS con informazioni sulle 
minacce e dati provenienti da altri strumenti, filtrare falsi allarmi‌ e assegnare priorità agli incidenti per la correzione.


\subsubsection{IDS e IPS (sistemi di prevenzione delle intrusioni)}
Un IPS monitora il traffico di rete alla ricerca di attività sospette, come gli IDS, e intercetta le minacce in tempo reale, 
interrompendo automaticamente le connessioni o attivando altri strumenti di sicurezza.

Alcune organizzazioni implementano un IDS e un IPS come soluzioni separate. Più spesso, IDS e IPS vengono combinati in un unico 
sistema di rilevamento e prevenzione delle intrusioni (IDPS) che rileva le intrusioni, le registra, 
avvisa i team addetti alla sicurezza e risponde in modo automatico.

\subsection{IDS e firewall}
IDS e firewall sono complementari. 
I firewall sono rivolti all'esterno della rete e hanno funzione di barriera utilizzando set di 
regole predefinite per consentire o impedire il traffico. 
Gli IDS spesso si trovano vicino ai firewall e aiutano a fermare tutto ciò che li supera. 
Alcuni firewall, in particolare i firewall di nuova generazione, dispongono di funzioni IDS e IPS integrate.


%https://cyberment.it/sicurezza-informatica/ids-intrusion-detection-system-di-cosa-si-tratta/ 
L’Intrusion Detection System è un sistema di sicurezza informatica basato su controllo e monitoraggio continuo dell’infrastruttura da proteggere.
Quindi l’IDS, più che uno scudo vero e proprio, è una sentinella, in grado di lanciare l’allarme affinché si intervenga manualmente per bloccare l’intrusione.

In sostanza, l’IDS analizza il traffico di rete, i log di sistema o altre fonti di dati per identificare:
pattern
comportamenti anomali
firme di attacchi noti

Quando viene rilevata un’attività sospetta, l’IDS può generare avvisi o segnalazioni per notificare le 
anomalie agli amministratori di sistema o altri responsabili della sicurezza.

Questa tecnologia consente di avere piena visibilità su infrastrutture tecnologiche e reti, 
semplificando la redazione di audit e l’adeguamento alle normative di sicurezza.

Qualsiasi sistema di rilevamento delle intrusioni può essere composto tanto da una componente software, 
quanto da un dispositivo hardware (o da una combinazione di queste soluzioni).
Solitamente gli IDS vengono posizionati dopo i firewall perimetrali, i quali si occupano propriamente della 
“schermatura” dei pacchetti di dati tramite il controllo del traffico in entrata in uscita.


L’intrusion detection è basato sulle tecniche di analisi comportamentale
Queste cercano di descrivere e classificare il comportamento normale o anomalo di un utente.
Le tecniche di analisi comportamentale si basano principalmente su due metriche di rilevazione, la misuse detection e la anomaly detection:
La differenza principale tra i due sistemi di prevenzione è la generazione di falsi positivi, e quindi la rilevazione di intrusioni false.

L’anomaly detection incappa maggiormente in falsi positivi. 
La misuse detection, dal canto suo, produce un numero molto basso di falsi positivi ma, allo stesso tempo, 
fa registrare un aumento di falsi negativi, e quindi di attacchi informatici non rilevati o riconosciuti.


\subsection{Tipologie di IDS}
La classificazione degli IDS è basata sia sulle differenti tecniche di rilevamento che sul posizionamento dei sensori.
Ci sono prevalentemente tre tipologie di rilevamento IDS:
$\bullet$ basato su firma (SIDS): questo approccio si basa sul confronto delle firme degli attacchi conosciuti con 
il traffico di rete o i file in arrivo. 
Quando l’IDS trova una corrispondenza tra una firma conosciuta e il traffico in analisi, 
segnala l’evento come un possibile attacco. Questo metodo è efficace nel rilevare attacchi noti, 
ma può essere inefficace nel rilevare attacchi nuovi o modificati che non corrispondono alle firme esistenti
$\bullet$ basato su comportamento (BIDS): approccio basato sull’identificazione di comportamenti 
anomali rispetto a un modello di comportamento normale. 
L’IDS crea un profilo delle attività normali sulla rete o sul sistema e confronta 
i dati in ingresso con questo modello. 
Se viene rilevata una deviazione significativa rispetto al modello normale, l’IDS segnala un possibile attacco. 
Questo metodo può rilevare attacchi nuovi o sconosciuti, ma può anche generare falsi positivi se le deviazioni sono causate da attività legittime ma inconsuete
$\bullet$ analisi dei protocolli di stato: che consiste nell’analisi dei protocolli di 
comunicazione e dei loro stati per identificare attività anomale. 
L’IDS monitora i flussi di comunicazione e controlla se gli stati e i comportamenti sono 
conformi alle definizioni di attività normali. 
Questo metodo può rilevare attacchi che sfruttano vulnerabilità nei protocolli o che cercano di manipolare gli stati di comunicazione
$\bullet$ IDS ibridi: questi IDS combinano più approcci, come la firma, l’analisi del comportamento e 
la rilevazione delle anomalie, per ottenere una copertura più ampia nel rilevamento degli attacchi. 
Possono utilizzare sia dati di rete che dati dell’host per analizzare l’intero ecosistema IT e 
fornire una visione più completa delle minacce.


%https://www.fortinet.com/resources/cyberglossary/intrusion-detection-system
An intrusion detection system (IDS) is an application that monitors network traffic and 
searches for known threats and suspicious or malicious activity. 
The IDS sends alerts to IT and security teams when it detects any security risks and threats.
Most IDS solutions simply monitor and report suspicious activity and traffic when they detect an anomaly. 
However, some can go a step further by taking action when it detects anomalous activity, 
such as blocking malicious or suspicious traffic.

The answer to "what is intrusion" is typically an attacker gaining unauthorized access to a device, network, or system. 
Cyber criminals use increasingly sophisticated techniques and tactics to infiltrate organizations without being discovered. 
This includes common techniques like:
\begin{itemize}
    \item Address spoofing: The source of an attack is hidden using spoofed, misconfigured, and poorly secured proxy servers, which makes it difficult for organizations to discover attackers.
    \item Fragmentation: Fragmented packets enable attackers to bypass organizations’ detection systems.
    \item Pattern evasion: Hackers adjust their attack architectures to avoid the patterns that IDS solutions use to spot a threat.
    \item Coordinated attack: A network scan threat allocates numerous hosts or ports to different attackers, making it difficult for the IDS to work out what is happening.
\end{itemize}

IDS solutions come in a range of different types and varying capabilities. Common types of intrusion detection systems (IDS) include:
\begin{itemize}
    \item Network intrusion detection system (NIDS): A NIDS solution is deployed at strategic points within an organization’s network to monitor incoming and outgoing traffic. This IDS approach monitors and detects malicious and suspicious traffic coming to and going from all devices connected to the network.
    \item Host intrusion detection system (HIDS): A HIDS system is installed on individual devices that are connected to the internet and an organization’s internal network. This solution can detect packets that come from inside the business and additional malicious traffic that a NIDS solution cannot. It can also discover malicious threats coming from the host, such as a host being infected with malware attempting to spread it across the organization’s system.
    \item Signature-based intrusion detection system (SIDS): A SIDS solution monitors all packets on an organization’s network and compares them with attack signatures on a database of known threats. 
    \item Anomaly-based intrusion detection system (AIDS): This solution monitors traffic on a network and compares it with a predefined baseline that is considered "normal." It detects anomalous activity and behavior across the network, including bandwidth, devices, ports, and protocols. An AIDS solution uses machine-learning techniques to build a baseline of normal behavior and establish a corresponding security policy. This ensures businesses can discover new, evolving threats that solutions like SIDS cannot. 
    \item Perimeter intrusion detection system (PIDS): A PIDS solution is placed on a network to detect intrusion attempts taking place on the perimeter of organizations’ critical infrastructures.
    \item Virtual machine-based intrusion detection system (VMIDS): A VMIDS solution detects intrusions by monitoring virtual machines. It enables organizations to monitor traffic across all the devices and systems that their devices are connected to.
    \item Stack-based intrusion detection system (SBIDS): SBIDS is integrated into an organization’s Transmission Control Protocol/Internet Protocol (TCP/IP), which is used as a communications protocol on private networks. This approach enables the IDS to watch packets as they move through the organization’s network and pulls malicious packets before applications or the operating system can process them.
\end{itemize}

An IDS works by looking for the signature of known attack types or detecting activity that deviates from a prescribed normal. It then alerts or reports these anomalies and potentially malicious actions to administrators so they can be examined at the application and protocol layers.

This enables organizations to detect the potential signs of an attack beginning or being carried out by an attacker. IDS solutions do this through several capabilities, including:
\begin{itemize}
    \item Monitoring the performance of key firewalls, files, routers, and servers to detect, prevent, and recover from cyberattacks
    \item Enabling system administrators to organize and understand their relevant operating system audit trails and logs that are often difficult to manage and track
    \item Providing an easy-to-use interface that allows staff who are not security experts to help with the management of an organization’s systems
    \item Providing an extensive database of attack signatures that can be used to match and detect known threats
    \item Providing a quick and effective reporting system when anomalous or malicious activity occurs, which enables the threat to be passed up the stack
    \item Generating alarms that notify the necessary individuals, such as system administrators and security teams, when a breach occurs
    \item In some cases, reacting to potentially malicious actors by blocking them and their access to the server or network to prevent them from carrying out any further action
\end{itemize}


While IDS solutions are important tools in monitoring and detecting potential threats, they are not without their challenges. These include:
\begin{itemize}
    \item False alarms: Also known as false positives, these leave IDS solutions vulnerable to identifying potential threats that are not a true risk to the organization. To avoid this, organizations must configure their IDS to understand what normal looks like, and as a result, what should be considered as malicious activity.
    \item False negatives: This is a bigger concern, as the IDS solution mistakes an actual security threat for legitimate traffic. An attacker is allowed to pass into the organization’s network, with IT and security teams oblivious to the fact that their systems have been infiltrated. 
\end{itemize}
As the threat landscape evolves and attackers become more sophisticated, it is preferable for IDS solutions to provide false positives than false negatives. In other words, it is better to discover a potential threat and prove it to be wrong than for the IDS to mistake attackers for legitimate users.


An IDS solution is typically limited to the monitoring and detection of known attacks and activity that deviates from a baseline normal prescribed by an organization.
An intrusion prevention system (IPS) goes beyond this by blocking or preventing security risks. An IPS can both monitor for malicious events and take action to prevent an attack from taking place.


Firewalls and intrusion detection systems (IDS) are cybersecurity tools that can both safeguard a network or endpoint. Their objectives, however, are very different from one another.
\begin{itemize}
    \item IDS: Intrusion detection systems are passive monitoring tools that identify possible threats and send out notifications to analysts in security operations centers (SOCs). In this way, incident responders can promptly look into and address the potential event.
    \item Firewall: A firewall, on the other hand, analyzes the metadata contained in network packets and decides whether to allow or prohibit traffic into or out of the network based on pre-established rules. A firewall essentially creates a barrier that stops certain traffic from crossing through it. 
\end{itemize}
 An IDS is focused on detecting and generating alerts about threats, while a firewall inspects inbound and outbound traffic, keeping all unauthorized traffic at bay.  



%https://www.ionos.it/digitalguide/server/sicurezza/intrusion-detection-system-ids/
Si distingue tra tre tipi di IDS: possono essere basati su host (HIDS), basati su rete (NIDS) o ibridi (ovvero combinare HIDS e NIDS).

\textbf{HIDS: Intrusion Detection System basati su host}
L’Intrusion Detection System basato su host è la forma più vecchia del sistema di sicurezza. L’IDS viene installato direttamente sul relativo sistema. Analizza i dati direttamente a livello di registro e di kernel, controllando anche altri file di sistema. Per poter essere usato nelle stazioni di lavoro autonome, l’Intrusion Detection System basato su host ricorre ad agenti di monitoraggio che prefiltrano il traffico di dati e inoltrano le informazioni così acquisite al server centrale. Il metodo è molto preciso ed esteso, ma può essere aggirato dagli attacchi DoS e DDoS. Inoltre, questo sistema di rilevamento delle intrusioni dipende dal sistema operativo.

\textbf{NIDS: Intrusion Detection System basati su rete}
Un IDS basato su rete scansiona i pacchetti di dati che vengono inviati e ricevuti all’interno di una rete. Così, i modelli insoliti o discordanti vengono rilevati e segnalati rapidamente. Da questo punto di vista, il volume di dati inviati può risultare problematico. Se supera le capacità dell’Intrusion Detection System, il monitoraggio non riesce più a essere capillare.

\textbf{Intrusion Detection System ibrido}
Oggi si punta molto sugli Intrusion Detection System ibridi, che coniugano i due diversi approcci. Questi sistemi sono costituiti da sensori basati su host, sensori basati su rete e un livello di gestione in cui confluiscono i risultati, che vengono poi analizzati approfonditamente. Anche il comando parte da questo livello.



\textbf{Monitoraggio dati}
Con il monitoraggio dati, appositi sensori raccolgono tutti i dati pertinenti e li filtrano in base alla rilevanza. Si tratta sia di informazioni da parte dell’host sia di file di registro e dati di sistema. Vengono considerati anche i pacchetti di dati inviati attraverso la rete. L’IDS rileva e ordina anche indirizzi di origine e di destinazione e altre caratteristiche importanti. Il presupposto più importante è che i dati raccolti provengano da una fonte affidabile o dallo stesso Intrusion Detection System.

\textbf{Analisi}
Il secondo componente dell’Intrusion Detection System è l’analizzatore, che valuta tutti i dati ottenuti e prefiltrati sfruttando diversi modelli.
L’analizzatore può contare su due metodi diversi:
\begin{itemize}
    \item Misuse Detection: con la tecnica del Misuse Detection (traducibile con “rilevamento per uso improprio”), l’analizzatore tenta di riconoscere tra i dati ricevuti schemi di attacco già noti. Questi sono salvati in un database separato, che viene aggiornato regolarmente. Gli attacchi sferrati con una firma già registrata possono così essere rilevati tempestivamente. Quelli che invece non sono ancora noti al sistema non vengono rilevati.
    \item Anomaly Detection: questa tecnica (traducibile con “rilevamento per anomalia”) si basa sull’osservazione dell’intero sistema. Non appena una o più fasi di lavoro deviano dalla norma, viene segnalata l’anomalia, ad esempio quando il carico della CPU supera un valore prestabilito o gli accessi a una pagina aumentano in modo inusuale. L’Intrusion Detection System può anche controllare la successione temporale degli eventi per rilevare gli schemi di attacco sconosciuti. Talvolta vengono però segnalate anche anomalie innocue.
\end{itemize}

%https://www.redbooks.ibm.com/redpapers/pdfs/redp4226.pdf
%file:///D:/Tesi%20Magistrale/file/mitigazione/ids_ibm.pdf

%https://ieeexplore.ieee.org/stamp/stamp.jsp?tp=&arnumber=9623451
%file:///D:/Tesi%20Magistrale/file/mitigazione/nids_iee.pdf

%https://www.sciencedirect.com/science/article/pii/S1877050922004422
%file:///D:/Tesi%20Magistrale/file/mitigazione/ids_w_supervised_ML.pdf

%https://docs.trendmicro.com/all/ent/apex-one/2019/en-us/apexOne_2019_agent_olh/Intrusion-Detection-.html

%https://www.sciencedirect.com/science/article/abs/pii/S1084804512001944
%https://www.sciencedirect.com/science/article/pii/S2665917423001630
%file:///D:/Tesi%20Magistrale/file/mitigazione/ids_w_AI.pdf

%https://onlinelibrary.wiley.com/doi/full/10.1002/ett.4150?msockid=0d2c7404210469333f0b6091207c68dd
%file:///D:/Tesi%20Magistrale/file/mitigazione/Transactions%20on%20Emerging%20Telecommunications%20Technologies%20-%202020%20-%20Ahmad%20-%20Network%20intrusion%20detection%20system%20%20A%20systematic.pdf

%https://www.frontiersin.org/journals/computer-science/articles/10.3389/fcomp.2024.1387354/full
%file:///D:/Tesi%20Magistrale/file/mitigazione/fcomp-06-1387354.pdf

%https://www.mdpi.com/2078-2489/16/7/515
%file:///D:/Tesi%20Magistrale/file/mitigazione/information-16-00515-v2.pdf

%https://www.academia.edu/Documents/in/Intrusion_Detection_System

%file:///D:/Tesi%20Magistrale/file/mitigazione/NCRTCA-PID-428.pdf

%https://www.degruyterbrill.com/document/doi/10.1515/jisys-2023-0248/html
%file:///D:/Tesi%20Magistrale/file/mitigazione/10.1515_jisys-2023-0248.pdf

%https://www.researchgate.net/publication/316599266_INTRUSION_DETECTION_SYSTEM
%https://www.researchgate.net/publication/339551603_INTRUSION_DETECTION_SYSTEM_-A_STUDY
%https://ijsrset.com/paper/6331.pdf
%file:///D:/Tesi%20Magistrale/file/mitigazione/nids_using_AI.pdf
%https://www.academia.edu/5234791/INTRUSION_DETECTION_SYSTEMS_IDS_AND_INTRUSION_PREVENTION_SYSTEMS_IPS_FOR_NETWORK_SECURITY_A_CRITICAL_ANALYSIS
%https://www.researchgate.net/publication/339551603_INTRUSION_DETECTION_SYSTEM_-A_STUDY
%file:///D:/Tesi%20Magistrale/file/mitigazione/ids_review.pdf
%file:///D:/Tesi%20Magistrale/file/mitigazione/ids_based_AI.pdf
%https://www.researchgate.net/publication/364200354_Intrusion_Detection_System_using_Machine_Learning_Techniques_A_Review
%

%---------------------
%https://www.proofpoint.com/it/threat-reference/intrusion-detection-system-ids
\section{Tipologie di Intrusion Detection Systems (IDS)}

Gli IDS (Intrusion Detection Systems) utilizzano varie tecniche di rilevamento per identificare le attività sospette all’interno di una rete. Mentre i primi due (sotto) sono i tipi principali di rilevamento IDS, per ambienti specifici vengono utilizzati metodi alternativi:

\textbf{Rilevamento basato sulle firme}
Il rilevamento basato sulle firme è uno dei metodi di rilevamento più comuni e si basa su un database di modelli di attacco noti, spesso definiti “firme”. Quando il traffico in entrata corrisponde a uno di questi schemi, viene generato un avviso. Pur essendo efficace contro le minacce note, non è in grado di rilevare nuove minacce non registrate in precedenza.

\textbf{Rilevamento basato sulle anomalie}
A differenza dei sistemi basati sulle firme, gli IDS basati sulle anomalie si concentrano sulla definizione di una linea di base del comportamento “normale” della rete. Se il traffico in entrata si discosta significativamente da questa linea di base, viene emesso un avviso. Questo approccio è vantaggioso per rilevare minacce nuove o sconosciute, ma a volte può produrre falsi positivi.

\textbf{Rilevamento basato su euristica}
Gli IDS basati su euristica utilizzano algoritmi e analisi avanzate per prevedere la prossima mossa di un attaccante in base ai suoi modelli di comportamento. Può adattarsi e imparare dal traffico osservato, proteggendo dalle minacce nuove e in evoluzione.

\textbf{Analisi dei protocolli Stateful}
Questo metodo prevede la comprensione e il monitoraggio dello stato dei protocolli di rete in uso. Identifica le deviazioni che potrebbero indicare un attacco confrontando gli eventi osservati con profili predeterminati di definizioni generalmente accettate di attività benigne.

\textbf{Rilevamento basato sulle policy}
Questa tipologia di IDS funziona in base a una serie di criteri o regole definite dall’amministratore di rete. Qualsiasi attività che violi tali criteri attiva un avviso. Si tratta di un approccio proattivo che richiede l’aggiornamento periodico dei criteri per essere sempre attuale.

\textbf{Rilevamento delle honeypot}
Non si tratta di una tecnica di rilevamento tradizionale, gli honeypot sono sistemi esca che attirano potenziali aggressori. Distolgono l’attaccante dai sistemi reali e raccolgono informazioni sui suoi metodi. Le informazioni ricavate dagli honeypot possono informare altri IDS sui modelli di minaccia emergenti.

La comprensione dei diversi tipi di rilevamento è fondamentale per la scelta dell’IDS adatto a specifici ambienti di rete. L’approccio migliore spesso combina più metodi di rilevamento per garantire un livello di protezione completo contro un’ampia gamma di minacce.


\section{Intrusion Detection Systems vs. Intrusion Prevention Systems}
Gli Intrusion Detection Systems (IDS) e gli Intrusion Prevention Systems (IPS) sono strumenti essenziali per la sicurezza di rete, progettati per identificare e contrastare le attività dannose o le violazioni dei criteri all’interno di una rete. La loro distinzione principale risiede nelle rispettive reazioni alle minacce percepite.

Funzionalità e risposta:

IDS: funziona principalmente come un meccanismo di sorveglianza, monitorando da vicino il traffico di rete. Quando rileva attività sospette o anomale, genera avvisi, fungendo da dispositivo di “solo ascolto” senza la capacità di intervenire autonomamente.
IPS: agisce in modo più proattivo. Al di là del semplice rilevamento, un IPS reagisce in tempo reale alle minacce in corso adottando misure per bloccarle, assicurando che non raggiungano mai i bersagli previsti nella rete.
 

Applicazioni e vantaggi:

IDS: oltre alle funzioni principali di rilevamento, gli IDS sono fondamentali per quantificare e classificare i tipi di attacchi. Queste informazioni possono consentire alle organizzazioni di rafforzare le misure di sicurezza, individuare le vulnerabilità o correggere eventuali anomalie di configurazione nei dispositivi di rete.
IPS: essendo uno strumento prevalentemente preventivo, le capacità dell’IPS vanno oltre il semplice rilevamento delle minacce. Cerca attivamente di bloccare o attenuare qualsiasi azione dannosa, fungendo da solida barriera protettiva contro le potenziali intrusioni.
 

Sebbene IDS e IPS abbiano ruoli distinti, spesso funzionano meglio se utilizzati in combinazione. L’IDS garantisce che nulla passi inosservato e l’IPS impedisce alle minacce rilevate di causare danni.

\section{IDS vs. Firewalls}
Gli Intrusion Detection System e i firewall sono entrambi componenti integrali della sicurezza di rete. Tuttavia, hanno scopi diversi, principalmente in base alla loro funzionalità e al meccanismo di risposta.

Funzionalità:

IDS: principalmente uno strumento di monitoraggio, l’IDS scansiona la rete alla ricerca di attività sospette e avvisa gli amministratori quando queste vengono rilevate. Agisce come una telecamera di sorveglianza, che osserva e segnala costantemente.
Firewalls: si tratta di barriere di rete che filtrano il traffico in entrata e in uscita in base a regole predefinite. Sono come dei gatekeeper che decidono che traffico può entrare o uscire da una rete.
 

Meccanismo di risposta:

IDS: Sebbene gli IDS siano in grado di rilevare e segnalare il traffico dannoso, non lo bloccano intrinsecamente.
Firewalls: Bloccano in modo proattivo il traffico non conforme alle regole impostate, offrendo una prima linea di difesa contro le potenziali minacce.
 

Mentre i firewall controllano il flusso di traffico in base a parametri prestabiliti, gli IDS monitorano la rete per identificare e segnalare le anomalie. Per ottenere una solida posizione di sicurezza, l’uso congiunto di entrambi offre una protezione a più livelli, con i firewall che filtrano il traffico indesiderato e gli IDS che assicurano un monitoraggio continuo.

\section{IDS vs. SIEM}
 Mentre l’IDS è uno strumento specializzato nel rilevamento delle minacce, il Security Information and Event Management (SIEM) fornisce una piattaforma completa di analisi e gestione dei dati di sicurezza. Ognuno di questi strumenti opera a diverso titolo all’interno di un quadro di sicurezza di rete.

Funzionalità:

IDS: Concentrandosi principalmente sul rilevamento di attività sospette o anomale in una rete, l’IDS avvisa gli amministratori una volta identificate tali attività. È come una sentinella vigile, sempre alla ricerca di potenziali minacce.
SIEM: il SIEM non si limita al semplice rilevamento, ma raccoglie e centralizza i log e gli eventi provenienti da varie fonti dell’ambiente IT. È un centro di intelligence che consolida i dati per offrire una visione olistica del panorama della sicurezza.
 

Obiettivo:

IDS: il suo compito è generalmente limitato al rilevamento di potenziali minacce basate su schemi o anomalie note.
SIEM: grazie alla sua portata più ampia, il SIEM non solo rileva, ma correla anche i dati, aiuta nell’analisi forense e supporta i rapporti di conformità.
 

Il SIEM opera come centro di controllo principale, offrendo una visione a 360 gradi dello stato della sicurezza, delle tendenze e delle minacce. È la controparte analitica e integrativa della vigilanza dell’IDS. L’utilizzo di entrambi in sinergia garantisce un rapido rilevamento delle minacce, unito a una visione approfondita e a una difesa a più livelli.


\section{Tattiche di evasione IDS}
Con l’evoluzione dei sistemi di rilevamento delle intrusioni, si evolvono anche le tattiche degli attori delle minacce. Molti hacker hanno ideato tecniche per aggirare o eludere il rilevamento da parte degli IDS. La comprensione di questi metodi è fondamentale per rafforzare le difese e mantenere una sicurezza solida. Ecco alcune tecniche di elusione degli IDS comunemente utilizzate e le relative spiegazioni:

Frammentazione: gli hacker dividono i payload dannosi in pacchetti più piccoli o frammenti. Frammentando i dati dannosi in pezzi che non sembrano dannosi di per sé, possono eludere il rilevamento. Una volta all’interno della rete, questi frammenti vengono riassemblati per eseguire l’attacco.

Shellcode polimorfico: il polimorfismo consiste nell’alterare l’aspetto del codice dannoso in modo che la sua firma cambi, ma la sua funzione rimanga la stessa. In questo modo, gli hacker possono rendere il loro codice irriconoscibile alle soluzioni IDS basate sulla firma.

Offuscamento: gli attori delle minacce utilizzano questa tecnica per modificare il payload dell’attacco in modo che il computer di destinazione lo inverta, ma l’IDS no. L’offuscamento può essere utilizzato per sfruttare l’host finale senza allertare l’IDS.

Crittografia e tunneling: criptando il payload dell’attacco o creando un tunnel attraverso un protocollo legittimo (come HTTP o DNS), gli aggressori possono mascherare il loro traffico dannoso, rendendo difficile per gli IDS rilevare il contenuto nascosto.

Attacchi low-and-slow: alcuni aggressori distribuiscono le loro attività su periodi prolungati o limitano la frequenza delle loro richieste, rimanendo di fatto “sotto il radar”. Questi attacchi prolungati e a bassa frequenza possono passare inosservati ai sistemi IDS che rilevano azioni rapide ed eccessivamente sospette.

Splicing di sessione: simile alla frammentazione, lo splicing di sessione comporta la distribuzione di payload dannosi in più sessioni o pacchetti TCP. L’intento è quello di introdurre il payload in modo lento e poco appariscente, evitando così i trigger di rilevamento.

Per combattere queste tattiche di evasione, le organizzazioni e le aziende devono aggiornare e configurare regolarmente i loro IDS. Inoltre, gli intrusion detection system dovrebbero essere integrati con altri strumenti di sicurezza, in quanto la combinazione di più livelli di sicurezza e il mantenimento della vigilanza possono contribuire a mitigare il rischio di tali tecniche di evasione.


%https://intrusa.io/magazine/intrusion-detection-system-ids/
\section{Rilevare le intrusioni: le diverse tipologie di detection}

L’efficacia degli Intrusion Detection System si basa naturalmente sulla loro capacità di rilevare le minacce. Questi sistemi possono in realtà essere distinti in due categorie principali, a seconda della posizione dei sensori dedicati al rilevamento delle intrusioni. Gli IDS possono infatti essere “posizionati” sugli endpoint oppure sulla rete. Una differenza, questa, non certo secondaria, in quanto determina che cosa si intende monitorare di specifico all’interno del sistema informatico aziendale.

NIDS o Network intrusion Detection System: i sistemi di rilevamento delle intrusioni di rete, noti come NIDS, analizzano i pacchetti IP e monitorano il traffico in entrate e in uscita su tutta la rete. Posizionati strategicamente, spesso direttamente dietro i firewall, i NIDS possono rilevare il traffico dannoso che è riuscito a penetrare nella rete, attività anomale come accessi non autorizzati, propagazioni di software malevoli e l’acquisizione abusiva di privilegi utente. Fondamentali per fornire informazioni sulle scansioni di rete, evidenziare errori di configurazione e vulnerabilità, i NIDS monitorano anche il comportamento degli utenti interni e segnalano eventuali malfunzionamenti di altre misure di sicurezza.

HIDS o Host based Intrusion Detection System: i sistemi di rilevamento delle intrusioni basati sugli host sono installati su specifici endpoint (laptop, router, server, etc.). Il loro compito è quello di monitorare l’attività del singolo dispositivo, incluso il traffico in entrata e in uscita, eseguendo snapshot periodici dei file critici del sistema operativo e confrontandoli nel tempo. Alla base di questi sistemi si trova la certezza che chi entra illecitamente nel sistema lascia necessariamente una traccia del proprio passaggio. Ne sono degli esempi l’installazione di programmi deputati alla raccolta di informazioni e alla gestione remota della macchina. Una volta rilevati i cambiamenti, dalla modifica dei file di log alle alterazioni delle configurazioni, gli HIDS avvisano il team di sicurezza.


%https://www.paloaltonetworks.com/cyberpedia/what-is-an-intrusion-detection-system-ids
The following table summarizes the differences between the IPS and the IDS deployment.

\begin{table}
    \begin{tabular}{c|c|c}
   & Intrusion Prevention System &	IDS Deployment \\
   Placement in Network Infrastructure &		Part of the direct line of communication (inline) &		Outside direct line of communication (out-of-band) \\
System Type	 &	Active (monitor \& automatically defend) and/or passive	 &	Passive (monitor \& notify) \\
Detection Mechanisms	 &	1. Statistical anomaly-based detection 2. Signature detection:- Exploit-facing signatures- Vulnerability-facing signatures	 &	1. Signature detection:- Exploit-facing signatures \\
    \end{tabular}
\end{table}

The two most common types of IDS are:

\textbf{Network-based intrusion detection system (NIDS)}
A network IDS monitors a complete protected network. It is deployed across the infrastructure at strategic points, such as the most vulnerable subnets. The NIDS monitors all traffic flowing to and from devices on the network, making determinations based on packet contents and metadata.
\textbf{Host-based intrusion detection system (HIDS)}
A host-based IDS monitors the computer infrastructure on which it is installed. In other words, it is deployed on a specific endpoint to protect it against internal and external threats. The IDS accomplishes this by analyzing traffic, logging malicious activity and notifying designated authorities.


The remaining three types can be described as such:
\textbf{Protocol-based (PIDS)}
A protocol-based intrusion detection system is usually installed on a web server. It monitors and analyzes the protocol between a user/device and the server. A PIDS normally sits at the front end of a server and monitors the behavior and state of the protocol.
\textbf{Application protocol-based (APIDS)}
An APIDS is a system or agent that usually sits inside the server party. It tracks and interprets correspondence on application-specific protocols. For example, this would monitor the SQL protocol to the middleware while transacting with the web server.
\textbf{Hybrid intrusion detection system}
A hybrid intrusion detection system combines two or more intrusion detection approaches. Using this system, system or host agent data combined with network information for a comprehensive view of the system. The hybrid intrusion detection system is more powerful compared to other systems. One example of Hybrid IDS is Prelude.

There is also a subgroup of IDS detection methods, the two most common variants being:
\textbf{Signature-based}
A signature-based IDS monitors inbound network traffic, looking for specific patterns and sequences that match known attack signatures. While it is effective for this purpose, it is incapable of detecting unidentified attacks with no known patterns.
\textbf{Anomaly-based}
The anomaly-based IDS is a relatively newer technology designed to detect unknown attacks, going beyond the identification of attack signatures. This type of detection instead uses machine learning to analyze large amounts of network data and traffic.

Anomaly-based IDS creates a defined model of normal activity and uses it to identify anomalous behavior. However, it is prone to false positives. For example, if a machine demonstrates rare, but healthy behavior, it is identified as an anomaly. This results in a false alarm.

\subsection{IDS Evasion Techniques}

There are numerous techniques intruders may use to avoid detection by IDS. These methods can create challenges for IDSes, as they are meant to circumvent existing detection methods:

\textbf{Fragmentation}
Fragmentation divides a packet into smaller, fragmented packets. This allows an intruder to remain hidden, as there will be no attack signature to detect.

Fragmented packets are later reconstructed by the recipient node at the IP layer. They are then forwarded to the application layer. Fragmentation attacks generate malicious packets by replacing data in constituent fragmented packets with new data.
\textbf{Flooding}
This attack is designed to overwhelm the detector, triggering a failure of control mechanism. When a detector fails, all traffic will then be allowed.

A popular way to cause flooding is by spoofing the legitimate User Datagram Protocol (UDP) and Internet Control Message Protocol (ICMP). The traffic flooding is then used to camouflage the anomalous activities of the perpetrator. As a result, the IDS would have great difficulty finding malicious packets within an overwhelming volume of traffic.
\textbf{Obfuscation}
Obfuscation can be used to avoid being detected by making a message difficult to understand, thereby hiding an attack. The terminology of obfuscation means altering program code in such a way which keeps it functionally indistinguishable.

The objective is to reduce detectability to reverse engineering or static analysis process by obscuring it and compromising readability. Obfuscating malware, for instance, allows it to evade IDSes.
\textbf{Encryption}
Encryption offers multiple security capabilities including data confidentiality, integrity and privacy. Unfortunately, malware creators use security attributes to conceal attacks and evade detection.

For instance, an attack on an encrypted protocol cannot be read by an IDS. When the IDS cannot match encrypted traffic to existing database signatures, the encrypted traffic is not encrypted. This makes it very difficult for detectors to identify attacks.













































