Il rilevamento e la mitigazione dei covert Channel richiede quindi un rigoroso monitoraggio  delle anomalie, l'isolamento delle risorse e tecniche per introdurre rumore. 
\vspace{1ex} \newline 
La difesa richiede una combinazione di rinforzo delle politiche, gestione delle risorse e tecniche di monitoraggio. 
Mentre la mitigazione richiede una sicurezza multi livello fra hardware, OS, applicazioni e reti. 

\begin{center} 
\begin{longtable}{|p{0.25\textwidth}|p{0.55\textwidth}|} 
    \hline
    \textbf{Tipologie di difesa} & \textbf{Sottotipologie} \\
    \hline 
    Difese basate sul Sistema e sule Politiche(Policy) & 
    Politiche di controllo degli accessi. \newline 
    Controllo del flusso di informazioni. \newline
    Separazione e isolamento dei processi.
    \\
    \hline 
    Protezioni basate sulla gestione delle risorse e dei tempi & 
    Tecniche di Randomizzazione. \newline 
    Limitazione della velocità e controllo della larghezza di banda.
    \\
    \hline 
    Protezioni basate sulla sicurezza della rete & 
    Ispezione e filtraggio dei pacchetti. \newline 
    Analisi del traffico e rilevamento delle anomalie.
    \\
    \hline 
    Verifica e test dei Covert Channel & 
    
    \\
    \hline  
\caption{Tipologie di difese} 
\label{tabella:difese:tipologie} 
\end{longtable} 
\end{center}  

\subsection{Difese basate sul Sistema e sule Politiche(Policy)} 

\subsubsection*{Politiche di controllo degli accessi}  
Applicare un forte controllo degli accessi (MAC, RBAC) per evitare interazioni non autorizzate con i processi. 
Limitare i permessi implementando il minimo privilegio e il controllo obbligatorio dell'accesso (MAC) per limitare e/o prevenire la comunicazione non autorizzata tra i processi (e quindi lo scambio di informazioni non autorizzato). 
Utilizzare sandbox e compartimentazione per isolare i processi.

\subsubsection*{Controllo del flusso di informazioni}  
Utilizzare obbligatoriamente modelli di controllo del flusso di dati (Bell-LaPadula, Biba) 
per evitare fughe di informazioni e impedire così che i processi ad alta 
sicurezza perdano dati ai processi a bassa sicurezza.

\subsubsection*{Separazione e isolamento dei processi}  
Disattivare le risorse condivise non necessarie (ad esempio, comunicazione tra processi, memoria condivisa).
Utilizzare la virtualizzazione e la containerizzazione per separare i processi. 
Applicare l'air-gapping per i sistemi altamente sensibili.

\subsection*{Protezioni basate sulla gestione delle risorse e dei tempi} 

\subsubsection*{Tecniche di Randomizzazione} 
Introdurre rumore (Noise Injection) nelle risposte del sistema (ad esempio, randomizzando i tempi di 
esecuzione, aggiungendo ritardi) per interrompere i Covert Channel basati sul tempo. 
Utilizzare tecniche di randomizzazione o svuotamento della cache per 
prevenire attacchi side-channel basati sulla cache.
    

\subsubsection*{Limitazione della velocità e controllo della larghezza di banda} 
Limitare la CPU, la memoria o la larghezza di banda della rete per limitare la capacità di un canale nascosto. 
Implementare meccanismi di throttling (limitazione) per le risorse condivise. 
E analizzare i comportamenti del sistema per rilevare anomalie. 

\subsection*{Protezioni basate sulla sicurezza della rete} 

\subsubsection*{Ispezione e filtraggio dei pacchetti} 
Monitoraggio del Traffico utilizzando la Deep Packet Inspection (DPI) per rilevare schemi anomali nel traffico di rete.
Bloccare o sanificare i campi inutilizzati dei protocolli (ad esempio, le intestazioni TCP/IP). 

\subsubsection*{Analisi del traffico e rilevamento delle anomalie} 
Applicare la segmentazione della rete per limitare i flussi di dati non autorizzati.
Utilizza il monitoraggio basato sull'intelligenza artificiale per rilevare modelli di comunicazione insoliti.
Utilizza sistemi di rilevamento delle intrusioni (IDS) e analisi dei log per identificare attività sospette.

\subsection*{Verifica e test dei Covert Channel} 
Eseguire regolarmente analisi dei canali nascosti nei test di penetrazione.
Utilizzare strumenti di rilevamento dei Covert Channel (ad esempio, analisi del flusso di rete, monitoraggio del comportamento del sistema). 


\subsection{Mitigazioni}
Per prevenire gli attacchi basati su ICMP; buone misure di sicurezza sono: 
\begin{itemize}
    \item limitare e filtrare l'utilizzo di ICMP tramite i firewall 
    \item la limitazione della velocità 
    \item il monitoraggio del traffico (tramite strumenti di sicurezza) per rilevare le anomalie
\end{itemize} 

\begin{center} 
\begin{longtable}{|p{0.3\textwidth}|p{0.5\textwidth}|} 
    \hline
    \textbf{Tipologie di mitigazioni} & \textbf{Descrizione} \\
    \hline 
    Regole del firewall & 
    Limitare o bloccare il traffico ICMP non necessario.
    \\
    \hline 
    Limitazione della velocità & 
    Limitare la velocità delle richieste ICMP.
    \\
    \hline 
    Monitoraggio della rete e Rilevamento & 
    Utilizzare sistemi di rilevamento delle intrusioni (IDS) per monitorare attività ICMP sospette. 
    \\
    \hline 
    %Rafforzamento del sistema & 
    %
    %\\
    %\hline  
\caption{Tipologie di mitigazioni} 
\label{tabella:mitigazioni:tipologie} 
\end{longtable} 
\end{center}  

\begin{minipage}{\linewidth}
    \begin{tabular}{|p{0.45\linewidth}|p{0.45\linewidth}|}
        \hline
        \textbf{Metodo di mitigazione} & \textbf{Effetti} \\
        \hline \hline 
        Disattiva ICMP se non necessario & Impedisce la maggior parte degli attacchi basati su ICMP \\
        \hline 
        Limita ICMP ai tipi necessari & blocca i vettori di attacco non necessari \\
        \hline 
        Limitazione della velocità & Impedisce il flooding e il tunneling ICMP \\ %Rileva richieste ICMP eccessive &
        \hline 
        Regole del firewall & Blocca l'ICMP in uscita dai sistemi critici \\ % Contrassegna le richieste ICMPS non autorizzate &
        \hline 
        Blocca ICMP in uscita dai firewall & Impedisce perdite di dati tramite ICMP \\
        \hline 
        Endpoint Security (EDR) & Previene l'esecuzione dannosa di ICMP \\ %Rileva malware tramite Covert Channel ICMP & 
        \hline 
    \end{tabular} 
    \captionof{table}{Strategie di mitigazione}
\end{minipage} 

\subsection*{Regole del firewall} 
Limitare o bloccare il traffico ICMP non necessario. %sui firewall. 
\begin{itemize}
    \item Bloccare le richieste Echo di ICMP da reti esterne, a meno che non siano necessarie.
    \item Disabilitare le risposte a ICMP Timestamp e Address Mask per impedire la ricognizione.
    \item Consentire solo i messaggi di errore ICMP necessari (ad esempio, Destinazione non raggiungibile).
    \item Eliminare i messaggi di reindirizzamento ICMP per impedire la manipolazione dell'instradamento (del routing).
\end{itemize} 
Disattivare ICMP sui server e sugli endpoint a meno che non sia esplicitamente necessario e 
bloccare il traffico ICMP proveniente da fonti non attendibili. %sul firewall.
Configurare firewall e router in modo tale da consentire solo i messaggi ICMP necessari 
(e.g Destinazione non raggiungibile, Tempo Scaduto).
Disattivare le richieste/risposte di eco ICMP sui sistemi critici. 
%
Impedisci il traffico ICMP in uscita dalle reti interne per fermare l'esfiltrazione.
Consenti ICMP solo per scopi diagnostici interni.

\subsection*{Limitazione della velocità} 
Limitare la velocità delle richieste ICMP. %per evitare di essere sopraffatti
\begin{itemize}
    \item Limita il numero di pacchetti ICMP al secondo per prevenire la sovrastazione. 
    \item Configura i criteri di limitazione della velocità ICMP su router e firewall.
\end{itemize} 
Limitare la frequenza e la dimensione dei pacchetti ICMP per evitare il trasferimento nascosto di dati.
Configurare i firewall in modo da consentire solo un numero specifico di pacchetti ICMP al secondo.

\subsection*{Monitoraggio della rete e Rilevamento} 
%Utilizzare sistemi di rilevamento delle intrusioni (IDS) per monitorare attività ICMP sospette. 
\begin{itemize}
    \item Utilizzare i sistemi di rilevamento delle intrusioni (IDS/IPS) per rilevare abusi del protocollo ICMP.
    \item Analizza i registri di rete per attività ICMP insolite (ad esempio, pacchetti ICMP di grandi dimensioni, ping frequenti).
    \item Implementa l'ispezione approfondita dei pacchetti (DPI) per identificare il Tunneling ICMP.
\end{itemize} 
Analizzare il volume e le dimensioni dei pacchetti ICMP (ad esempio, payload insolitamente grandi) per eventuali anomalie.
Rileva il traffico ICMP ad alta frequenza verso host esterni sconosciuti.
Verificare la presenza di pacchetti ICMP con payload insolitamente grandi (e.g tentativi di esfiltrazione dei dati) o con 
schemi irregolari (e.g valori TTL variabili). 
Pacchetti ICMP con modifiche costanti del payload potrebbero indicare il trasferimento di dati nascosti. 

\subsubsection*{Deep Packet Inspection (DPI)} 
Esaminare il contenuto del payload ICMP per rilevare eventuali dati incorporati insoliti (messaggi codificati, crittografia o anomalie).
Contrassegna i pacchetti ICMP che contengono risposte non standard (e.g, una risposta Echo contenente dati inaspettati).
Identificare schemi di comunicazione con indirizzi IP esterni tramite ICMP. 

\subsubsection*{Sistemi di rilevamento e prevenzione delle intrusioni (IDS/IPS)} 
Utilizzare Snort, Suricata o Zeek per rilevare e segnalare attività ICMP sospette

\subsubsection*{Rilevamento basato su anomalie}
Rilevare il traffico ICMP che potrebbe indicare una comunicazione C2 implementando analisi 
comportamentali che possano rilevare un utilizzo anomalo di ICMP.
Utilizzare strumenti di apprendimento automatico o SIEM (Security Information and Event 
Management) per segnalare deviazioni nell'utilizzo di ICMP. 

\begin{minipage}{\linewidth}
    \begin{tabular}{|p{0.3\linewidth}|p{0.3\linewidth}|p{0.3\linewidth}|}
        \hline 
        \textbf{Tecnica} & \textbf{Rilevamento} & \textbf{Mitigazione} \\
        \hline \hline 
        Analisi del traffico di rete & Identifica anomalie nel volume e nei pattern ICMP & Limita i tipi ICMP non necessari \\
        \hline 
        Deep Packet Inspection (DPI) & Rileva l'esfiltrazione e il tunneling dei dati & Blocca i pacchetti ICMP con payload inattesi \\
        \hline 
        IDS/IPS (Snort, Zeek) & Segnala comportamenti ICMP insoliti & Blocca le richieste ICMP sospette \\
        %Utilizza la Deep Packet Inspection (DPI) & identifica i dati nascosti nei pacchetti ICMP. \\
        %Implementa regole IDS/IPS per ICMP & avvisi su attività ICMP sospette \\ 
        \hline 
    \end{tabular} 
    \captionof{table}{Strategie di rilevamento}
\end{minipage} 

\subsection*{Rafforzamento del sistema} 
\begin{itemize}
    \item Mantieni aggiornati i sistemi e il firmware per correggere le vulnerabilità ICMP note
    \item Disattivare i servizi ICMP sui sistemi critici se non necessari.
    \item Utilizzare soluzioni di sicurezza degli endpoint per rilevare malware che utilizzano ICMP per la comunicazione
\end{itemize} 

\subsubsection{Utilizza la crittografia per prevenire la fuga di dati}
Implementa la crittografia TLS/SSL per tutte le comunicazioni legittime così da impedire agli 
aggressori di utilizzare ICMP per l'esfiltrazione. 
Inoltre bloccare le trasmissioni non autorizzate di testo in chiaro su ICMP. 

\subsubsection{Sicurezza degli endpoint \& Antivirus}
Implementare strumenti antivirus e soluzioni EDR (Endpoint Detection \& Response) per 
rilevare le minacce informatiche che utilizzano i covert channel ICMP per comunicare.








