%NOT USED
%
%https://www.geeksforgeeks.org/computer-science-fundamentals/what-is-an-ip-address/ 
%https://www.geeksforgeeks.org/computer-networks/introduction-to-subnetting/ 
%https://www.geeksforgeeks.org/computer-organization-architecture/introduction-of-ports-in-computers/ 
%
%
%https://www.geeksforgeeks.org/linux-unix/linux-firewall/
\subsection{Firewall in Linux} 
What is Linux Firewall (firewalld)
The security system in Linux OS is known as Linux Firewall, which monitors and governs the network traffic 
(outbound/inbound connections). It can be used to block access to different IP addresses, Specific subnets, 
ports (virtual points where network connections begin and end), and services.
%
We have a daemon's name called Firewalld which is used to maintain the firewall policies. 
A dynamically managed firewall tool in a Linux system is known as Firewalld, it can be updated in real-time 
if there are any changes in the network environment.
%
This Firewalld works in concepts of zones (segments). We can check whether our firewall services are 
running or not by using the commands sudo (user access) and systemctl (use to control and manage the 
status of services).
\begin{lstlisting}[language=bash, caption=Check the status of firewall service] 
    sudo systemctl status firewalld
\end{lstlisting}
To protect our system from unauthorized access and to control network traffic (incoming and outgoing). 
We can do customization in ports, addresses, protocols, etc.
%
Rule 1: Allowing SSH (Secure Shell or Secure Socket Shell) traffic
By using this we can allow all incoming traffic on the SHH port so that we can connect to the system remotely.
\begin{lstlisting}[language=bash, caption=Allowing SSH traffic] 
    sudo firewall-cmd --zone=public --add-services=ssh --permanent 
    sudo firewall-cmd --reload
\end{lstlisting}
%
Rule 2: Allowing incoming traffic on a specific port
We are allowing traffic on a specific TCP port 8080 you can replace it with requirements.
\begin{lstlisting}[language=bash, caption=Allowing SSH traffic] 
    sudo firewall-cmd --zone=public --add-port=8080/tcp --permanent
    sudo firewall-cmd --reload
\end{lstlisting}
%
Rule 3: Blocking incoming traffic on a specific IP address
We are blocking incoming traffic on IP 192.168.52.1 you can replace it with your requirements.
\begin{lstlisting}[language=bash, caption=Allowing SSH traffic] 
sudo firewall-cmd --zone=public --add-rich='rule family="ipv4" source address="192.168.52.1" reject'
sudo firewall-cmd --reload
\end{lstlisting}


\subsubsection{Types of Linux Firewalls}
There is more than one Linux firewall option available. When we come to drop down and research, 
we have a few popular names IPCop, iptables, Shorewall, and UFW But one of the most popular is 
the "iptables" firewall.
%
\subsection{Iptables Working}
Linux-based software that performs manipulation functions, packet filtering, and NAT 
(network address translation) is known as Iptables. With the help of Iptables which 
allows system administrators to control incoming and outgoing traffic by setting up the rules. 
%
When a packet is received in a Linux base system, it has to go through the chains and tables in 
the iptables firewall. The most commonly used tables are filter and nat but we have five predefined 
tables in iptables (raw, nat, filter, security, and mangle).

\subsubsection{Types of Tables} 
We will discuss five predefined tables:
\begin{itemize}
    \item Security Table: It is often used in conjunction with other security tools like SELinux, 
    it is also used for MAC (Mandatory Access Control) rules, which can further be used to set rules 
    related to security labels and access controls. 
    \vspace{2ex} \newline
    It has four built-in chains: OUTPUT, FORWARD, INPUT, and SECMARK. 
    \item Mangle Table: It is used to modify packets by setting the packet's ToS/DSCP field, altering packet header 
    fields, and changing packet marks. 
    \vspace{2ex} \newline
    It has Five built-in chains: POSTROUTING, FORWARD, OUTPUT, PREROUTING, and INPUT. 
    \item Nat Table: It stands for network address translation, which helps in sharing a single 
    public IP address between multiple devices. 
    \vspace{2ex} \newline
    It has two built-in chains: PREROUTING and POSTROUTING. 
    \item Raw Table: It is used for the configuration of low-level packet processing. 
    \vspace{2ex} \newline
    It has limited built-in chains, but the user can create additional chains if required. 
    \item Filter Table: It is used for packet filtering. 
    \vspace{2ex} \newline
    It has three built-in chains. INPUT, OUTPUT, and FORWARD.
\end{itemize}
Here filters are responsible for filtering the packets on the defined rules based on the source and destination 
of the IP address, port number, and protocol type. And Chains there are three different types of built-in chains.

\subsubsection{Types of Chains}
Chain Rule: Rules that are described for a particular task. Subdivided into three types: 
\begin{itemize}
    \item INPUT: Filter incoming traffic in the local system.
    \item OUTPUT: Filter Outgoing traffic for the local system.
    \item FORWARD: Packets forwarded from one system to another go throw it.
\end{itemize}

\subsection{Configure a Firewall on Linux OS}
We will be configuring iptables in our operating system.
\begin{lstlisting}[language=bash, caption=Allowing SSH traffic] 
To install iptables
sudo dnf install iptables 

Basic Syntax for using iptables
sudo iptables [option] CHAIN-rule [-j target]
\end{lstlisting}
\begin{itemize}
    \item Output Chains: Traffic going through local machines has to pass through these output chains. 
    \item Input Chains: Traffic has to go from every rule that has been assigned within input chains.
    \item Forward Chains: Traffic going from the arising network location to another network location 
    has to pass through forward chains. 
\end{itemize}
\begin{center} 
\begin{longtable}{|p{0.3\textwidth}|p{0.5\textwidth}|} 
    \hline 
    \textbf{Options} & \textbf{Descriptions} \\
    \hline 
    -C	& [CHECK]: This is to check and find a rule that matches the requirements of the string. \\
    \hline 
    -D	& [DELETE]: This is used to delete a specific rule. \\
    \hline 
    -A	& [APPEND]: This is used to append or add rules. \\
    \hline 
    -I	& [INSERT]: This can add a rule to a particular position in a string. \\
    \hline 
    -L	& [LIST]: To display all the rules we can use this. \\
    \hline 
    -v	& [VERBOSE]: This is used to get more information in the list option. \\
    \hline 
    -X	& [DELETE CHAIN]: This deletes the entire supplied string. \\
    \hline 
    -p	& [Protocol\_name]: It is used to define the name of the protocol. \\
    \hline 
    -N	& [NEW CHAIN]: To create a new chain. \\
    \hline 
    -j	& [job]: It tells what operation has to be done with the packet. \\
    \hline 
    -F	& [Flush]: It is to delete all rules. \\
    \hline 
    -s	& [specify]: It is a flag used to specify the source of the packet.  \\
    \hline 
\caption{Tipologie di mitigazioni} 
\label{tabella:mitigazioni:tipologie} 
\end{longtable} 
\end{center}  
We have three basic Policies. Let's discuss Some Basic Operations and their Syntax
\begin{itemize}
    \item DROP: It can block an incoming signal, which basically states that the firewall is blocked for that particular IP.
    \item ACCEPT: It allows the IP we provide to make users go into the system.
    \item REJECT: It works similarly to Drop, but in 'drop' the sender is blocked without any notification whereas in 'reject' a message states the reason for not being able to connect.
\end{itemize}
%
Creating our first rule
The first rule to allow incoming ICMP (ping) traffic on the INPUT chain:
\begin{lstlisting}[language=bash, caption=Allowing SSH traffic] 
sudo iptables -A INPUT -p icmp -j ACCEPT
\end{lstlisting}
Uses '-A' to append the rule at the end of the INPUT chain. '-p icmp' tells that rule is applying to ICMP 
traffic. '-j ACCEPT' tells you to accept(allow) any traffic that matches the rule.
%
The syntax for using policies: 
\# Refer context mentioned above to see the use-case of [ -I , -A , -p , -s ,-j ] 
sudo iptables -I/-A name\_chain -s source\_ip -p protocol\_name --dport port\_number -j action\_to\_do 
%
Example:
Accept Rule: If we have to accept an IP (source) 192.168.160.51 on port number 22 using TCP protocol. 
\begin{lstlisting}[language=bash, caption=Allowing SSH traffic] 
sudo iptables -A INPUT -s 192.168.160.51 -p tcp --dport 22 -j ACCEPT
\end{lstlisting}
%
Drop Rule: If we have to Drop an IP (source) 192.168.160.51.
\begin{lstlisting}[language=bash, caption=Allowing SSH traffic] 
sudo iptables -A/-I chain_name -s source_ip -j action_to_do
\end{lstlisting}
%
Reset Rule: To reset all iptables rule we use -F.
\begin{lstlisting}[language=bash, caption=Allowing SSH traffic] 
sudo iptables -F
\end{lstlisting}

%https://www.geeksforgeeks.org/linux-unix/iptables-command-in-linux-with-examples/ 
%https://www.geeksforgeeks.org/linux-unix/ipv6-iptables-rules/
%
%NOT USED
%https://en.wikipedia.org/wiki/Mandatory_access_control
\subsection{iptables command in Linux with Examples}
The iptables command in Linux is a powerful tool that is used for managing the firewall rules and network traffic. 
It's essential for securing servers and networks by selectively permitting or denying specific types of traffic 
based on defined rules and conditions. This flexibility makes iptables a fundamental component in 
Linux networking and security configurations. 

\subsubsection{What are iptables?}
It is a command line interface used to set up and maintain tables for the Netfilter firewall for IPv4, 
included in the Linux kernel. The firewall matches packets with rules defined in these tables and then 
takes the specified action on a possible match. 
\begin{itemize}
    \item Tables is the name for a set of chains.
    \item Chain is a collection of rules.
    \item Rule is condition used to match packet.
    \item Target is action taken when a possible rule matches. Examples of the target are ACCEPT, DROP, QUEUE.
    \item Policy is the default action taken in case of no match with the inbuilt chains and can be ACCEPT or DROP.
\end{itemize}
\begin{lstlisting}
    iptables --table TABLE -A/-C/-D... CHAIN rule --jump Target
\end{lstlisting}
There are five possible tables as follows: 
\begin{itemize}
    \item filter: Default used table for packet filtering. It includes chains like INPUT, OUTPUT and FORWARD.
    \item nat : Related to Network Address Translation. It includes PREROUTING and POSTROUTING chains.
    \item mangle : For specialised packet alteration. Inbuilt chains include PREROUTING and OUTPUT.
    \item raw : Configures exemptions from connection tracking. Built-in chains are PREROUTING and OUTPUT.
    \item security : Used for Mandatory Access Control
\end{itemize}
There are few built-in chains that are included in tables. They are: 
\begin{itemize}
    \item INPUT : A set of rules for packets destined to localhost sockets.
    \item FORWARD :for packets routed through the device.
    \item OUTPUT : It is locally generated packets, meant to be transmitted outside.
    \item PREROUTING : It is used for modifying packets as they arrive.
    \item POSTROUTING : IIt helps in modifying packets as they are leaving.
\end{itemize}

\subsubsection{User-defined Chains}
User-defined chains can also be created. The following are the some of the possible one with examples:
%
1. -A, --append : Append to the chain provided in parameters.
\begin{lstlisting}
    iptables [-t table] --append [chain] [parameters]
\end{lstlisting}
\begin{lstlisting}
    iptables -t filter --append INPUT -j DROP
\end{lstlisting}
%
2. -D, --delete : Delete rule from the specified chain.
\begin{lstlisting}
    iptables [-t table] --delete [chain] [rule_number]
\end{lstlisting}
\begin{lstlisting}
    iptables -t filter --delete INPUT 2
\end{lstlisting}
%
-C, --check :Check if a rule is present in the chain or not. It returns 0 if the rule exists and
returns 1 if it does not. Syntax:
\begin{lstlisting}
    iptables [-t table] --check [chain] [parameters]
\end{lstlisting}
\begin{lstlisting}
    iptables -t filter --check INPUT -s 192.168.1.123 -j DROP
\end{lstlisting}

\subsubsection{Examples of Iptables Commands with Common Parameters}
The iptables command uses parameters to match packets and define actions. Key parameters include -p or --proto, 
which specify the protocol of the packet, such as tcp, udp, icmp, ssh, etc. This parameter allows administrators 
to selectively filter or handle packets based on their communication protocol. The common parameters are: 
%
1. -p, --proto : is the protocol that the packet follows. Possible values maybe: tcp, udp, icmp, ssh etc. 
\begin{lstlisting}
    iptables [-t table] -A [chain] -p {protocol_name} [target]
\end{lstlisting}
\begin{lstlisting}
    iptables -t filter -A INPUT -p udp -j DROP
\end{lstlisting}
%
2. -s, --source: is used to match with the source address of the packet.
\begin{lstlisting}
    iptables [-t table] -A [chain] -s {source_address} [target]
\end{lstlisting}
\begin{lstlisting}
    iptables -t filter -A INPUT -s 192.168.1.230 -j ACCEPT
\end{lstlisting}
%
3. -d, --destination : is used to match with the destination address of the packet.
\begin{lstlisting}
    iptables [-t table] -A [chain] -d {destination_address} [target]
\end{lstlisting}
\begin{lstlisting}
    iptables -t filter -A OUTPUT -d 192.168.1.123 -j DROP
\end{lstlisting}
%
4. -i, --in-interface : matches packets with the specified in-interface and takes the action. 
\newline
5. -o, --out-interface : matches packets with the specified out-interface.
\begin{lstlisting}
    iptables [-t table] -A [chain] -i {interface} [target]
\end{lstlisting}
\begin{lstlisting}
    iptables -t filter -A INPUT -i wlan0 -j DROP
\end{lstlisting}
%
6. -j, --jump : this parameter specifies the action to be taken on a match.
\begin{lstlisting}
    iptables [-t table] -A [chain] [parameters] -j {target}
\end{lstlisting}
\begin{lstlisting}
    iptables -t filter -A FORWARD -j DROP
\end{lstlisting}

\subsubsection{Filters of Iptables}
1. While trying out the commands, you can remove all filtering rules and user created chains. 
\begin{lstlisting}
sudo iptables --flush
\end{lstlisting}
%
2. To save the iptables configuration use:
\begin{lstlisting}
sudo iptables-save
\end{lstlisting}
%
3. Restoring iptables config can be done with:
\begin{lstlisting}
sudo iptables-restore
\end{lstlisting}
%
There are other interfaces such ipv6 tables which are used to manage filtering tables for IPv6. 
To block IPv6 traffic on a Linux system, you can use ip6tables, the IPv6 counterpart of iptables 

\subsection{Why use Iptables in Linux?}
The following are the some of the reasons to use Iptables in Linux: 
\begin{itemize}
    \item Firewall Configuration: It helps in enabling the precise control over the netowrk traffic to protect against unauthorized access and attacks.
    \item Packet Filtering: It allows in filtering based on the criteria like protocol, IP addresses and prots providing the security.
    \item Network Address translation (NAT): It facilitates with seamless communication between different network segments.
    \item Logging and Monitoring: It provides the insights into the network activity for providing feature sof security auditing and troubleshooting. 
\end{itemize}
The following are the benefits of using iptable command: 
\begin{itemize}
    \item Robust Firewall Capabilities: It facilitates with configuration of firewall rules to control incoming and outgoing traffic, enhancing network security.
    \item Precise Packet Filtering: It provides the filtering based on criteria such as protocol, source/destination IP addresses, and ports, ensuring only authorized traffic passes through.
    \item Network Address Translation (NAT): It supports NAT functionality for translating IP addresses and ports, essential for network connectivity and management.
    \item Logging and Monitoring: It provides logging capabilities to monitor and analyze network traffic, aiding in security auditing and troubleshooting.
\end{itemize}
The following are the some of the features of Iptables: 
\begin{itemize}
    \item Packet Filtering: Iptables facilitates with providing filtering features for network packets based on various criteria such as source and destination IP addresses and ports.
    \item NAT: Iptables supports the NAT by allowing for the translation of the private IP address to public address making an essential for devices within a private network to establish the communication with external networks.
    \item Stateful Inspection: Through stateful inspection, iptables helps in tracking the state of network connections with providing the enhanced security by legitimating the traffic that is only allowed.
\end{itemize}
The following are the some of the usecases of Iptables: 
\begin{itemize}
    \item Firewall Protection: Iptables can be configured through blocking the unauthorized access and allow legitimate traffic. It facilitates with providing a robust firewall to secure a network or individual system.
    \item Traffic Shaping and Control: By setting rules, iptables we can manage and prioritize network traffic, ensuring critical services maintain performance and reducing congestion during peak usage times.
    \item Network Address Translation (NAT): iptables facilitates NAT, allowing multiple devices on a private network to access external networks using a single public IP address, essential for home and business networks.
    \item Port Forwarding: With iptables, administrators can redirect traffic from one port to another, enabling access to services running on different ports or internal servers from external networks.
\end{itemize}

%https://www.geeksforgeeks.org/linux-unix/how-to-setup-firewall-in-linux/
%
%
\subsection{How to Configure your Linux Firewall - 3 Methods}
Linux offers multiple firewall management tools, including iptables and firewalld, both of which can be 
used to manage and secure your network.
Setting up a firewall in Linux is an essential step to protect your system from unauthorized access 
and potential threats. Firewalls act as a barrier between your internal network and external connections, 
filtering traffic based on predefined rules. These firewalls help your system or server by keeping it safe and secure.

\subsubsection{What is a Firewall?}
A firewall is a security system designed to monitor and control incoming and outgoing network traffic. 
Acting as a barrier between trusted internal networks and untrusted external connections, firewalls enforce predefined security policies. Firewalls can be implemented in both hardware and software, with the main goal being to:
\begin{itemize}
    \item Restrict unauthorized access.
    \item Allow legitimate communication.
    \item Prevent data breaches.
\end{itemize}
\textbf{iptables} \newline 
iptables is the most widely used firewall tool in Linux. It works by defining chains of rules that filter 
network traffic at various points (such as incoming or outgoing traffic). iptables operates at the network 
layer (Layer 3) and transport layer (Layer 4).
\newline
\textbf{firewall} \newline 
firewalld is a more modern firewall management tool for Linux, available on distributions like CentOS, RHEL, 
and Fedora. It provides a more dynamic and user-friendly approach to managing firewall rules using zones.
\newline 
\textbf{nftables} \newline 
nftables is the successor to iptables and provides an improved, more efficient way to filter network traffic. 
It is designed to replace iptables in newer Linux distributions.


\subsection{How Does a Linux Firewall Work?} 
A Linux firewall filters network traffic based on a series of rules. These rules specify which types of 
network packets (data sent over the network) are allowed or denied based on factors like:
\begin{itemize}
    \item IP Address: The source or destination address of the packet.
    \item Port Number: The communication port the packet is trying to reach (e.g., port 80 for HTTP or port 22 for SSH).
    \item Protocol: The type of network protocol used (TCP, UDP, ICMP, etc.).
    \item Connection State: Whether the packet is part of an established connection or is a new connection request.
\end{itemize}
When a packet enters or exits the system, the firewall checks it against its rules to determine whether it 
should be allowed to pass or blocked. If a packet matches an "allow" rule, it is allowed to pass. 
If it matches a "deny" rule, it is blocked.

\subsection{Understanding Firewall Tools}
iptables: A powerful command-line tool that filters network traffic. It works by defining chains of rules for different types of network traffic.
\newline
ufw (Uncomplicated Firewall): A user-friendly frontend for iptables, simplifying configuration.
\newline
firewalld: A dynamic firewall management tool, offering flexible configuration. It uses zones to define trust levels for network connections and interfaces, providing a simpler method to manage firewall settings compared to iptables.
\newline
CSF (ConfigServer Security \& Firewall): A complete security solution, including firewall features.
\newline
ClearOS and OPNsense: Firewall-focused operating systems providing web-based interfaces.

\subsection{Configuring Firewall with iptables}
iptables is a powerful tool for configuring packet filtering and NAT rules. It's ideal for experienced Linux users and system administrators managing complex environments.

iptables operates on a three-tiered system:
1. Tables: These are categorized groups of rules. Each table handles a specific type of packet:
\begin{itemize}
    \item INPUT: Incoming packets destined for the local machine.
    \item OUTPUT: Outgoing packets originating from the local machine.
    \item FORWARD: Packets routed through the machine.
\end{itemize}
2. Chains: These are sequences of rules within a table. Packets are processed through a chain until a matching rule is found, determining the packet's fate.
\newline
3. Rules: These are the individual instructions within a chain. Each rule has conditions (matching criteria) and targets (actions to take). Common actions include:
\begin{itemize}
    \item ACCEPT: Allow the packet to pass.
    \item DROP: Discard the packet silently.
    \item REJECT: Discard the packet and send an error message.
    \item LOG: Log information about the packet.
    \item JUMP: Redirect the packet to another chain.
\end{itemize} 
Step 1: Check Current Rules
Run sudo iptables -L to list current firewall rules.
This lists all rules for INPUT (incoming), FORWARD (forwarding), and OUTPUT (outgoing) chains.
\begin{lstlisting}
sudo iptables -L
\end{lstlisting}
%
Step 2: Clear Existing Rules
If you want to clear/flush out all the existing rules. Run the following command 
if you want to Reset all current rules to start fresh:
\begin{lstlisting}
sudo iptables -F
\end{lstlisting}
%
Step 3: Changing the Default Policy of Chains
the default policy of each of the chain is ACCEPT.
In order to change the policy of forwarding to drop:
\begin{lstlisting}
sudo iptables -P FORWARD DROP
\end{lstlisting}
The above command will stop any traffic to be forwarded through your system. 
That means no other system can your system as an intermediary to pass the data. 
%
Step 4: Implementing a DROP Rule 
\begin{lstlisting}
sudo iptables -A/-I chain_name -s source_ip -j action_to_take
\end{lstlisting}
The following command can be used:
\begin{lstlisting}
    Let's assume we want to block the traffic coming from an IP address 192.168.1.3.
    sudo iptables -A INPUT -s 192.168.1.3 -j DROP
\end{lstlisting}
-A INPUT: The flag -A is used to append a rule to the end of a chain. This part of the command 
tells the iptable that we want to add a rule to the end of the INPUT chain. 
\newline 
-I INPUT: In this flag the rules are added to the top of the chain.
%
Step 5: Implementing a ACCEPT Rule
If you want to add rules to specific ports of your network, then the following commands can be used. 
\begin{lstlisting}
sudo iptables -A/-I chain_name -s source_ip -p protocol_name --dport port_number -j Action_to_take
\end{lstlisting} 
\begin{lstlisting}
sudo iptables -A INPUT -s 192.168.1.3 -p tcp --dport 22 -j ACCEPT
\end{lstlisting} 
%
The Rules you set in the iptables are checked from the top to the bottom. Whenever a packet is processed to one of the top rules, it is not checked with the lower rules.
%
Deleting a Rule from the iptable (Optional)
If you want to delete the rule which accepts the traffic, 
Please follow the below example to understand it properly:
\begin{lstlisting}
sudo iptables -D chain_name rule_number
\end{lstlisting} 
%
Step 6: Saving your Configuration 
it's always better to save your configurations. There are a lot of ways to do this, but the easiest way 
you could find is with iptables-persistent package.
You can download the package from Ubuntu's default repositories:
\begin{lstlisting}
sudo apt-get update
sudo apt-get install iptables-persistent
\end{lstlisting} 
Once the installation is complete, you can save your configuration using the command:-
\begin{lstlisting}
sudo invoke-rc.d iptables-persistent save
\end{lstlisting} 

\subsection{Method 2: Configuring Firewall with firewalld}
firewalld simplifies managing rules by grouping them into zones (e.g., public, work, home).
%
Step 1: Install firewalld 
\begin{lstlisting}
sudo apt-get install firewalld 
sudo systemctl start firewalld 
sudo systemctl enable firewalld
\end{lstlisting} 
%
Step 2: Assign Network Interfaces to Zones 
\begin{lstlisting}
Assign eth0 to the "public" zone: 
sudo firewall-cmd --zone=public --add-interface=eth0 --permanent
\end{lstlisting} 
%
Step 3: Allow Services in Zones
\begin{lstlisting}
Permit HTTP traffic in the "public" zone: 
sudo firewall-cmd --zone=public --add-service=http --permanent 
sudo firewall-cmd --reload
\end{lstlisting} 
%
Step 4: View Active Zones and Rules 
\begin{lstlisting}
sudo firewall-cmd --get-active-zones 
sudo firewall-cmd --list-all
\end{lstlisting} 

\subsection{Method 3: Configuring Firewall with UFW (Uncomplicated Firewall)}
UFW is beginner-friendly and ideal for quick firewall setups. 
%
Step 1: Enable UFW
\begin{lstlisting}
    sudo ufw enable
\end{lstlisting} 
%
Step 2: Allow Specific Services
\begin{lstlisting}
    Permit SSH: 
    sudo ufw allow ssh
\end{lstlisting} 
%
Step 3: Block Traffic to Specific Ports
\begin{lstlisting}
    Block traffic to port 8080:
    sudo ufw deny 8080
\end{lstlisting} 
%
Step 4: View Status
\begin{lstlisting}
sudo ufw status
\end{lstlisting} 
%
\subsection{Common Mistakes to Avoid}
Not Saving Rules: Forgetting to save changes leads to loss of configurations after a reboot. 
\newline
Over-Blocking Traffic: Be cautious when setting DROP rules to avoid locking yourself out.
\newline
Misapplying Zones (firewalld): Ensure interfaces are assigned to the correct zone.







