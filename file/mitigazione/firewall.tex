%NOT RED
%https://www.bing.com/search?q=linux%20how%20to%20allow%20or%20drop%20packets&qs=n&form=QBRE&sp=-1&ghc=1&lq=0&pq=linux%20how%20to%20allow%20or%20drop%20packets&sc=12-34&sk=&cvid=BB9922DC160D445C8B5C10B7FCD20E9B 
%https://www.lanzagiuseppe.it/guida_firewall_41.html 
%https://linuxvox.com/blog/linux-firewalls/ 
%https://linuxvox.com/blog/linux-firewalls/ 
%https://allthings.how/set-up-a-firewall-using-ufw-or-firewalld-in-linux/ 

%NOT USED
%
%https://www.geeksforgeeks.org/computer-science-fundamentals/what-is-an-ip-address/ 
%https://www.geeksforgeeks.org/computer-networks/introduction-to-subnetting/ 
%https://www.geeksforgeeks.org/computer-organization-architecture/introduction-of-ports-in-computers/ 
%
%
%https://www.geeksforgeeks.org/linux-unix/linux-firewall/  

%https://www.geeksforgeeks.org/linux-unix/how-to-setup-firewall-in-linux/
%
% 
A firewall is a network security system, available as hardware or software, designed to monitor and control 
incoming and outgoing network traffic. %based on predefined rules. 
%By combining prevention, monitoring, and control, 
%firewalls provide customizable and comprehensive protection against both external and internal threats.
It is essential for protecting your system from unauthorized access and attacks. 
And by configuring firewall rules, you can control the flow of traffic into and out of your system. 
%A firewall inspects all incoming and outgoing traffic and decide whether to allow or block it. 
It can filter the data packets to either:
\begin{itemize}
    \item Accept: Allow the traffic.
    \item Reject: Block with an error response.
    \item Drop: Block silently without response. 
\end{itemize} 
A Linux firewall act as a barrier between trusted internal networks and untrusted external connections, filtering traffic based on predefined rules. 
With them, you can enforce predefined security policies by restricting unauthorized access and allowing legitimate communication. 
So setting up a firewall in Linux is an essential step to protect your system from unauthorized access and potential threats or data breaches. 
%These firewalls help your system or server by keeping it safe and secure.
%A firewall is the first line of defense in cybersecurity, acting as a security barrier between internal systems and external networks. 
%It forces all traffic through a single checkpoint. 
%It's a checkpoint, where data packets are monitored, filtered, and either allowed or blocked based on predefined rules. 
%\vspace{2ex} \newline 
%
%, it is allowed; if it is suspicious, blacklisted, or 
%contains malicious content, it is blocked.
%Blocked or unusual traffic is recorded in logs, and real-time alerts may be generated for serious threats.
\vspace{2ex} \newline 
%When a packet enters or exits the system, the firewall checks it against its rules to determine whether it 
%should be allowed to pass or blocked.
%All data packets entering or leaving the network must first pass through the firewall.
The firewall examines each packet against predefined security rules set by the organization and check 
if the packet matches one of this rules. 
%A Linux firewall filters network traffic based on a series of rules. 
If a packet matches an "allow" rule, it is allowed to pass. If it matches a "deny" rule, it is blocked. 
Since it is not possible to define every rule, the firewall applies a default policy (accept, reject, or drop). 
Setting the default policy to drop or reject is considered best practice to prevent unauthorized access.
\vspace{2ex} \newline 
%These rules specify which types of network packets (data sent over the network) are allowed or denied based on factors like: 
The rules are based on factors like: 
\begin{itemize}
    \item IP Address: The source or destination address of the packet. 
    IP Address: The source or destination address of the packet.
    \item Port Number: The communication port the packet is trying to reach 
    (e.g., port 80 for HTTP or port 22 for SSH). 
    Port Number: The communication port the packet is trying to reach (e.g., port 80 for HTTP or port 22 for SSH).
    \item Protocol: The type of network protocol used (TCP, UDP, ICMP, etc.). 
    Protocol: The type of network protocol used (TCP, UDP, ICMP, etc.).
    \item Connection State: Whether the packet is part of an established connection or is a 
    new connection request. 
    Connection State: Whether the packet is part of an established connection or is a new connection request.
\end{itemize} 
Firewalls are essential because they:
\begin{itemize}
    \item Prevent Unauthorized Access: Like a locked door with a guard, only trusted users and traffic are allowed through.
    \item Block Malicious Traffic: Harmful data such as viruses, phishing attempts, or denial-of-service (DoS) attacks are stopped before reaching the system.
    \item Protect Sensitive Information: Safeguards personal and business data from theft or accidental leaks.
    \item Control Network Usage: Enforces policies such as parental controls, workplace restrictions, or government filtering.
    \item Mitigate Insider Risks: Detects suspicious applications or data exfiltration attempts from within the network.
\end{itemize} 


%https://www.ninjaone.com/it/blog/come-configurare-un-firewall-linux/# 
%https://www.cybersecurity360.it/soluzioni-aziendali/iptables-cose-a-cosa-serve-come-configurarlo-e-gli-usi-avanzati-per-il-firewalling/ 
\subsection{Come configurare un firewall Linux: la guida completa} 
Prima di dedicarsi alle configurazioni, è necessario iniziare con una politica di sicurezza ben articolata. 
Definisci quali dati e servizi devono essere protetti e da quali minacce, 
per garantire che il firewall Linux funga da prima linea di difesa efficace. 
Non si tratta solo di ciò che decidi di bloccare e da chi; 
anche le considerazioni sulla sicurezza delle informazioni, come i metodi di blocco, entrano in gioco. 
\vspace{2ex} \newline  
Per una sicurezza ottimale di Linux, adotta una posizione di minima esposizione. 
Una buona pratica consiste nel negare tutti per impostazione predefinita e aprire 
le vie di accesso solo quando necessario, il che è essenzialmente un’architettura zero-trust. 
In questo modo si minimizzano le superfici di attacco potenziali e si riduce l’esposizione involontaria. 
Un caso d’uso comune è la distinzione tra regole di “deny” e “drop” nel firewall Linux: 
le prime inviano attivamente una risposta di RICHIESTA RIFIUTATA al mittente dei pacchetti, 
mentre le seconde scartano silenziosamente i pacchetti di richiesta senza risposta. 
Chiunque abbia mai gestito un server di produzione su Internet aperto può dirti quanto 
sia saggio non pubblicizzare il tuo indirizzo IP a scansioni di porte casuali, ad esempio.
\vspace{2ex} \newline 
Sebbene l’attenzione sia rivolta a contrastare il traffico in entrata non richiesto, 
il monitoraggio ed il controllo del traffico in uscita sono altrettanto importanti. 
Ciò assicura che i sistemi potenzialmente compromessi all’interno della rete non diventino canali 
per l’esfiltrazione dei dati o altre attività dannose. 
Le infezioni degli endpoint sui laptop degli utenti possono contaminare altri computer all’interno della rete, 
come l’archiviazione di rete o i server aziendali, più facilmente di quanto possano fare i malintenzionati, 
in parte a causa del livello di fiducia esistente. 
\vspace{2ex} \newline  
Configurazione di regole specifiche per i servizi
Servizi diversi presentano vulnerabilità diverse.
Adatta le regole del firewall Linux per soddisfare queste sfumature, assicurando che ogni servizio, 
sia esso SSH, HTTP o FTP, abbia il suo scudo protettivo su misura. 
I servizi di watchdog come fail2ban possono anche osservare i log dei servizi alla ricerca di 
segnali di attacchi alla sicurezza, intervenendo generalmente in modo appropriato regolando le regole 
del firewall e le blacklist.
\vspace{2ex} \newline  
Verifica e aggiornamenti regolari del firewall Linux
Il panorama delle minacce è in continua evoluzione, e così anche le tue difese. 
La verifica e l’aggiornamento regolari delle configurazioni del firewall garantiscono 
che queste rimangano solide, pertinenti e rispondenti all’ambiente attuale delle minacce. 
In sostanza, la configurazione di un firewall Linux non consiste solo nell’impostare le regole, 
ma si tratta di costruire un sistema di protezione completo che sia in linea con gli standard di 
sicurezza dell’organizzazione. 
\vspace{4ex} \newline 
Firewalls can be categorized based on their generation.
\begin{enumerate}
    \item Network Placement
Packet Filtering Firewall
Stateful Inspection Firewall
Proxy Firewall (Application Level)
Circuit-Level Gateway
Web Application Firewall (WAF)
Next-Generation Firewall (NGFW)
\item Systems Protected
Network Firewall
Host-Based Firewall
\item Data Filtering Method
Perimeter Firewall
Internal Firewall
Distributed Firewall
\item Form Factors
Hardware Firewall
Software Firewall 
\end{enumerate} 

%https://tekneed.com/how-to-configure-firewall-in-linux-systems/ 
\subsection{Hardware vs. Software Firewalls}
There are two primary types of firewalls:
Hardware-based firewalls are physical devices designed to protect an entire network.
Software-based firewalls (like those we use on Linux) protect individual systems.


\subsection{How To Prevent Data Breaching}
For Enterprises
1. Vulnerability Management - Using a vulnerability tool or at the very least complete a vulnerability assessment will help you identify the gaps, weaknesses, and security miss configurations within your physical and virtual environments. It can continuously monitor your infrastructure and IT assets for vulnerabilities and compliance weaknesses and configuration best practices.
2. End-user security awareness - End-user security awareness training when done, is a huge benefit. But only when it changes the culture of the company to be more security-minded. Training insiders may help to eliminate mistakes that lead to the breach as well as notice odd behavior by malicious insiders or fraudsters.
3. Update software regularly - Keep software updated, install patches, Operating system must update regularly as out-dated software may contain bugs that can prevent attackers to get access to your data easily. This is an easy and cost-effective way to strengthen your network and stop attacks before they happen.
4. Limit access to your valuable data - In old days employees have access to all the data of the company. Now the company is limiting the critical data for employee access because there is no need to show financial data or personal data to the employees.

For Employees
1. Securing Devices - While using any device we should ensure that we have installed genuine antivirus, we are using the password on our device, and all the software is updated.
2. Securing accounts - We should change the password of our account after a short span of time so that an attacker cannot get easy access to the account.
3. Beware of social engineering - Whenever you are surfing on the internet be aware of fraud links and sites do not open any site or don't provide any crucial information to anyone it can be so harmful.
4. Keep checking bank receipt - You should daily check your bank transaction for ensuring that there is no fraud transaction.

\subsection{Tips for Effective Firewall Management}
1. Understand Your Network Needs
Identify which ports and services are required for your system and block the rest.

2. Use Logging for Monitoring
Enable logging to track allowed and blocked traffic for better troubleshooting.

For firewalld
sudo firewall-cmd --set-log-denied=all
For iptables
Use the LOG target to record dropped packets.

3. Test Firewall Rules
Use tools like nmap to scan your system and verify that only intended ports are open.

4. Automate Rule Application
Write startup scripts or use tools like Ansible to automate firewall configurations.   


\subsection{Common Mistakes to Avoid}
Not Saving Rules: Forgetting to save changes leads to loss of configurations after a reboot. 
\newline
Over-Blocking Traffic: Be cautious when setting DROP rules to avoid locking yourself out.
\newline
Misapplying Zones (firewalld): Ensure interfaces are assigned to the correct zone.


\subsection{Linux's firewall maanagment tools}
Linux offers multiple firewall management tools, including iptables and firewalld, 
both of which can be used to manage and secure your network.
%
There is more than one Linux firewall option available. 
When we come to drop down and research, we have a few popular names IPCop, iptables, Shorewall, and UFW; 
but one of the most popular is the "iptables" firewall. 
%
Though iptables is still available, Red Hat distributions have moved towards 
using firewalld and nftables for better usability and scalability.
\begin{center} 
\begin{longtable}{|c|c|c|c|} 
    \hline
    Feature	& iptables & firewalld & UFW \\
    \hline 
    Ease of Use	& Moderate & Easy & Very Easy \\
    \hline 
    Best For & Advanced users & Zone-based management & Beginners \\
    \hline 
    Dynamic Rules & No & Yes & Limited \\
    \hline 
    GUI Available & No	& Yes (GUI plugins)	& Yes (Gufw) \\
    \hline 
%\caption{Tipologie di mitigazioni} 
%\label{tabella:mitigazioni:tipologie} 
\end{longtable} 
\end{center}  


%https://www.geeksforgeeks.org/linux-unix/iptables-command-in-linux-with-examples/ 
%https://www.geeksforgeeks.org/linux-unix/ipv6-iptables-rules/
%
%NOT USED
%https://en.wikipedia.org/wiki/Mandatory_access_control
\subsection{iptables} 
Linux-based software that performs manipulation functions, packet filtering, and NAT (network address translation). 
%iptables is the most widely used firewall tool in Linux. 
It works by defining chains of rules that filter network traffic at various points (such as incoming or outgoing traffic). 
%And it operates at the network layer (Layer 3) and transport layer (Layer 4).
%A powerful command-line tool that filters network traffic.  
%The iptables command in Linux is a powerful tool that is used for managing the firewall rules and network traffic. 
Iptables allows system administrators to control incoming and outgoing traffic by setting up the rules. 
When a packet is received in a Linux base system, it has to go through the chains and 
tables in the iptables firewall.  
%It's essential for securing servers and networks by selectively permitting or denying specific types 
%of traffic based on defined rules and conditions. 
%This flexibility makes iptables a fundamental component in Linux networking and security configurations. 
\vspace{3ex} \newline 
There are other interfaces such ipv6 tables which are used to manage filtering tables for IPv6. 
To block IPv6 traffic on a Linux system, you can use ip6tables, the IPv6 counterpart of iptables 
%
\vspace{3ex} \newline 
The following are the some of the reasons to use Iptables in Linux: 
\begin{itemize}
    \item Firewall Configuration: It helps in enabling the precise control over the netowrk traffic to protect against unauthorized access and attacks.
    Robust Firewall Capabilities: It facilitates with configuration of firewall rules to control incoming and outgoing traffic, enhancing network security.
    Firewall Protection: Iptables can be configured through blocking the unauthorized access and allow legitimate traffic. It facilitates with providing a robust firewall to secure a network or individual system.
    \item Packet Filtering: It allows in filtering based on the criteria like protocol, IP addresses and prots providing the security.
    Packet Filtering: Iptables facilitates with providing filtering features for network packets based on various criteria such as source and destination IP addresses and ports.
    Precise Packet Filtering: It provides the filtering based on criteria such as protocol, source/destination IP addresses, and ports, ensuring only authorized traffic passes through.
    Traffic Shaping and Control: By setting rules, iptables we can manage and prioritize network traffic, ensuring critical services maintain performance and reducing congestion during peak usage times.
    \item Network Address translation (NAT): It facilitates with seamless communication between different network segments.
    NAT: Iptables supports the NAT by allowing for the translation of the private IP address to public address making an essential for devices within a private network to establish the communication with external networks.
    Network Address Translation (NAT): It supports NAT functionality for translating IP addresses and ports, essential for network connectivity and management.
    Network Address Translation (NAT): iptables facilitates NAT, allowing multiple devices on a private network to access external networks using a single public IP address, essential for home and business networks.
    \item Logging and Monitoring: It provides the insights into the network activity for providing feature sof security auditing and troubleshooting. 
    Logging and Monitoring: It provides logging capabilities to monitor and analyze network traffic, aiding in security auditing and troubleshooting.
    \item Stateful Inspection: Through stateful inspection, iptables helps in tracking the state of network connections with providing the enhanced security by legitimating the traffic that is only allowed.
    \item Port Forwarding: With iptables, administrators can redirect traffic from one port to another, enabling access to services running on different ports or internal servers from external networks.
\end{itemize} 
%Here filters are responsible for filtering the packets on the defined rules based on the source and destination of the IP address, port number, and protocol type. 


\subsection{Configuring Firewall with iptables}
iptables operates on a three-tiered system:
%The firewall matches packets with rules defined in these tables and then takes the specified action on a possible match. 
\begin{itemize}
    \item Tables is the name for a set of chains.
    \item Chain is a collection of rules.
    \item Rule is condition used to match packet.
    \item Target is action taken when a possible rule matches. Examples of the target are ACCEPT, DROP, QUEUE.
    \item Policy is the default action taken in case of no match with the inbuilt chains and can be ACCEPT or DROP.
\end{itemize}
\begin{lstlisting}
    iptables --table TABLE -A/-C/-D... CHAIN rule --jump Target

    #The syntax for using policies:  
    sudo iptables -I/-A name_chain -s source_ip -p protocol_name --dport port_number -j action_to_do 
\end{lstlisting}  

\subsubsection{Tables}
These are categorized groups of rules. 
Each table handles a specific type of packet. 
The most commonly used tables are filter and nat but we have five predefined tables:  
%There are five possible tables as follows: We will discuss five predefined tables: 
\begin{center} 
\begin{longtable}{|p{0.3\textwidth}|p{0.5\textwidth}|} 
    \hline
    \textbf{Tipologie di tabella} & \textbf{Descrizione} \\
    \hline 
    Security Table & 
    It is often used in conjunction with other security tools like SELinux, 
    it is also used for MAC (Mandatory Access Control) rules, which can further be used to set rules 
    related to security labels and access controls. 
    %\vspace{2ex} \newline
    %It has four built-in chains: OUTPUT, FORWARD, INPUT, and SECMARK. 
    \\
    \hline 
    Mangle Table & 
    It is used to modify packets by setting the packet's ToS/DSCP field, altering packet header 
    fields, and changing packet marks. 
    %For specialised packet alteration. Inbuilt chains include PREROUTING and OUTPUT.
    %\vspace{2ex} \newline
    %It has five built-in chains: POSTROUTING, FORWARD, OUTPUT, PREROUTING, and INPUT. 
    \\
    \hline 
    Nat Table & 
    It stands for network address translation, which helps in sharing a single 
    public IP address between multiple devices. 
    %\vspace{2ex} \newline
    %It has two built-in chains: PREROUTING and POSTROUTING.  
    \\
    \hline 
    Raw Table & 
    It is used for the configuration of low-level packet processing.  
    It has limited built-in chains, but the user can create additional chains if required. 
    %raw : Configures exemptions from connection tracking. Built-in chains are PREROUTING and OUTPUT.
    \item 
    \\
    \hline 
    Filter Table & 
    It is used for packet filtering. 
    %\vspace{2ex} \newline
    %It has three built-in chains. INPUT, OUTPUT, and FORWARD. 
    \\
    \hline 
\caption{Types of Tables} 
\label{tabella:iptables:tipologie:tabelle} 
\end{longtable} 
\end{center}  

\subsubsection{Chains} 
These are sequences of rules within a table. 
Packets are processed through a chain until a matching rule is found, determining the packet's fate. 
%There are few built-in chains that are included in tables. They are: 
%Chain Rule: Rules that are described for a particular task. Subdivided into three types: 
\begin{center} 
\begin{longtable}{|p{0.25\textwidth}|p{0.55\textwidth}|} 
    \hline
    \textbf{Tipologie di difesa} & \textbf{Sottotipologie} \\
    \hline 
    INPUT Chains & 
    Filter incoming traffic in the local system. 
    Traffic has to go from every rule that has been assigned within input chains.
    INPUT : A set of rules for packets destined to localhost sockets. 
    INPUT: Incoming packets destined for the local machine.
    \\
    \hline 
    OUTPUT Chains & 
    Filter Outgoing traffic for the local system. 
    Traffic going through local machines has to pass through these output chains. 
    OUTPUT : It is locally generated packets, meant to be transmitted outside. 
    OUTPUT: Outgoing packets originating from the local machine.
    \\
    \hline 
    FORWARD Chains & 
    Packets forwarded from one system to another go throw it. 
    Traffic going from the arising network location to another network location has to pass through forward chains. 
    FORWARD :for packets routed through the device. 
    FORWARD: Packets routed through the machine.
    \\
    \hline 
    PREROUTING & It is used for modifying packets as they arrive. \\ \hline 
    POSTROUTING & IIt helps in modifying packets as they are leaving. \\ \hline 
\caption{Types of Chains} 
%\label{tabella:difese:tipologie} 
\end{longtable} 
\end{center} 

\subsubsection{Rules} 
These are the individual instructions within a chain. 
Each rule has conditions (matching criteria) and targets (actions to take). Common actions include:
%We have three basic Policies. Let's discuss Some Basic Operations and their Syntax
\begin{itemize}
    \item ACCEPT: Allow the packet to pass. 
    ACCEPT: It allows the IP we provide to make users go into the system.
    \item DROP: Discard the packet silently. 
    DROP: It can block an incoming signal, which basically states that the firewall is blocked for that particular IP.
    \item REJECT: Discard the packet and send an error message. 
    REJECT: It works similarly to Drop, but in 'drop' the sender is blocked without any notification whereas in 'reject' a message states the reason for not being able to connect.
    \item LOG: Log information about the packet.
    \item JUMP: Redirect the packet to another chain.
\end{itemize} 

\subsubsection{Example}
\begin{lstlisting} 
    #Check Current Rules by list current firewall rules
    sudo iptables -L 

    #Clear Existing Rules: Reset all current rules to start fresh
    sudo iptables -F 
    sudo iptables --flush

    #Changing the Default Policy of each Chains
    sudo iptables -P FORWARD DROP 
    sudo iptables -P Chain_name Action_to_be_taken
\end{lstlisting} 
\begin{lstlisting} 
    #Basic Syntax for using iptables
    sudo iptables [option] CHAIN-rule [-j target] 

    #allow incoming ICMP (ping) traffic on the INPUT chain:
    sudo iptables -A INPUT -p icmp -j ACCEPT

    #Implementing a DROP Rule 
    sudo iptables -A/-I chain_name -s source_ip -j action_to_take
    #block the traffic coming from 192.168.1.3.
    sudo iptables -A INPUT -s 192.168.1.3 -j DROP
    
    #Implementing a ACCEPT Rule
    sudo iptables -A/-I chain_name -s source_ip -p protocol_name --dport port_number -j Action_to_take

    #rules to specific ports of your network
    sudo iptables -A INPUT -s 192.168.1.3 -p tcp --dport 22 -j ACCEPT
    
    #Deleting a Rule from the iptable (Optional)
    sudo iptables -D chain_name rule_number 
\end{lstlisting} 
\vspace{3ex} 
It's always better to save your configurations. There are a lot of ways to do this, but the easiest way 
you could find is with iptables-persistent package.
%You can download the package from Ubuntu's default repositories:
\begin{lstlisting} 
    #Saving your Configuration 
    sudo apt-get update
    sudo apt-get install iptables-persistent
    
    #save your configuration using the command
    sudo invoke-rc.d iptables-persistent save 
    #To save the iptables configuration
    sudo iptables-save

    #Restoring iptables config 
    sudo iptables-restore
\end{lstlisting} 

\begin{center} 
\begin{longtable}{|p{0.3\textwidth}|p{0.5\textwidth}|} 
    \hline 
    \textbf{Options} & \textbf{Descriptions} \\
    \hline 
    -C	& [CHECK]: This is to check and find a rule that matches the requirements of the string. \\
    \hline 
    -D	& [DELETE]: This is used to delete a specific rule. \\
    \hline 
    -A	& [APPEND]: This is used to append or add rules. \\
    \hline 
    -I	& [INSERT]: This can add a rule to a particular position in a string. \\
    \hline 
    -L	& [LIST]: To display all the rules we can use this. \\
    \hline 
    -v	& [VERBOSE]: This is used to get more information in the list option. \\
    \hline 
    -X	& [DELETE CHAIN]: This deletes the entire supplied string. \\
    \hline 
    -p	& [Protocol\_name]: It is used to define the name of the protocol. \\
    -p, --proto & is the protocol that the packet follows. Possible values maybe: tcp, udp, icmp, ssh etc. \\
    \hline 
    -N	& [NEW CHAIN]: To create a new chain. \\
    \hline 
    -j	& [job]: It tells what operation has to be done with the packet. \\ 
    -j, --jump & this parameter specifies the action to be taken on a match. \\
    \hline 
    -F	& [Flush]: It is to delete all rules. \\
    \hline 
    -s	& [specify]: It is a flag used to specify the source of the packet.  \\ 
    -s, --source & is used to match with the source address of the packet. \\ 
    -d, --destination & is used to match with the destination address of the packet.\\
    -i, --in-interface & matches packets with the specified in-interface and takes the action. \\
    \hline 
\caption{Tipologie di mitigazioni} 
\label{tabella:mitigazioni:tipologie} 
\end{longtable} 
\end{center} 



%https://www.redhat.com/en/blog/firewalld-linux-firewall 
\subsection{firewalld}
\textbf{firewall} \newline 
firewalld is a more modern firewall management tool for Linux, available on distributions like CentOS, RHEL, and Fedora. 
It provides a more dynamic and user-friendly approach to managing firewall rules using zones.
A dynamic firewall management tool, offering flexible configuration that supports both IPv4 and IPv6
It uses zones to define trust levels for network connections and interfaces, providing a simpler method to manage firewall settings. 
%Linux Firewall (firewalld) monitors and governs the network traffic (outbound/inbound connections). 
%It can be used to block access to different IP addresses, specific subnets, ports (virtual points where network connections begin and end), and services.
It works in concepts of zones (segments) to protect our system from unauthorized access and to control network traffic (incoming and outgoing).  
%firewalld simplifies managing rules by grouping them into zones (e.g., public, work, home). 
%
%FirewallD is a dynamic firewall daemon that manages firewall rules in real-time. 
%It provides flexibility by allowing the use of zones, which are predefined sets of rules governing network traffic. 
%These zones help you manage different types of traffic (incoming and outgoing) based on your system’s network interfaces.

Firewalld: Simplified Firewall Management
Firewalld is the default firewall tool in many Red Hat distributions. 
It simplifies the process of how to configure firewall in Linux systems. It enables management of rules by allowing 
dynamic rule changes without requiring a firewall restart. 
Firewalld is structured around the concept of zones that define a level of trust for network connections.


Managing Firewall Zones
FirewallD operates on the concept of zones, which group firewall rules together. 
Every system has an active zone, and by default, this is usually the public zone. 


Understanding Firewall Targets
Before modifying the rules, it’s important to understand the targets available in FirewallD. 
A target is essentially the action you want the firewall to take on a connection or packet. FirewallD offers the following targets:
\begin{itemize}
    \item Accept: Allows all traffic.
    \item Drop: Silently drops all traffic without notifying the sender. 
    \item Reject: Drops all traffic but notifies the sender of the rejection. 
    \item Default: Similar to reject, but used as the fallback action.
\end{itemize}


Firewalld can restrict access to services, ports, and networks. You can block specific subnets and IP addresses.

As with any firewall, firewalld inspects all traffic traversing the various interfaces on your system. 
The traffic is allowed or rejected if the source address network matches a rule.

Firewalld uses the concept of zones to segment traffic that interacts with your system. 
A network interface is assigned to one or more zones, and each zone contains a list of allowed ports and services. 
A default zone is also available to manage traffic that does not match any zones.

Firewalld is the daemon's name that maintains the firewall policies. 
Use the firewall-cmd command to interact with the firewalld configuration.

Persistent vs. Runtime Rules
One critical aspect of knowing how to configure firewall in Linux systems using FirewallD is understanding 
the difference between runtime and persistent rules.

Runtime rules are temporary and will be lost upon reboot.
Persistent rules are saved and will be applied on every reboot.
To ensure your rules are persistent, always add the --permanent flag when creating rules:

\begin{lstlisting}
    sudo apt-get install firewalld 
    sudo systemctl start firewalld 
    sudo systemctl enable firewalld 

    #Get a more detailed view of the FirewallD service
    systemctl status firewalld

    Check that firewalld is running:
    sudo firewall-cmd --state 

    Stop the firewall: 
    sudo systemctl stop firewalld 
\end{lstlisting} 

\begin{lstlisting}[language=bash, caption=Allowing SSH traffic] 
    #Allow SSH (Secure Shell or Secure Socket Shell) traffic in the "public" zone
    sudo firewall-cmd --zone=public --add-services=ssh --permanent   

    #Allow traffic from any IP on a specific TCP port 8080
    sudo firewall-cmd --zone=public --add-port=8080/tcp --permanent 

    #Remove a port rule
    firewall-cmd --remove-port=44/tcp --permanent 

    #Allow traffic from a specific IP
    firewall-cmd --add-source=192.168.1.100 --permanent

    #Blocking incoming traffic on a specific IP address 
    sudo firewall-cmd --zone=public --add-rich='rule family="ipv4" source address="192.168.52.1" reject' 

    #Assign eth0 to the "public" zone: 
    sudo firewall-cmd --zone=public --add-interface=eth0 --permanent  

    #Assign traffic coming from the network 172.16.1.0/24 to the internal zone 
    #Allow the Jenkins service:
    sudo firewall-cmd --zone=internal --add-source=172.16.1.0/24 --permanent 
    sudo firewall-cmd --add-service=jenkins --permanent   

    #Reload the firewall for them to take effect
    sudo firewall-cmd --reload 
\end{lstlisting}  

\begin{lstlisting} 
    #View all zones on a system
    sudo firewall-cmd --get-zones

    #View Active Zones and Rules 
    sudo firewall-cmd --get-active-zones 
    sudo firewall-cmd --list-all  

    #View all settings for all zones
    sudo firewall-cmd --list-all-zones 

    #List all active ports 
    firewall-cmd --list-ports 

    #view all predefined services on your system:
    firewall-cmd --get-services 

    #Display the default zone
    sudo firewall-cmd --get-default-zone

    #Changing the Active Zone
    firewall-cmd --set-default-zone=<zone>   

    #Reload firewalld and all permanent rules:
    sudo firewall-cmd --reload 
\end{lstlisting}   


\subsection{nftables}
High-performance firewall tool designed to provide more flexible handling of complex firewall rules.
%It is the successor to iptables and provides an improved, more efficient way to filter network traffic. 
%It is designed to replace iptables in newer Linux distributions. 
%nftables provides advanced management of firewall rules and offers greater flexibility compared to firewalld. 
It supports complex firewall rules while maintaining high performance, which makes it suitable for managing large-scale systems.
\begin{lstlisting} 
    View existing rules:
    nft list ruleset
    
    Add a rule to allow traffic on port 80:
    nft add rule inet filter input tcp dport 80 accept
    
    Delete a rule:
    nft delete rule inet filter input handle <rule-handle-number>
\end{lstlisting} 

\subsection{ufw (Uncomplicated Firewall)}
User-friendly frontend for iptables, simplifying configuration.
UFW is beginner-friendly and ideal for quick firewall setups. 
\begin{lstlisting} 
    Step 1: Enable UFW
    sudo ufw enable

    Allow Specific Services (permit SSH)
    sudo ufw allow ssh

    Block Traffic to Specific Ports (port 8080)
    sudo ufw deny 8080

    View Status
    sudo ufw status
\end{lstlisting} 


\subsection{CSF (ConfigServer Security \& Firewall)}
CSF (ConfigServer Security \& Firewall) is a complete security solution, including firewall features.  

\subsection{ClearOS and OPNsense}
ClearOS and OPNsense is a Firewall-focused operating systems providing web-based interfaces.  

