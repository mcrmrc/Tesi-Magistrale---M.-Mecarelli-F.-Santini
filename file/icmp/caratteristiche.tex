\subsection{Caratteristiche del protocollo}
\begin{minipage}{0.5\textwidth}
    \begin{itemize}
        \item Opera al Livello di rete (Livello 3) nel modello OSI.
        \item Funziona con IP per fornire feedback sui problemi di rete.
        \item Non stabilisce una sessione (Stateless e Connectionless).
        \item Nessun numero di porta (a differenza di TCP e UDP).
        \item Utilizzato per la risoluzione dei problemi di rete (e.g. esempio, ping, traceroute).
        \item Supporta IPv4 (ICMPv4) e IPv6 (ICMPv6) con funzionalità avanzate in ICMPv6.
    \end{itemize}
\end{minipage}
\begin{minipage}{0.5\textwidth}
    \centering
    \includegraphics[width=0.9\textwidth]{./img/iso_osi-model.jpg}
    \captionof{figure}{Modello ISO/OSI}
\end{minipage}
\vspace{2ex} 
\subsubsection*{Utilizzi di ICMP}
\begin{itemize}
    \item \textbf{Segnalazione errori}: informa il mittente sui problemi di rete (ad esempio, destinazione non raggiungibile, perdita di pacchetti).
    \item \textbf{Diagnostica di rete}: aiuta nella risoluzione dei problemi di rete utilizzando strumenti come ping e traceroute.
    \item \textbf{Messaggistica di controllo}: gestisce la congestione della rete e gli aggiornamenti di routing in alcuni casi.
\end{itemize}
\subsubsection*{Strumenti che utilizzano ICMP} 
\begin{enumerate}
    \item Ping (Richiesta Echo ICMP e Risposta Echo) 
    \vspace{1ex} \newline 
    Il comando \textbf{ping}, invia pacchetti ICMP Echo Request per testare la connettività.
    \begin{itemize}
        \item Invia delle richieste Eco ICMP a una destinazione per verificare la connettività.
        \item Se l'host è raggiungibile, risponde con un ICMP Echo Reply.
    \end{itemize}
    \item Traceroute (tracert su Windows, traceroute su Linux/macOS) 
    \vspace{1ex} \newline 
    Il comando \textbf{traceroute}, utilizza messaggi ICMP Time Exceeded per mappare il percorso dei pacchetti.
    \begin{itemize}
        \item Tramite i messaggi ICMP Time Exceeded traccia il percorso che i pacchetti seguono attraverso una rete
        \item Il valore TTL (Time-To-Live) viene incrementato per determinare ciascun router lungo il percorso.
    \end{itemize}
    \item Scoperta del percorso MTU (PMTUD) 
    \vspace{1ex} \newline 
    La \textbf{PMTUD}, utilizza messaggi ICMP Fragmentation Needed per ottimizzare le dimensioni dei pacchetti. 
    Ovvero per trovare la dimensione ottimale del pacchetto per un percorso di rete.
\end{enumerate} 
% ICMP in IPv6 (ICMPv6)
% ICMPv6 extends ICMP functionality for IPv6 networks, including:
%\begin{itemize}
%    \item Neighbor Discovery Protocol (NDP) – Replaces ARP for IPv6 
%    \item Router Advertisements & Solicitation – Helps configure IPv6 addresses. 
%    \item MLD (Multicast Listener Discovery) – Manages multicast group memberships
%\end{itemize}

\subsection{Struttura di un pacchetto ICMP} 
I messaggi ICMP vengono inviati utilizzando l'intestazione IP di base. 
Il primo ottetto della porzione dati del datagramma è un campo di tipo ICMP; 
il valore di questo campo determina il formato dei dati rimanenti. 
Qualsiasi campo etichettato come "non utilizzato" è riservato per future estensioni e deve essere zero quando 
inviato, ma i destinatari non dovrebbero utilizzare questi campi (eccetto per includerli nel checksum). 
Salvo diverso avviso nelle singole descrizioni del formato, 
i valori dei campi dell'intestazione internet sono i seguenti: 
\begin{itemize} 
    \item Versione 4 
    \item IHL: lunghezza dell'intestazione internet in parole da 32 bit. 
    \item Tipo di Servizio 0 
    \item Lunghezza Totale: lunghezza dell'intestazione internet e dei dati in ottetti. 
    \item Identificazione, Flag, Offset del frammento (utilizzato nella frammentazione dei pacchetti). 
    \item Tempo di vita in secondi; poiché questo campo viene decrementato in ciascun dispositivo in cui il 
    datagramma viene elaborato, il valore in questo campo dovrebbe essere almeno grande quanto il numero di 
    gateway che questo datagramma attraverserà. 
    \item Protocollo ICMP = 1
    \item Checksum dell'intestazione: Il complemento a 16 bit della somma del complemento a uno di tutte le 
    parole a 16 bit nell'intestazione. Per calcolarlo, il campo di checksum deve essere zero. 
    Questo checksum potrebbe essere sostituito in futuro.
    \item Indirizzo di origine. L'indirizzo del gateway o dell'host che compone il messaggio ICMP.
    Se non diversamente specificato, può trattarsi di uno qualsiasi degli indirizzi di un gateway.
    \item Indirizzo di destinazione L'indirizzo del gateway o dell'host a cui deve essere inviato il messaggio.
\end{itemize} 
\begin{minipage}{0.5\textwidth}
    La parte del messaggio relativa a ICMP è composto dai seguenti campi:
    \begin{itemize}
        %\item \textbf{Intestazione IP}: primi 20 ottetti di un pacchetto ICMP. 
        \item \textbf{Tipo}: Identifica il tipo di messaggio (ad esempio, Echo Request, Destinazione irraggiungibile).
        \item \textbf{Codice}: Fornisce dettagli aggiuntivi sul tipo di messaggio.
        \item \textbf{Checksum}: Garantisce l'integrità dei dati.
        \item \textbf{Dati}: Opzionale, può contenere parte del pacchetto IP originale che ha causato l'errore.
    \end{itemize}
\end{minipage}
\hspace{2ex}
\begin{minipage}{0.5\textwidth}
    \centering
    \includegraphics[width=\textwidth]{./img/ICMP-packet-structure.png}
    \captionof{figure}{Struttura pachhetto ICMP/IP} 
    \vspace{2ex} 
    Nel caso si usi il protocolllo ICMP, il campo \textit{protocol} nell'intestazione IP avrà valore 1
\end{minipage} 
\newline
I messaggi ICMP sono classificati o come messaggi di errore o come messaggi informativi
\vspace{1ex} 
\begin{itemize}
    \item \textbf{Messaggi di errore} - Segnalano problemi nella comunicazione di rete [Table:\ref{table:icmpv4:errormessage}] [Table:\ref{table:icmpv6:errormessage}].
    \item \textbf{Messaggi informativi} - Utilizzati per scopi diagnostici e di controllo [Table:\ref{table:icmpv4:infomessage}] [Table:\ref{table:icmpv6:infomessage}].
\end{itemize}  
%\begin{minipage}{\linewidth}
    \subsubsection*{Messaggi di Errore}
    \begin{tabular}{|p{0.3\linewidth}|p{0.3\linewidth}|p{0.3\linewidth}|p{0.4\linewidth}|}
        \hline 
        \textbf{Type} & \textbf{Code} & \textbf{Meaning} \\
        \hline
        3 & 0-5 & Destination Unreachable (e.g., net, host, port, \dots) \\
        \hline 
        4 & 0 & Source Quench (indica la congestione nella rete) \\
        \hline 
        5 & 0-3 & Redirect Message (suggerisce strada migliore) \\
        \hline 
        11 & 0-1 & Time Exceeded (TTL scaduto, \dots) \\
        \hline 
        12 & 0-1 & Parameter Problem (intestazione IP non valida) \\ 
        \hline 
    \end{tabular}
    \captionof{table}{ICMP v4 messaggi di errore} 
    \label{table:icmpv4:errormessage}
    %
    \begin{tabular}{|p{0.3\linewidth}|p{0.3\linewidth}|p{0.3\linewidth}|p{0.4\linewidth}|}
        \hline 
        \textbf{Type} & \textbf{Code} & \textbf{Meaning} \\
        \hline
        1 & 0-6 & Destination Unreachable (e.g., net, host, port, \dots) \\
        \hline 
        2 & 0 & Packet Too Big  (indica la congestione nella rete) \\ 
        \hline 
        3 & 0-1 & Time Exceeded (TTL scaduto, \dots) \\
        \hline 
        4 & 0-2 & Parameter Problem (intestazione IP non valida) \\ 
        \hline 
    \end{tabular}
    \captionof{table}{ICMP v6 messaggi di errore} 
    \label{table:icmpv6:errormessage}
%\end{minipage} 
%\begin{minipage}{\linewidth}
    \subsubsection*{Messaggi di Informazione} 
    \begin{tabular}{|p{0.3\linewidth}|p{0.3\linewidth}|p{0.3\linewidth}|p{0.4\linewidth}|}
        \hline 
        Type & Code & Meaning \\
        \hline 
        128 & 0 & Echo Request Message: invia dati a un destinatario (e.g. ping)\\
        \hline 
        129 & 0 & Echo Reply Message: replica i dati ricevuti mandandoli al mittente (e.g risposta a ping)\\
        \hline  
    \end{tabular} 
    \captionof{table}{ICMP v6 messaggi di informazione}
    \label{table:icmpv6:infomessage}
    %
    \begin{tabular}{|p{0.3\linewidth}|p{0.3\linewidth}|p{0.3\linewidth}|p{0.4\linewidth}|}
        \hline 
        Type & Code & Meaning \\
        \hline 
        8 & 0 & Echo Request Message: invia dati a un destinatario (e.g. ping)\\
        \hline 
        0 & 0 & Echo Reply Message: replica i dati ricevuti mandandoli al mittente (e.g risposta a ping)\\
        \hline 
        13 & 0 & Timestamp Request Message (invia un timestamp a un destinatario)\\
        \hline 
        14 & 0 & Timestamp Request Message (replica i il timestamp ricevuta al mittente)\\
        \hline 
        15 & 0 & information request message (invia un timestamp a un destinatario)\\
        \hline 
        16 & 0 & information reply message (replica i il timestamp ricevuta al mittente)\\
        \hline 
        %9 & 0 & Router Advertisement & Routers announce themselves to hosts. \\
        %\hline
        %10 & 0 & Router Solicitation & Hosts request router advertisements. \\
        %\hline 
    \end{tabular} 
    \captionof{table}{ICMP v4 messaggi di informazione}
    \label{table:icmpv4:infomessage}
%\end{minipage} 

%\begin{minipage}{0.5\textwidth}
    
%\end{minipage}