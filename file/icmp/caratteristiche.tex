\subsection{Caratteristiche del protocollo}
\begin{minipage}{0.5\textwidth}
    \begin{itemize}
        \item Opera al Livello di rete (Livello 3) nel modello OSI.
        \item Funziona con IP per fornire feedback sui problemi di rete.
        \item Non stabilisce una sessione (Stateless e Connectionless).
        \item Nessun numero di porta (a differenza di TCP e UDP).
        \item Utilizzato per la risoluzione dei problemi di rete (e.g. esempio, ping, traceroute).
        \item Supporta IPv4 (ICMPv4) e IPv6 (ICMPv6) con funzionalità avanzate in ICMPv6.
    \end{itemize}
\end{minipage}
\begin{minipage}{0.5\textwidth}
    \centering
    \includegraphics[width=0.9\textwidth]{./img/iso_osi-model.jpg}
    \captionof{figure}{Livelli del modello ISO/OSI}
\end{minipage}
\vspace{2ex} 
\subsubsection*{Utilizzi di ICMP}
\begin{itemize}
    \item \textbf{Segnalazione errori}: informa il mittente sui problemi di rete (ad esempio, destinazione non raggiungibile, perdita di pacchetti).
    \item \textbf{Diagnostica di rete}: aiuta nella risoluzione dei problemi di rete utilizzando strumenti come ping e traceroute.
    \item \textbf{Messaggistica di controllo}: gestisce la congestione della rete e gli aggiornamenti di routing in alcuni casi.
\end{itemize}
\subsection*{Struttura di un messaggio ICMP} 
\subsubsection{Analysis of the packets} 
Message Formats
ICMP messages are sent using the basic IP header. 
The first octet of the data portion of the datagram is a ICMP type field; 
the value of this field determines the format of the remaining data. 
Any field labeled "unused" is reserved for later extensions and must be zero when sent, but receivers should not use these fields (except to include them in the checksum). 
Unless otherwise noted under the individual format descriptions, the values of the internet header fields are as follows:
\begin{itemize}
    \item Version 4
    \item IHL Internet header length in 32-bit words.
    \item Type of Service 0
    \item Total Length Length of internet header and data in octets.
    \item Identification, Flags, Fragment Offset Used in fragmentation.
    \item Time to Live
    Time to live in seconds; 
    as this field is decremented at each machine in which the datagram is processed, the value in this field should be at least as great as the number of gateways which this datagram will traverse.
    \item Protocol ICMP = 1
    \item Header Checksum
    The 16 bit one's complement of the one's complement sum of all 16 bit words in the header.  
    For computing the checksum, the checksum field should be zero.
    This checksum may be replaced in the future.
    \item Source Address
    The address of the gateway or host that composes the ICMP message.
    Unless otherwise noted, this can be any of a gateway's addresses.
    \item Destination Address
    The address of the gateway or host to which the message should be sent.
\end{itemize} 
\begin{minipage}{0.5\textwidth}
    Ogni messaggio ICMP è composto da:
    \begin{itemize}
        \item \textbf{Intestazione IP}: primo ottetto di un pacchetto ICMP e determina il valore dei successivi campi
        \item \textbf{Tipo}: Identifica il tipo di messaggio (ad esempio, Echo Request, Destinazione irraggiungibile).
        \item \textbf{Codice}: Fornisce dettagli aggiuntivi sul tipo di messaggio.
        \item \textbf{Checksum}: Garantisce l'integrità dei dati.
        \item \textbf{Dati}: Opzionale, può contenere parte del pacchetto IP originale che ha causato l'errore.
    \end{itemize}
\end{minipage}
\hspace{1ex}
\begin{minipage}{0.5\textwidth}
    \centering
    \includegraphics[width=\textwidth]{./img/ICMP-packet-structure.png}
    \captionof{figure}{Struttura di un pachhetto ICMP/IP} 
    \vspace{2ex} 
    Nel caso si usi il protocolllo ICMP, l'omonimo campo nell'intestazione IP avrà valore 1
\end{minipage} 
I messaggi ICMP sono classificati o come messaggi di errore o come messaggi informativi
\vspace{1ex} 
\begin{itemize}
    \item \textbf{Messaggi di errore} - Segnalano problemi nella comunicazione di rete.
    \item \textbf{Messaggi informativi} - Utilizzati per scopi diagnostici e di controllo.
\end{itemize} 
\begin{minipage}{\linewidth}
    \subsubsection*{Messaggi di errore}
    \begin{tabular}{|p{0.3\linewidth}|p{0.3\linewidth}|p{0.3\linewidth}|p{0.4\linewidth}|}
        \hline 
        \textbf{Type} & \textbf{Code} & \textbf{Meaning} \\
        \hline
        3 & 0-5 & Destination Unreachable (e.g., net, host, port, \dots) \\
        \hline 
        4 & 0 & Source Quench (indica la congestione nella rete) \\
        \hline 
        5 & 0-3 & Redirect Message (suggerisce strada migliore) \\
        \hline 
        11 & 0-1 & Time Exceeded (TTL scaduto, \dots) \\
        \hline 
        12 & 0-1 & Parameter Problem (intestazione IP non valida) \\ 
        \hline 
    \end{tabular}
    \captionof{table}{ICMP messaggi di errore} 
    \subsubsection*{Messaggi di Informazione} 
    \begin{tabular}{|p{0.3\linewidth}|p{0.3\linewidth}|p{0.3\linewidth}|p{0.4\linewidth}|}
        \hline 
        Type & Code & Meaning \\
        \hline 
        8 & 0 & Echo Request Message: invia dati a un destinatario (e.g. ping)\\
        \hline 
        0 & 0 & Echo Reply Message: replica i dati ricevuti mandandoli al mittente (e.g risposta a ping)\\
        \hline 
        13 & 0 & Timestamp Request Message (invia un timestamp a un destinatario)\\
        \hline 
        14 & 0 & Timestamp Request Message (replica i il timestamp ricevuta al mittente)\\
        \hline 
        15 & 0 & information request message (invia un timestamp a un destinatario)\\
        \hline 
        16 & 0 & information reply message (replica i il timestamp ricevuta al mittente)\\
        \hline 
        %9 & 0 & Router Advertisement & Routers announce themselves to hosts. \\
        %\hline
        %10 & 0 & Router Solicitation & Hosts request router advertisements. \\
        %\hline 
    \end{tabular} 
    \captionof{table}{ICMP messaggi di informazione}
\end{minipage} 

%\begin{minipage}{0.5\textwidth}
    
%\end{minipage}