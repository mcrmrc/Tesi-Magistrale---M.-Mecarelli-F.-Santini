\subsubsection{Utilizzi di ICMP nelle reti} 
\begin{enumerate}
    \item Ping (Richiesta Echo ICMP e Risposta Echo) 
    \vspace{1ex} \newline 
    Il comando \textbf{ping}, invia pacchetti ICMP Echo Request per testare la connettività.
    \begin{itemize}
        \item Invia delle richieste Eco ICMP a una destinazione per verificare la connettività.
        \item Se l'host è raggiungibile, risponde con un ICMP Echo Reply.
    \end{itemize}
    \item Traceroute (tracert su Windows, traceroute su Linux/macOS) 
    \vspace{1ex} \newline 
    Il comando \textbf{traceroute}, utilizza messaggi ICMP Time Exceeded per mappare il percorso dei pacchetti.
    \begin{itemize}
        \item Tramite i messaggi ICMP Time Exceeded traccia il percorso che i pacchetti seguono attraverso una rete
        \item Il valore TTL (Time-To-Live) viene incrementato per determinare ciascun router lungo il percorso.
    \end{itemize}
    \item Scoperta del percorso MTU (PMTUD) 
    \vspace{1ex} \newline 
    La \textbf{PMTUD}, utilizza messaggi ICMP Fragmentation Needed per ottimizzare le dimensioni dei pacchetti. 
    Ovvero per trovare la dimensione ottimale del pacchetto per un percorso di rete.
\end{enumerate} 
% ICMP in IPv6 (ICMPv6)
% ICMPv6 extends ICMP functionality for IPv6 networks, including:
%\begin{itemize}
%    \item Neighbor Discovery Protocol (NDP) – Replaces ARP for IPv6 
%    \item Router Advertisements & Solicitation – Helps configure IPv6 addresses. 
%    \item MLD (Multicast Listener Discovery) – Manages multicast group memberships
%\end{itemize}