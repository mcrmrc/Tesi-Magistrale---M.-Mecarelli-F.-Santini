\subsection{Cos'è il protocollo ICMP}
ICMP (Internet Control Message Protocol) è un protocollo a livello rete utilizzato 
per la diagnostica, per la segnalazione di errori, per ottenere informazioni di controllo e per la 
risoluzione dei problemi nelle reti. 
Aiuta i dispositivi (come i router e gli host) a comunicare, gestire e risolvere i 
problemi della rete ma al contrario di TCP o UDP non è utilizzato per la trasmissione di dati.
\vspace{2ex} \newline
Sebbene è essenziale per la diagnostica di rete e la segnalazione degli errori, può essere utilizzato in modo improprio per degli attacchi o per la ricognizione della rete (network reconnaissance). 
%Gli aggressori possono utilizzare ICMP per attacchi DDoS, di ricognizione, di esfiltrazione di dati o di covert channel.
%Le regole del firewall e la limitazione della velocità aiutano a bilanciare usabilità e sicurezza. 
\vspace{2ex} \newline
\begin{minipage}{\linewidth}
    \begin{tabular}{|p{0.25\linewidth}|p{0.25\linewidth}|p{0.25\linewidth}|p{0.25\linewidth}|}
        \hline
        \textbf{} & \textbf{ICMP} & \textbf{TCP} & \textbf{UDP} \\
        \hline \hline
        \textbf{Scopo} & Segnalazione di errori e diagnostica & Trasferimento di dati affidabile & Trasferimento di dati veloce e senza connessione \\
        \hline
        \textbf{Orientato alla connessione?} & No & Sì & No \\
        \hline
        \textbf{Numero di porta?} & No & Sì & Sì \\
        \hline
        \textbf{Affidabilità} & No & Sì (Acknowledgments) & No \\
        \hline
        \textbf{Utilizzato per} & Ping, Traceroute, PMTUD & HTTP, FTP, Email & DNS, VoIP, Streaming \\
        \hline
    \end{tabular}
    \captionof{table}{Differenze tra ICMP, TCP e UDP}
\end{minipage}
