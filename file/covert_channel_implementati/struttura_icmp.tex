\subsection{Struttura di un pacchetto ICMP} 
I messaggi ICMP vengono inviati utilizzando l'intestazione IP di base. 
In essi i primi venti byte indicano l'intestazione IP mentre il primo ottetto, della porzione dati del datagramma, riguarda l'intestazione ICMP. 
I campi relativa al protocollo ICMP sono i seguenti:
\begin{itemize} 
    \item \textbf{Tipo}: Identifica il tipo di messaggio (ad esempio, Echo Request, Destinazione irraggiungibile). 
    In base al valore di questo campo verrà determinato il formato dei rimanenti dati. 
    \item \textbf{Codice}: Fornisce dettagli aggiuntivi sul tipo di messaggio.
    \item \textbf{Checksum}: complemento a 16 bit relativo alla somma del messaggio ICMP. Garantisce l'integrità dei dati. 
    \item \textbf{Dati}: Campo opzionale, può contenere la parte del pacchetto IP originale che ha causato il messaggio o altre tipologie di dati. 
\end{itemize} 
\begin{minipage}{\textwidth}
    \centering
    \includegraphics[width=0.7\textwidth]{./img/ICMP-packet-structure.png}
    \captionof{figure}{Struttura pacchetto ICMPv4/IPv4} 
\end{minipage} 
\vspace{1ex} \newline
Nell'intestazione, nel caso si usi il protocolllo ICMP, il campo \textit{protocol} avrà valore 1
%Qualsiasi campo etichettato come "non utilizzato" è riservato per future estensioni e deve essere zero quando 
%inviato, ma i destinatari non dovrebbero utilizzare questi campi (eccetto per includerli nel checksum). 

%\subsection*{Tipologie di Messaggi ICMP}
I messaggi presenti in ICMP sono classificati o come messaggi di errore o come messaggi informativi. 
I primi segnalano problemi nella comunicazione di rete 
%(Destination Unreachable, source Quench, Redirect Message,Time Excedeed, Parameter Problem, Packet too Big) 
mentre i secondi vengono utilizzati per scopi diagnostici e di controllo. 
%(Echo Request, Timestamp Request, information Request) 
%Di seguito la struttura delle varie tipolgie. 

%\import{./covert_channel_implementati/icmp_packet}{tipologia_msg_ICMP} 