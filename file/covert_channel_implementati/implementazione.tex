\subsection{Implementazione}  
In un Covert Channel ICMP (Internet Control Message Protocol) verranno utilizzati messaggi ICMP 
%(di solito richieste e risposte Eco) 
per nascondere 
i dati all'interno dei campi che normalmente vengono ignorati o non monitorati. 
%Il canale sarà possibile siccome il protocollo ICMP consente agli attaccanti di trasferire dati 
%aggirando le politiche di sicurezza ed eludendo il rilevamento.
L'implementazione del canale è possibile siccome è un protocollo che, dati i suoi utilizzi, non può essere del tutto disabilitato.
In sinergia con il protocollo IP, permette di informare il mittente 
sui problemi di rete, di aiutare nella risoluzione dei problemi di rete e di gestire la congestione della rete. %oltre gli aggiornamenti di routing (in alcuni casi)
%Opera a livello rete (livello 3 modello ISO/OSI). 
%ICMP provides support for the IP protocol by operating at the network layer. 
%The protocol has a connectionless structure and does not require a connection to the target device before sending a message 
%ICMP protocol stands out as an indispensable component for the healthy functioning of modern network infrastructure

%For network administrators and system experts, the comprehensive diagnostic and monitoring capabilities offered by the ICMP 
%protocol are essential for the rapid detection and resolution of problems. When configured in accordance with security policies, 
%ICMP will continue to be an indispensable part of modern network infrastructures. The detailed error reports and performance metrics 
%provided by the protocol increase system reliability by supporting proactive approaches to network management.
\begin{center} 
\begin{longtable}{|p{0.4\textwidth}|p{0.4\textwidth}|} 
    \hline
    \textbf{Utilizzi} & \textbf{Descrizione} \\
    \hline
    Diagnostica della rete & 
    %Utilizzando strumenti come ping e traceroute 
    %Crea l'infrastruttura per strumenti diagnostici di rete di base come ping e traceroute 
    Il protocollo fornisce metriche critiche per il monitoraggio continuo delle prestazioni della rete 
    %(es tempo di risposta, pacchettiu persi, banda utilizzata, \dots)
    %Response times: Measurement of network latencies
    %Packet Losses Detection of transmission failures
    %Bandwidth Utilisation: Analysing network traffic
    %Availability Status: Control of the operating status of the systems 
    %
    %Traceroute: Used to determine the path packets take across routers to reach the destination. 
    %Ping: Sends echo-request and echo-reply messages to measure round-trip time and test connectivity.
    \\
    \hline 
    Segnalazione di errori & 
    %ad esempio, destinazione non raggiungibile, perdita di pacchetti 
    %Rileva e segnala errori di comunicazione tra dispositivi sulla rete 
    Il protocollo rileva e segnala i problemi riscontrati durante la trasmissione dei dati tra i dispositivi sulla rete
    %(Destination Unreachable Notifications, TTL Overrun Messages, Parameter Problems, Parameter Problems)
    %Destination Unreachable Notifications: When packets fail to reach their destination
    %TTL Overrun Messages: When the lifetime of the packages expires
    %Parameter Problems: When an error is detected in the IP header
    %Source Suppression: In cases of network congestion
    %
    %Error reporting: When data packets cannot reach their destination due to issues such as unreachable hosts, timeouts or fragmentation errors. 
    %If a message cannot be delivered, ICMP informs the source about the failure. 
    %Example: If a packet is too large and cannot be forwarded, the receiver drops the packet and sends an ICMP error message to the sender.
    \\
    \hline
    Risoluzione dei problemi & 
    Gli amministratori di rete possono ricevere avvisi in tempo reale e rispondere rapidamente ai problemi di rete grazie agli strumenti di monitoraggio basati su ICMP.
    %Questi strumenti consentono una gestione proattiva della rete fornendo notifiche istantanee sullo stato delle applicazioni e dei servizi web critici. 
    \\
    \hline
    Ottenimento di informazioni & 
    Può inviare messaggi senza la necessità di una connessione preventiva permettendo così di ottenere informazioni. %tramite la messagistica di controllo 
    \\
    \hline
    %detect and report errors on the network &  The protocol checks whether the data reaches the destination on time and transmits information to the sending device in case of any problems \\
    %
\caption{Utilizzi del protocollo ICMP} 
\label{tabella:utilizzi:ICMP} 
\end{longtable} 
\end{center} 
%\vspace{2ex} \newline 
%Molti firewall e dispositivi di sicurezza consentono il traffico ICMP per la diagnostica della rete.
%I pacchetti ICMP possono trasportare dati (payload) nascosti senza destare sospetti.
%I sistemi di sicurezza tradizionali si concentrano sul traffico TCP/UDP, trascurando ICMP.
%Sebbene sia essenziale per la diagnostica di rete e la segnalazione degli errori, può essere comunque utilizzato in 
%modo improprio per degli attacchi o per la ricognizione della rete (network reconnaissance). 
%Gli aggressori possono utilizzare ICMP per attacchi DDoS, di ricognizione, di esfiltrazione di dati o di covert channel.
%Le regole del firewall e la limitazione della velocità aiutano a bilanciare usabilità e sicurezza. 
%Data la sua necessita per la diagnostica di rete e la segnalazione degli errori, 
%può essere utilizzato in modo improprio per mettere a segno degli attacchi o per studiare la rete (ricognizione della rete). %network reconnaissance 
%Siccome ICMP è spesso consentito nei firewall, gli aggressori lo utilizzano per aggirare le politiche di sicurezza. 
