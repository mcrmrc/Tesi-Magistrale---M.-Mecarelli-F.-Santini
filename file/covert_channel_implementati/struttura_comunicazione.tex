La struttura del programma è definita nel seguente modo: 
sono presenti tre entita, che verranno definite come \textbf{attaccante}, \textbf{proxy} e \textbf{vittima}. 
Il caso standard sarà quello rappresentato in [Fig.\ref{fig:covertchannel:struttura:1}]; 
tuttavià sarà possibile, per l'attaccante, comunicare direttamente con la vittima. 
In questo caso l'entità proxy non verrà utilizzata ed alcune funzioni presenti in essa verranno eseguite 
dall'attaccante (e.g connettersi alla vittima). 
\begin{center} 
    \includegraphics[width=0.4\textwidth]{../img/implementazione/struttura attacker.proxy.victim 3.jpg} 
    \captionof{figure}{Diagramma delle entità presenti}
    \label{fig:covertchannel:struttura:1} 
\end{center}   
Come avviene la comunicazione fra le entità è definito dal diagramma in [Fig.\ref{fig:covertchannel:struttura:2}]. 
Le entità inizializzano i parametri necessari, in particolare l'attaccante leggerà un file di configurazione in cui si indicano determinate cose. 
In particolare si indicano i proxy se si vorrà utilizzare per ò'attacco, l'indirizzo IP della macchina vittima 
oltre alla tipologia di attacco che si vuole effettuare. 
Dopodichè l'attaccante si connetterà ad ogni proxy specificato nel file per testare se la connesisone è possibile. 
La connesisone viene ritenuta possibile se il proxy e l'attaccante riescono a comunicare fra di loro e se anche il proxy e la vittima riescono a comunicare fra di loro. 
Identificati i proxy sfruttabili, si inserirà un comando e, dopo averlo inviato ad uno dei proxy, si aspetteranno che i proxy ritornino i dati relativi all'operazione. 
Per l'invio del comando si utilizzerà un singolo proxy; 
quindi a tutti gli altri verrà notificato di aspettare direttamente i dati. 
Infine aspetta che tutti gli altraccanti e poi la analizza, per poterli riordinare. 
%finto potrà decidere se inviare un ulteriore comando o terminare la comunicazione. 
\vspace{2ex} \newline 
Il proxy invece, al momento dell'inizializzazione, imposta un server su cui l'attaccante si connetterà. 
%Si potrebbe impostare una comunicazione tramite ICMP (come nel caso con la vittima) ma questo renderebbe la comunicaizone instabile. 
%Invece, siccome il proxy può ritenersi sicuro, un canale TCP/IP può essere utilizzato. 
Successivamente cercherà di connettersi alla vittima inviandogli un messaggio ICMP e poi aggiorna l'attaccante se la connesisone è stata stabilita o no. 
Dopodichè rimarrà in attesa del comando che l'attaccante potrà inviare. 
Se lo riceve provvederà ad inoltrarlo altrimenti rimane in attesa dei dati dalla vittima. 
Una volta ricevuti tutti i dati dalla vittima, procederà ad inoltrarli all'attaccante. 
Infine rimarra in attesa di aggiornamenti da parte dell'attaccante, 
il quale può decidere se inviare un altro comando o termianre la comunicazione. 
\vspace{2ex} \newline 
La vittima, durante la sua inizializzazione, specifica il numero minimo di proxy necessari per la comunicazione. 
Quindi si metterà in ascolto dei proxy che vogliono connettersi; 
quando si connette ad un proxy, gli manderà un messaggio di conferma. 
Se entro un certo tempo il numero non viene raggiunto, si può scegliere se continuare ad aspettare altri proxy o avanzare con l'attacco. 
Dopodichè la vittima rimane in attesa del comando dell'attaccante e una volta arrivato, lo esegue e ricava i dati restituiti in caso dall'operazione. 
Infine, per esfiltare i dati, invierà a ciascuno dei proxy connessi una porzione delle informazioni ricavate. 
Terminato di farlo, si metterà in ascolto del prossimo comando dell'attaccante, che determinerà la prossima operazione o la terminazione della comunicazione. 
\begin{center} 
    \includegraphics[width=0.5\textwidth]{../img/implementazione/diagramma_struttura_entita.jpg} 
    \captionof{figure}{Diagramma del dlusso di comunicazione fra le entita}
    \label{fig:covertchannel:struttura:2} 
\end{center} 


