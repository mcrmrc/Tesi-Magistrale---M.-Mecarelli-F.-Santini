Le entita coinvolte sono l'attaccante, il proxy e la vittima. 
\begin{center} 
\begin{longtable}{|p{0.25\textwidth}|p{0.55\textwidth}|} 
    \hline
    \textbf{Entità} & \textbf{Descrizione} \\
    \hline
    Attaccante &  
    %Il primo avrà bisogno di 
    Carica un file di configurazione nel quale viene definito l'indirizzo IP della vittima, 
    il metodo di attacco e i proxy utilizzabili per farlo. 
    %e il comando da eseguire. 
    Inoltre inserirà il comando che la vittima dovrà eseguire o il dato che gli si vuole mandare.  
    %dovo averlo inoltrato, tramite i proxy, aspetta i dati relativi ad esso. 
    Nel caso l'attaccante utilizzasse dei proxy, aspettera da essi i dati che la vittima ha restituito. 
    \\ 
    \hline 
    Proxy &  
    Ha bisongo di sapere qual'è l'indirizzo IP dell'attaccante. 
    %Questo perchè 
    Deve potersi connettere ad esso ed ottenere l'indirizzo IP della vittima e il comando da inoltrare. 
    Una volta stabilita una connesisone sia con l'attaccante che con la vittima; 
    si occuperà di inoltrare i dati che le due entità si inviano. 
    \\
    \hline 
    Vittima &  
    Aspetta le connessioni o dall'attaccante o dai proxy se vengonoutilizzati. 
    Dopodichè aspetterà un comando (o un dato). 
    Nel primo caso lo eseguirà e invierà i dati ricavati dall'esecuzione. 
    \\
    \hline  
\caption{Entita presenti nell'attacco} 
\label{tabella:attacco:entitaPresenti} 
\end{longtable} 
\end{center} 
\vspace{2ex}  
Il flusso di come avviene la comunicazione fra le entità è definito dal diagramma seguente. 
Il canale di comunicazione che il proxy e l'attaccante protrano avere dipende dall'affidabilità del proxy. 
Nel caso si reputi il proxy insicuro, si potrà utilizare una comunicazione tramite ICMP (ovvero la stessa che il proxy avrà con la vittima) 
altrimenti si potra usare il protocollo TCP; che permetterà una comunicazione maggiormente stabile. 
\begin{center} 
    \includegraphics[width=0.9\textwidth]{../img/implementazione/diagramma_struttura_entita.jpg} 
    \captionof{figure}{Diagramma del flusso di comunicazione fra le entita}
    \label{fig:covertchannel:struttura:2} 
\end{center}  
Per la comunicazione l'attaccante potrà usare uno o più proxy per comunicare con la vittima. 
Il caso standard sarà quello rappresentato nella figura tuttavià sarà possibile, per l'attaccante, comunicare direttamente con la vittima. 
In questo caso l'entità proxy non verrà utilizzata ed alcune funzioni presenti in essa verranno eseguite 
dall'attaccante (e.g connettersi alla vittima) e quindi l'entità proxy potrà essere vista come l'entità attaccante. 
\begin{center} 
    \includegraphics[width=0.7\textwidth]{../img/implementazione/struttura attacker.proxy.victim 2.jpg} 
    \captionof{figure}{Struttura delle entità presenti e come dialogano}
    \label{fig:covertchannel:struttura:1} 
\end{center} 
%\begin{center} 
%    \includegraphics[width=0.4\textwidth]{../img/implementazione/struttura attacker.proxy.victim 3.jpg} 
%    \captionof{figure}{Diagramma delle entità presenti}
%    \label{fig:covertchannel:struttura:1} 
%\end{center} 

\subsection{Struttura dell'attaccante} 
%Quando si esegue il programma, tramite linea di comando, devono essere passati degli argomenti; 
%in particolare verrà utilizzato 
Tramite un parser rilevare se nella linea di comando, è stato inserito un path per il file di configuraizone 
da caricare. 
In questo file JSON viene specificato l'indirizzo IP della vittima, la metodologia di attacco e 
la lista dei proxy a cui ci si vuole connettere. 
%Un ulteriore dato che si ricaverà è l'indirizzo IP dell'host stesso. 
\vspace{1ex} \newline 
Dopodichè per ogni proxy specificato 
%ci si connette ai proxy definiti nella lista precedente per 
verificare quali sono disponibili; 
un proxy si può ritenere disponibile se è connesso sia all'attaccante che alla vititma. 
%Quindi si creeerà un socket per ogni proxy connettendosi all'indirizzo e alla porta in cui si sono messi 
%in ascolto e tramite esso, si invia un messaggio di conferma. 
Quindi si connette al socket in cui il proxy è in ascolto e gli invierà un messaggio in cui specifica
%Questo messaggio specificherà 
l'indirizzo IP della vittima e la metodologia di attacco scelta.  
Successivamente rimarrà inattesa che il proxy confermi la connessione con la vittima. 
%Il contentuto sarà l'indirizzo IP della vittima ricevuto oltre che all'indirizzo IP del proxy stesso. 
%Infine rimane in attesa che il proxy invii un aggiornamento sul suo stato di coneessione con la vittima. 
%Se il risultato delle seguenti operazioni risulterà positivo, il proxy verrà considerato disponibile altrimenti verrà consdiderato 
%non disponibile e di conseguenza scartato (questo significherà che verrà rimosso dalla lista dei proxy connessi). 
%
%WAIT PROXY UPDATE:
%Eseguito su ogni thread. 
%Tra il proxy e l'attaccante ci può essere un canale TCP/IP oppure ICMP. 
%Aspetta di ricevere dal proxy un messaggio di conferma; 
%se la conferma aspettata non combacia co nquiella ricevuta il proxy viene scartato e il socjket chiuso. 
%Nel messaggio di conferma il proxy indica se è connesso alla vititma. 
\vspace{2ex} \newline 
Prima di inviare il comando, l'attaccante definirà un thread per ogni proxy. 
Il codice eseguito nei thread si occuperà di ricevere i dati che il proxy inoltrerà. 
Dopodichè si inserirà e si invierà il comando alla vittima. 
%Ai restanti proxy verrà notificato di ascoltare direttamente i dati. 
%Dopodichè si aspetta che i thread ricevano i dati e 
Una volta ricevuti tutti i dati, inoltrati dai proxy, li si riordina così da ricavare il messaggio. 
Finito ciò si reimposteranno le variabili necessarie e se voluto il ciclo continua 
(tramite l'inserimento di un altro comando). 
Altrimenti si può decidere di interrompere la comunicazione e in quel caso i proxy ne verranno aggiornati. 
%
%WAIT DATA FROM PROXY 
%1- Ricava i lsocket associato al proxy 
%2- Aspetta finchè non riceve dei dati; aquel punto aggiunge i dati ricevuti alla lista di quelli già ricevuti.
%quando riceve l'ultimo messaggio smette di aspettare. 
%
%RESTET VARIABLES 
%Le variabili che devono essere resettate sono: 
%-gli eventi associati ai proxy: il loro valore viene reimpostato 
%-il dizionario contenente i dati viene ricreato con le liste vuote 
%-i thread vengono ridefiniti e il dizionario che li contiene aggiornaot. 
%
%file:///D:/Tesi%20Magistrale/main%20v0.1.pdf pag 81 definizione variabili 


\subsection{Struttura del Proxy} 
Il proxy come argomento richiede l'indirizzo IP dell'attaccante; 
e dopo averlo ricavato tramite il parser, ricaverà il proprio indirizzo IP. 
Dopodichè definisce un socket in cui rimarra in ascolto per la connessione dell'attaccante. 
Per esserne sicuro confronta l'indirizzo di chi si è connesso con quello passato e, in aggiunta, controlla se il messaggio ricevuto successivamente 
corrisponde ad un messaggio di conferma. Nel caso non fosse il socket viene chiuso e si ritorna in ascolto; 
altrimenti dal messaggio si ricava l'indirizzo Ip della vittima e la metodologia di attacco. 
Dopodichè, tramite il socket definito, invierà all'attaccante il messagigo di conferma della connessione. 
In questo messaggio indicherà l'indirizzo della vititma ricevuto oltre al proprio indirizzo IP. 
%Le principali variabili utilizzate sono: 
%ip\_attaccante: che indica l'indirzzo IP dell'attaccante
%ip\_host: che indica l'indirzzo IP del host
%ip\_vittima: che indica l'indirzzo IP della vittima.  
%attack\_function: indica la tipologia di attacco  
\vspace{1ex} \newline 
\textbf{NB} nel caso il proxy e l'attaccante combaciassero, in questo caso non verrà definito alcun socket ma, come per l'attaccante, 
si ricaveranno i dati necessari dal file di configurazione. 
\vspace{3ex} \newline 
Per la connessione con la vittima, il proxy comunicherà tramite pacchetti ICMP; 
quindi, prima di inviare alcun dato, imposta un thread il cui scopo è analizzare il traffico e catturare il pacchetto che 
la vittima manderà per confermare la connesisone. Se questo viene fatto dopo, il pacchetto potrebbe andare e perso. 
Dopo aver fatto partire il thread, si procederà codificando l'attacco scelta nel campo identifier del messaggio ICMP Echo Request 
mentre nel payload verrà ionserito il messaggio che richiede la connessione.  
Infine si aspetta che il thread termini e ritorni lo stato della connessione con la vititma. 
Dopodichè si procede ad aggiornare l'attaccante sul risultato della connesisone che si ha con la vittima e, 
nel caso il risultato sia negativo il programma viene terminato (siccome non potrà essere utilizzato per l'inoltro delle informazioni).  
Se invece si è stabilita una connessione con la vittima, si rimarra in attesa di messaggi da aprte dell'attaccante. 
\vspace{3ex} \newline 
A questo punto i messaggi che un proxy può ricevere dall'attacante sono tre: 
nel primo caso, il pacchetto indicherà il comando che il proxy dovrà inoltrare alla vittima. 
La seconda tipologia di messaggio invece indica al proxy che non ha ricevuto il comando e 
che quindi può iniziare subito ad aspettare i dati dalla vittima. 
L'ultimo invece indica la volonta, da parte dell'attaccante, di terminare la comunicazione. 
In questo caso il proxy aggiornerà la vittima della cosa. 
\vspace{1ex} \newline
Siccome lo scambio di messaggi avviene fra l'attaccante e il proxy, l'invio e la ricezione avverrà tramite 
il socket definito all'inizio della comunicazione. 
Tuttavia questo non varrà per l'inoltro del comando; infatti la comunicazione avverrà fra la vittima e 
il proxy e per questo si invieranno (e riceveranno) i dati in base alla tipologia di attacco definita.  
Inoltre non essendo presente alcun tipo di socket fra le due entità, 
il thread che si occuperà di ricevere i dati dovrà partire prima di inviare il comando.  
\vspace{1ex} \newline
Infine quando il thread avrà ricevuto i dati dalla vittima, si procederà ad inoltrarli all'attaccante così come li si è ricevuti.  
%
%ASPETTA IL COMANDO 
%1- Rimane in ascolto tramite il socket finchè non riceve un messaggio. 
%Una volta ricevuti rileva se il dato indica la volontà di chiudere la comunicazione. 
%2- Se si chiude la comunicazione si notifica la vittima della cosa sennò si continuerà normalmente. 
%3- Nel caso si continuasse, se definisce un thread che ha lo scopo di aspettare i dati che la vittima manda e poi 
%il proxy controlla se ha ricevuto il comando. 
%Se così non fosse rimane in attesa dei dati della vittima. 
%altrimenti invia il ocmando alla vititma. 
%4- Una volta ricevuti i dati dalla vittima li inoltra all'attaccante 
%5- Infine rimaniamo in attesa di un altro comando 
\vspace{1ex} \newline
\textbf{NB} nel caso il proxy e l'attaccante combaciassero, in questo caso non si aspetterà alcun comando lo lo si richiederà in input. 
Inoltre i dati non verranno inoltrati. 
%Connessione con la vittima
%La connessione con la vittima avviene in questo modo:
%1. Si imposta un thread che si occuperà di ricevere i dati mandati dalla vittima e lo si fà partire.
%2. Successivamente si codifica la modalità di attacco così da poterla inserire nel campo 'identifier' del pacchetto ICMP che si manderà.
%3.Dopodichè si invia un pacchetto tramite il protocollo ICMP alla vittima.
%4. Successivamente si aspetta che la vittima confermi la connessione.
%5. Fatto questo si aggiornerà l'attaccante se il proxy è connesso alla vittima.  


\subsection{Struttura Vittima} 
Quando si eseguirà il programma, dovranno essere definitiil numero di proxy necessarri. 
Ciò indicherà il numero minimo di proxy necessari che serviranno per l'esecuzione dell'attacco. 
Successivamente si ricaverà il prorpio indirizzo IP e si andrà in attessa dei proxy che si vorranno connettere. 
\vspace{1ex} \newline
Questo viene fatto rimanendo in attessa e monitornado il flusso di rete, filtrando i messaggi ICMP destinati alla vittima e 
al cui interno sia presente un messaggio di richeista di connessione. 
Se viene ricavato un messaggio del genere, si risponde con un messaggio di conferma della connessione e l'indirizzo IP del mittente viene inserito nella lista dei proxy connesisone. 
Da questo messaggio la vittima ricaverà la metodologia di attacco scelta. 
La vititma smetterà di monitorare il traffico finche o non verrà raggiunto il nuemro minimo di proxy o finche il tmier non scade. 
In ogni caso la lista dei proxy connessi verrà controllata per verificare se ci sia alemo un proxy connesso. 
Se così non fosse (e il numero risulti zero) termina immediatamente la connessione altrimenti se il numero è inferiore a quello richiesto, chiese se si vuole procedere ciò nonostante.
\vspace{1ex} \newline 
Avendo la tipologia di attacco con cui il comando verrà inviato, la vittima si metterà in ascolto; 
in attesa che uno dei proxy lo mandi. 
Una volta ricevuto, aprirà una shell dove poterlo eseguire e ne ricaverà i dati. 
I dati ricavati non saranno solo quelli legato all'output ma anche quelli legati ad eventuali errori che possono essere accaduti durante l'esecuzione. 
Questi dati saranno poi inoltrati ai proxy. 
\vspace{1ex} \newline 
Per ogni porxy connesso, si definirà una lista indicante i dati che dovrà ricevere. 
Il modo in cui verranno distribuiti è sequenziale. 
Un esmepio è quando, in un gioco di carte, si distribuisce a tutti i giocatori una singola carta fino ad esaurimento del mazzo, 
piuttosto che distribuire ad ogni giro X carte a ciascun giocatore. 
Quindi il primo proxy riceverà la prima porzione, il seocndo proxy la seconda e così via fino a quando non si è ritornati al 
proxy di partenza e i dati da mandare non siano terminati. 
L'ulyimo messaggio che verrà inviato ai proxy sarà quello in cui si indica che tutti i dati sono stati mandati. 
%
%4- Eeguito il comando e ricavati i dati, li manderà ai proxy. 
%Per ognuno di loro definirà una lista che conterrà i dati che dovrà ricevere.  
%I dati che un proxy deve mandare, non verrnno mandati immediatamente; 
%ma si ciclerà in modo circolare fra tutti i proxy vinchè tutti hanno mandato i dati. 
%Dopodichè si invierà un messaggio a tutti i proxy per indicare che tutti i dati sono stati mandati.  





%Per l'invio del comando si utilizzerà un singolo proxy; 
%quindi a tutti gli altri verrà notificato di aspettare direttamente i dati. 

%Il proxy invece, al momento dell'inizializzazione, imposta un server su cui l'attaccante si connetterà. 
%Si potrebbe impostare una comunicazione tramite ICMP (come nel caso con la vittima) ma questo renderebbe la comunicaizone instabile. 
%Invece, siccome il proxy può ritenersi sicuro, un canale TCP/IP può essere utilizzato. 

%Infine, per esfiltare i dati, invierà a ciascuno dei proxy connessi una porzione delle informazioni ricavate. 
%Terminato di farlo, si metterà in ascolto del prossimo comando dell'attaccante, che determinerà la prossima operazione o la 
%terminazione della comunicazione. 

