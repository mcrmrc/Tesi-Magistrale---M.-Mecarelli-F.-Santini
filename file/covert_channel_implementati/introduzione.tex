Un \textbf{Covert Channel} è un attacco che permette (in ambienti ritenuti sicuri) la capacità di comunicare e/o trasferire dati in maniera non autorizzata e non voluta.  
%Ciò avverrà fra processi e/o entità comunicanti senza che vengano rivelati ed evitando (se non violando) le normali politiche di sicurezza. 
Solitamente operano al di fuori degli usuali meccanismi di comunicazioni sfruttando vulnerabilità o comportamenti non previsti nei sistemi. 
%Non usando i normali protocolli e/o canali di comunicazione (es network sockets, emails) 
%ciò gli permette di non generare segnali di un uso improprio del sistema. 
Ciò gli permette di non generare segnali di un uso improprio del sistema ed inoltre, 
nascondendosi all'interno dei normali processi del sistema, sono difficili da rilevare e/o identificare. 
\vspace{1ex} \newline  
%Da notare inoltre che 
Qualsiasi risorsa condivisa può essere utilizzata come canale nascosto. 
Questo permetterà ai Covert Channel di esistere in qualsiasi sistema. 
%In aggiunta, qualsiasi risorsa condivisa può essere utilizzata per la creazione di un canale nascosto, 
%e di conseguenza possono esistere in qualunque sistema. 
E per questo la loro esistenza rappresenta un problema che spesso rimane non notato. %dagli amministratori o dai tipici strumenti di monitoraggio. 
%Questi attacchi sono un problema significativo in tutti quegli ambienti dove una fuoriuscita di informazioni può avere conseguenze gravi (es ambienti militari, governativi,\dots).  
%Siccome l'suo imporprio che si fà di queste risorse: porta alla fuoriuscita (o scambio) dei dati. 
\vspace{2ex} \newline  
%\textbf{Stealthiness} (furtività)
%Siccome un covert channel deve poter aggirare i controlli in maniera nascosta dovrà avere alcune caratteristiche. 
%Tuttavia, uno dei maggiori problemi nell'implementazione di un canale nascosto è l'uso eccessivo delle risorse. 
%La necessità cardine, 
Un Covert Channel possiede determinate caratterisitche. 
L'\textbf{indistinguibilità} è la principale; è estremamente importante riuscire a trasmettere informazioni mantenendo conforme lo stato del sistema. 
L'obbiettivo è rendere il canale indistinguibile rispetto alle altre risorse presenti nel sistema così da risultare invisibili ai sistemi di monitoraggio. 
%Fra queste, la proprietà maggiormente importante è quella di riuscire a trasmettere informazioni mantenendo conforme lo stato del sistema; 
%così da rendere il canale \textbf{indistinguibile} rispetto alle altre risorse presenti nel sistema e di conseguenza invisibili ai sistemi di monitoraggio. 
%dalla risorsa sfruttata e di conseguenza invisibili ai sistemi di monitoraggio. 
%In generale il canale è incorporato in operazioni di sistema legittime per poter mascherare la trasmissione dei dati. 
%(e.g carico della CPU, accesso alla memoria, traffico della rete, metadati del file systema). 
%
%    Di solilto si sfruttano servizi e/o risorse già presenti e quindi non sospette; 
%    uno dei maggiori problemi nell'implementazione di un canale nascosto è il “rumore” che potrebbe attirare l'attenzione 
%    da parte degli amministratori (es. se si sfruttano eccessivamente le risorse). 
%    La necessità è quella di riuscire a trasmette attraverso un canale nascosto mantenendo conforme e inalterato il 
%    funzionamento della risorsa utilizzata così da rendersi “indistinguibili” dalla risorsa autorizzata e di 
%    conseguenza invisibili ai sistemi di monitoraggio.
%    Per evitare la rilevazione, il canale è incorporato in operazioni di sistema legittime per poter mascherare la trasmissione dei dati. 
%    (e.g carico della CPU, accesso alla memoria, traffico della rete, metadati del file systema). 
%\vspace{2ex} \newline
%Per evitare la presenza dei Covert Channel, bisogna quindi prestare attenzione all'\textbf{uso involontario delle risorse} 
%ed evitare che esse possano essere usate (in maniere non previste) per la comunicazione. %(e.g memoria condivisa, uso della CPU, attributi dei file)
%Questo perchè, permetterebbero lo scambio non non autorizzato di informazioni; 
%potenzialmente violando le politiche di sicurezza; 
%nelle quali i principali requisiti sono: la confidenzialità dei dati, l'integrità di quest'ultimi o la disponibilità delle risorse e/o servizi. 
%Di conseguenza un metodo per la \textbf{rilevazione delle violazioni} delle politiche di sicurezza, risulta necessaria. 
\vspace{1ex} \newline
\begin{center} 
\begin{longtable}{|p{0.4\textwidth}|p{0.4\textwidth}|} 
    \hline
    \textbf{Caratteristica} & \textbf{Descrizione} \\
    \hline
    \textbf{Furtività} & 
        Evitare di attirare le attenzioni sia degli amministratori che degli strumenti utilizzati per il rilevamento degli attacchi. 
        %Si devono poter aggirare i controlli in maniera nascosta.
    \\
    \hline 
    \textbf{Capacità di trasmissione} &
        Espressa in termini di \textbf{throughput} ($\frac{dati}{tempo}$). 
        Più dati il canale trasmette per un determinato intervallo di tempo, maggiore sarà il rischio che venga scoperto 
        siccome potrebbe rendere anomalo il funzionamento delle altre risorse.
        %e un valore maggiore permetterà di scambiare un maggior numero di informazioni. 
        %Tuttavia se la trasmissione risultasse anomala o eccessiva, potrebbe essere rilevata siccome un eccessivo carico di informazioni, 
        %potrebbe rendere anomalo il funzionamento delle risorse.
        %Quindi se la lunghezza di banda (Bandwith) risultasse limitata;  
        %Quindi il throughput è inversamente correlato alla segretezza di un canale: 
        %più dati un canale trasmette in un determinato periodo di tempo, maggiore è il rischio che il canale venga scoperto
    \\
    \hline
    \textbf{Uso delle risorse} &
        Un uso delle risorse improprio o sproporzionato aumenta il rischio di essere individuati. 
        Siccome il canale potrebbe andare in conflitto con le risorse legittime presenti nel sistema. 
        %I Covert channels sfruttano le risorse del sistema (e.g memoria cxondivisa, uso della CPU, attributi dei file) 
        %in maniere non previste per la comunicazione. 
    \\
    \hline
    \textbf{Rumore} &
        Sfruttando servizi e/o risorse già presenti nel sistema, si potrebbe alterare il loro funzionamento. 
        L'alterazione del comportamento della risorsa o del servizio sfruttato potrebbe attirare l'attenzione da parte degli amministratori. 
    \\
    \hline
    \textbf{Indistinguibilità} & 
        Si ha la necessità di riuscire a trasmette i dati sempre mantenendo conforme e inalterato il 
        funzionamento della risorsa utilizzata. 
        L'obbietivo è quello di rendersi indistinguibili dalla risorsa autorizzata e di conseguenza invisibili ai sistemi di monitoraggio.
    \\ 
    \hline
    %Violazione delle politiche di sicurezza &
    %    Permettono lo scambio non autorizzato di informazioni, potenzialmente violando 
    %    i requisiti di confidenzialità, di integrità o quelli di disponibilità.  
    %\\
    %\hline 
\caption{Caratterisitche di un Covert Channel} 
\label{tabella:caratteristiche:covert:channel} 
\end{longtable} 
\end{center} 
%\vspace{2ex} %\newline
Nei seguenti capitoli verrà illustrato come è stato implementato.  
%come può essere sfruttato per la creazione di un Covert Channel.   
%un attacco che permette (in ambienti ritenuti sicuri) la capacità di comunicare e/o trasferire dati in maniera non autorizzata e non voluta. 
%L'attacco opera al di fuori degli usuali meccanismi di comunicazioni %Inoltre sfruttando vulnerabilità o comportamenti non previsti nei sistemi,  
%e per questo risulta difficile da rilevare e/o identificare. 
%Sia dagli amministratori che dai tipici strumenti di monitoraggio. 
%Infine, siccome qualsiasi risorsa condivisa può essere utilizzata per la sua creazione, può esistere in qualunque sistema. 
Principalmente le risorse condivise, e manipolate, saranno i pacchetti ICMP e i dati verranno codificati all'interno dei campi presenti. 
%I messaggi verranno poi usati per creare una comunicazione fra una vittima e il suo attaccante, 
%in cui potrebbero essere presenti in caso dei proxy intermedi. 
%Il mittente manipola una risorsa di sistema condivisa (osservabile dal destinatario) per codificare i dati segreti (eg tramite il tempo o la memoria).  
%\textbf{Codifica dei dati}: gli aggressori incorporano messaggi nascosti all'interno di pacchetti ICMP, come richieste di Eco (ping) o risposte di Eco.
%\textbf{Evasione del firewall}: poiché ICMP è spesso consentito nei firewall, gli aggressori lo utilizzano per aggirare le politiche di sicurezza.
%\textbf{Comunicazione furtiva}: malware e botnet utilizzano poi ICMP per comunicare segretamente con un attaccante remoto.
Altre metodologie sfrutteranno gli intervalli di tempo fra un pacchetto e il successivo
oppure la tipologia di messaggio ICMP ricevuto. 
%Nel primo caso in base alla distanze di tempo fra un pacchetto e l'altro, viene codificato l'informazione. 
%Nel secondo caso la codifica dei dati verrà associata alla particolare tipologia di messaggio ricevuto. 


%Nelle prossime pagine, si analizzeranno le varie tipologie di messaggi 
%(divise in \textbf{Messaggi di errore} \footnote{segnalano problemi nella comunicazione di rete} e \textbf{Messaggi informativi} \footnote{utilizzati per scopi diagnostici e di controllo})
%che il protocollo può mandare per definire un canale di comunicazione nascosto per uno scambio di informaizoni (non consentito). 

