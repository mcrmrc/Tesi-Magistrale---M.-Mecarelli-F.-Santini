Implicano la scrittura di dati su un'area di memoria condivisa accessibile da entrambi i processi. 
%(e.g attributi del file, i bit di memoria, gli stati della cache,\dots)
I veicoli saranno tutte quelle risorse che consentono la scrittura (diretta o indiretta) da parte di un processo e 
la lettura (diretta o indiretta) da parte di un altro. 
%Coinvolgono quindi la scrittura di dati su un'area di memoria condivisa accedibile da entrambi i processi  
Quindi in uno Storage Covert channel, un processo scrive su una risorsa condivisa mentre un altro processo legge da essa. 
%Nei covert channel di archiviazione un processo scrive su una risorsa condivisa, mentre un altro processo legge da essa. 
%Possono essere quindi utilizzati da processi all'interno di un singolo computer o tra più computer in una rete.
%\vspace{2ex} \newline
%\begin{esempio}
%    Un esempio di un canale di archiviazione è la condivisione di un file. 
%    Supponendo che l'utente A (con privilegi sufficenti) voglia trasmettere segretamente dei dati riservati all'utente B (con un livello di sicurezza inferiore). 
%    Utilizzando un file di testo in cui apparentemente verranno scritte informazioni non classificate, mentre in verità occulterà l'informazione riservata. 
%\end{esempio}
%\vspace{2ex} 
%\begin{esempio}
%    Variare deliberatamente il tempo fra delle azioni (es trasmissione di network packet, patter di uso della CPU) 
%    oppure codificando dati nella temporalizzazione dell'esecuzione dei processi o delay di risposta. 
%\end{esempio} 
In questo caso come risosrsa condivisa verranno utilizzate le tipologie di messaggi presenti nel protocollo ICMP.  
Il mittente codificherà i dati nei campi presenti e il destinatario, una volta ricevuto il pacchetto, li decodificherà.  
Nelle tabelle sono stati indicati quali tipologie di messaggi verranno sfruttati e, per ogni tipologia di messaggio ICMP, 
quali campi sono stati utilizzati oltre alla quantità di byte inseribili in un singolo messaggio. 
%Inoltre è stato indicato anche quant'è il peso di un singolo pacchetto di quella tipologia. 
\vspace{2ex} \newline 
Nel caso dei messaggi Echo Request e Echo Reply, si avranno diverse marianti. 
Questo perchè il messaggio può trasportare un payload al suo interno. 
Le varianti implicheranno quindi l'utliizzazione o meno di questo campo. 
Nel caso lo si utilizzi il payload o conterrà 32 byte di informazione oppure 56 byte; 
il primo caso verrà usato perchè è il valore di defualt usato nei sistemi Windows il secondo è invece il sistema usato nei sistemi Linux. 
Anche il TTL varierà, il suo valore sarà 128 nei sistemi windows mentre 64 nei sistemi Linux. 
%\vspace{2ex} \newline 
%Altre varianti potranno essere presenti per qeulle tipologie che richiedono il pacchetto che ha causato l'errore (Redirect, Destination Unreachable). 
%In questi casi si potranno usare diverse tipologie di pacchetti purchè la loro lunghezza non superi i 28 bit. 
%Possibili alternative trovate sono con IP/ICMP e con TCP/IP con entrambe che permettono l'invio di due byte. 
\begin{longtable}{|p{0.2 \textwidth}|p{0.15\textwidth}|p{0.25\textwidth}|p{0.25\textwidth}|}
    \hline 
    \textbf{Tipologia} & \textbf{Byte trasmessi} & \textbf{Campi Utilizzati} & \textbf{Descrizione} \\ %messaggi di errore
    \hline
    Destination Unreachable & 3-8 & unused, header+64 bits & Una destinazione (rete, porta,\dots) risulta non disponibile  \\ %messaggi di errore
    \hline 
    Source Quench & 3-8 & unused, header+64 bits & Un gateway notifica il buffer di memoria per i pacchetti pieno (indica la congestione nella rete). \\ %messaggi di errore
    \hline 
    Redirect Message & 3-4 & header+64 bits & Suggerisce un reindirizzamento del pacchetto verso un percorso migliore. \\ %messaggi di errore
    \hline 
    Time Exceeded & 3-8 & unused, header+64 bits & Il gateway notifica che il TTL del pacchetto ricevuto risulta zero. \\ %messaggi di errore
    \hline 
    Parameter Problem & 4-8 & pointer, unused, header+64 bits & Il gateway rileva dei problemi nei campi dell'intestazione \\ %messaggi di errore
    \hline 
    Echo Request & 2 & identifier, data  &  Usato per inviare un dato a un destinatario e ricevere una risposta indietro.\\ %messaggi di informazione
    \hline 
    Echo Reply & 2 & identifier, data  &  Replica i dati ricevuti nella richiesta rimandandoli al mittente \\ %messaggi di informazione
    \hline 
    Timestamp Request & 5 & identifier,timestamp, data  & Usato per inviare una serie di timestamp a un destinatario e ricevere una risposta indietro.\\ %messaggi di informazione
    \hline 
    Timestamp Reply & 5 & identifier,timestamp, data  &  Replica i timestamp ricevuti nella richiesta rimandandoli al mittente \\ %messaggi di informazione
    \hline 
    Information Request & 2 & identifier & Permete di scoprire se l'host si trova nella stessa rete di chi ha risposto. Nel mandare il pacchetto lascia il campo \textit{destinazione} vuoto. \\ %messaggi di informazione
    \hline 
    Information Reply & 2 & identifier & Risponde alla richiesta con tutti i campi compilati correttamente. In particolare quello relativo al proprio indirizzo IP. \\ %messaggi di informazione
    \hline 
    %9 & 0 & CC  & Router Advertisement & Routers announce themselves to hosts. \\
    %\hline
    %10 & 0 & CC  & Router Solicitation & Hosts request router advertisements. \\
    %\hline 
\end{longtable}
\captionof{table}{Tipologie di messaggi ICMPv4} 
\label{table:icmpv4:tipologie} 
\vspace{4ex}
\begin{longtable}{|p{0.2\textwidth}|p{0.15\textwidth}|p{0.25\textwidth}|p{0.25\textwidth}|} 
    \hline 
    \textbf{Tipologia} & \textbf{Codici} & \textbf{Campi Sfruttatti} & \textbf{Uso} \\
    \hline
    Destination Unreachable & 4-8 & unused, invoking packet & Una destinazione (rete, porta,\dots) risulta non disponibile \\ %messaggi di errore
    \hline 
    Packet Too Big & 8 & mtu, invoking packet & Un router notifica l'impossibilità nell'inoltrare un pacchetto (indica la congestione nella rete) \\  %messaggi di errore
    \hline 
    Time Exceeded & 4-8 & unused, invoking packet & Il gateway notifica che il TTL del pacchetto ricevuto risulta zero. \\ %messaggi di errore
    \hline 
    Parameter Problem & 8 & pointer, invoking packet & Il gateway rileva dei problemi nei campi dell'intestazione \\  %messaggi di errore
    \hline 
    Echo Request & 2 & identifier, data & Usato per inviare un dato a un destinatario e ricevere una risposta indietro.\\ %messaggi di informazione
    \hline 
    Echo Reply & 2 & identifier, data  & Replica i dati ricevuti nella richiesta rimandandoli al mittente \\ %messaggi di informazione
    \hline 
\end{longtable}
\captionof{table}{Tipologie di messaggi ICMPv6} 
\label{table:icmpv6:tipologie}

%\import{./icmp}{messaggi_errore}  

%\import{./icmp}{messaggi_informativi}  

