When most people think of reverse shells, they imagine TCP or UDP — like the classic Meterpreter or Netcat.
But there’s a lesser-known method lurking in plain sight: ICMP. 
\vspace{1ex} \newline
ICMP (Internet Control Message Protocol) is typically used for network diagnostics (think ping or traceroute) but it can also be abused as a covert tunnel to send and receive commands.
%ICMP is mainly used to ping computers and appliances across networks. 
Because it’s essential for basic troubleshooting, many organizations either don’t block ICMP or don’t monitor it closely. 
This creates an opportunity: if TCP and UDP ports are filtered, an attacker can still exfiltrate data or remotely control a host by embedding commands into ICMP “echo request” (type 8) and retrieving responses via “echo reply” (type 0).


\section{ICMP Door}
\import{./tools}{icmpdoor} \newpage 
%
\section{ICMPExfil}
\import{./tools}{ICMPExfil} \newpage 
%
\section{Icmpsh}
\import{./tools}{Icmpsh} \newpage 
%
\section{ICMP Shell}
\import{./tools}{icmpShell} \newpage 
%
\section{ICMPtunnel}
\import{./tools}{icmptunnel} \newpage 
%
%\section{}
%\import{./tools}{} \newpage 
%



%https://github.com/DhavalKapil/icmptunnel
%https://github.com/jamesbarlow/icmptunnel
%https://dhavalkapil.com/icmptunnel/
