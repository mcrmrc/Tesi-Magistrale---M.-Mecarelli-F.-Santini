%https://scapy.readthedocs.io/en/latest/index.html
%https://github.com/secdev/scapy
\subsection{Scapy}
Scapy is packet manipulation framework written in python. 
You can forge a lot of kind of packets (http, tcp, ip, udp, icmp, etc...)
\vspace{2ex}\newline
Scapy is a powerful Python-based interactive packet manipulation program and library.
\vspace{1ex}\newline
It is able to forge or decode packets of a wide number of protocols, send them on the wire, capture them, store or read them using pcap files, match requests and replies, and much more. 
It is designed to allow fast packet prototyping by using default values that work.
\vspace{1ex}\newline
It can easily handle most classical tasks. 


\subsection{Scapy}
Scapy is a Python program that enables the user to send, sniff, dissect and forge network packets. 
This capability allows construction of tools that can probe, scan or attack networks.
Scapy mainly does two things: sending packets and receiving answers. 
You define a set of packets, it sends them, receives answers, matches requests with answers and returns a list of packet couples (request, answer) and a list of unmatched packets.
%This has the big advantage over tools like Nmap or hping that an answer is not reduced to open, closed, or filtered, but is the whole packet.


The send() function will send packets at layer 3. 
That is to say, it will handle routing and layer 2 for you.
%
The sendp() function will work at layer 2. 
It’s up to you to choose the right interface and the right link layer protocol.


Sniffing
We can easily capture some packets or even clone tcpdump or tshark. 
Either one interface or a list of interfaces to sniff on can be provided. 
If no interface is given, sniffing will happen on conf.iface



