Dopo aver analizzato gli strumenti che già hanno provato a sviluppare un covert channel trmaite ICMP e 
dopo aver analizzato metodologie sfruttabili per poter nasocndere i dati; 
procediamo nel svilupparne uno. 
%Prima di sviluppare un covert Channel Channel che potesse esfiltrare i dati dalla macchina vittima; 
%si sono analizzati strumenti già presenti per studiarne il comportamento. 
\vspace{2ex} \newline
Dagli strumenti analizzati; sono stati ricavati alcuni punti sulla struttra dell'insieme. 
%ciò si è arrivati alle seguenti conclusioni: 
Si è realizzata la necessita di \textbf{proxy intermediari} fra la vitima e l'attaccante così da poter nascondere l'identità di quest'ultimo. 
Quindi sebbene sia possibile anche nessun utilizzo di questi proxy, e quindi si utilizzerà solo la machcina dell'attaccante; 
nel caso li si volesse utilizzare, si deve tenere conto di una \textbf{distribuzione omogenea del traffico} così da non dirigere il throughput creato solo verso una singola macchina. 
Se il traffico generato, venisse instradato attraverso un singolo proxy alla volta, il numero di messaggi scambiati genererebbe anomalie che potrebbero essere notato dai meccanismi di difesa. 
%del traffico generato e non generare un throughput elevato (dato il numero di messaggi scambiati)
%l'attaccante si può connettere ai proxy. 
%Si può quindi utilizzare una tipologia di comunicazione meno nascosta e più diretta.
%Si possono quindi creare dei Socket
\vspace{1ex} \newline 
Ciò porta l'attenzione anche alla quantità di dati che si vuole inviare e quindi al possibile \textbf{limite} della loro dimensione. 
Se mai si volesse esfiltrare un file contenente una grande quantità di dati; questo potrebbe generare rumore e destare sospetti. 
Quindi anche un \textbf{periodo di riposo} fra le varie richieste deve essere presente. 
Maggiore è il numero continuativo di richieste, maggiore sarà la possibilità di far notare la presenza del Covert Channel (e quindi di essere scoperti). 
Questo soglia può non essere direttamente legata a un fattore temporale, 
ma può variare anche in base a quanti messaggi sono stati scambiati fra le due parrti.
\vspace{1ex} \newline 
Inoltre bisogna considerare che trasmettere i dati in chiaro permettera a chiunque riceva i pacchetti di poterli leggere. 
Un sistema di difesa, se effettuasse un'ispezione approfondita di questi pacchetti, potrebbe rilevare lo comunicazione invalida. 
Tuttavia anche mandarli cifrati potrebbe destare sospetti nei meccanismi di difesa. 
Un metodo che potrebbe funzionare è l'inserimento di numeri; 
per esempio nel campo \textit{identifier}, che richiede  interi, l'inserimento di numeri non risulterebbe strano. 
Tuttavia questo non varrà per il campo \textit{data}. 
Quindi un metodo per poter inserire dati codificati ma che risultino normali è necessario. 
\vspace{1ex} \newline 
Un ulteriore fattore da considerare è l'utilizzo di tipologie di messaggi che richiedano una risposta. 
Siccome ad ogni Echo Request viene mandata in risposta una Echo Reply (avente gli stessi dati ricevuti) 
si avranno due messaggi identici tranne per il campo che indica il tipo di messaggio. 
Ciò potrebbe non risultare un problema se la dimensione del campo dei dati non sia eccessiva. 
Una possibile soluzione a questo problema è l'evitare l'invio delle Echo Request ma inviare solo Echo Reply. 
Tuttavia se il numero di richieste non combaciasse con il numero di risposte, ciò potrebbe destare dei sospetti. 
Un ulteriore possibilità èquella, se possibile, di disabilitare l'invio da parte del sistema delle risposte e mandare una propria personale Echo Reply. 
Così quando al sistema arriverà una richiesta, non invieerà una risposta ma la nostra che 
non ripeterà il contenuto mandato dall'attaccante ma un'altro (possibilmente di dimensione minore). 
%\vspace{1ex} \newline 
%\item Evitare un numero importante di pacchetti tutti verso la stessa deastinazione. 
%Per esempio non mandare più di 5 pacchetti, in un determinato istante, verso la stessa destinazione. 
%\uppercase{è} preferibile mandare meno e poi aspettare che mandarli tutti e subit. 
%In questo modo si evita di essere scoperti facilmente.




