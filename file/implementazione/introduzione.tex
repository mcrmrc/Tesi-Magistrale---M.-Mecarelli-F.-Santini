Dopo aver analizzato gli strumenti già sviluppati riguardo l'argomento, procediamo nel crearne uno.
Prima di sviluppare un covert Channel Channel che potesse esfiltrare i dati dalla macchina vittima; 
si sono analizzati strumenti già presenti per studiarne il comportamento. 
\vspace{2ex} \newline
Da ciò si è arrivati alle seguenti conclusioni: 
\begin{enumerate}
    \item Il bisogno di un proxy intermediario fra la vittima e l'attaccante così da nasconderne l'entita. 
    \item Un numero di proxy che permetta la distribuzione omogenea del traffico generato e non generare un throughput elevato (dato il numero di messaggi scambiati)
    \item  Un limite alla dimensione dei dati inviati. 
    Se mai si volesse esfiltrare un file contenente una grande quantità di dati; 
    questo potrebbe generare rumore e destare sospetti. 
    \item Un periodo di riposo randomico dopo ogni richiesta fatta dall'attaccante. 
    Maggiore è il numero di richieste maggiore questo valore dovrà crescere così da non far notare la propria presenza. 
    Questo valore può variare anche in base a quanti messaggi sono stati scambiati con la vittima.
    \item Il testo scambiato non deve essere in chiaro. 
    Averlo in chiaro permetterebbe di leggere il contenuto di ciò che viene mandato da sistemi di monitoraggio. 
    \item usare il campo relativo all'id o alla sequenza per trasmettere i dati mentre il payload  sarà un testo usato comunemente per testare la rete. 
    Questo per evitare che analizzando il payload si scopra la natura ilelcita della comunicazione. 
    \footnote{
        Il lato negativo è che può contenere solo 2 byte. 
        \uppercase{è} quindi preferibile per mandare il comando piuttosto che per ricevere i dati.
    } 
    \item Evitare un numero importante di pacchetti tutti verso la stessa deastinazione. 
    Per esempio non mandare più di 5 pacchetti, in un determinato istante, verso la stessa destinazione. 
    \uppercase{è} preferibile mandare meno e poi aspettare che mandarli tutti e subit. 
    In questo modo si evita di essere scoperti facilmente.
    \item Possibilità di utilizzare solo le Echo Reply così da non avere il doppio dei messaggi. 
    \footnote{
        Ad ogni Echo Request viene associata una Echo Reply in cui la risposta manda gli stessi dati ricevuti
    }
    Tuttavia avere troppe reply senza una request potrebbe destare sospetti. 
    Una soluzione potrebbe essere di disabilitare sulla macchina vittima le Echo Reply; 
    così che quando arriva una richiesta non si invia una risposta. 
    \item Un possibile approccio è disabilitare le risposte ma mandarle comunque, ma cambiando il contenuto al loro interno. 
    Quindi la risposta non ripeterà il contenuto mandato dall'attaccante ma un'altro (possibilmente di diemsnione minore), così da avere per ogni Richiesta una Rispoassta
    %\item 
\end{enumerate}
%l'attaccante si può connettere ai proxy. 
%Si può quindi utilizzare una tipologia di comunicazione meno nascosta e più diretta.
%Si possono quindi creare dei Socket


