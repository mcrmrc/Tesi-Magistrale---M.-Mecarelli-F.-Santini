\import{./covert_channel}{introduzione}  
\import{./covert_channel}{tipologie}  
%\import{./covert_channel}{attacchi}  -> meglio in tipologia
%\import{./covert_channel}{struttura}  
\import{./covert_channel}{caratteristiche}  
\import{./covert_channel}{vulnerabilita}  
%\import{./covert_channel}{strategie}  

% Characteristics 
%Un Covert Channel viene chiamato così perchè è nascosoto dai meccanismi di controllo degli accessi dei sistemi operativi siccome non usa il meccanismo di trasferimento dati legittimo del sistema (lettura/scrittura) 
%e quindi non possono essere identificati o controllati dai meccanismi di sicurezza alla base della sicurezza dei sistemi operativi. 
%\vspace{1ex} \newline
%I Covert Channel sono eccessivamente difficili da installare nei sistemi reali e spesso possono essere identificati monitorando le performance del sistema. 
%In aggiunta soffrono da un basso 'signal-to-noise ratio' e 'low data rates'. 
%Possono essere anche rimossi manualmente con un alto grado di fiducia da sistemi sicuri, da strategie di analisi ben definite. 
%\vspace{1ex} \newline 
%
%I Covert Channel possono passare attraverso sistemi operativi sicuri a richiedono misure speciali per poterli controllare; 
%in particolare l'analisi dei Covert Channel è l'unico modo provato per poterli gestire. 
%Al contrario, sistemi operativi sicuri possono facilemnte identificare e/o prevenire l'uso improprio di canali legittimi.
%
%\subsubsection*{Identifying Covert Channels}  
%\subsubsection*{Data hiding in OSI model}  
%\subsubsection*{DATA hiding in LAN environment by covert channels}  
%\subsubsection*{Data hiding in TCP/IP Protocols suite}   
