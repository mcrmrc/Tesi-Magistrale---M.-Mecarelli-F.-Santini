\subsection{Cos'è un Covert Cahnnel?}
Un \textbf{Covert Channel} è un attacco che permette (in ambienti ritenuti sicuri) la capacità di comunicare/trasferire dati, in maniera non autorizzata e non voluta, 
fra processi/entità comunicanti spesso senza essere rivelati e spesso evitando (se non violando) le normali politiche di sicurezza.  
\vspace{2ex} \newline
Solitamente operano al di fuori dei soliti meccanismi di comunicazioni. 
Quindi non usando i normali protocolli/canali di comunicazione (es network sockets, emails) non generano segnali di un uso improprio del sistema. 
Ciò li rende difficili da rilevare usando i tipici strumenti di monitoraggio. %Inoltre non usando spesso protocolli di comunicazione standard (es TCP/IP),  
Inoltre questi canali sfruttano le vulnerabilità o comportamenti non previsti nei sistemi. 
\vspace{2ex} \newline
In un Covert Channel, qualsiasi risorsa condivisa può essere utilizzata come canale nascosto ed è per questo che possono esistere 
in qualunque sistema (che abbia delle risorse condivise). 
Lo sfruttamento di queste risorse porta alla fuoriuscita/scambio di dati. %non autorizzata.  
\vspace{2ex} \newline 
L'attacco è un problema siccome sono estremamente difficili da identificare e controllare. 
La loro esistenza spesso rimane non notata dagli amministratori (di sistema) siccome si nascondono all'interno dei 
normali processi del sistema. 
Sono inoltre un problema significativo in tutti quegli ambienti altamente sicuri (es ambienti militari, governativi,\dots) 
dove una fuoriuscita di informazioni può avere conseguenze gravi.  
%
\subsection{Principali categorie di Covert Channel}
Le principali categorie di canali nascosti sono: 
\begin{itemize}
    \item Covert Channel Timing (Temporizzazione): 
    \newline 
    coinvolgono la scrittura di dati a un'area di memoria condivisa in cui entrambi i processi possono accedere 
    \begin{esempio}
        Modificare i permessi dei file o i metadati per codificare informazioni. 
        Oppure modificare variabili condivise o buffer
    \end{esempio}
    \item Covert Channel Storage (Archiviazione): 
    \newline 
    manipolano la temporalizzazione o l'ordine di eventi per codificare informazioni. 
    \begin{esempio}
        Variare deliberatamente il tempo fra delle azioni (es trasmissione di network packet, patter di uso della CPU) 
        oppure codificando dati nella temporalizzazione dell'esecuzione dei processi o delay di risposta. 
    \end{esempio}
    \item Covert Channel Behavioral (Comportamentali) 
\end{itemize}
\subsubsection{Covert Channel Timing (temporizzazione)}
I canali nascosti di temporizzazione sono metodi di comunicazione che permettono ad un osservatore (umano o processo) di acquisire 
informazioni attraverso il cambiamento nel tempo di risposta di una risorsa.
Essenzialmente qualsiasi metodo che utilizza un orologio o una misurazione del tempo per segnalare il valore inviato sul canale.
\href{https://www.youtube.com/watch?v=QIvsmQQ6vu8}{Esempio}
%
\subsubsection{Covert Channel Storage (Archiviazione)}
Nei canali nascosti di archiviazione un processo scrive su una risorsa condivisa, mentre un altro processo legge da 
essa. I canali di archiviazione possono essere utilizzati tra processi all’interno di un singolo computer o tra più 
computer in una rete.
\vspace{1ex} \newline 
I veicoli dell'attacco sono tutte quelle risorse che consentono la scrittura, diretta o indiretta, di una risorsa 
da parte di un processo e la sual lettura, diretta o indiretta, da parte di un altro.
\begin{esempio}
    Un esempio di canale di archiviazione è la condivisione di un file. 
    Supponiamo che l’utente A con privilegi di autorizzazione elevati voglia trasmettere in segreto, dati riservati all’utente B con un livello di sicurezza inferiore. 
    Per farlo, utilizzerà un file word apparentemente contenente informazioni non classificate, dove invece occulterà l’informazione riservata. 
\end{esempio}
%
\subsubsection{Covert Channel Behavioral (Comportamentali)}
I canali nascosti comportamentali operano trasmettendo dati in base all’assegnazione di diversi eventi di processi, sistemi e 
applicazioni, generalmente suddividendo e trasmettendo i dati in pacchetti più piccoli.
%\vspace{1ex} \newline
%Tra i più affascinanti canali nascosti comportamentali, in cui mi sono imbattuto, quello che utilizza il protocollo 
%ICMP (Internet Control Message Protocol), sfrutta appieno caratteristiche quali la rapidità di attivazione, semplicità d’uso e 
%bypassa molte delle policy e standard di sicurezza con estrema nonchalance, manifestando in maniera ottimale le caratteristiche di 
%Stealthiness e Indistinguishability.
%L’ICMP è progettato per fornire feedback su problemi di comunicazione di una rete TCP/IP. 
%L’ ICMP si affida al supporto di base dell’IP come parte di protocollo di livello superiore. 
%A causa di questa dipendenza, sia ICMPv4 che ICMPv6 esistono per entrambe le versioni di IP. 
%Applicazioni come ad es. traceroute e ping utilizzano i messaggi ICMP per raccogliere informazioni e diagnosticare eventuali 
%problemi di rete. 
%\vspace{1ex} \newline
%I canali nascosti spesso sfruttano per i propri flussi di informazioni, alcune caratteristiche tecniche incorporate nelle reti 
%IEEE 802, caratteristiche che normalmente non vengono “viste” a livello di rete più alto perché considerate di servizio.
%\vspace{1ex} \newline
%L’idea di utilizzare l’ICMP come canale nascosto è quindi quella di sfruttare una comunicazione standard con un protocollo inferiore 
%rispetto a TCP o UDP. Questo avrà un “ingombro” ridotto in tutto il traffico di rete e potrà passare inosservato agli amministratori 
%di rete e agli analizzatori di traffico, in quanto, come detto, normalmente è utilizzato per la diagnostica e manutenzione della rete 
%e degli host connessi, non per il trasporto dati, quindi difficilmente bloccato da policy di sicurezza.
%Questo rende il protocollo ICMP un canale nascosto decisamente praticabile, l’uso di campi dati o payload all’interno di determinati 
%messaggi ICMP permette di incorporare il messaggio del canale nascosto facilmente e trasforma paradossalmente l’ICMP stesso in un 
%canale nascosto. Questi semplici fattori consentono all’ICMP di essere di fatto un traffico invisibile, vediamo un esempio. 
%
%---
%
\subsection{Struttura/Caratteristiche dei Covert Channel} 
Tipicamente è costituito da due principali componenti: 
\begin{itemize}
    \item \textbf{Mittente} (Covert Transmitter): è l'entita che codifica e trasmette le informazioni nascote usando una risorsa di sistema condivisa.  
    \item \textbf{Destinatario} (Covert Listener): è l'entità che rileva e decifra l'informazione segreta dalla risorsa condivisa. 
\end{itemize}
%
\subsubsection*{Come funzionano i Covert Channel?} 
Il mittente inserisce informazioni segrete in un componente del sistema che è osservabile da un destinatario. 
Il destinatario decifra i dati trasmessi di nascosto monitorando i cambiamenti nel comportamento del sistema. 
\vspace{1ex} \newline
Le informaizoni vengono inserite sfruttano gli effetti collaterali delle normali operazione del sistema 
senza un esplicito intento di comunicare.  %(delay o scrittura su file condivisi) 
\vspace{1ex} \newline  
Un Covert Channel quindi opera cifrando dati nascosti nei comportamenti del sistema che i controlli di sicurezza 
tipicamente non monitorano così da permettere la comunicazione segreta fra due entità. 
%è strutturato come un sistema di comunicazione segreta che bypassa i normali meccanismi di sicurezza. 
%
\subsubsection*{Caratteristiche Chiave dei covert Channel} 
Le principali caratteristiche di un Covert Channel sono:  \newline
$\bullet$ \textbf{Stealthiness} (furtività):  \newline
si devono poter aggirare i controlli in maniera nascosta
\vspace{2ex} \newline
$\bullet$ \textbf{Bandwith} (capacità di trasmissione):  \newline
la capacità di trasmissione dei dati che è generalmente bassa in termini di dati/tempo (throughput). 
Un eccessivo carico di informazioni, potrebbe rendere anomalo il funzionamento di quelle risorse o delle normali 
strutture dati. %Si stanno utilizzando altre risorse
Nei canali nascosti generalmente il throughput è inversamente correlato alla segretezza di un canale.
\begin{center}
    Più dati un canale trasmette in un determinato periodo di tempo, maggiore è il rischio che il canale venga scoperto
\end{center}
\vspace{2ex}
$\bullet$ \textbf{Indistinguishability} (Indistinguibilità):  \newline
di solilto si sfruttano servizi e/o risorse già presenti e quindi non sospette. 
\vspace{1ex} \newline 
Uno dei maggiori problemi nell’implementazione di un canale nascosto è il “rumore”
(es. sfruttando eccessivamente le risorse alterando e/o danneggiando il corretto funzionamento delle stesse)
che potrebbe attirare l'attenzione da parte degli amministratori di sistema. 
La necessità è quella di riuscire a trasmette attraverso un canale nascosto mantenendo conforme e inalterato il 
funzionamento della risorsa utilizzata così da rendersi “indistinguibili” dalla risorsa autorizzata e quindi invisibili ai sistemi di monitoraggio.
\vspace{4ex} \newline
Ulteriori caratteristiche sono: 
\vspace{2ex} \newline
$\bullet$ \textbf{Uso involontario delle risorse}: \newline
i Covert channels sfruttano le risorse del sistema (e.g memoria cxondivisa, uso della CPU, attributi dei file) 
in maniere che non fossero previste per la comunicazione.
\newline
$\bullet$ \textbf{Comunicazione nascosta}: \newline 
sono progettati per evitare la rilevazione; spesso sfruttando operazioni di sistema legittime per mascherare la trasmissione dei dati. 
\vspace{2ex} \newline
$\bullet$ \textbf{Violazione delle politiche di sicurezza}: \newline 
permettono lo scambio non autorizzato di informazioni, potenzialmente violando i requisiti di confidenzialità, di integrità o quelli di disponibilità. 
\vspace{2ex} \newline
$\bullet$ \textbf{Mezzo di comunicazione nascosoto}: \newline 
il canale è incorporato in operazioni di sistema legittime (e.g carico della CPU, accesso alla memoria, 
traffico della rete, metadati del file systema). 
\newline
\begin{esempio}{\quad\newline}
    cache della CPU, intestazioni TCP/IP, consumo energetico, temporizzazione/tempistica dei pacchetti. 
\end{esempio} 
\vspace{2ex} \noindent
$\bullet$  \textbf{Meccanismi di Codifica}: \newline 
Il mittente manipola una risorsa di sistema condivisa per codificare dati. 
\vspace{1ex} \newline
Tecniche comuni: 
\begin{itemize}
    \item \underline{Codifica basata sul Tempo}: \newline
    usa degli intervalli di tempo (e.g. ritardi fra i pacchetti di rete) 
    \item \underline{Codifica basata sulla Memoria}: \newline
    modifica degli attributi del file, i bit di memoria oppure gli stati della cache 
    \item \underline{Abuso del Protocollo}: \newline
    alterazione dei flag TCP, dei numeri di sequenza oppure dei bit inutilizzati nelle intestazione dei pacchetti 
\end{itemize}  
$\bullet$  \textbf{Meccanismo di comunicazione}: \newline 
il mittente modifica continuamente i comportamenti del sistema per trasmettere bit di informazione. 
Questo può essere fatto introducendo ritardi, cambiando il carico di lavoro della CPU, o modificando gli stati 
della memoria in maniera controllata. 
\vspace{2ex} \newline
$\bullet$  \textbf{Meccanismi di Decodifica}: \newline 
il destinatario monitora la risorsa condivisa per rilevare e ricostruire i dati trasmessi. 
\begin{esempio}{\quad \newline}
    Misurazione dele variazioni del tempo di esecuzione per dedurre i dati segreti.
\end{esempio} 
\vspace{2ex} \noindent
$\bullet$  \textbf{Sincronizzazione e Correzione degli Errori}: \newline 
il mittente e il destinatario devono sincronizzarsi (e.g. utilizzando segnali di temporizzazione pre-concordati). 
I meccanismi di rilevamento degli errori (come bit di parità o checksum) garantiscono un recupero accurato dei dati. 
\begin{esempio}{Esempio di un Covert channel in una rete} \newline
   \underline{Mittente}: modifica il campo TTL (time-to-live) nei pacchetti IP per rappresentare dati binari (e.g. TTL=65$\rightarrow$bit 1, TTL=128$\rightarrow$bit 0)
    \vspace{1ex} \newline
    \underline{Destinatario}: osserva i valori TTL dei pacchetti in arrivo per ricostruire il messaggio nascosoto 
\end{esempio}
%
\subsection{Vulnerabilità Utilizzate}
I Covert Channel sfruttano le vulnerabilità nel design del sistema, nelle politiche di sicurezza e nei protocolli di comunicazione per trasferire informazioni segretamente. 
Sfruttando queste vulnerabilità, gli attaccanti possono stabilire Covert Channel che evitano il controllo degli standard  sicurezza, 
permettendo esfiltrazione non autorizzata di dati o comunicazione fra processi interni (inter-process comunicaizone). 
\vspace{2ex}\newline
La loro mitigazione richiede controllo degli accessi, randomizzazione dei tempi, iniezione di rumore e una sicurezza hardware migliore. 
\subsubsection*{Principali vulnerabilità usate dai covert Channel} 
\begin{enumerate}
    \item \textbf{Sfruttamento delle risorse condivise} 
    \begin{itemize}
        \item \textbf{Scheduling della CPU}: l'attaccante può modulare l'uso della CPU per diffondere informazioni. 
        \item \textbf{Memoria Cache}: gli attacchi side-channel alla cache sfruttano le differenze nei tempi di accesso per dedurre i dati. 
        \item \textbf{Accesso al File System}: i processi possono dedurre informazioni in base ai lock dei file, timestamp o sull'attività del disco
    \end{itemize}
    \item \textbf{Vulnerabilità basate sulla temporizzazione} 
    \begin{itemize}
        \item \textbf{Variabilità del tempo di risposta}: \newline
        l'attacante misura i tempi di risposta del sistema per estrarre segreti. 
        \item \textbf{Ritardi nell'esecuzione delle istruzioni}: \newline
        le differenze del tempo di esecuzione tra le operazioni privilegiate e non possono causare la fuoriuscita di dati. 
        \item \textbf{Tempistica dei pacchetti}: \newline
        le informazioni possono essere codificate negli intervalli durante la trasmissione dei pacchetti 
        \item \textbf{Manipolazione delle intestazioni}: \newline
        campi come TTL, sequenza dei numeri o bit non utilizzati possono essere utilizzati per codificare i dati 
        \item \textbf{Pattern del traffico}: \newline
        le variazioni nel flusso del traffico (es burst size) si possono comportare come un Covert Channel. 
    \end{itemize}
    \item \textbf{Manipolazione della Memoria e dello Stato della CPU} 
    \begin{itemize}
        \item \textbf{Previsione delle ramificazioni ed esecuzione speculativa}: \newline
        sfruttato in attacchi come Spectre e Meltdown 
        \item \textbf{Analisi del consumo energeticos}:  \newline
        i canali secondari possono rilevare chiavi crittografiche 
    \end{itemize}
    \item \textbf{Falle nel sistema operativo e nella Virtualizzazione} 
    \begin{itemize}
        \item \textbf{Abuso della comunicazione fra processi (Inter-Process Communication IPC)}: \newline 
        i processi posono ricavare i dati tramite la memoria condivisa o il passaggio di messaggi
        \item \textbf{Debolezze dell'hypervisor}: \newline
        le macchine virtuali posso far trapelare informazioni tra le guest instances
    \end{itemize}
    \item \textbf{Vulnerabilità Hardware} 
    \begin{itemize}
        \item \textbf{Emissioni elettromagnetiche}: \newline 
        dati sensibili possono essere divulgati tramite dei segnali EM (attacco TEMPEST)
        Sensitive data can be leaked via EM signals (TEMPEST attacks).
        \item \textbf{Canali laterali acustici}: \newline 
        è possibile analizzare i suoni/rumori della tastiera, le variazioni della velocità della ventola o il rumore dell'alimentatore.
    \end{itemize} 
\end{enumerate} 
%
\subsection{Applicazione dei Covert Channel}
I Covert Channel sono spesso applicati in: 
\begin{itemize}
    \item \textbf{Malware and Spionaggio}: usati per esfiltrare dati sensibili. 
    \item \textbf{Test di sicurezza}: identificare e mitigare i Covert Channel è una parte fondamentale nel stabilire la sicurezza del sistema. 
    \item \textbf{Ricerca}: espolare i Covert Channel aiuta a capire potenziali vulnerabilità in sistemi complessi. 
\end{itemize} 
%
\subsection{Tipologie di attacchi Covert Channel} 
Gli attachi tramite Covert Channel sfruttano vulnerabilita nel design del sistema, nelle risorse condivise e nelle
politiche di sicurezza per trasmettere segretamente dati fra processi o sistemi aggirando i tradizionali 
controlli di sicurezza. 
Questi attacchi sono spessi usati per l'esfiltrazione dei dati, privilege escaletion o comunicazioni 
silenzione tra delle componenti malware.  
\newline
\begin{enumerate}
    \item \textbf{Covert Channel basati sulla memoria}: 
    Questi attacchi manipolano le risorse di sistema condivise per memorizzare e recuperare informazioni nascoste.
    \begin{esempio}{\quad \newline}
        \underline{Manipolazione degli attributi dei file}: \newline
        il malware altera i metadati dei file (e.g. timestamp, permessi) per codificare i messaggi.
        \vspace{1ex} \newline 
        \underline{Sfruttamento della memoria condivisa}: \newline 
        i processi comunicano modificando le regioni di memoria condivise. 
        \vspace{1ex} \newline 
        \underline{Segnali tramite l'utilizzo del disco}: \newline 
        un processo scrive o elimina i dati mentre un altro processo rileva le modifiche.
        Disk Usage Signaling: One process writes or deletes data, and another process detects changes. 
        \vspace{1ex} \newline 
        \underline{Campi nell intestazione TCP/IP}: \newline 
        gli attaccani codificano i dati in campi inutilizzati o facoltativi dei pacchetti di rete 
        (e.g. ID IP, numeri di sequenza o valori TTL). 
    \end{esempio}
    \item \textbf{Covert Channels basati sulla temporizzazione}: \newline
    Questi attacchi manipolano la tempistica o le prestazioni del sistema per trasmettere informazioni nascoste.
    \begin{esempio}{\quad\newline}
        \underline{Fluttuazione del carico della CPU}: 
        il malware altera gli schemi di utilizzo della CPU, che un altro processo misura per decodificare le informazioni. 
        \vspace{1ex} \newline 
        \underline{Temporizzazione dei pacchetti di rete}: 
        il mittente trasmette i pacchetti a intervalli di tempo specifici per codificare i dati binari. 
        \vspace{1ex} \newline 
        \underline{Attacchi basati sulla cache}: 
        gli aggressori utilizzano i tempi di accesso alla cache (e.g. Flush+Reload, Prime+Probe) per far trapelare segreti
        \vspace{1ex} \newline 
        \underline{Analisi del consumo energetico}: 
        i dati sensibili vengono estratti analizzando le variazioni del consumo energetico 
        (utilizzate negli attacchi crittografici side-channel). 
    \end{esempio} 
\end{enumerate} 
\subsubsection*{Esempi reali di attacchi Covert Channel} 
\begin{itemize}
    \item Attacchi basati sui Malware: \newline 
    Duqu 2.0 (2015) utilizzava canali TCP/IP occulti per esfiltrare i dati evitando il rilevamento
    \item Attacchi di tunneling DNS: \newline 
    il malware nasconde i dati all'interno delle query DNS (ad esempio, comunicazione C2 per le botnet). 
    \item Covert Channels basati sul Cloud e sulla Virtualizatione: \newline 
    Hypervisor Covert Channels: 
    Le macchine virtuali (VM) sullo stesso host fisico perdono dati attraverso la cache o la memoria della CPU condivisa.
    \vspace{1ex} \newline 
    Cloud Timing Attacks: Cloud tenants use execution timing differences to infer co-resident VM activities. 
\end{itemize}
\subsubsection*{Menzione a notevoli attacchi Covert Channel} 
\begin{minipage}{\linewidth}
    \begin{tabular}{|p{0.3\linewidth}|p{0.3\linewidth}|p{0.3\linewidth}|}
        \hline
        Nome Attacco & Tipo & Descrizione \\
        \hline 
        Spectre and Meltdown & Timing (Cache) & Exploit speculative execution to leak memory contents \\ 
        \hline 
        Flush+Reload & Timing (Cache) & Attacker flushes shared memory and reloads it to observe access patterns. \\
        \hline 
        Prime+Probe & Timing (Cache) & Attacker fills cache and monitors eviction patterns to infer secret data. \\
        \hline 
        Packet Timing Attack & Timing (Network) & Varies packet transmission timing to send hidden messages. \\ 
        \hline 
        Keystroke Timing Attack & Timing (Human Interaction) & Infers typed keys based on timing variations between keystrokes. \\
        \hline 
        TCP Covert Channel & Storage (Network) & Encodes data in TCP packet fields (e.g., sequence numbers, flags). \\
        \hline 
        File Lock Covert Channel & Storage (Filesystem) & Uses file locking/unlocking as a signaling mechanism. \\
        \hline 
    \end{tabular}
\end{minipage}
%
\subsection{Strumenti di Mitigazione e Protezione}
\subsubsection*{Mitigation Strategies}
Gli attacchi tramite Covert Channel sfruttano le debolezze, del timing del sistema, delle risorse condivise e dei protocolli di rete, 
per trasmettere dati nascosti. 
Pongono una seria minaccia nella comunicazione fra malware, esfiltrazione dei dati e il cyber-spionaggio. 
Difese efficaci implicano l'isolamento delle risorse, iniezione di rumore e rilevamento delle anomalie così da disturbare questi attacchi. 
\subsubsection*{Protezione contro i Covert Channel}
Il loro rilevamento e la loro mitigazione richiede un rigoroso monitoraggio, l'isolamento delle risorse e 
tecniche per introdurre rumore. 
%
I Covert channel sfruttano le vulnerabilità del sistema per trasmettere segretamente dei dati. 
Proteggersi da loro, richiede una combinazione di rinforzo delle politiche, gestione delle risorse e tecniche di monitoraggio. 
%
Mitigare i Covert channel richiede una sicurezza multi livello fra hardware, OS, applicazioni e reti. 
Siccome la completa eliminazione è difficle, strategie di rilevazione e minimizzazione sono essenziali 
(es randomizzazione, rigoroso controllo degli accessi delle risorse, rilevamento delle anomalie). 
\subsubsection*{Eliminating Covert Channels} 
Le possibilità di un Covert Channel non possono essere eliminate sebbene possano essere significatamente ridotte da un design e analisi attenti. 
La rilevazione di un Covert Channel può essere resa maggiormente difficile usando caratteristiche del medium di comunicazione per il canale legittimo che non sono mai controllati o esaminati da utenti legittimi. 
\begin{esempio}
    Un file può essere aperto e chiuso da un progremma in modo specifico pattern temporale così che possa essere rilevato da un altro programma; 
    lo schema potrà essere poi interpretato come una stringa di bit formando così un Covert Channel. 
    Di conseguenza, siccome è improbabile che l'utente legittimo controlli i pattern relativi alla chiusura/apertura dei file; 
    questo tipo ti Covert Channel può rimanere non identificato per un lungo periodo. 
\end{esempio}
\vspace{3ex} \noindent
Le strategie di difesa possono essere: \newline
\textbf{Difese a livello di sistema}
\begin{itemize}
    \item Applicare un forte controllo degli accessi (MAC, RBAC) per evitare interazioni non autorizzate con i processi. 
    \item Utilizzare obbligatoriamente modelli di controllo del flusso di dati (Bell-LaPadula, Biba) per evitare fughe di informazioni. 
    \item Disattivare le risorse condivise non necessarie (ad esempio, comunicazione tra processi, memoria condivisa).
\end{itemize}
\textbf{Difese di rete}
\begin{itemize}
    \item Implementate la deep packet inspection (DPI) e il rilevamento delle anomalie per identificare i dati nascosti nel traffico di rete. 
    \item Applicare la segmentazione della rete per limitare i flussi di dati non autorizzati.
\end{itemize}
\textbf{Difese hardware e OS}
\begin{itemize}
    \item Randomizzare i tempi di esecuzione e iniettare rumore nelle risposte del sistema (per interrompere gli attacchi basati sulla temporizzazione). 
    \item Implementare operazioni crittografiche a tempo costante per prevenire i canali laterali di temporizzazione. 
    \item Svuotare e partizionare le cache della CPU per prevenire gli attacchi alla cache cross-process.
\end{itemize}
%
\subsection*{Principali strategie per la mitigazione} 
\subsubsection*{Protezioni basate sul Sistema e sule Politiche(Policy)}
\begin{enumerate}
    \item \textbf{Politiche di controllo degli accessi}: \newline 
    Implementare il minimo privilegio e il controllo obbligatorio dell'accesso (MAC) per limitare la comunicazione non autorizzata tra i processi. 
    Utilizzare sandbox e compartimentazione per isolare i processi.
    \item \textbf{Controllo del flusso di informazioni}: \newline 
    Applicare le politiche sul flusso dei dati così da impedire che i processi ad alta sicurezza perdano dati ai processi a bassa sicurezza (modello Bell-LaPadula, Biba). 
    \item \textbf{Separazione e isolamento dei processi}: \newline 
    Utilizzare la virtualizzazione e la containerizzazione per separare i processi. 
    Applicare l'air-gapping per i sistemi altamente sensibili.
\end{enumerate}
\subsubsection*{Protezioni di gestione delle risorse e dei tempi} 
\begin{itemize}
    \item \textbf{Tecniche di Randomizzazione}
    \newline 
    Introdurre rumore nelle risposte del sistema (ad esempio, randomizzando i tempi di esecuzione, aggiungendo 
    ritardi) per interrompere i §covert Channel basati sul tempo.   
    Utilizzare tecniche di randomizzazione o svuotamento della cache per prevenire attacchi side-channel basati sulla cache.
    \item \textbf{Limitazione della velocità e controllo della larghezza di banda} 
    \newline 
    Limitare la CPU, la memoria o la larghezza di banda della rete per limitare la capacità di un canale nascosto. 
    Implementare meccanismi di throttling (limitazione) per le risorse condivise. 
\end{itemize}
\subsubsection*{Protezioni di sicurezza della rete} 
\begin{itemize}
    \item \textbf{Ispezione e filtraggio dei pacchetti}: 
    \newline 
    Utilizzare la Deep Packet Inspection (DPI) per rilevare schemi anomali nel traffico di rete. 
    Bloccare o sanificare i campi inutilizzati dei protocolli (ad esempio, le intestazioni TCP/IP). 
    \item \textbf{Analisi del traffico e rilevamento delle anomalie}: 
    \newline 
    Utilizza il monitoraggio basato sull'intelligenza artificiale per rilevare modelli di comunicazione insoliti.
    Utilizza sistemi di rilevamento delle intrusioni (IDS) e analisi dei log per identificare attività sospette.
\end{itemize} 
\subsubsection*{Miglioramenti della sicurezza hardware e software} 
\begin{itemize}
    \item Progettazione hardware sicura
    \newline 
    Implementare operazioni crittografiche a tempo costante per prevenire attacchi basati sulla temporizzazione. 
    Utilizzare enclave sicuri (ad esempio, Intel SGX, ARM TrustZone) per proteggere i calcoli sensibili. 
    \item Protezioni a livello di sistema operativo 
    \newline 
    Applicare l'isolamento della memoria e disabilitare la memoria condivisa quando non è necessaria. 
    Implementare algoritmi di pianificazione sicuri per prevenire fuoriusicte di dati tramite la temporizzazione basata sui processi.
\end{itemize} 
\subsubsection*{Verifica e test dei Covert Channel} 
\begin{itemize}
    \item Eseguire regolarmente analisi dei canali nascosti nei test di penetrazione.
    \item Utilizzare strumenti di rilevamento dei Covert Channel (ad esempio, analisi del flusso di rete, monitoraggio del comportamento del sistema).
\end{itemize}
%
\textbf{Strategie di Mitigazione} 
\newline 
Controllo sugli Accessi: \newline
limnitare i permessi per prevenire scambio di informazioni non autorizzato 
Monitoraggio del Traffico: \newline
analizzare i comportamenti del sistema per rilevare anomalie 
Aggiunta di Rumore (Noise Injection): \newline
introdurre casualità nei pattern temporali o di accesso alla memoria per rendere il prelevamento dei dati difficile.  
%
Strategie di mitigazione: 
\begin{itemize}
    \item System Design: Minimize shared resources and unnecessary communication paths. 
    \item Monitoring: Detect unusual patterns in resource usage or timing. 
    \item Access Controls: Restrict access to critical resources. 
    \item Noise Introduction: Add random delays or variations to disrupt timing-based channels.
\end{itemize}
\vspace{2ex}  
% Characteristics 
%Un Covert Channel viene chiamato così perchè è nascosoto dai meccanismi di controllo degli accessi dei sistemi operativi siccome non usa il meccanismo di trasferimento dati legittimo del sistema (lettura/scrittura) 
%e quindi non possono essere identificati o controllati dai meccanismi di sicurezza alla base della sicurezza dei sistemi operativi. 
%\vspace{1ex} \newline
%I Covert Channel sono eccessivamente difficili da installare nei sistemi reali e spesso possono essere identificati monitorando le performance del sistema. 
%In aggiunta soffrono da un basso 'signal-to-noise ratio' e 'low data rates'. 
%Possono essere anche rimossi manualmente con un alto grado di fiducia da sistemi sicuri, da strategie di analisi ben definite. 
%\vspace{1ex} \newline 
%
%I Covert Channel possono passare attraverso sistemi operativi sicuri a richiedono misure speciali per poterli controllare; 
%in particolare l'analisi dei Covert Channel è l'unico modo provato per poterli gestire. 
%Al contrario, sistemi operativi sicuri possono facilemnte identificare e/o prevenire l'uso improprio di canali legittimi.
%
%\subsubsection*{Identifying Covert Channels}  
%\subsubsection*{Data hiding in OSI model}  
%\subsubsection*{DATA hiding in LAN environment by covert channels}  
%\subsubsection*{Data hiding in TCP/IP Protocols suite}   

\subsection{Aree di ricerca sui covert Channel} 
Una significante area di ricerca riguarda lo sviluppo di meccanismi di comunicazione nascosta in ambienti wireless.
Ad esempio, uno studio ha introdotto un metodo di comunicazione unidirezionale wireless nascosto che utilizza 
gli intervalli di beacon dei punti di accesso nelle reti IEEE 802.11.
\vspace{1ex} \newline 
Questo metodo, noto come canale di temporizzazione nascosto ping-pong (PPCTC), mira a ridurre al minimo le 
possibilità di rilevamento garantendo al contempo una trasmissione dati affidabile, anche in presenza di errori.
\vspace{1ex} \newline  
Questa innovazione dimostra il potenziale dei Covert Channel per essere efficacemente integrati nei protocolli 
di rete esistenti con modifiche minime.
\vspace{3ex} \newline 
Un altro aspetto critico dei Covert Channel è la loro individuazione. 
Poiché le comunicazioni segrete diventano sempre più avanzate e più difficili da identificare, i ricercatori 
stanno esplorando le tecniche di apprendimento automatico (ML) per migliorare le capacità di rilevamento. 
\vspace{2ex} \newline
Una revisione ha evidenziato vari tipi di Covert Channel e l'efficacia di diversi approcci ML 
nell'identificazione di queste minacce nascoste. 
Lo studio ha sottolineato la necessità di una ricerca continua per migliorare i metodi di rilevamento, 
poiché le misure di sicurezza tradizionali spesso non riescono a riconoscere le comunicazioni nascoste
\vspace{2ex} \newline
Inoltre, l'uso di protocolli Internet of Things (IoT) per l'esfiltrazione dei dati ha attirato l'attenzione.
La ricerca ha dimostrato che protocolli come MQTT e AMQP sono efficaci per i trasferimenti di 
dati nascosti grazie alla loro progettazione per una larghezza di banda ridotta e un consumo energetico 
ridotto. %rendendoli adatti agli ambienti IoT.
Uno strumento software sviluppato per questo scopo ha dimostrato come questi protocolli potrebbero essere 
sfruttati per trasferimenti di dati non autorizzati, sottolineando la necessità di meccanismi di rilevamento 
robusti nelle reti IoT.
\vspace{2ex} \newline
Inoltre, un'analisi a lungo termine della suscettibilità di Internet ai Covert Channel ha rivelato che 
l'evoluzione dei protocolli di rete ha influenzato l'efficacia delle tecniche di occultamento delle informazioni.
Questo studio ha suggerito che il monitoraggio continuo e la quantificazione delle capacità dei canali nascosti dovrebbero essere
integrali alle strategie di sicurezza informatica
\vspace{2ex} \newline 
Infine, un'analisi specifica delle minacce si è concentrata sull'uso delle scansioni delle porte come 
copertura per canali di comando e controllo nascosti. 
Questa ricerca ha proposto un nuovo metodo per nascondere le informazioni all'interno delle scansioni delle 
porte TCP e dei messaggi syslog, fornendo intuizioni su potenziali indicatori di compromesso e strategie di mitigazione
