\subsection{Cos'è un Covert Cahnnel?}
Un \textbf{Covert Channel} è un attacco che permette (in ambienti ritenuti sicuri) la capacità di comunicare/trasferire dati, in maniera non autorizzata e non voluta, 
fra processi/entità comunicanti spesso senza essere rivelati e spesso evitando (se non violando) le normali politiche di sicurezza.  
\vspace{2ex} \newline
Solitamente operano al di fuori dei soliti meccanismi di comunicazioni. 
Quindi non usando i normali protocolli/canali di comunicazione (es network sockets, emails) non generano segnali di un uso improprio del sistema. 
Ciò li rende difficili da rilevare usando i tipici strumenti di monitoraggio. %Inoltre non usando spesso protocolli di comunicazione standard (es TCP/IP),  
Inoltre questi canali sfruttano le vulnerabilità o comportamenti non previsti nei sistemi. 
\vspace{2ex} \newline
In un Covert Channel, qualsiasi risorsa condivisa può essere utilizzata come canale nascosto ed è per questo che possono esistere 
in qualunque sistema (che abbia delle risorse condivise). 
Lo sfruttamento di queste risorse porta alla fuoriuscita/scambio di dati. %non autorizzata.  
\vspace{2ex} \newline 
L'attacco è un problema siccome sono estremamente difficili da identificare e controllare. 
La loro esistenza spesso rimane non notata dagli amministratori (di sistema) siccome si nascondono all'interno dei 
normali processi del sistema. 
Sono inoltre un problema significativo in tutti quegli ambienti altamente sicuri (es ambienti militari, governativi,\dots) 
dove una fuoriuscita di informazioni può avere conseguenze gravi.  
%
\subsection{Principali categorie di Covert Channel}
Le principali categorie di canali nascosti sono: 
\begin{itemize}
    \item Covert Channel Timing (Temporizzazione): 
    \newline 
    coinvolgono la scrittura di dati a un'area di memoria condivisa in cui entrambi i processi possono accedere 
    \begin{esempio}
        Modificare i permessi dei file o i metadati per codificare informazioni. 
        Oppure modificare variabili condivise o buffer
    \end{esempio}
    \item Covert Channel Storage (Archiviazione): 
    \newline 
    manipolano la temporalizzazione o l'ordine di eventi per codificare informazioni. 
    \begin{esempio}
        Variare deliberatamente il tempo fra delle azioni (es trasmissione di network packet, patter di uso della CPU) 
        oppure codificando dati nella temporalizzazione dell'esecuzione dei processi o delay di risposta. 
    \end{esempio}
    \item Covert Channel Behavioral (Comportamentali) 
\end{itemize}
\subsubsection{Covert Channel Timing (temporizzazione)}
I canali nascosti di temporizzazione sono metodi di comunicazione che permettono ad un osservatore (umano o processo) di acquisire 
informazioni attraverso il cambiamento nel tempo di risposta di una risorsa.
Essenzialmente qualsiasi metodo che utilizza un orologio o una misurazione del tempo per segnalare il valore inviato sul canale.
\href{https://www.youtube.com/watch?v=QIvsmQQ6vu8}{Esempio}
%
\subsubsection{Covert Channel Storage (Archiviazione)}
Nei canali nascosti di archiviazione un processo scrive su una risorsa condivisa, mentre un altro processo legge da 
essa. I canali di archiviazione possono essere utilizzati tra processi all’interno di un singolo computer o tra più 
computer in una rete.
\vspace{1ex} \newline 
I veicoli dell'attacco sono tutte quelle risorse che consentono la scrittura, diretta o indiretta, di una risorsa 
da parte di un processo e la sual lettura, diretta o indiretta, da parte di un altro.
\begin{esempio}
    Un esempio di canale di archiviazione è la condivisione di un file. 
    Supponiamo che l’utente A con privilegi di autorizzazione elevati voglia trasmettere in segreto, dati riservati all’utente B con un livello di sicurezza inferiore. 
    Per farlo, utilizzerà un file word apparentemente contenente informazioni non classificate, dove invece occulterà l’informazione riservata. 
\end{esempio}
%
\subsubsection{Covert Channel Behavioral (Comportamentali)}
I canali nascosti comportamentali operano trasmettendo dati in base all’assegnazione di diversi eventi di processi, sistemi e 
applicazioni, generalmente suddividendo e trasmettendo i dati in pacchetti più piccoli.
%\vspace{1ex} \newline
%Tra i più affascinanti canali nascosti comportamentali, in cui mi sono imbattuto, quello che utilizza il protocollo 
%ICMP (Internet Control Message Protocol), sfrutta appieno caratteristiche quali la rapidità di attivazione, semplicità d’uso e 
%bypassa molte delle policy e standard di sicurezza con estrema nonchalance, manifestando in maniera ottimale le caratteristiche di 
%Stealthiness e Indistinguishability.
%L’ICMP è progettato per fornire feedback su problemi di comunicazione di una rete TCP/IP. 
%L’ ICMP si affida al supporto di base dell’IP come parte di protocollo di livello superiore. 
%A causa di questa dipendenza, sia ICMPv4 che ICMPv6 esistono per entrambe le versioni di IP. 
%Applicazioni come ad es. traceroute e ping utilizzano i messaggi ICMP per raccogliere informazioni e diagnosticare eventuali 
%problemi di rete. 
%\vspace{1ex} \newline
%I canali nascosti spesso sfruttano per i propri flussi di informazioni, alcune caratteristiche tecniche incorporate nelle reti 
%IEEE 802, caratteristiche che normalmente non vengono “viste” a livello di rete più alto perché considerate di servizio.
%\vspace{1ex} \newline
%L’idea di utilizzare l’ICMP come canale nascosto è quindi quella di sfruttare una comunicazione standard con un protocollo inferiore 
%rispetto a TCP o UDP. Questo avrà un “ingombro” ridotto in tutto il traffico di rete e potrà passare inosservato agli amministratori 
%di rete e agli analizzatori di traffico, in quanto, come detto, normalmente è utilizzato per la diagnostica e manutenzione della rete 
%e degli host connessi, non per il trasporto dati, quindi difficilmente bloccato da policy di sicurezza.
%Questo rende il protocollo ICMP un canale nascosto decisamente praticabile, l’uso di campi dati o payload all’interno di determinati 
%messaggi ICMP permette di incorporare il messaggio del canale nascosto facilmente e trasforma paradossalmente l’ICMP stesso in un 
%canale nascosto. Questi semplici fattori consentono all’ICMP di essere di fatto un traffico invisibile, vediamo un esempio. 
%
%---
%
\subsection{Struttura/Caratteristiche dei Covert Channel} 
Tipicamente è costituito da due principali componenti: 
\begin{itemize}
    \item \textbf{Mittente} (Covert Transmitter): è l'entita che codifica e trasmette le informazioni nascote usando una risorsa di sistema condivisa.  
    \item \textbf{Destinatario} (Covert Listener): è l'entità che rileva e decifra l'informazione segreta dalla risorsa condivisa. 
\end{itemize}
%
\subsubsection*{Come funzionano i Covert Channel?} 
Il mittente inserisce informazioni segrete in un componente del sistema che è osservabile da un destinatario. 
Il destinatario decifra i dati trasmessi di nascosto monitorando i cambiamenti nel comportamento del sistema. 
\vspace{1ex} \newline
Le informaizoni vengono inserite sfruttano gli effetti collaterali delle normali operazione del sistema 
senza un esplicito intento di comunicare.  %(delay o scrittura su file condivisi) 
\vspace{1ex} \newline  
Un Covert Channel quindi opera cifrando dati nascosti nei comportamenti del sistema che i controlli di sicurezza 
tipicamente non monitorano così da permettere la comunicazione segreta fra due entità. 
%è strutturato come un sistema di comunicazione segreta che bypassa i normali meccanismi di sicurezza. 
%
\subsubsection*{Caratteristiche Chiave dei covert Channel} 
Le principali caratteristiche di un Covert Channel sono:  \newline
$\bullet$ \textbf{Stealthiness} (furtività):  \newline
si devono poter aggirare i controlli in maniera nascosta
\vspace{2ex} \newline
$\bullet$ \textbf{Bandwith} (capacità di trasmissione):  \newline
la capacità di trasmissione dei dati che è generalmente bassa in termini di dati/tempo (throughput). 
Un eccessivo carico di informazioni, potrebbe rendere anomalo il funzionamento di quelle risorse o delle normali 
strutture dati. %Si stanno utilizzando altre risorse
Nei canali nascosti generalmente il throughput è inversamente correlato alla segretezza di un canale.
\begin{center}
    Più dati un canale trasmette in un determinato periodo di tempo, maggiore è il rischio che il canale venga scoperto
\end{center}
\vspace{2ex}
$\bullet$ \textbf{Indistinguishability} (Indistinguibilità):  \newline
di solilto si sfruttano servizi e/o risorse già presenti e quindi non sospette. 
\vspace{1ex} \newline 
Uno dei maggiori problemi nell’implementazione di un canale nascosto è il “rumore”
(es. sfruttando eccessivamente le risorse alterando e/o danneggiando il corretto funzionamento delle stesse)
che potrebbe attirare l'attenzione da parte degli amministratori di sistema. 
La necessità è quella di riuscire a trasmette attraverso un canale nascosto mantenendo conforme e inalterato il 
funzionamento della risorsa utilizzata così da rendersi “indistinguibili” dalla risorsa autorizzata e quindi invisibili ai sistemi di monitoraggio.
\vspace{4ex} \newline
Ulteriori caratteristiche sono: 
\vspace{2ex} \newline
$\bullet$ \textbf{Unintended Use of Resources}: Covert channels exploit system resources 
(e.g shared memory, CPU usage, or file attributes) in ways that were not intended for communication. 
\newline
$\bullet$ \textbf{Hidden Communication}: They are designed to avoid detection, often leveraging 
legitimate system operations to mask the transmission of data. 
\vspace{2ex} \newline
$\bullet$ \textbf{Violation of Security Policies}: They allow unauthorized exchange of information, 
potentially breaching confidentiality, integrity, or availability requirements.   
\vspace{2ex} \newline
$\bullet$ \textbf{Hidden Communication Medium}: 
\newline 
The channel is embedded within legitimate system operations, such as CPU load, memory access, network traffic, or file system metadata. 
Examples: CPU cache, TCP/IP headers, power consumption, packet timing.
\vspace{2ex} \newline
$\bullet$  \textbf{Encoding Mechanism}: 
\newline 
The sender manipulates a shared system resource to encode data. 
Common techniques:
\begin{itemize}
    \item Timing-Based Encoding: Using time intervals (e.g., delays between network packets). 
    \item Storage-Based Encoding: Modifying file attributes, memory bits, or cache states. 
    \item Protocol Abuse: Altering TCP flags, sequence numbers, or unused bits in packet headers. 
\end{itemize} 
$\bullet$  \textbf{Transmission Mechanism}: 
\newline 
The sender continuously alters system behavior to transmit bits of information. 
This can be done by introducing delays, changing CPU load, or modifying memory states in a controlled manner. 
\newline
$\bullet$  \textbf{Decoding Mechanism}: 
\newline 
The receiver monitors the shared resource to detect and reconstruct the transmitted data. 
Example: Measuring execution time variations to infer secret data. 
\newline
$\bullet$  \textbf{Synchronization and Error Correction}: 
\newline 
The sender and receiver must synchronize (e.g., using pre-agreed timing signals). 
Error detection mechanisms (such as parity bits or checksums) ensure accurate data retrieval.
\begin{esempio}{Esempio di un Covert channel in una rete} \newline
    $\bullet$ Sender: Modifies the Time-to-Live (TTL) field in IP packets to represent binary data (e.g., TTL=64 → bit 1, TTL=128 → bit 0). 
    \newline
    $\bullet$ Receiver: Observes incoming packet TTL values to reconstruct the hidden message.
\end{esempio}
%
\subsection{Vulnerabilità Utilizzate}
I Covert Channel sfruttano le vulnerabilità nel design del sistema, nelle politiche di sicurezza e nei protocolli di comunicazione per trasferire informazioni segretamente. 
Sfruttando queste vulnerabilità, gli attaccanti possono stabilire Covert Channel che evitano il controllo degli standard  sicurezza, 
permettendo esfiltrazione non autorizzata di dati o comunicazione fra processi interni (inter-process comunicaizone). 
\vspace{2ex}\newline
La loro mitigazione richiede controllo degli accessi, randomizzazione dei tempi, iniezione di rumore e una sicurezza hardware migliore. 
\subsubsection*{Principali vulnerabilità usate dai covert Channel} 
$\bullet$ \textbf{Shared Resource Exploitation} 
\begin{itemize}
    \item CPU Scheduling: l'attaccante può modulare l'uso della CPU per diffondere informazioni. 
    \item Cache Memory: gli attacchi side-channel alla cache sfruttano le differenze nei tempi di accesso per dedurre i dati. 
    \item File System Access: i processi possono dedurre informazioni in base ai lock dei file, timestamp o sull'attività del disco
\end{itemize}
\vspace{1ex}  
$\bullet$ \textbf{Timing-Based Vulnerabilities} 
\begin{itemize}
    \item \textbf{Response Time Variability}: l'attacante misura i tempi di risposta del sistema per estrarre segreti. 
    \item \textbf{Instruction Execution Delays}: le differenze del tempo di esecuzione tra le operazioni privilegiate e non possono causare la fuoriuscita di dati. 
\end{itemize}
\vspace{1ex}  
$\bullet$ \textbf{Timing-Based Vulnerabilities} 
\begin{itemize}
    \item \textbf{Packet Timing}: le informazioni possono essere codificate negli intervalli durante la trasmissione dei pacchetti 
    \item \textbf{Header Manipulation}: campi come TTL, sequenza dei numeri o bit non utilizzati possono essere utilizzati per codificare i dati 
    \item \textbf{Traffic Patterns}: le variazioni nel flusso del traffico (es burst size) si possono comportare come un Covert Channel. 
\end{itemize}
\vspace{1ex}  
$\bullet$ \textbf{Memory and CPU State Manipulation} 
\begin{itemize}
    \item Branch Prediction and Speculative Execution: Exploited in attacks like Spectre and Meltdown.
    \item Power Consumption Analysis: Side channels can reveal cryptographic keys
\end{itemize}
\vspace{1ex}  
$\bullet$ \textbf{Operating System and Virtualization Flaws} 
\begin{itemize}
    \item Inter-Process Communication (IPC) Abuse: Processes can infer data through shared memory or message passing.
    \item Hypervisor Weaknesses: Virtual machines can leak information across guest instances.
\end{itemize}
\vspace{1ex}  
$\bullet$ \textbf{Hardware Vulnerabilities} 
\begin{itemize}
    \item Electromagnetic Emissions: Sensitive data can be leaked via EM signals (TEMPEST attacks).
    \item Acoustic Side-Channels: Keyboard sounds, fan speed variations, or power supply noise can be analyzed.
\end{itemize}
%
\subsection{Applicazione dei Covert Channel}
I Covert Channel sono spesso applicati in: 
\begin{itemize}
    \item Malware and Espionage: usati per esfiltrare dati sensibili. 
    \item Security Testing: identificare e mitigare i Covert Channel è una parte fondamentale nel stabilire la sicurezza del sistema. 
    \item Ricerca: espolare i Covert Channel aiuta a capire potenziali vulnerabilità in sistemi complessi. 
\end{itemize} 
%
\subsection{Covert Channel Attacks} 
Gli attachi tramite Covert Channel sfruttano vulnerabilita nel design del sistema, nelle risorse condivise e nelle
politiche di sicurezza per trasmettere segretamente dati fra processi o sistemi aggirando i tradizionali 
controlli di sicurezza. 
Questi attacchi sono spessi usati per l'esfiltrazione dei dati, privilege escaletion o comunicazioni 
silenzione tra delle componenti malware. 
\subsubsection*{Types of Covert Channel Attacks} 
\begin{enumerate}
    \item \textbf{Storage-Based Covert Channels}: 
    These attacks manipulate shared system resources to store and retrieve hidden information.
    \begin{esempio}{\quad1\newline}
        File Attribute Manipulation: Malware alters file metadata (e.g., timestamps, permissions) to encode messages. 
        \newline 
        Shared Memory Exploitation: Processes communicate by modifying shared memory regions. 
        \newline 
        Disk Usage Signaling: One process writes or deletes data, and another process detects changes. 
        \newline 
        TCP/IP Header Fields: Attackers encode data in unused or optional fields of network packets (e.g., IP ID, sequence numbers, or TTL values).
    \end{esempio}
    \item \textbf{Timing-Based Covert Channels}: 
    These attacks manipulate system timing or performance to transmit hidden information. 
    \begin{esempio}{\quad\newline}
        CPU Load Fluctuation: Malware alters CPU usage patterns, which another process measures to decode information.
        \newline 
        Network Packet Timing: The sender transmits packets at specific time intervals to encode binary data.
        \newline 
        Cache-Based Attacks: Attackers use cache access times (e.g., Flush+Reload, Prime+Probe) to leak secrets
        \newline 
        Power Consumption Analysis: Sensitive data is extracted by analyzing power usage variations (used in side-channel cryptographic attacks).
    \end{esempio} 
\end{enumerate}
\subsubsection*{Notable Covert Channel Attacks} 
\begin{table}
    \begin{tabular}{|p{0.3\linewidth}|p{0.3\linewidth}|p{0.3\linewidth}|}
        \hline
        Attack Name & Type & Description \\
        \hline 
        Spectre and Meltdown & Timing (Cache) & Exploit speculative execution to leak memory contents \\ 
        \hline 
        Flush+Reload & Timing (Cache) & Attacker flushes shared memory and reloads it to observe access patterns. \\
        \hline 
        Prime+Probe & Timing (Cache) & Attacker fills cache and monitors eviction patterns to infer secret data. \\
        \hline 
        TCP Covert Channel & Storage (Network) & Encodes data in TCP packet fields (e.g., sequence numbers, flags). \\
        \hline 
        File Lock Covert Channel & Storage (Filesystem) & Uses file locking/unlocking as a signaling mechanism. \\
        \hline 
        Packet Timing Attack & Timing (Network) & Varies packet transmission timing to send hidden messages. \\ 
        \hline 
        Keystroke Timing Attack & Timing (Human Interaction) & Infers typed keys based on timing variations between keystrokes. \\
        \hline 
    \end{tabular}
\end{table}
\subsubsection*{Real-World Examples of Covert Channel Attacks}
\begin{itemize}
    \item Malware-Based Attacks: 
    \newline 
    Duqu 2.0 (2015): Used covert TCP/IP channels to exfiltrate data while avoiding detection.
    DNS Tunneling Attacks: Malware hides data inside DNS queries (e.g., C2 communication for botnets). 
    \item Cloud \& Virtualization-Based Covert Channels: 
    \newline 
    Hypervisor Covert Channels: Virtual machines (VMs) on the same physical host leak data via shared CPU cache or memory. 
    Cloud Timing Attacks: Cloud tenants use execution timing differences to infer co-resident VM activities. 
\end{itemize}
\subsubsection*{Mitigation Strategies}
Gli attacchi tramite Covert Channel sfruttano le debolezze, del timing del sistema, delle risorse condivise e dei protocolli di rete, 
per trasmettere dati nascosti. 
Pongono una seria minaccia nella comunicazione fra malware, esfiltrazione dei dati e il cyber-spionaggio. 
Difese efficaci implicano l'isolamento delle risorse, iniezione di rumore e rilevamento delle anomalie così da disturbare questi attacchi. 
\textbf{System-Level Defenses}
\begin{itemize}
    \item Enforce strong access control (MAC, RBAC) to prevent unauthorized process interactions. 
    \item Use mandatory data flow control models (Bell-LaPadula, Biba) to prevent information leaks. 
    \item Disable unnecessary shared resources (e.g., inter-process communication, shared memory).
\end{itemize}
\textbf{Network Defenses}
\begin{itemize}
    \item Deploy deep packet inspection (DPI) and anomaly detection to identify hidden data in network traffic. 
    \item Enforce network segmentation to limit unauthorized data flows.
\end{itemize}
\textbf{Hardware \& OS Defenses}
\begin{itemize}
    \item Randomize execution times and inject noise into system responses (to disrupt timing-based attacks). 
    \item Implement constant-time cryptographic operations to prevent timing side channels. 
    \item Flush and partition CPU caches to prevent cross-process cache attacks. 
\end{itemize}
%
\subsection{Protezione contro i Covert Channel}
Il loro rilevametno e la loro mitigazione richiede un rigoroso monitoraggio, l'isolamento delle risorse e 
tecniche per introdurre rumore. 
%
I Covert channel sfruttano le vulnerabilità del sistema per trasmettere segretamente dei dati. 
Proteggersi da loro, richiede una combinazione di rinforzo delle politiche, gestione delle risorse e tecniche di monitoraggio. 
%
Mitigare i Covert channel richiede una sicurezza multi livello fra hardware, OS, applicazioni e reti. 
Siccome la completa eliminazione è difficle, strategie di rilevazione e minimizzazione sono essenziali 
(es randomizzazione, rigoroso controllo degli accessi delle risorse, rilevamento delle anomalie). 
%
Le effettive strategie per la mitigazione sono: 
\subsubsection*{System and Policy-Based Protections}
\begin{enumerate}
    \item \textbf{Strict Access Control Policies}: 
    \newline 
    Implement least privilege and mandatory access control (MAC) to restrict unauthorized communication between processes. 
    Use sandboxing and compartmentalization to isolate processes. 
    \item \textbf{Information Flow Control}: 
    \newline 
    Enforce data flow policies to prevent high-security processes from leaking data to lower-security processes (Bell-LaPadula, Biba model). 
    \item \textbf{Process Separation and Isolation}: 
    \newline 
    Use virtualization and containerization to separate processes. 
    Apply air-gapping for highly sensitive systems
\end{enumerate}
\subsubsection*{Timing and Resource Management Protections} 
\begin{itemize}
    \item \textbf{Randomization Techniques }
    \newline 
    Introduce noise in system responses (e.g., randomizing execution times, adding delays) to disrupt timing-based covert channels. 
    Use cache randomization or flush techniques to prevent cache-based side-channel attacks. 
    \item \textbf{Rate Limiting and Bandwidth Control} 
    \newline 
    Restrict CPU, memory, or network bandwidth to limit the capacity of a covert channel. 
    Implement throttling mechanisms for shared resources.
\end{itemize}
\subsubsection*{Network Security Protections} 
\begin{itemize}
    \item \textbf{Packet Inspection and Filtering}: 
    \newline 
    Use Deep Packet Inspection (DPI) to detect anomalous patterns in network traffic. 
    Block or sanitize unused fields in protocols (e.g., TCP/IP headers). 
    \item \textbf{Traffic Analysis and Anomaly Detection}: 
    \newline 
    Employ AI-based monitoring to detect unusual communication patterns. 
    Use intrusion detection systems (IDS) and log analysis to identify suspicious activities.
\end{itemize} 
\subsubsection*{Hardware and Software Security Enhancements} 
\begin{itemize}
    \item Secure Hardware Design 
    \newline 
    Implement constant-time cryptographic operations to prevent timing-based attacks. 
    Use secure enclaves (e.g., Intel SGX, ARM TrustZone) to protect sensitive computations. 
    \item OS-Level Protections 
    \newline 
    Enforce memory isolation and disable shared memory where unnecessary. 
    Implement secure scheduling algorithms to prevent process-based timing leaks. 
\end{itemize} 
\subsubsection*{Covert Channel Auditing and Testing} 
\begin{itemize}
    \item Regularly perform covert channel analysis in penetration tests. 
    \item Use covert channel detection tools (e.g., network flow analysis, system behavior monitoring). 
\end{itemize}
%
\subsection{Strategie di mitigazione}
\textbf{Strategie di Mitigazione} 
\newline 
Controllo sugli Accessi: limnitare i permessi per prevenire scambio di informazioni non autorizzato 
Monitoraggio del Traffico: analizzare i comportamenti del sistema per rilevare anomalie 
Aggiunta di Rumore (Noise Injection): introdurre casualità nei pattern temporali o di accesso alla memoria per rendere il prelevamento dei dati difficile.  
%
Strategie di mitigazione: 
\begin{itemize}
    \item System Design: Minimize shared resources and unnecessary communication paths. 
    \item Monitoring: Detect unusual patterns in resource usage or timing. 
    \item Access Controls: Restrict access to critical resources. 
    \item Noise Introduction: Add random delays or variations to disrupt timing-based channels.
\end{itemize}
\vspace{2ex}  
% Characteristics 
%Un Covert Channel viene chiamato così perchè è nascosoto dai meccanismi di controllo degli accessi dei sistemi operativi siccome non usa il meccanismo di trasferimento dati legittimo del sistema (lettura/scrittura) 
%e quindi non possono essere identificati o controllati dai meccanismi di sicurezza alla base della sicurezza dei sistemi operativi. 
%\vspace{1ex} \newline
%I Covert Channel sono eccessivamente difficili da installare nei sistemi reali e spesso possono essere identificati monitorando le performance del sistema. 
%In aggiunta soffrono da un basso 'signal-to-noise ratio' e 'low data rates'. 
%Possono essere anche rimossi manualmente con un alto grado di fiducia da sistemi sicuri, da strategie di analisi ben definite. 
%\vspace{1ex} \newline 
I Covert Channel possono passare attraverso sistemi operativi sicuri a richiedono misure speciali per poterli controllare; 
in particolare l'analisi dei Covert Channel è l'unico modo provato per poterli gestire. 
Al contrario, sistemi operativi sicuri possono facilemnte identificare e/o prevenire l'uso improprio di canali legittimi.
\vspace{2ex} \newline 
\subsubsection*{Identifying Covert Channels} 
PASS 
\subsubsection*{Eliminating Covert Channels} 
Le possibilità di un Covert Channel non possono essere eliminate sebbene possano essere significatamente ridotte da un design e analisi attenti. 
La rilevazione di un Covert Channel può essere resa maggiormente difficile usando caratteristiche del medium di comunicazione per il canale legittimo che non sono mai controllati o esaminati da utenti legittimi. 
\begin{esempio}
    Un file può essere aperto e chiuso da un progremma in modo specifico pattern temporale così che possa essere rilevato da un altro programma; 
    lo schema potrà essere poi interpretato come una stringa di bit formando così un Covert Channel. 
    Di conseguenza, siccome è improbabile che l'utente legittimo controlli i pattern relativi alla chiusura/apertura dei file; 
    questo tipo ti Covert Channel può rimanere non identificato per un lungo periodo. 
\end{esempio}
\vspace{2ex} 
\subsubsection*{Data hiding in OSI model} 
PASS 
\vspace{2ex} \newline 
\subsubsection*{DATA hiding in LAN environment by covert channels} 
PASS 
\vspace{2ex} \newline 
\subsubsection*{Data hiding in TCP/IP Protocols suite} 
PASS 
%






\subsection{Aree di ricerca sui covert Channel}

One significant area of research involves the development of covert communication mechanisms in wireless 
environments. 
For instance, a study introduced a covert wireless unidirectional communication method that utilizes the beacon 
intervals of access points in IEEE 802.11 networks. This method, known as the ping-pong covert timing channel 
(PPCTC), aims to minimize the chances of detection while ensuring reliable data transmission, even in the presence 
of errors[1]. 
This innovation demonstrates the potential for covert channels to be effectively integrated into existing network 
protocols with minimal modifications.

Another critical aspect of covert channels is their detection. As covert communications become more advanced and 
harder to identify, researchers are exploring machine learning (ML) techniques to enhance detection capabilities. 
A review highlighted various types of covert channels and the effectiveness of different ML approaches in 
identifying these hidden threats. The study emphasized the need for ongoing research to improve detection methods, 
as traditional security measures often fail to recognize covert communications[2].
Additionally, the use of Internet of Things (IoT) protocols for data exfiltration has garnered attention. 
Research has shown that protocols like MQTT and AMQP are particularly effective for covert data transfers due 
to their design for low bandwidth and power consumption, making them suitable for IoT environments. 
A software tool developed for this purpose demonstrated how these protocols could be exploited for unauthorized 
data transfers, underscoring the need for robust detection mechanisms in IoT networks[3].
Furthermore, a long-term analysis of the Internet's susceptibility to covert channels revealed that the evolution 
of network protocols has influenced the effectiveness of information hiding techniques. 
This study suggested that continuous monitoring and quantification of covert channel capabilities should be 
integral to cybersecurity strategies[4].
Lastly, a specific threat analysis focused on using port scans as a cover for covert command and control channels. 
This research proposed a novel method for hiding information within TCP port scans and syslog messages, providing 
insights into potential indicators of compromise and mitigation strategies[5].


























