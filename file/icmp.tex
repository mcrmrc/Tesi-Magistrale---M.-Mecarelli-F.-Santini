ICMP (Internet Control Message Protocol) è un protocollo a livello rete utilizzato 
per la diagnostica, per la segnalazione di errori, per ottenere informazioni di controllo e per la 
risoluzione dei problemi nelle reti. 
Aiuta i dispositivi (come i router e gli host) a comunicare, gestire e risolvere i 
problemi della rete ma non è utilizzato per la trasmissione di dati (come TCP o UDP).
\vspace{2ex} \newline
Sebbene sia essenziale per la diagnostica di rete e la segnalazione di errori; 
può essere utilizzato in modo improprio per degli attacchi o per la ricognizione della rete (network reconnaissance). 
Gli aggressori possono utilizzare ICMP per attacchi DDoS, di ricognizione, di esfiltrazione di dati o di covert channel.
%Le regole del firewall e la limitazione della velocità aiutano a bilanciare usabilità e sicurezza. 
%
\subsubsection*{Differenze tra ICMP, TCP e UDP}
\begin{minipage}{\linewidth}
    \begin{tabular}{|p{0.25\linewidth}|p{0.25\linewidth}|p{0.25\linewidth}|p{0.25\linewidth}|}
        \hline
        \textbf{Funzionalità} & \textbf{ICMP} & \textbf{TCP} & \textbf{UDP} \\
        \hline \hline
        Scopo & Segnalazione di errori e diagnostica & Trasferimento di dati affidabile & Trasferimento di dati veloce e senza connessione \\
        \hline
        Orientato alla connessione? & No & Sì & No \\
        \hline
        Numeri di porta? & No & Sì & Sì \\
        \hline
        Affidabilità & No & Sì (Acknowledgments) & No \\
        \hline
        Utilizzato da & Ping, Traceroute, PMTUD & HTTP, FTP, Email & DNS, VoIP, Streaming \\
        \hline
    \end{tabular}
\end{minipage}
%
\subsection*{Caratteristiche di ICMP}
\begin{itemize}
    \item Opera al Livello 3 (Livello di rete) nel modello OSI.
    \item Funziona con IP per fornire feedback sui problemi di rete.
    \item Non stabilisce una sessione (Stateless e Connectionless).
    \item Nessun numero di porta (a differenza di TCP e UDP).
    \item Utilizzato per la risoluzione dei problemi di rete (e.g. esempio, ping, traceroute).
    \item Supporta IPv4 (ICMPv4) e IPv6 (ICMPv6) con funzionalità avanzate in ICMPv6.
\end{itemize}
\vspace{2ex} 
ICMP è utilizzato principalmente per:
\begin{itemize}
    \item Segnalazione errori: informa il mittente sui problemi di rete (ad esempio, destinazione non raggiungibile, perdita di pacchetti).
    \item Diagnostica di rete: aiuta nella risoluzione dei problemi di rete utilizzando strumenti come ping e traceroute.
    \item Messaggistica di controllo: gestisce la congestione della rete e gli aggiornamenti di routing in alcuni casi.
\end{itemize}
\subsection*{Struttura di un messaggio ICMP} 
Ogni messaggio ICMP è composto da:
\begin{itemize}
    \item Tipo - Identifica il tipo di messaggio (ad esempio, Echo Request, Destinazione irraggiungibile).
    \item Codice - Fornisce dettagli aggiuntivi sul tipo di messaggio.
    \item Checksum - Garantisce l'integrità dei dati.
    \item Dati - Opzionale, può contenere parte del pacchetto IP originale che ha causato l'errore.
\end{itemize}
\subsubsection*{Formato dell'intestazione ICMP}
\begin{center}
    +-+-+-+-+-+-+-+-+-+-+-+-+-+-+-+-+ \\
    | Type | Code | Checksum        | \\
    +-+-+-+-+-+-+-+-+-+-+-+-+-+-+-+-+ \\
    | Additional Data (if required)  | \\
    +-+-+-+-+-+-+-+-+-+-+-+-+-+-+-+-+
\end{center}
\vspace{2ex}  
I messaggi ICMP sono classificati o come messaggi di errore o come messaggi informativi
\vspace{1ex} 
\begin{itemize}
    \item \textbf{Messaggi di errore} - Segnalano problemi nella comunicazione di rete.
    \item \textbf{Messaggi informativi} - Utilizzati per scopi diagnostici e di controllo.
\end{itemize}
\vspace{2ex} 
\begin{minipage}{\linewidth}
    \subsubsection*{Error Messages}
    \begin{tabular}{|p{0.3\linewidth}|p{0.3\linewidth}|p{0.3\linewidth}|p{0.4\linewidth}|}
        \hline 
        Type & Code & Meaning \\
        \hline
        3 & 0-15 & Destination Unreachable (e.g., no route to host, port unreachable) \\
        \hline 
        4 & 0 & Source Quench (deprecated, used to indicate congestion) \\
        \hline 
        5 & 0-3 & Redirect Message (suggesting a better route) \\
        \hline 
        11 & 0-1 & Time Exceeded (TTL expired, used in traceroute) \\
        \hline 
        12 & 0-1 & Parameter Problem (invalid IP header) \\ 
        \hline 
    \end{tabular}
\end{minipage} 
\begin{minipage}{\linewidth}
    \subsubsection*{Error Messages}
    \begin{tabular}{|p{0.25\linewidth}|p{0.25\linewidth}|p{0.25\linewidth}|p{0.25\linewidth}|}
        \hline 
        Type & Code & Message Name & Description \\
        \hline
        3 & 0 & Network Unreachable & No route to destination network. \\
        \hline
        3 & 1 & Host Unreachable & No route to specific host. \\
        \hline
        3 & 3 & Port Unreachable & Destination port is closed. \\
        \hline
        3 & 4 & Fragmentation Needed & Packet needs fragmentation, but DF bit is set. \\
        \hline
        4 & 0 & Source Quench (Deprecated) & Indicates network congestion. \\
        \hline
        5 & 0-3 & Redirect Message & Suggests a better route for packets. \\
        \hline
        11 & 0 & Time Exceeded & TTL expired before reaching the destination (used in traceroute). \\
        \hline
        12 & 0-1 & Parameter Problem & Invalid IP header field. \\
        \hline
    \end{tabular}
\end{minipage} 
\vspace{3ex} \newline 
\begin{minipage}{\linewidth}
    \subsubsection*{Informational Messages}
    \begin{tabular}{|p{0.25\linewidth}|p{0.25\linewidth}|p{0.25\linewidth}|p{0.25\linewidth}|}
        \hline 
        Type & Code & Message Name & Description \\
        \hline
        0 & 0 & Echo Reply & Response to a ping request. \\
        \hline
        8 & 0 & Echo Request & Used by ping to test connectivity. \\
        \hline
        9 & 0 & Router Advertisement & Routers announce themselves to hosts. \\
        \hline
        10 & 0 & Router Solicitation & Hosts request router advertisements. \\
        \hline 
    \end{tabular}
\end{minipage} 
%
\subsubsection*{Utilizzi di ICMP nelle reti} 
\begin{enumerate}
    \item Ping (Richiesta Echo ICMP e Risposta Echo) 
    \vspace{1ex} \newline 
    Il comando \textbf{ping}, invia pacchetti ICMP Echo Request per testare la connettività.
    \begin{itemize}
        \item Invia delle richieste Eco ICMP a una destinazione per verificare la connettività.
        \item Se l'host è raggiungibile, risponde con un ICMP Echo Reply.
    \end{itemize}
    \item Traceroute (tracert su Windows, traceroute su Linux/macOS) 
    \vspace{1ex} \newline 
    Il comando \textbf{traceroute}, utilizza messaggi ICMP Time Exceeded per mappare il percorso dei pacchetti.
    \begin{itemize}
        \item Tramite i messaggi ICMP Time Exceeded traccia il percorso che i pacchetti seguono attraverso una rete
        \item Il valore TTL (Time-To-Live) viene incrementato per determinare ciascun router lungo il percorso.
    \end{itemize}
    \item Scoperta del percorso MTU (PMTUD) 
    \vspace{1ex} \newline 
    La \textbf{PMTUD}, utilizza messaggi ICMP Fragmentation Needed per ottimizzare le dimensioni dei pacchetti. 
    Ovvero per trovare la dimensione ottimale del pacchetto per un percorso di rete.
\end{enumerate} 
% ICMP in IPv6 (ICMPv6)
% ICMPv6 extends ICMP functionality for IPv6 networks, including:
%\begin{itemize}
%    \item Neighbor Discovery Protocol (NDP) – Replaces ARP for IPv6 
%    \item Router Advertisements & Solicitation – Helps configure IPv6 addresses. 
%    \item MLD (Multicast Listener Discovery) – Manages multicast group memberships
%\end{itemize}
%
\subsection{Possibili attacchi tramite ICMP}
\subsection*{Attacchi Denial-of-Service (DoS/DDoS)}
\subsubsection*{ICMP Flood (Ping Flood)}
Sopraffare un bersaglio con richieste Echo
\begin{itemize}
    \item Attacco: \newline 
    L'attaccante invia un gran numero di richieste di ICMP Echo (richieste di ping) a un sistema bersaglio.
    Se il sistema risponde con risposte ICMP Echo, consuma potenza di elaborazione e larghezza di banda. 
    Se più macchine attaccano contemporaneamente, si parla di un attacco DDoS (Distributed DoS) ICMP Flood.
    \item Mitigazione: \newline 
    Limitare la velocità del traffico ICMP su firewall e router.
    Disattivare le richieste di eco ICMP dalle reti esterne se non necessarie. 
    Utilizzare sistemi di rilevamento delle intrusioni (IDS) per monitorare le richieste di ping eccessive.
\end{itemize} 
\subsubsection*{Attacco Smurf} 
Richieste ICMP contraffatte amplificano il traffico verso una vittima.
\begin{itemize}
    \item Attacco: \newline 
    L'aggressore invia richieste ICMP Echo con un IP sorgente falsificato (l'IP della vittima). 
    Le richieste vengono inviate a un indirizzo broadcast, provocando la risposta di tutti gli host della rete.
    La vittima viene sommersa da risposte ICMP Echo, che portano a una condizione DoS.
    \item Mitigazione: \newline 
    Disabilitare le richieste di broadcast ICMP sui router (nessuna trasmissione diretta IP)
    Implementare filtri in ingresso per bloccare i pacchetti con indirizzi di origine falsificati. 
    Utilizzare le regole del firewall per bloccare il traffico ICMP non necessario.
\end{itemize} 
\subsubsection*{Ping della morte (attacco storico)} 
invio di pacchetti ICMP di grandi dimensioni per mandare in crash i sistemi
\begin{itemize}
    \item Attacco: 
    L'attaccante invia un pacchetto ICMP sovradimensionato ($>$ 65.535 byte) 
    causano crash da buffer overflow nei sistemi vulnerabili.
    I sistemi operativi più vecchi potrebbero crashare, bloccarsi o riavviarsi quando gestiscono tali pacchetti.
    \item Mitigazione: 
    I sistemi moderni rifiutano i pacchetti di dimensioni eccessive.
    Applicare aggiornamenti e patch di sistema per prevenire questa vulnerabilità.
\end{itemize} 
\subsubsection*{ICMP Unreachable Flood}
\begin{itemize}
    \item Attacco: \newline 
    L'attaccante invia un numero massiccio di messaggi ICMP Destination Unreachable.
    Può sovraccaricare i dispositivi di rete e causare un denial of service. 
    \item Mitigazione: \newline 
    Configurare limiti di velocità per i messaggi di errore ICMP.
    Implementare regole firewall per eliminare il traffico ICMP eccessivo
\end{itemize}
%
\subsection*{Attacchi di ricognizione}
\subsubsection*{ICMP Ping Sweep}
\begin{itemize}
    \item Attacco: \newline 
    L'aggressore invia richieste ICMP Echo a più host su una rete.
 Sulla base delle risposte, l'attaccante identifica gli host attivi per ulteriori attacchi.
    \item Mitigazione: \newline 
    Blocca le richieste ICMP Echo da fonti esterne. 
    Utilizzare sistemi di prevenzione delle intrusioni (IPS) per rilevare e bloccare attività di scansione sospette.
\end{itemize} 
\subsubsection*{Attacco Timestamp ICMP}
\begin{itemize}
    \item Attacco: \newline 
    Le richieste ICMP Timestamp (tipo 13) consentono agli aggressori di determinare il tempo di attività del sistema.
    Queste informazioni aiutano gli aggressori a individuare i sistemi vulnerabili o riavviati di recente.
    \item Mitigazione: \newline 
    Disattivare le richieste di timestamp ICMP su firewall e router.
 Utilizzare protocolli di sincronizzazione temporale (NTP) con autenticazione anziché query orarie basate su ICMP.
\end{itemize} 
\subsubsection*{Attacco ICMP che maschera l'indirizzo}
\begin{itemize}
    \item Attacco: \newline 
    L'aggressore invia una richiesta di mascheramento dell'indirizzo ICMP (tipo 17) a un bersaglio.
    Se l'obiettivo risponde con la sua maschera di sottorete (subnet mask), rivela i dettagli della rete all'attaccante.
    \item Mitigazione: \newline 
    Disattivare le risposte ICMP Address Mask a meno che non siano necessarie per le operazioni di rete.
    Utilizzare i firewall per filtrare il traffico ICMP proveniente da fonti non attendibili.
\end{itemize} 
\subsection*{Attacchi ICMP Tunneling e Covert Channel}
\subsubsection*{ICMP Tunneling} 
Covert Channel che utilizzano pacchetti ICMP per aggirare i firewall.
\begin{itemize}
    \item Attacco: \newline 
    Gli attaccanti incapsulano dati dannosi all'interno delle richieste e delle risposte ICMP Echo.
    I dati sono incorporati nei pacchetti ICMP per poter aggirare i firewall che consentono il traffico ICMP 
    (ma bloccano le connessioni TCP/UDP) ed esfiltrare così le informazioni.
    Spesso utilizzato per comunicazioni segrete in malware e canali C2 (comando e controllo). 
    \item Esempi di strumenti: 
    \begin{itemize}
        \item Icmpsh - Crea una reverse shell utilizzando ICMP. 
        \item PingTunnel - Incanala il traffico TCP attraverso pacchetti ICMP.
    \end{itemize} 
    \item Mitigazione: \newline 
    Ispezione approfondita dei pacchetti (DPI) per rilevare ICMP Tunneling. 
    Blocca le richieste/risposte di ICMP Echo da reti non attendibili.
    Monitorare il traffico di rete per individuare modelli ICMP insoliti.
\end{itemize} 
\subsubsection*{Esfiltrazione ICMP (furto di dati tramite ICMP)}
\begin{itemize}
    \item Attacco: \newline 
    Gli attaccanti inseriscono dati sensibili (password, file, comandi) all'interno dei pacchetti ICMP.
    I dati vengono inviati a un server esterno controllato dall'attaccante.
    \item Mitigazione: \newline 
    Monitorare e registrare il traffico ICMP per rilevare attività anomale.
    Utilizzare i firewall per limitare il traffico ICMP solo ai dispositivi necessari.
    Utilizzare soluzioni DLP (Data Loss Prevention) per rilevare i tentativi di esfiltrazione.
\end{itemize}
\subsubsection*{ICMP Covert Channels}
\begin{itemize}
    \item Attacco: \newline
    Malware e attaccanti utilizzano pacchetti ICMP per stabilire un canale di comunicazione nascosto.
    Spesso utilizzato nella comunicazione C2 per botnet o operazioni di malware furtive.
    \item Mitigazione: \newline
    Monitorare il traffico ICMP per individuare modelli di utilizzo insoliti.
    Utilizzare i sistemi di rilevamento delle intrusioni di rete (NIDS) per rilevare Covert Channel.
    Limitare la comunicazione ICMP tra reti interne ed esterne.
\end{itemize}
\subsubsection*{Attacco di reindirizzamento ICMP}
\begin{itemize}
    \item Messaggi di reindirizzamento ICMP non autorizzati reindirizzano il traffico 
    verso un gateway dannoso.
    \item Mitigazione: disabilitare il reindirizzamento ICMP.
\end{itemize}
%
\subsection{Buona practice di sicurezza}
Per prevenire gli attacchi basati su ICMP; buone misure di sicurezza sono: 
\begin{itemize}
    \item limitare e filtrare l'utilizzo di ICMP tramite i firewall 
    \item la limitazione della velocità 
    \item il monitoraggio del traffico (tramite strumenti di sicurezza) per rilevare le anomalie
\end{itemize} 
\subsubsection*{Regole del firewall}
Limitare o bloccare il traffico ICMP non necessario. %sui firewall. 
%Consentire solo i messaggi ICMP strettamente necessari (e.g. Echo Reply, Destinazione non raggiungibile, ma non Redirect).
%Bloccare i tipi di ICMP non necessari sui firewall (ad esempio, Redirect, Timestamp, Source Quench). 
\begin{itemize}
    \item Bloccare le richieste Echo di ICMP da reti esterne, a meno che non siano necessarie.
    \item Disabilitare le risposte a ICMP Timestamp e Address Mask per impedire la ricognizione.
    \item Consentire solo i messaggi di errore ICMP necessari (ad esempio, Destinazione non raggiungibile).
    \item Eliminare i messaggi di reindirizzamento ICMP per impedire la manipolazione dell'instradamento (del routing).
\end{itemize}
\subsubsection*{Limitazione della velocità} 
Limitare la velocità delle richieste ICMP. %per evitare di essere sopraffatti
\begin{itemize}
    \item Limita il numero di pacchetti ICMP al secondo per prevenire la sovrastazione. 
    \item Configura i criteri di limitazione della velocità ICMP su router e firewall.
\end{itemize}
\subsubsection*{Monitoraggio della rete e Rilevamento}
%Utilizzare sistemi di rilevamento delle intrusioni (IDS) per monitorare attività ICMP sospette. 
\begin{itemize}
    \item Utilizzare i sistemi di rilevamento delle intrusioni (IDS/IPS) per rilevare abusi del protocollo ICMP.
    \item Analizza i registri di rete per attività ICMP insolite (ad esempio, pacchetti ICMP di grandi dimensioni, ping frequenti).
    \item Implementa l'ispezione approfondita dei pacchetti (DPI) per identificare il Tunneling ICMP.
\end{itemize}
\subsubsection*{Rafforzamento del sistema}
\begin{itemize}
    \item Mantieni aggiornati i sistemi e il firmware per correggere le vulnerabilità ICMP note
    \item Disattivare i servizi ICMP sui sistemi critici se non necessari.
    \item Utilizzare soluzioni di sicurezza degli endpoint per rilevare malware che utilizzano ICMP per la comunicazione
\end{itemize} 
