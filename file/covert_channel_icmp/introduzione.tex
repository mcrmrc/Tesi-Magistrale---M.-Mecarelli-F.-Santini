\subsection{Cos'è un covert Channel ICMP}
I Covert Channel ICMP utilizzano pacchetti ICMP (tipicamente richieste e risposte di eco) per nascondere 
i dati all'interno di campi che normalmente vengono ignorati o non monitorati. 
Ciò può essere sfruttato per comunicazioni nascoste per esfiltrare dati, per operazioni di comando e controllo (C2),\dots.
\vspace{1ex} \newline
Ciò consente agli attaccanti di trasferire dati in modo da aggirare le politiche di sicurezza (e.g i firewall) ed eludono il rilevamento.
Quest'ultimo punto è possibile siccome:
\begin{itemize}
    \item Molti firewall e dispositivi di sicurezza consentono il traffico ICMP per la diagnostica della rete.
    \item I pacchetti ICMP possono trasportare dati (payload) nascosti senza destare sospetti.
    \item I sistemi di sicurezza tradizionali si concentrano sul traffico TCP/UDP, trascurando ICMP.
\end{itemize}
\vspace{1ex} 
Pongono seri rischi per la sicurezza, consentendo l'esfiltrazione furtiva dei dati, il tunneling e la 
comunicazione di malware. 
Implementando rigide restrizioni ICMP, l'ispezione approfondita dei pacchetti, le regole del firewall e il 
rilevamento delle anomalie, le organizzazioni possono rilevare e mitigare efficacemente queste minacce.
%Le organizzazioni devono monitorare il traffico ICMP, limitarne l'uso e utilizzare strumenti di sicurezza per 
%rilevare e bloccare efficacemente i covert channel. 
\vspace{2ex} \newline
La comunicazione in questi tipi di attacchi avviene in questo modo:
\begin{itemize}
    \item \textbf{Codifica dei dati}: gli aggressori incorporano messaggi nascosti all'interno di pacchetti ICMP, come richieste di Eco (ping) o risposte di Eco.
    \item \textbf{Evasione del firewall}: poiché ICMP è spesso consentito nei firewall, gli aggressori lo utilizzano per aggirare le politiche di sicurezza.
    \item \textbf{Comunicazione furtiva}: malware e botnet utilizzano poi ICMP per comunicare segretamente con un attaccante remoto.
\end{itemize} 
