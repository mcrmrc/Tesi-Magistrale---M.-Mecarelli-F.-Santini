\subsection{Strategie di rilevamento} 
\begin{minipage}{\linewidth}
    \begin{tabular}{|p{0.3\linewidth}|p{0.3\linewidth}|p{0.3\linewidth}|}
        \hline 
        \textbf{Tecnica} & \textbf{Rilevamento} & \textbf{Mitigazione} \\
        \hline \hline 
        Analisi del traffico di rete & Identifica anomalie nel volume e nei pattern ICMP & Limita i tipi ICMP non necessari \\
        \hline 
        Deep Packet Inspection (DPI) & Rileva l'esfiltrazione e il tunneling dei dati & Blocca i pacchetti ICMP con payload inattesi \\
        \hline 
        IDS/IPS (Snort, Zeek) & Segnala comportamenti ICMP insoliti & Blocca le richieste ICMP sospette \\
        %Utilizza la Deep Packet Inspection (DPI) & identifica i dati nascosti nei pacchetti ICMP. \\
        %Implementa regole IDS/IPS per ICMP & avvisi su attività ICMP sospette \\ 
        \hline 
    \end{tabular} 
    \captionof{table}{Strumenti di rilevamento}
\end{minipage}
\subsubsection{Monitoraggio del traffico di rete} 
Analizzare il volume e le dimensioni dei pacchetti ICMP (ad esempio, payload insolitamente grandi) per eventuali anomalie.
Rileva il traffico ICMP ad alta frequenza verso host esterni sconosciuti.
Verificare la presenza di pacchetti ICMP con payload insolitamente grandi (e.g tentativi di esfiltrazione dei dati) o con 
schemi irregolari (e.g valori TTL variabili). 
Pacchetti ICMP con modifiche costanti del payload potrebbero indicare il trasferimento di dati nascosti. 
%
\subsubsection{Deep Packet Inspection (DPI)}
Esaminare il contenuto del payload ICMP per rilevare eventuali dati incorporati insoliti (messaggi codificati, crittografia o anomalie).
Contrassegna i pacchetti ICMP che contengono risposte non standard (e.g, una risposta Echo contenente dati inaspettati).
Identificare schemi di comunicazione con indirizzi IP esterni tramite ICMP
%
\subsubsection{Sistemi di rilevamento e prevenzione delle intrusioni (IDS/IPS)}  
Utilizzare Snort, Suricata o Zeek per rilevare e segnalare attività ICMP sospette
\begin{esempio}{Regola Snort per il rilevamento del tunneling ICMP} 
    \begin{lstlisting}{shell}
        alert icmp any any -> any any ( 
            msg:"ICMP tunnel detected"; 
            content:"malicious_payload"; 
            sid:100001; 
        )
    \end{lstlisting}
\end{esempio}  
%
\subsubsection{Rilevamento basato su anomalie} 
Rilevare il traffico ICMP che potrebbe indicare una comunicazione C2 implementando analisi 
comportamentali che possano rilevare un utilizzo anomalo di ICMP.
Utilizzare strumenti di apprendimento automatico o SIEM (Security Information and Event 
Management) per segnalare deviazioni nell'utilizzo di ICMP. 
%  
\subsection{Strategie di mitigazione}
\begin{minipage}{\linewidth}
    \begin{tabular}{|p{0.5\linewidth}|p{0.5\linewidth}|}
        \hline
        \textbf{Metodo di mitigazione} & \textbf{Effetti} \\
        \hline \hline 
        Disattiva ICMP se non necessario & Impedisce la maggior parte degli attacchi basati su ICMP \\
        \hline 
        Limita ICMP ai tipi necessari & blocca i vettori di attacco non necessari \\
        \hline 
        Limitazione della velocità & Impedisce il flooding e il tunneling ICMP \\ %Rileva richieste ICMP eccessive &
        \hline 
        Regole del firewall & Blocca l'ICMP in uscita dai sistemi critici \\ % Contrassegna le richieste ICMPS non autorizzate &
        \hline 
        Blocca ICMP in uscita dai firewall & Impedisce perdite di dati tramite ICMP \\
        \hline 
        Endpoint Security (EDR) & Previene l'esecuzione dannosa di ICMP \\ %Rileva malware tramite Covert Channel ICMP & 
        \hline 
    \end{tabular} 
    \captionof{table}{Metodologie di mitigazione}
\end{minipage}
\subsubsection{Restringere/ Limitare il traffico ICMP}
Disattivare ICMP sui server e sugli endpoint a meno che non sia esplicitamente necessario e 
bloccare il traffico ICMP proveniente da fonti non attendibili. %sul firewall.
Configurare firewall e router in modo tale da consentire solo i messaggi ICMP necessari 
(e.g Destinazione non raggiungibile, Tempo Scaduto).
Disattivare le richieste/risposte di eco ICMP sui sistemi critici.
\begin{esempio}{Regola del firewall per bloccare il traffico ICMP}\newline
    \small
    \begin{lstlisting}{bash}
        iptables -A INPUT -p icmp --icmp-type echo-request -j DROP
    \end{lstlisting}
\end{esempio}
%
\subsubsection{Limitazione della velocità del traffico ICMP}
Limitare la frequenza e la dimensione dei pacchetti ICMP per evitare il trasferimento nascosto di dati.
Configurare i firewall in modo da consentire solo un numero specifico di pacchetti ICMP al secondo.
%Esempio: Configurare i firewall per consentire solo un certo numero di richieste ICMP al secondo.
\begin{esempio}{Regola del firewall per limitare il traffico ICMP}
    \small
    \begin{lstlisting}{bash}
        iptables -A INPUT -p icmp -m limit --limit 1/second -j ACCEPT
    \end{lstlisting}
\end{esempio}
\subsubsection{Utilizza la crittografia per prevenire la fuga di dati}
%Impedire agli aggressori di intercettare dati sensibili crittografando tutte le 
%comunicazioni legittime (ad esempio tramite VPN, TLS).
Implementa la crittografia TLS/SSL per tutte le comunicazioni legittime così da impedire agli 
aggressori di utilizzare ICMP per l'esfiltrazione. 
Inoltre bloccare le trasmissioni non autorizzate di testo in chiaro su ICMP.
%
\subsubsection{Blocca ICMP su interfacce esterne}
Impedisci il traffico ICMP in uscita dalle reti interne per fermare l'esfiltrazione.
Consenti ICMP solo per scopi diagnostici interni.
%
\subsubsection{Sicurezza degli endpoint \& Antivirus}
%Utilizzare firewall basati sull'host per bloccare le comunicazioni ICMP sospette.
Implementare strumenti antivirus e soluzioni EDR (Endpoint Detection \& Response) per 
rilevare le minacce informatiche che utilizzano i covert channel ICMP per comunicare.
%Aggiorna regolarmente il software antivirus per identificare e bloccare le minacce note
%
\subsection*{Implementa ICMP Proxy Filtering}
Utilizza proxy ICMP per ispezionare, sanificare e bloccare payload ICMP inaspettati.
Consenti solo il passaggio di traffico ICMP diagnostico legittimo