\subsection{Attacchi Covert Channel ICMP} 
\begin{minipage}{\linewidth}  
    \begin{tabular}{|p{0.5\linewidth}|p{0.5\linewidth}|} 
        \hline
        \textbf{Tipo di attacco} & \textbf{Descrizione} \\
        \hline \hline 
        Tunneling ICMP & Incapsulamento del traffico TCP/IP all'interno di pacchetti ICMP per eludere le restrizioni del firewall \\
        \hline 
        Esfiltrazione dati ICMP & Invio di dati rubati nascosti all'interno di payload ICMP a un server esterno. \\
        \hline 
        Comando e controllo (C2) con ICMP & Malware che riceve comandi da un aggressore tramite ICMP. \\
        \hline 
        ICMP Reverse Shell & Una backdoor che consente a un aggressore di controllare una macchina da remoto tramite ICMP. \\
        \hline 
    \end{tabular}
    \captionof{table}{Esempi di attacchi Covert Channel ICMP}
    \label{table:esempi:attacchiCC_ICMP}
\end{minipage} 
%
\subsubsection{ICMP Tunneling}
Il tunneling ICMP consente agli aggressori di incapsulare i dati all'interno dei pacchetti ICMP, creando un canale di comunicazione nascosto.
\begin{enumerate}
    \item L'attaccante inserisce istruzioni di comando e controllo (C2) nei pacchetti ICMP.
    \item Questi pacchetti vengono inviati a un sistema compromesso dietro un firewall.
    \item Il sistema estrae le istruzioni nascoste e le esegue.
    \item Le risposte vengono inviate tramite ICMP Echo Replies
\end{enumerate}
\begin{esempio}{Esempio di un caso d'uso}\newline
    I malware (ad esempio le botnet) utilizzano il protocollo ICMP per aggirare i firewall e ricevere comandi da aggressori remoti.
    Gli attaccanti stabiliscono una reverse shell tramite ICMP, controllando una macchina compromessa. 
\end{esempio}
\begin{esempio}{Esempi di Strumenti per il tunneling ICMP}\newline
    Icmpsh - Crea una shell inversa tramite ICMP.
    PingTunnel – Incanala il traffico TCP attraverso richieste e risposte di eco ICMP.
    Ptunnel-NG – Versione avanzata di PingTunnel per aggirare i firewall
\end{esempio}
%
\subsubsection{Esfiltrazione dei dati ICMP}
Gli aggressori possono rubare dati (password, file, informazioni sensibili) incorporandoli nei pacchetti ICMP 
e inviandoli a un server esterno.
\begin{enumerate}
    \item L'aggressore codifica dati sensibili (ad esempio numeri di carte di credito, chiavi di crittografia) in pacchetti ICMP.
    \item I pacchetti vengono inviati a un server esterno controllato dall'aggressore.
    \item L'aggressore estrae e decodifica i dati rubati dal traffico ICMP.
\end{enumerate}
\begin{esempio}{Esempio di caso d'uso}\newline
    Una minaccia interna estrae dati classificati tramite richieste ICMP Echo.
    Un'infezione da malware trasmette keylog o screenshot tramite pacchetti ICMP
\end{esempio}
\begin{esempio}{Esempio di strumenti per l'esfiltrazione di dati con ICMP}\newline
    icmptx - Codifica e trasferisce dati tramite pacchetti ICMP.
    LOKI - Nasconde i dati nelle risposte ICMP Echo.
    Hans - Utilizza ICMP per il trasferimento di dati criptati.
\end{esempio}
%
\subsubsection{Comando e controllo (C2) della botnet basato su ICMP}
Alcune botnet e malware utilizzano ICMP per comunicare con i loro server %di comando e controllo (C2)
\begin{enumerate}
    \item L'attaccante inserisce i comandi C2 nei pacchetti ICMP.
    \item Il bot infetto legge il comando e lo esegue.
    \item Il bot invia i risultati dell'esecuzione tramite risposte ICMP
\end{enumerate}
\begin{esempio}{Esempio di malware che utilizzano ICMP per la comunicazione C2}\newline
    Duqu – Utilizza ICMP per inviare dati crittografati.
    Pingback - Un malware che riceve comandi tramite ICMP.
    Trojan.Medo - Utilizzava ICMP come canale backdoor.
\end{esempio}  
%\subsection{Esempio reale di attacco tramite covert channel ICMP}
%Caso di studio: Duqu Malware (2011)
%\begin{itemize}
%    \item Cosa è successo?
%    Duqu, un malware sofisticato, utilizza pacchetti ICMP per esfiltrare dati dai sistemi infetti
%    \item Come funziona:
%    Incorpora dati rubati all'interno di richieste ICMP Echo inviate a un server remoto.
%    Gli strumenti di sicurezza non sono riusciti a rilevarlo perché ICMP era considerato innocuo
%    \item Mitigazione:
%    Le organizzazioni hanno imparato a monitorare il traffico ICMP e a bloccare i messaggi ICMP non necessari per prevenire futuri attacchi
%\end{itemize}  
%