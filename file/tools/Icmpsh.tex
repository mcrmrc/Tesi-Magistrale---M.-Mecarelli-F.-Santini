%https://github.com/skmt/icmpsh NO

%https://www.kitploit.com/2015/06/icmpsh-simple-reverse-icmp-shell.html
%https://github.com/nelsonmaligro/icmpshell 
%https://github.com/hemp3l/icmpsh  
%https://github.com/bdamele/icmpsh 

%La versione di python usata è la 2 non funziona

\subsection{Description}
icmpsh is a simple reverse ICMP shell with a win32 slave and a POSIX compatible master in C, Perl or Python. 
The main advantage over the other similar open source tools is that it does not require administrative privileges to run onto the target machine.
The tool is clean, easy and portable. 
The slave (client) runs on the target Windows machine, it is written in C and works on Windows only whereas the master (server) can run on any platform on the attacker machine as it has been implemented in C and Perl. 

\subsection{Features}
\begin{itemize}
    \item Open source software - primarily coded by Nico, forked by me.
    \item Client/server architecture.
    \item The master is portable across any platform that can run either C, Perl or Python code.
    \item The target system has to be Windows because the slave runs on that platform only for now.
    \item The user running the slave on the target system does not require administrative privileges.
\end{itemize} 

\textbf{Running the master} 
The master is straight forward to use. 
There are no extra libraries required for the C and Python versions. 
The Perl master however has the following dependencies:
\begin{itemize}
    \item IO::Socket
    \item NetPacket::IP
    \item NetPacket::ICMP
\end{itemize}
When running the master, don't forget to disable ICMP replies by the OS. 
For example: sysctl -w net.ipv4.icmp\_echo\_ignore\_all=1
If you miss doing that, you will receive information from the slave, but the slave is unlikely to receive commands send from the master.

\textbf{Running the slave} 
The slave comes with a few command line options as outlined below:
-t host            host ip address to send ping requests to. This option is mandatory!
-r                 send a single test icmp request containing the string "Test1234" and then quit. 
                   This is for testing the connection.
-d milliseconds    delay between requests in milliseconds 
-o milliseconds    timeout of responses in milliseconds. 
    If a response has not received in time, the slave will increase a counter of blanks. 
    If that counter reaches a limit, the slave will quit.
    The counter is set back to 0 if a response was received.
-b num             limit of blanks (unanswered icmp requests before quitting)
-s bytes           maximal data buffer size in bytes
In order to improve the speed, lower the delay (-d) between requests or increase the size (-s) of the data buffer.
\vspace{2ex} \newline
icmp-slave-complete.c : Hard coded values
For the ease of execution, I have hard coded the values of target, delay, timeout, data buffer. 
It will help to execute the binary directly without command line arguments.
Check line number 186 to 197.
\begin{itemize}
    \item target: IP address of attacker's machine
    \item delay: delay between requests in milliseconds
    \item timeout: timeout in milliseconds
    \item max\_blanks: maximal number of blanks (unanswered icmp requests)
    \item max\_data\_size: maximal data buffer size in bytes
\end{itemize}


\begin{lstlisting}
    git clone https://github.com/bdamele/icmpsh

    cd icmpsh
\end{lstlisting}


\section{icmpshell}
icmpsh is a simple reverse ICMP shell with a win32 slave and a POSIX compatible master in C, Perl or Python. 
The main advantage over the other similar open source tools is that it does not require administrative privileges to run onto the target machine.

The tool is clean, easy and portable. 
The slave (client) runs on the target Windows machine, it is written in C and works on Windows only whereas the master (server) can run on any platform on the attacker machine as it has been implemented in C and Perl by Nico Leidecker and I have ported it to Python too, hence this GitHub fork.


Running the master
The master is straight forward to use. 
There are no extra libraries required for the C and Python versions. 
The Perl master however has the following dependencies:
\begin{itemize}
    \item IO::Socket
    \item NetPacket::IP
    \item NetPacket::ICMP
\end{itemize}
When running the master, don't forget to disable ICMP replies by the OS. 
For example: sysctl -w net.ipv4.icmp\_echo\_ignore\_all=1
If you miss doing that, you will receive information from the slave, but the slave is unlikely to receive commands send from the master.


Running the slave
The slave comes with a few command line options as outlined below:
-t host            host ip address to send ping requests to. This option is mandatory!
-r                 send a single test icmp request containing the string "Test1234" and then quit. 
                   This is for testing the connection.
-d milliseconds    delay between requests in milliseconds 
-o milliseconds    timeout of responses in milliseconds. If a response has not received in time, 
                   the slave will increase a counter of blanks. If that counter reaches a limit, the slave will quit.
                   The counter is set back to 0 if a response was received.
-b num             limit of blanks (unanswered icmp requests before quitting)
-s bytes           maximal data buffer size in bytes
In order to improve the speed, lower the delay (-d) between requests or increase the size (-s) of the data buffer.


icmp-slave-complete.c : Hard coded values
For the ease of execution, I have hard coded the values of target, delay, timeout, data buffer. 
It will help to execute the binary directly without command line arguments.

Check line number 186 to 197.
\begin{itemize}
    \item target: IP address of attacker's machine
    \item delay: delay between requests in milliseconds
    \item timeout: timeout in milliseconds
    \item max\_blanks: maximal number of blanks (unanswered icmp requests)
    \item max\_data\_size: maximal data buffer size in bytes
\end{itemize}