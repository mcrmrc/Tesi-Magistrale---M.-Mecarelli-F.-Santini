Analizziamo dei programmi già presenti per identificare le loro caratterisitche e per capire quali metodi vengnao utilizzatiu per la creazione del covert Channel e lo scambio di messaggi.
Per ognuno di essi, dove possibile, si effettuerà una richiesta alla vittima;  
prima partendo da un comando leggero (e.g \textbf{pwd}) per poi passare a comandi sempre più difficili e quindi maggiormente rilevabili (e.g \textbf{cd /path; ls -l} oppure \textbf{cat ./file\_grande}, \dots)
\vspace{1ex} \newline
Gli strumenti analizzati sono i seguenti:
%\begin{itemize}
%	\item C
%\end{itemize}
\subsubsection*{$\bullet$ICMP Door} 
\begin{itemize}
    \item[] \textbf{PRO}: Tramite delle Echo Reply crea un canale di comunicazione tra attaccante e vittima. 
    In particolare realizza una reverse shell sulla quale inviare comandi e ricevere risposte.
    \item[] \textbf{CONS}: Vengono trasmesse in sequenza molteplici Echo Reply, e il campo Data contiene la risposta in chiaro e quindi facilmente rilevabile. 
    Inoltre l'identificatore della sessione rimane invariato. 
    %Quindi se mai la comunicazione venisse scoperta la vuittima potrebbe filtrare i pacchetti con quello specifico ID
\end{itemize}
Se un agente monitorasse il flusso dei dati, in quel momento, noterebbe il numero di Echo Reply sospetto soprattuto se non è presente una Echo Request analoga.
Inoltre non essendo i dati crittografati, gli basta analizzare il pacchetto per vedere cosa viene inviato.
Potrebbe quindi facilmente scoprire il canale nascosto. 
\vspace{1ex} \newline
Di solito per ogni Echo Request corrisponde una singola Echo Reply in cui la risposta rimanda i dati ricevuti. 
Il campo Data di solito o è vuoto o contiene frasi gia preimpostate (e.g '\textit{helloworld}'). 

\subsubsection*{$\bullet$ICMP Exfil}  
\begin{itemize}
    \item[] \textbf{PRO}: Tramite la temporalizzazione delle Echo Request crea un canale di comunicazione tra attaccante e vittima. 
    Anche se un IDS analizzasse il pacchetto non troverebbe anomalie in esso. 
    \item[] \textbf{CONS}: Più lento rispetto alla classica modifica dei dati del pacchetto ma acnhe più silensioso. 
    %Quindi se mai la comunicazione venisse scoperta la vuittima potrebbe filtrare i pacchetti con quello specifico ID
\end{itemize}
Testandolo non funzionava: non si riusciva a passare i i dati sebbene fossero lettere e numeri. 
Tuttavia la teoria è valida.

\subsubsection*{$\bullet$icmpsh}  
\begin{itemize}
    \item[] \textbf{PRO}: Non richiede i privilegi di amministratore. Crea un canale tramite le Echo Reply
    \item[] \textbf{CONS}: Gira solo su Windows. Ergo la machcina vittima deve avere Windows
    %Quindi se mai la comunicazione venisse scoperta la vuittima potrebbe filtrare i pacchetti con quello specifico ID
\end{itemize}
Scritto in C, Python e Perl.

\subsubsection*{$\bullet$ICMP Shell}  

\subsubsection*{$\bullet$ICMP Tunnel}  
\begin{itemize}
    \item[] \textbf{PRO}: Tramite la delle Echo Reply crea un canale di comunicazione tra attaccante e vittima. 
    Inoltre un proxy comunica con la vittima (al posto dell'attaccante) e i dati vengono cifrati.
    \item[] \textbf{CONS}: Le Echo Reply, se numerose, destano sospetti. 
    %Quindi se mai la comunicazione venisse scoperta la vuittima potrebbe filtrare i pacchetti con quello specifico ID
\end{itemize}


