Prima di sviluppare un covert Channel Channel che potesse esfiltrare i dati dalla macchina vittima; 
si sono analizzati strumenti già presenti per studiarne il comportamento. 
\vspace{2ex} \newline
Da ciò si è arrivati alle seguenti conclusioni: 
\begin{enumerate}
    \item Il bisogno di un proxy intermediario fra la vittima e l'attaccante così da nasconderne l'entita. 
    \item Un numero di proxy che permetta la distribuzione omogenea del traffico generato e non generare un throughput elevato (dato il numero di messaggi scambiati)
    \item  Un limite alla dimensione dei dati inviati. 
    Se mai si volesse esfiltrare un file contenente una grande quantità di dati; 
    questo potrebbe generare rumore e destare sospetti. 
    \item Un periodo di riposo randomico dopo ogni richiesta fatta dall'attaccante. 
    Maggiore è il numero di richieste maggiore questo valore dovrà crescere così da non far notare la propria presenza. 
    Questo valore può variare anche in base a quanti messaggi sono stati scambiati con la vittima.
    \item Il testo scambiato non deve essere in chiaro. 
    Così da non permettere di leggere da sistemi di monitoraggio, ciò che viene mandato. 
    %\item 
\end{enumerate}
%l'attaccante si può connettere ai proxy. 
%Si può quindi utilizzare una tipologia di comunicazione meno nascosta e più diretta.
%Si possono quindi creare dei Socket


%https://noxenius.medium.com/icmpdoor-how-i-used-an-icmp-reverse-shell-12814ca1b0e7
\subsection{Detection and Mitigation}
Network administrators and security engineers should limit or deny ICMP traffic as much as possible. 
When this is not feasible due to protocol requirements or network planning, scope the accepted source and destination of ICMP packets. 
This blog post elaborates on how to configure this with iptables.
%TO-DO