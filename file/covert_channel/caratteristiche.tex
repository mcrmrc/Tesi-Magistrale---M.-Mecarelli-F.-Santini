\subsection{Caratteristiche chiave dei covert Channel} 
$\bullet$ \textbf{Stealthiness} (furtività):  \newline
Si devono poter aggirare i controlli in maniera nascosta.
\vspace{2ex} \newline
$\bullet$ \textbf{Bandwith} (capacità di trasmissione):  \newline
La capacità di trasmissione viene espressa in termini di \textbf{throughput} ($\frac{dati}{tempo}$). 
E un eccessivo carico di informazioni, potrebbe rendere anomalo il funzionamento delle risorse.
\vspace{1ex} \newline
Quindi il throughput è inversamente correlato alla segretezza di un canale: 
più dati un canale trasmette in un determinato periodo di tempo, maggiore è il rischio che il canale venga scoperto
\vspace{2ex} \newline
$\bullet$ \textbf{Indistinguishability} (Indistinguibilità):  \newline
Di solilto si sfruttano servizi e/o risorse già presenti e quindi non sospette; 
uno dei maggiori problemi nell'implementazione di un canale nascosto è il “rumore” che potrebbe attirare l'attenzione 
da parte degli amministratori (es. se si sfruttano eccessivamente le risorse). 
\vspace{1ex} \newline
La necessità è quella di riuscire a trasmette attraverso un canale nascosto mantenendo conforme e inalterato il 
funzionamento della risorsa utilizzata così da rendersi “indistinguibili” dalla risorsa autorizzata e di 
conseguenza invisibili ai sistemi di monitoraggio.
Per evitare la rilevazione, il canale è incorporato in operazioni di sistema legittime per poter mascherare la trasmissione dei dati. 
(e.g carico della CPU, accesso alla memoria, traffico della rete, metadati del file systema). 
\vspace{4ex} \newline
Ulteriori caratteristiche sono: 
\vspace{2ex} \newline
$\bullet$ \textbf{Uso involontario delle risorse}: \newline
I Covert channels sfruttano le risorse del sistema (e.g memoria cxondivisa, uso della CPU, attributi dei file) 
in maniere non previste per la comunicazione. 
\vspace{2ex} \newline
$\bullet$ \textbf{Violazione delle politiche di sicurezza}: \newline 
Permettono lo scambio non autorizzato di informazioni, potenzialmente violando 
i requisiti di confidenzialità, di integrità o quelli di disponibilità. 