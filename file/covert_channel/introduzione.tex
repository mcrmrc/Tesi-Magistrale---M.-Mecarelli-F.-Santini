\subsection{Cos'è un Covert Channel?}
Un \textbf{Covert Channel} è un attacco che permette (in ambienti ritenuti sicuri) la capacità di comunicare e/o trasferire dati (in maniera non autorizzata e non voluta), 
fra processi e/o entità comunicanti spesso senza essere rivelati ed evitando (se non violando) le normali politiche di sicurezza.  
\vspace{2ex} \newline
Solitamente operano al di fuori dei soliti meccanismi di comunicazioni sfruttando vulnerabilità o comportamenti non previsti nei sistemi.
Non usando i normali protocolli e/o canali di comunicazione (es network sockets, emails) ciò gli permette di non generare segnali di un uso improprio del sistema. 
Nascondendosi all'interno dei normali processi del sistema; sono difficili da rilevare e/o identificare usando i tipici strumenti di monitoraggio. 
Ciò comporta che la loro esistenza è un problema che spesso rimane non notata dagli amministratori. 
\vspace{2ex} \newline
Da notare chequalsiasi risorsa condivisa può essere utilizzata come canale nascosto e per questo i Covert Channel possono esistere in qualunque sistema. 
Poichè lo sfruttamento di queste risorse porta alla fuoriuscita (o scambio) dei dati; 
questi attacchi sono un problema significativo in tutti quegli ambienti dove una fuoriuscita di informazioni può avere conseguenze gravi (es ambienti militari, governativi,\dots).  
\vspace{2ex} \newline
Tipicamente sono costituiti da due principali componenti: 
\begin{itemize}
    \item \textbf{Mittente} (Covert Transmitter): è l'entita che codifica e trasmette le informazioni nascote usando una risorsa di sistema condivisa.  
    \item \textbf{Destinatario} (Covert Listener): è l'entità che rileva e decifra l'informazione segreta dalla risorsa condivisa. 
\end{itemize}