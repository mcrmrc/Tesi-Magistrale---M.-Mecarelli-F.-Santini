ICMP (Internet Control Message Protocol) è un protocollo che opera al livello di rete (livello 3 nel modello ISO/OSI). 
e permette la \textbf{segnalazione errori}, la \textbf{diagnostica di rete} e la \textbf{messaggistica di controllo}. 
%utilizzato per la diagnostica, per la segnalazione di errori, per ottenere informazioni di controllo
Proprio per questo che viene utilizzato per il monitoraggio dello stato di una rete e per la risoluzione dei problemi che avvengono in essa. 
%A differenza di altri protocolli (eg TCP/IP, UDP,\dots) non è utilizzato per la trasmissione di dati e di conseguenza non stabilirà una connessione e non presenterà un numero di porta specifico. 
%Si avrà quindi una comunicazione Stateless e Connectionless che può avvenire senza specificare alcun tipo di porta. 
%
%I principali strumenti di rete che sfruttano questo protocollo sono: \textbf{Ping} e \textbf{Traceroute}
%\begin{itemize}
%    \item \textbf{Ping}: tramite lo scambio di Echo Reqeuste e Echo Reply fra due macchine, testa la connettività fra di loro. 
%    L'host mittente, per verificare la connesisone con il dispositivo di destinazione, invia una richieste Echo. 
%    Ciò porterà l'host interessato, se raggiungibile, a rispondere con una Echo Reply.  
%    \item \textbf{Traceroute}: permette di trovare il percorso che i pacchetti seguono per una determinata destinazione. 
%    Ciò viene effettuato tramite l'utilizzo dei messaggi di tipo Time Exceeded; 
%    il quale verrà mandato allo scadere del TTL (time to live) del pacchetto. 
%    Infatti, per poter mappare l'intero percorso e quindi i router presenti in esso, il campo verrà incrementato progressivamente.  
    %\item \textbf{PMTUD}: Scoperta del percorso MTU che utilizza messaggi ICMP Fragmentation Needed per ottimizzare le dimensioni dei pacchetti. 
    %La \textbf{PMTUD}, utilizza messaggi ICMP Fragmentation Needed per ottimizzare le dimensioni dei pacchetti. 
    %Ovvero per trovare la dimensione ottimale del pacchetto per un percorso di rete.
    %Ovvero per trovare la dimensione ottimale del pacchetto per un percorso di rete. 
    %\item[] ICMP in IPv6 (ICMPv6) ICMPv6 extends ICMP functionality for IPv6 networks, including:
    %\item Neighbor Discovery Protocol (NDP) – Replaces ARP for IPv6 
    %\item Router Advertisements & Solicitation – Helps configure IPv6 addresses. 
    %\item MLD (Multicast Listener Discovery) – Manages multicast group memberships
%\end{itemize}  
%\vspace{2ex} \newline
Data la sua necessita per la diagnostica di rete e la segnalazione degli errori, 
può essere utilizzato in modo improprio per mettere a segno degli attacchi o per studiare la rete (ricognizione della rete). %network reconnaissance
\vspace{2ex} \newline
Nei seguenti capitoli verrà illustrato come può essere sfruttato per la creazione di un Covert Channel; 
un attacco che permette (in ambienti ritenuti sicuri) la capacità di comunicare e/o trasferire dati in maniera non autorizzata e non voluta. 
L'attacco opera al di fuori degli usuali meccanismi di comunicazioni %Inoltre sfruttando vulnerabilità o comportamenti non previsti nei sistemi,  
e per questo risulta difficile da rilevare e/o identificare. 
Sia dagli amministratori che dai tipici strumenti di monitoraggio. 
Infine, siccome qualsiasi risorsa condivisa può essere utilizzata per la sua creazione, può esistere in qualunque sistema. 


