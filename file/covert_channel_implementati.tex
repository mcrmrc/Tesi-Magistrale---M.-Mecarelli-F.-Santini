\import{./covert_channel_implementati}{introduzione} \newpage  

%Nelle prossime pagine, si analizzeranno le varie tipologie di messaggi 
%(divise in \textbf{Messaggi di errore} \footnote{segnalano problemi nella comunicazione di rete} e \textbf{Messaggi informativi} \footnote{utilizzati per scopi diagnostici e di controllo})
%che il protocollo può mandare per definire un canale di comunicazione nascosto per uno scambio di informaizoni (non consentito). 


Le principali categorie di Covert Channel sono: 
\begin{itemize}
    \item Timing Covert Channel (Temporizzazione): \newline
    Sfruttano gli intervalli di tempo o l'ordine degli eventi per codificare informazioni (e.g. ritardi fra i pacchetti di rete,\dots). 
    \item Storage Covert Channel (Archiviazione): \newline 
    Implicano la scrittura di dati su un'area di memoria condivisa accedibile da entrambi i processi. 
    I veicoli saranno tutte quelle risorse che consentono la scrittura (diretta o indiretta) da parte di un processo e la lettura (diretta o indiretta) da parte di un altro.
    Un processo scrive su una risorsa condivisa, mentre un altro processo legge da essa. 
    %\item Behavioral Covert Channel (Comportamento): \newline 
    %AAAA
\end{itemize}

\subsection{Timing Covert Channel}
\import{./covert_channel_implementati}{timing_covert_channel}  

\subsection{Storage Covert Channel}
\import{./covert_channel_implementati}{storage_covert_channel}  

%\subsection{Behavioral Covert Channel}
%\import{./covert_channel_implementati}{behavioral_covert_channel} 


\subsection{Struttura della comunicazione fra le entità} 
\import{./covert_channel_implementati}{struttura_comunicazione}  
%\import{./implementazione}{introduzione} \newpage    

