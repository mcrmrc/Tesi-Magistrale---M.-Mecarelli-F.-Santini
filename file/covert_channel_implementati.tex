\import{./covert_channel_implementati}{introduzione} \newpage   

\subsection{Timing Covert Channel}
\subsubsection*{Categorie di Covert Channel implementate}  
\begin{itemize}
    \item Timing Covert Channel (Temporizzazione): \newline 
    %I covert channel di temporizzazione sono metodi di comunicazione che permettono ad un osservatore (un umano o un processo) di acquisire informazioni attraverso il cambiamento del tempo di rispostadi una risorsa. 
    Sfruttano gli intervalli di tempo o l'ordine degli eventi per codificare le informazioni (e.g. ritardi fra i pacchetti di rete,\dots). 
    Qualsiasi metodo che utilizza un orologio (o una misurazione del tempo) per segnalare il valore può implementarlo. 
    \item Storage Covert Channel (Archiviazione): \newline 
    Implicano la scrittura di dati su un'area di memoria condivisa accessibile da entrambi i processi. 
    I veicoli saranno tutte quelle risorse che consentono la scrittura (diretta o indiretta) da parte di un processo e 
    la lettura (diretta o indiretta) da parte di un altro.Un processo scrive su una risorsa condivisa, mentre un altro processo legge da essa. 
    %Possono essere quindi utilizzati da processi all'interno di un singolo computer o tra più computer in una rete.
    %\begin{esempio}
    %    Variare deliberatamente il tempo fra delle azioni (es trasmissione di network packet, patter di uso della CPU) 
    %    oppure codificando dati nella temporalizzazione dell'esecuzione dei processi o delay di risposta. 
    %\end{esempio}
    %\vspace{1ex} \noindent
    %Coinvolgono quindi la scrittura di dati su un'area di memoria condivisa accedibile da entrambi i processi (e.g attributi del file, i bit di memoria, gli stati della cache,\dots).
    %Di conseguenza, i veicoli sono tutte le risorse che consentono la scrittura (diretta o indiretta) da parte di un processo e la lettura (diretta o indiretta) da parte di un altro.
    \item Behavioral Covert Channel (Comportamento): \newline 
    I canali nascosti comportamentali operano trasmettendo dati in base all'avvenimento di diversi eventi. 
    %di processi, sistemi e applicazioni, generalmente suddividendo e trasmettendo i dati in pacchetti più piccoli. 
\end{itemize}


\import{./covert_channel_implementati}{timing_covert_channel}  

\subsection{Storage Covert Channel}
\import{./covert_channel_implementati}{storage_covert_channel}  

%\subsection{Behavioral Covert Channel}
%\import{./covert_channel_implementati}{behavioral_covert_channel} 


\subsection{Struttura della comunicazione fra le entità} 
\import{./covert_channel_implementati}{struttura_comunicazione}  
%\import{./implementazione}{introduzione} \newpage    

