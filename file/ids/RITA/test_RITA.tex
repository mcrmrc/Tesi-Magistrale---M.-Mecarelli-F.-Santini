%Zeek sites
%https://kifarunix.com/analyze-network-traffic-using-zeek/ 
%https://medium.com/@Mohamed-Medhat/zeek-commands-sheet-cheat-de4358bd277a 
%https://docs.zeek.org/en/master/logs/conn.html
%   %https://gist.github.com/donovanrodriguez/a6cb54c1e1a525a2810f3eb2b8b7ffaa
%   %https://community.zeek.org/t/icmp-not-in-conn-log/4990 
%   %https://github.com/zeek/zeek/issues/3915
%https://docs.zeek.org/en/current/install.html


Per testare se RITA fosse in grado di rilevare i Covert Channel implementati; 
tramite Wireshark si è monitorato il traffico e il risultato è stato salvato in dei file pcap. 
%I file del flusso di rete che rita deve analizzare sono in formato pcapng. 
Tuttavia RITA analizza solo i log generati da Zeek, non i file pcapp stessi. 
%non supporta questo formato, ma solo i log di Zeek. 
%Si dovrà quindi \href{https://www.youtube.com/watch?v=kwR3TjIgoCo&t=13s}{convertire} i file pcap in log Zeek e poi analizzarli con RITA. 
Tramite Zeek si sono quindi estratti (dal file pcap) i log necessari. %che sono stati poi analizzati da RITA.  
%\begin{lstlisting}
%    cd "/mnt/d/test_pacchetti_finali" 
%    zeek readpcap ./100_ciao_receive_ICMP_destination_unreachable.pcapng  ./zeek_log/100_ciao_receive_ICMP_destination_unreachable/
%    zeek readpcap ./source_path/file.pcapng  ./zeek_log/destination_path/
%    rita import -l "/mnt/d/test_pacchetti_finali/zeek_log/100_ciao_receive_ICMP_destination_unreachable" -d "ciaoDest100"
%    rita import -l "./zeek_log/log_path" -d database_name
%    rtia list
%    rita view ciaoDest100
%    rita view database_name 
%    zeek readpcap /mnt/d/test_pacchetti_finali/100_ciao_receive_ICMP_information.pcap /mnt/d/test_pacchetti_finali/zeek_log/100_ciao_information/
%\end{lstlisting}
%I databse che RITA ha creato saranno: ciaoDest100, ciaoTimeExc100, ciaoParam100, ciaoQuench100, ciaoRedirect100, ciaoTimeStmp100, ciaoInfo100
\vspace{1ex} \newline
Per testare se i pacchetti venissero rilevati, si sono inviate tre tipologie di dati: 
%L'odine di grandezza è dei Kilobytes; questo è stato scelto per non appesantire troppo il carico di rete ed evitare lunghi tempi di attesa. 
\begin{enumerate}
    \item Un testo corto e lineare. Per esempio un comando. 
    \item Un testo di lunghezza variabile. Per esempio il contenuto di un file. 
    \item Il terzo è invece un testo la cui lunghezza incrementa ogni volta. 
    Ciò è stato usato per testare se un flusso maggiore di dati venisse rilevato. 
    %RITA fosse in grado di rilevare un flusso di dati sempre più evidente.
\end{enumerate}  
I risultati sono divisi per la tipologia di messaggio inviato e la quantità di dati inviati. 
All'interno di ogni cella si indicherà il livello di gravità che RITA ha associato alla comunicazione. 
\vspace{2ex} \newline
Ed il risultato è insoddisfacente. Il riusltato è che RITA non rileva le connessioni ICMP. 
Sebbene si siano provate diverse tipologie di messaggio e diverse tipologie di messaggi RITA non rileva niente. 
All'inizio il problema erano i log di Zeek che non moritoravano le connesisoni ICMP; 
ma dopo aver dockerizzato Zeek così da avere l'ultima versione, il test è risultato Indefinibile. 
E in questa seconda prova zeek nei file di log dava una cossesisone ICMP. 
\vspace{1ex} \newline 
Si sono quindi provate vari livelli di connesioni, dalle più leggere a quelle che inviavano un costante flusso di dati. 
Per testare il caso di una comunicazione pesante, si è creato apposta un file contenente 2 MB di dati. 
Il risultato è ancora Indefinito sicocme i log di Zeek mostrano queste connessione sebbene i database creati da RITA risultino negativi. 
Data questa ambiguità il riusltato non può essere definito ne positivo ne negativo. 
\vspace{4ex} \newline
\begin{minipage}{\textwidth}
\centering
\begin{tabular}{|c|c|c|c|}
\hline 
\textbf{Tipologia Messaggio} & \multicolumn{3}{c|}{\textbf{Quantita di byte totali inviati}} \\ \hline
 & 33 bytes & 2,5 KB & 20 KB \\ \hline 
Information Reply & Non definito & Non definito & Non definito \\ \hline
Timestamp Reply & Non definito & Non definito & Non definito \\ \hline
Redirect & Non definito & Non definito & Non definito  \\ \hline 
Echo & Non definito & Non definito & Non definito  \\ \hline 
Source Quench & Non definito & Non definito & Non definito  \\ \hline 
Parameter Problem & Non definito & Non definito & Non definito  \\ \hline 
Time Exceeded & Non definito & Non definito & Non definito  \\ \hline 
Destination Unreachable & Non definito & Non definito & Non definito  \\ \hline 
Timing Channel 1bit & Non definito & Non definito & Non definito  \\ \hline 
Timing Channel 2bit & Non definito & Non definito & Non definito  \\ \hline 
\end{tabular}
\captionof{table}{Quali comunicazioni RITA ha rilevato}
\label{tab:rita_icmp:rilevati} 
\end{minipage}
\vspace{1ex} \newline 
%Zeek è stato utilizzato nella WSL Windows; 
%si è visto che la versione non fosse aggiornata siccome non rilevava nei log il protocollo ICMP. 
%Si è proceduto quindi a virtualizzarlo tramite docker; 
%la repository utilizzata era 'latest'. 
%Tuttavia ha ritornato risultati insoddisfacenti. 
%Sebbene ora venga rilevato il protocollo ICMP e l'indirizzo IP dell'attaccante RITA non rileva un attività sospetta. 
%\vspace{1ex} \newline 
%Si procederà ad effettuare un maggior numero di scambi con una singola tipologia. 


