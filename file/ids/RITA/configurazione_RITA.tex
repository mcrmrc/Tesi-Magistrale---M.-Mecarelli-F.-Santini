Prima di iniizare a testare RITA, è necessario configurarlo. 
Questo passaggio è fondamentale per il corretto funzionamento del sistema. %di rilevamento delle intrusioni. 
Senza una configurazione appropriata, il sitema non avrebbe rilevato l'attaccante in modo efficace.
\vspace{1ex} \newline
La configurazione di RITA avviene tramite il file di configurazione \textbf{config.hjson} 
che si trova nella cartella \textbf{/etc/rita/}.
%In questo file si possono settare diversi parametri che influenzano il funzionamento di RITA.

\subsubsection{Configurazione del campo \textit{filtering}} 
Il campo imposta i filtri che RITA utilizzerà durante lìanalisi dei log di Zeek. 
\begin{center} 
\begin{longtable}{|p{0.35\textwidth}|p{0.6\textwidth}|} 
    \hline
    \textbf{Sottocampo} & \textbf{Descrizione} \\
    \hline
    internal\_subnets &  
    Specifica gli indirizzi IP che si considereranno interni alla rete.
    %Questo perchè RITA analizza il traffico di rete in base alla distinzione tra host interni ed esterni. 
    %In questo caso è stato lasciato il valore di default ma si poteva specifcare o direttamente l'indirizzo IP 
    %della macchina vittima o la sottorete a cui appartiene. 
    %Si può specificare l'indirizzo di una sottorete (e.g 192.168.1.0/24) ma 
    Siccome l'attaccante e la vittima 
    sono all'interno della stesse rete; 
    nel campo è stato inserito solamente l'indirizzo IP della vititma. 
    %I valori di defualt sono stati rimossi. 
    \\ 
    \hline 
    always\_included\_subnets &  
    Specifica le reti che RITA dovrà sempre includere nell'analisi. 
    Siccome come rete interna si è specificato solo la vittima, il campo verrà lasciato al suo valore di default. 
    %Altrimenti si sarebbe potuta specificare la sottorete della macchina vittima (ovvero 192.168.1.0/24).
    %Il valore di default è un array vuoto, ma nel nostro caso si è aggiunta la sottorete della macchina vittima. 
    %Questo perchè sia l'attaccante che la vittima si trovano nella stessa sottorete: 192.168.1.0/24.
    \\
    \hline 
    filter\_external\_to\_internal &  
    Se impostato a true, fa sì che RITA ignori tutte le comunicazioni che avvengono da un host esterno ad 
    uno interno. In questo caso si è impostato a false, in modo che RITA analizzi anche queste comunicazioni.
    %Ciò è importante perchè l'attaccante si trova all'interno della rete e la vittima anche.
    \\
    \hline 
\caption{Sottocampi di \textit{filtering}} 
\label{tabella:RITA:config:filtering} 
\end{longtable} 
\end{center}   
\begin{lstlisting} [language=bash, showspaces=false, showstringspaces=false, showtabs=false]
"filtering": {
    "internal_subnets": [
        "192.168.1.3"
        #"192.168.1.11"
    ], 
    "always_included_subnets": [], 
    //Ignores external host to internal host communication 
    "filter_external_to_internal":false 
},
\end{lstlisting} 
\captionof{lstlisting}{Configurazione del campo \textit{filtering}} 
\label{lstlisting:RITA:config:filtering} 


\subsubsection{Configurazione del campo \textit{scoring}}
In questo campo si può impostare come RITA assegnerà i punteggi ad ogni tipologia di comunicazione sospetta. 
%In particolare definirà in base a quali parametri verràdefinito un beacon, 
%il punteggio che verrà assegnato in base alla durata della connessione ed altri.
\vspace{2ex} \newline 
In particolare il sottocampo \textbf{beacon} definisce i parametri che RITA utilizzerà 
per definire una connessione come beacon e quale punteggio assegnargli. 
%per assegnare un punteggio ad una connessione sospetta che potrebbe essere un beacon. 
Al suo interno si possono settare diversi parametri che influenzano il calcolo del punteggio finale da associare alla connessione.
%i seguenti sono quelli che determinano il punteggio 
%Il punteggio è attualmente composto da una media ponderata di 4 sotto-punteggi.
Il valore del punteggio è dato dalla media pesata di 4 sottocampi; 
il valore di default per ognuno di essi è 0.25, tuttavia possono essere modificati in base alle esigenze; 
a patto che la somma di tutti i pesi sia alla fine uguale a 1.
%A ognuno di questi campi può essere associato un peso, il cui valore dipenderà dalle esigenze. 
%\vspace{1ex} \newline 
%Il valore di default tassociato ad ognuno di questi campi è 0.25; 
%ciò significherà che tutti i campi hanno lo stesso peso e quindi la stessa importanza nel calcolo del punteggio. 
%I campi in questione sono: 

\begin{center} 
\begin{longtable}{|p{0.2\textwidth}|p{0.7\textwidth}|} 
    \hline
    \textbf{Test} & \textbf{Descrizione} \\
    \hline
    ICMP Test &  
    Comunicazione reale fra l'attacante e la vittima traimte il Covert Channel ICMP. 
    L'attaccante inizia chiedendo alla vititma il ocntenuto di un file; 
    una volta che la vittima ha inviato il contenuto del file, 
    l'attaccante procede a richiedere un nuovo file. 
    I file di testo hanno sempre una dimensione crescente: 
    con il primo che è una semplece descrizione del contenuto di una cartella 
    menre l'ultimo file ha una dimensione di 2 MB. 
    \\ 
    \hline 
    600 ping &  
    Si ha la \textbf{vittima} che pinga ogni 2 secondi l'attaccante per un totale di 600 ping. 
    %\newline 
    L'\textbf{attaccante} invece pinga la vittima ogni 2 secondi per un totale di 200 ping. 
    %\newline
    Questo scambio di messaggi è stato ripetuto altre 3 volte. 
    \vspace{1ex} \newline 
    Lo scambio è svolto a simulare uno scambio di messaggi tra i due host. 
    \\
    \hline 
\caption{Test effettuati con RITA} 
\label{tabella:RITA:testEffettuati} 
\end{longtable} 
\end{center}  

\subsubsection*{Prima configurazione}
La configurazione iniziale del campo \textit{scoring} è quella di default. 
Ovvero quella che RITA utilizza appena installato. 
%\begin{lstlisting} [language=bash, showspaces=false, showstringspaces=false, showtabs=false]
%"scoring": {
%        "beacon": { 
%            //min number of unique conn to qualify as beacon
%            "unique_connection_threshold": 4, 
            
%            "timestamp_score_weight": 0.25,
%            "datasize_score_weight": 0.25,
%            "duration_score_weight": 0.25,
%            "histogram_score_weight": 0.25, 
            
%            "duration_min_hours_seen": 6,
%            "duration_consistency_ideal_hours_seen": 12,
%            "histogram_mode_sensitivity": 0.05,
%            "histogram_bimodal_outlier_removal": 1,
%            "histogram_bimodal_min_hours_seen": 11,

%            "score_thresholds": {
%                // beacon score
%                "base": 50,
%                "low": 70,
%                "medium": 90,
%                "high": 100
%            }
%        }, 
%        "long_connection_score_thresholds": {
%            // duration, in seconds
%            "base": 3600, // 1 hour
%            "low": 14400, // 4 hours
%            "medium": 28800, // 8 hours
%            "high": 43200 // 12 hours
%        },
%        "c2_score_thresholds": {
%            // number of subdomains
%            "base": 100,
%            "low": 500,
%            "medium": 800,
%            "high": 1000
%        },
%        "strobe_impact": {
%            "category": "high" 
%        },
%        "threat_intel_impact": {
%            "category": "high" 
%        }
%    },
%\end{lstlisting} 
%\captionof{lstlisting}{Prima configurazione del campo \textit{scoring}} 
%\label{lstlisting:RITA:config:scoring:prima} 

\begin{center} 
\begin{longtable}{|c|c|c|c|} 
    \hline 
    \multicolumn{4}{|c|}{Comunicazione Vittima to Attaccante} \\ 
    \hline
    \textbf{Test} & \textbf{Risultato} & \textbf{Beacon}  & \textbf{Connessione}   \\ 
    \hline 
    ICMP test &   
    None &
    16\% & 
    19m 10s
    \\ 
    \hline 
    600 ping &   
    None &
    42.90\% & 
    1h 22m 2s
    \\
    \hline  
\caption{Risultati dei test effettuati con RITA} 
\label{tabella:RITA:risultatiTestEffettuati:vittimaAttaccante} 
\end{longtable} 
\end{center}  
%
\begin{center} 
\begin{longtable}{|c|c|c|c|} 
    \hline 
    \multicolumn{4}{|c|}{Comunicazione Attaccante to Vittima} \\ 
    \hline
    \textbf{Test} & \textbf{Risultato} & \textbf{Beacon}  & \textbf{Connessione}   \\ 
    \hline 
    ICMP test &   
    None &
    22\% & 
    1m 9s
    \\ 
    \hline 
    600 ping &   
    None &
    41.60\% & 
    34m 10s
    \\
    \hline 
\caption{Risultati dei test effettuati con RITA} 
\label{tabella:RITA:risultatiTestEffettuati:attaccanteVittima} 
\end{longtable} 
\end{center}  

\subsubsection*{Seconda configurazione} 
%ICMP covert channels typically show: very regular inter-packet timing (small jitter), consistent ICMP payload sizes, and often many repeated packets. 
%So you want to emphasize timing and histogram/size repeatability, keep duration moderate, and leave a little weight on datasize. 
%
%Tuttavia, siccome si vuole testare se un attacco Covert Channel viene rilevato da RITA, 
Siccome i test effettuati con la configurazione di base non hanno generato alcun avvertimento sulla presenza 
di un anomalia, si è deciso di modificare le soglie per il calcolo del punteggio finale. 
\vspace{1ex} \newline 
Siccome in questo tipo di attacco, la dimensione dei pacchetti rimangono costanti; 
%poichè i dati saranno inseriti all'interno dei campi stessi del pacchetto.  
i dati scambiati non sono così rilevanti. 
Si è deciso quindi di diminuire il peso del campo \textbf{datasize\_score\_weight} a \textbf{0.15}. 
Così come anche quello di \textbf{duration\_score\_weight}, che è stato diminuito a \textbf{0.1}.
\vspace{1ex} \newline 
Invece \textbf{timestamp\_score\_weight} è stato aumentato a \textbf{0.4}. 
Mentre il campo \textbf{histogram\_score\_weight} avrà un peso pari a \textbf{0.35}. 
%Un attacco Covert Channle può basarsi sulla regolarità temporale delle connessioni e siccome l'attacco cerca 
%di generare il meno rumore possibile la durata per inviare i dati può essere significativa. 
%siccome si vuole dare meno importanza alla distribuzione delle connessioni nel tempo 

\begin{center} 
\begin{longtable}{|p{0.3\textwidth}|p{0.65\textwidth}|} 
    \hline
    \textbf{Sottocampo} & \textbf{Descrizione} \\
    \hline
    timestamp\_score\_weight &  
    %peso del punteggio dato dalla regolarità temporale delle connessioni. 
    Parametro utilizzato per influenzare il modo in cui i timestamp vengono considerati nel punteggio delle potenziali minacce.
    Questo peso regola l'importanza dei pattern basati sul tempo, nell'analisi complessiva delle minacce.
    %This weight adjusts the importance of time-based patterns, such as beaconing intervals, in the overall threat analysis. 
    %Aumentare il peso potrebbe rendere RITA più sensibile a 
    Pattern basati sul tempo sono spesso indicativi di attività dannose come le comunicazioni di Comando e Controllo (C2). 
    %Increasing the weight might make RITA more sensitive to consistent time-based patterns, which are often indicative of malicious activity like Command and Control (C2) communication.
    %
    %timestamp_score_weight / interval — emphasizes regular timing (periodicity). If your covert channel pings at very regular intervals (very low jitter) this subscore will be high. 
    \\ 
    \hline 
    datasize\_score\_weight &  
    %peso del punteggio dato dalla dimensione dei dati scambiati. 
    Utilizzato per regolare il peso o l'importanza della dimensione dei dati nel calcolo dei punteggi per l'analisi delle minacce. 
    Modificandolo, è possibile enfatizzare o ridurre il ruolo della dimensione dei dati nel punteggio complessivo delle minacce. 
    %
    %datasize_score_weight — rewards a consistent connection payload/packet size (mode + low dispersion). ICMP covert channels frequently use consistent payload sizes → this helps. 
    \\
    \hline 
    duration\_score\_weight &  
    %peso del punteggio dato dalla durata della connessione. 
    Parametro utilizzato per regolare il peso del punteggio di durata della connessione nel calcolo dei punteggi.
    %Aiuta a identificare potenziali comportamenti di beaconing nel traffico di rete analizzando la durata delle connessioni.
    Modificandolo, è possibile influenzare l'impatto della durata delle connessioni sul punteggio complessivo. 
    \vspace{1ex} \newline
    Un peso maggiore: aumenta l'importanza della durata della connessione nel calcolo del punteggio beacon.
    %Higher weight: Increases the importance of connection duration in the beacon score calculation.  
    Un peso minore: riduce l'impatto della durata della connessione, dando maggiore importanza ad altri fattori come l'intervallo o il volume dei dati.
    %Lower weight: Reduces the impact of connection duration, giving more weight to other factors like interval or data volume.
    %
    %duration_score_weight — rewards long-lived connections / many samples. If each test only has a few pings, this subscore will be small. 
    \\
    \hline 
    histogram\_score\_weight &  
    %peso del punteggio dato dalla distribuzione delle connessioni nel tempo. 
    %Solitamente utilizzato per adattare l'importanza di diverse metriche durante l'analisi del traffico di rete o il rilevamento di anomalie.
    %typically used to adjust the importance of different metrics when analyzing network traffic or detecting anomalies. 
    Il punteggio dell'istogramma misura la regolarità nei tempi di comunicazione tra due host utilizzando un 
    istogramma dei tempi di interarrivo (ritardi tra pacchetti o flussi). 
    %Il punteggio dell'istogramma misura la regolarità degli intervalli di tempo tra le connessioni di rete verso la stessa destinazione.
    %Si basa su un istogramma statistico delle differenze di tempo tra connessioni consecutive.
    Da questo calcola quanto sia "grande" o "ripetitiva" la distribuzione: se una coppia di host scambia 
    pacchetti a intervalli molto regolari (ad esempio ogni 2 secondi), l'istogramma presenterà dei picchi 
    significativi e il punteggio dell'istogramma sarà alto. Se invece la comunicazione avviene a intervalli 
    casuali o variabili, l'istogramma è piatto, assegnando un punteggio basso. 
    %Una varianza inferiore (spaziatura più regolare) aumenta il punteggio dell'istogramma, suggerendo che è 
    %probabile che l'host stia effettuando del beaconing a un sistema remoto secondo una pianificazione 
    %prevedibile, una caratteristica comune nel traffico di comando e controllo dei malware. 
    Una varianza inferiore, ceh aumenta il punteggio dell'istogramma, suggerisce che l'host stia effettuando 
    del beaconing a un sistema remoto secondo una pianificazione prevedibile, una caratteristica comune nel 
    traffico di comando e controllo. 
    %
    %histogram_score_weight — emphasizes the shape/mode of the interval/size histograms (peakiness / repeatability). Good for perfectly repeating intervals or repeating fixed packet sizes. 
    \\
    \hline 
\caption{Sottocampi di \textit{beacon} relativi al punteggio} 
\label{tabella:RITA:config:scoring:beacon:score} 
\end{longtable} 
\end{center} 

\begin{lstlisting} [language=bash, showspaces=false, showstringspaces=false, showtabs=false]
"timestamp_score_weight": 0.4,
"datasize_score_weight": 0.15,
"duration_score_weight": 0.1,
"histogram_score_weight": 0.35, 
\end{lstlisting} 
\captionof{lstlisting}{Seconda configurazione dei campi in \textit{beacon}} 
\label{lstlisting:RITA:config:scoring:seconda} 
%These weights bias RITA to reward precise timing and repeated histogram modes (interval/size). 
%If your covert channel uses very consistent sizes (same ICMP payload every packet), you can bump datasize_score_weight to 0.25 and reduce duration_score_weight further. 
%If your covert channel intentionally adds jitter to timing but keeps size identical, move weight from timestamp → datasize/histogram. 

\begin{center} 
\begin{longtable}{|c|c|c|c|} 
    \hline 
    \multicolumn{4}{|c|}{Comunicazione Vittima to Attaccante} \\ 
    \hline
    \textbf{Test} & \textbf{Risultato} & \textbf{Beacon}  & \textbf{Connessione}   \\ 
    %\\[-2pt]   % negative moves up, positive moves down 
    %\\[4pt]      % adds vertical space below this row
    \hline 
    ICMP test 2 &   
    None &
    19.10\% & 
    19m 10s
    \\ 
    \hline 
    600 ping 2 &   
    None &
    43.60\% & 
    1h 22m 2s
    \\
    \hline  
\caption{Risultati dei test effettuati con RITA} 
\label{tabella:RITA:risultatiTestEffettuati:vittimaAttaccante} 
\end{longtable} 
\end{center}  
%
\begin{center} 
\begin{longtable}{|c|c|c|c|} 
    \hline 
    \multicolumn{4}{|c|}{Comunicazione Attaccante to Vittima} \\ 
    \hline
    \textbf{Test} & \textbf{Risultato} & \textbf{Beacon}  & \textbf{Connessione}   \\ 
    %\\[-2pt]   % negative moves up, positive moves down 
    %\\[4pt]      % adds vertical space below this row
    \hline 
    ICMP test 2 &   
    None &
    22.7\% & 
    1m 9s
    \\ 
    \hline 
    600 ping 2 &   
    None &
    41.50\% & 
    34m 10s
    \\
    \hline  
\caption{Risultati dei test effettuati con RITA} 
\label{tabella:RITA:risultatiTestEffettuati:attaccanteVittima} 
\end{longtable} 
\end{center}  

\subsubsection*{Terza configurazione} 
Dato il miglioramennto nella seconda configurazione si procede nel definire i parametri mancanti, 
oltre che nel raffinare ulteriormente i parametri già impostati. 
\vspace{1ex} \newline 
Si è provato ad abbassare il numero minimo di connessioni univoche richieste per analizzare una connessione 
come beacon. Tuttavia se il parametro risulta minore di 4 RITA non convalid a la configurazione.
%// Minimum number of unique connections before analyzing for beaconing.
%// Leave low to catch small ICMP patterns. 
%//Non possibile scendere sotto 4 altrimenti si sarebbe messo 3 
\vspace{1ex} \newline 
Si raffinano poi il peso da assegnare agli score per enfatizzare la regolarità dei tempo e gli istogrammi.
%la somiglianza degli istogrammi (intervallo e dimensione). 
%We emphasize timing regularity and histogram similarity (interval & size), 
Infatti gli attacchi Covert Channel ICMP tipicamente inviano pacchetti altamente regolari con dimensioni 
e strutture di payload quasi identiche. 
\vspace{2ex} \newline 
Si raffina poi la sensibilità alla durata. 
I beacon ICMP spesso si verificano in raffiche più brevi di 6 ore, quindi si  è abbassata la soglia entro cui 
iniziare a calcolare il punteggio. Così da permettere di calcoalre il punteggio anche delle connessioni più brevi.
Infine si è abbassata anche la soglia ideale per ottenere il punteggio massimo (\textit{duration\_consistency\_ideal\_hours\_seen}). 
%// ---- Duration sensitivity ----
%// ICMP beacons often occur in bursts shorter than 6 h, so lower thresholds.
\vspace{2ex} \newline 
Si è poi raffinato anche la sensibilità dell'istogramma. 
Si è deciso di rendere la rilevazione della modalità più sensibile (con bucket più stretti si avrà meno tolleranza) 
così che da distinguere chiarmanete gli intervalli ICMP. 
%// ---- Histogram tuning ----
%// Make the mode detection more sensitive (narrow buckets -> less forgiving)
%// so that tight ICMP intervals stand out clearly.
%// Default 0.05 -> tighter clustering
\vspace{2ex} \newline 
Infine si sono abbassati leggermente i punteggi delle soglie così che anche le comunicazioni ICMP più brevi 
appaiano nell'analisi almeno nelle categorie "bassa/media". 
%// ---- Beacon score thresholds ----
%// Lower thresholds slightly so shorter ICMP patterns still appear
%// in at least "low/medium" categories for analysis.

\begin{center} 
\begin{longtable}{|p{0.35\textwidth}|p{0.6\textwidth}|} 
    \hline
    \textbf{Sottocampo} & \textbf{Descrizione} \\  
    \hline  
    unique\_connection\_threshold &  
    Il campo  indica il minimo di connessioni necessarie affinche RITA consideri la connessione come una beacon. 
    %Numero minimo predefinito di connessioni univoche utilizzate per l'analisi dei beacon.
    Due host con un numero di connessioni inferiore a questo non verranno analizzati. 
    %È possibile aumentare questo valore in tutta sicurezza per migliorare le prestazioni se la lentezza dei beacon non rappresenta un problema. 
    \\
    \hline 
\caption{Sottocampi di \textit{beacon} relativi al numero minimo di connessioni} 
\label{tabella:RITA:config:scoring:beacon:connections:minNum} 
\end{longtable} 
\end{center} 
\begin{center} 
\begin{longtable}{|p{0.35\textwidth}|p{0.6\textwidth}|} 
    \hline
    \textbf{Sottocampo} & \textbf{Descrizione} \\ 
    duration\_min\_hours\_seen &  
    Il numero di ore visualizzate in una rappresentazione %grafica 
    delle connessioni di un beacon deve 
    essere superiore a questa soglia affinché venga calcolato un punteggio di durata complessivo.
    %Valore predefinito: 6 
    \\ 
    \hline 
    duration\_consistency\newline\_ideal\_hours\_seen &  
    Questo è il numero minimo di ore visualizzate in una rappresentazione grafica delle connessioni di un 
    beacon affinché il sottopunteggio di coerenza della durata raggiunga il 100\%.
    %Valore predefinito: 12 (mezza giornata) 
    \\ 
    \hline 
    histogram\_mode\_sensitivity &  
    %Il punteggio dell'istogramma ha un sottopunteggio che tenta di rilevare più sezioni piatte in un grafico 
    %di connessione rappresentante un beacon. 
    La variabile controlla la dimensione del bucket per il raggruppamento delle connessioni. 
    Questo valore è espresso come percentuale del numero massimo di connessioni. 
    %Ad esempio, se il numero massimo di connessioni è 400 e questa variabile è impostata a 0,05 (5\%), 
    %la dimensione del bucket sarà 20 (400*0,05=20). 
    Aumentando questa variabile, l'algoritmo diventa più tollerante alle variazioni. 
    %Valore predefinito: 0,05 
    \\ 
    \hline 
    histogram\_bimodal\newline\_outlier\_removal &  
    Questo è il numero di bucket che possono essere considerati valori anomali ed esclusi dal calcolo. 
    %Valore predefinito: 1 
    \\
    \hline 
    histogram\_bimodal\newline\_min\_hours\_seen &  
    Questo è il numero minimo di ore visualizzate in una rappresentazione %grafica 
    delle connessioni di un 
    beacon prima che venga utilizzato il punteggio del sottopunteggio bimodale. 
    %Valore predefinito: 11 (imposta la copertura minima a poco meno della metà della giornata) 
    \\
    \hline 
\caption{Sottocampi di \textit{beacon}} 
\label{tabella:RITA:config:scoring:beacon} 
\end{longtable} 
\end{center} 



\begin{lstlisting} [language=bash, showspaces=false, showstringspaces=false, showtabs=false]
"unique_connection_threshold": 4, 

// Timing regularity = strongest signal 
"timestamp_score_weight": 0.45, 
// Repeated interval/size histogram shapes 
"histogram_score_weight": 0.30, 
// Consistent ICMP payload size 
"datasize_score_weight": 0.20, 
// De-emphasize total connection time 
"duration_score_weight": 0.05, 

"duration_min_hours_seen": 2, 
"duration_consistency_ideal_hours_seen": 6, 

//Tighter clustering
"histogram_mode_sensitivity": 0.03, 
"histogram_bimodal_outlier_removal": 1, 
// Match duration window 
"histogram_bimodal_min_hours_seen": 6, 

"score_thresholds": {
    "base": 40,
    "low": 65,
    "medium": 85,
    "high": 100
}
\end{lstlisting} 
\captionof{lstlisting}{Terza configurazione dei campi in \textit{beacon}} 
\label{lstlisting:RITA:config:scoring:terza} 

\begin{center} 
\begin{longtable}{|c|c|c|c|} 
    \hline 
    \multicolumn{4}{|c|}{Comunicazione Vittima to Attaccante} \\ 
    \hline
    \textbf{Test} & \textbf{Risultato} & \textbf{Beacon}  & \textbf{Connessione}   \\ 
    %\\[-2pt]   % negative moves up, positive moves down 
    %\\[4pt]      % adds vertical space below this row
    \hline 
    ICMP test 2 &   
    None &
    24.2\% & 
    19m 10s
    \\ 
    \hline 
    600 ping 2 &   
    None &
    56\% & 
    1h 22m 2s
    \\
    \hline 
\caption{Risultati dei test effettuati con RITA} 
\label{tabella:RITA:risultatiTestEffettuati:vittimaAttaccante} 
\end{longtable} 
\end{center}  
%
\begin{center} 
\begin{longtable}{|c|c|c|c|} 
    \hline 
    \multicolumn{4}{|c|}{Comunicazione Attaccante to Vittima} \\ 
    \hline
    \textbf{Test} & \textbf{Risultato} & \textbf{Beacon}  & \textbf{Connessione}   \\ 
    %\\[-2pt]   % negative moves up, positive moves down 
    %\\[4pt]      % adds vertical space below this row
    \hline 
    ICMP test 2 &   
    None &
    29.6\% & 
    1m 9s
    \\ 
    \hline 
    600 ping 2 &   
    None &
    54.4\% & 
    34m 10s
    \\
    \hline 
\caption{Risultati dei test effettuati con RITA} 
\label{tabella:RITA:risultatiTestEffettuati:attaccanteVittima} 
\end{longtable} 
\end{center}  

\subsubsection*{Quarta configurazione} 
In questa configurazione si vanno a modificare i parametri esterni al sottocampo \textit{beacon}, 
ma sempre all'interno del campo \textit{scoring}. 
Tuttavia non si sono raggiunte modifiche sostanziali nei risultati finali.

\begin{center} 
\begin{longtable}{|p{0.35\textwidth}|p{0.6\textwidth}|} 
    \hline
    \textbf{Sottocampo} & \textbf{Descrizione} \\ 
    long\_connection\_score\_thresholds &  
    controls how RITA buckets long-lived connections into base/low/medium/high 
tiers by duration. Lowering these makes shorter connections count as “long” (so the “long connection” detector 
will flag them sooner). This can help if your ICMP tests are shorter than the defaults and you want them 
assessed as notable long connections.
    \\ 
    \hline 
    c2\_score\_thresholds &  
    applies to DNS / domain-subdomain analysis for C2 detection. Not directly relevant to 
ICMP, but lowering these will make small-scale domain churn look more suspicious (useful if you want more 
sensitivity to metadata-based C2). 
    \\ 
    \hline 
    strobe\_impact \& threat\_intel\_impact &  
    already set to "high". Any flow that RITA flags as a strobe 
(fast repeated connections) or that matches threat intel will be escalated as high — keep these at high 
to ensure flagged ICMP flows are prioritized.
    \\ 
    \hline 
\caption{Sottocampi di \textit{beacon}} 
\label{tabella:RITA:config:scoring:beacon} 
\end{longtable} 
\end{center} 

\begin{lstlisting} [language=bash, showspaces=false, showstringspaces=false, showtabs=false]
"long_connection_score_thresholds": {
    "base": 1800, // 30 min
    "low": 7200    // 2 hours
    "medium": 14400, // 4 hours
    "high": 28800, // 8 hours 
}, 
\end{lstlisting} 
\captionof{lstlisting}{Terza configurazione dei campi in \textit{beacon}} 
\label{lstlisting:RITA:config:scoring:terza} 

\begin{center} 
\begin{longtable}{|c|c|c|c|} 
    \hline 
    \multicolumn{4}{|c|}{Comunicazione Vittima to Attaccante} \\ 
    \hline
    \textbf{Test} & \textbf{Risultato} & \textbf{Beacon}  & \textbf{Connessione}   \\ 
    %\\[-2pt]   % negative moves up, positive moves down 
    %\\[4pt]      % adds vertical space below this row
    \hline 
    ICMP test 2 &   
    None &
    24.2\% & 
    19m 10s
    \\ 
    \hline 
    600 ping 2 &   
    None &
    56\% & 
    1h 22m 2s
    \\
    \hline 
\caption{Risultati dei test effettuati con RITA} 
\label{tabella:RITA:risultatiTestEffettuati:vittimaAttaccante} 
\end{longtable} 
\end{center}  
%
\begin{center} 
\begin{longtable}{|c|c|c|c|} 
    \hline 
    \multicolumn{4}{|c|}{Comunicazione Attaccante to Vittima} \\ 
    \hline
    \textbf{Test} & \textbf{Risultato} & \textbf{Beacon}  & \textbf{Connessione}   \\ 
    %\\[-2pt]   % negative moves up, positive moves down 
    %\\[4pt]      % adds vertical space below this row
    \hline 
    ICMP test 2 &   
    None &
    29.6\% & 
    1m 9s
    \\ 
    \hline 
    600 ping 2 &   
    None &
    54.4\% & 
    34m 10s
    \\
    \hline 
\caption{Risultati dei test effettuati con RITA} 
\label{tabella:RITA:risultatiTestEffettuati:attaccanteVittima} 
\end{longtable} 
\end{center}  


\subsubsection{Risultati della configurazione del campo \textit{scoring}}
\begin{center} 
\begin{longtable}{|c|c|c|c|} 
    \hline 
    \multicolumn{4}{|c|}{Comunicazione Vittima to Attaccante} \\ 
    \hline
    \textbf{Test} & \textbf{Risultato} & \textbf{Beacon}  & \textbf{Connessione}   \\
    \hline 
    \multicolumn{4}{|c|}{\textbf{Prima configurazione}} \\ 
    \hline 
    ICMP test &   
    None &
    16\% & 
    19m 10s
    \\ 
    \hline 
    600 ping &   
    None &
    42.90\% & 
    1h 22m 2s
    \\
    \hline 
    \multicolumn{4}{|c|}{\textbf{Seconda configurazione}} \\ 
    %\\[-2pt]   % negative moves up, positive moves down 
    %\\[4pt]      % adds vertical space below this row
    \hline 
    ICMP test 2 &   
    None &
    19.10\% & 
    19m 10s
    \\ 
    \hline 
    600 ping 2 &   
    None &
    43.60\% & 
    1h 22m 2s
    \\
    \hline 
    \multicolumn{4}{|c|}{\textbf{Terza configurazione}} \\ 
    %\\[-2pt]   % negative moves up, positive moves down 
    %\\[4pt]      % adds vertical space below this row
    \hline 
    ICMP test 2 &   
    None &
    24.2\% & 
    19m 10s
    \\ 
    \hline 
    600 ping 2 &   
    None &
    56\% & 
    1h 22m 2s
    \\
    \hline 
    \multicolumn{4}{|c|}{\textbf{Quarta configurazione}} \\ 
    %\\[-2pt]   % negative moves up, positive moves down 
    %\\[4pt]      % adds vertical space below this row
    \hline 
    ICMP test 2 &   
    None &
    24.2\% & 
    19m 10s
    \\ 
    \hline 
    600 ping 2 &   
    None &
    56\% & 
    1h 22m 2s
    \\
    \hline 
\caption{Risultati dei test effettuati con RITA} 
\label{tabella:RITA:risultatiTestEffettuati:vittimaAttaccante} 
\end{longtable} 
\end{center}  
%
\begin{center} 
\begin{longtable}{|c|c|c|c|} 
    \hline 
    \multicolumn{4}{|c|}{Comunicazione Attaccante to Vittima} \\ 
    \hline
    \textbf{Test} & \textbf{Risultato} & \textbf{Beacon}  & \textbf{Connessione}   \\
    \hline 
    \multicolumn{4}{|c|}{\textbf{Prima configurazione}} \\ 
    \hline 
    ICMP test &   
    None &
    22\% & 
    1m 9s
    \\ 
    \hline 
    600 ping &   
    None &
    41.60\% & 
    34m 10s
    \\
    \hline 
    \multicolumn{4}{|c|}{\textbf{Seconda configurazione}} \\ 
    %\\[-2pt]   % negative moves up, positive moves down 
    %\\[4pt]      % adds vertical space below this row
    \hline 
    ICMP test 2 &   
    None &
    22.7\% & 
    1m 9s
    \\ 
    \hline 
    600 ping 2 &   
    None &
    41.50\% & 
    34m 10s
    \\
    \hline 
    \multicolumn{4}{|c|}{\textbf{Terza configurazione}} \\ 
    %\\[-2pt]   % negative moves up, positive moves down 
    %\\[4pt]      % adds vertical space below this row
    \hline 
    ICMP test 2 &   
    None &
    29.6\% & 
    1m 9s
    \\ 
    \hline 
    600 ping 2 &   
    None &
    54.4\% & 
    34m 10s
    \\
    \hline 
    \multicolumn{4}{|c|}{\textbf{Quarta configurazione}} \\ 
    %\\[-2pt]   % negative moves up, positive moves down 
    %\\[4pt]      % adds vertical space below this row
    \hline 
    ICMP test 2 &   
    None &
    29.6\% & 
    1m 9s
    \\ 
    \hline 
    600 ping 2 &   
    None &
    54.4\% & 
    34m 10s
    \\
    \hline 
\caption{Risultati dei test effettuati con RITA} 
\label{tabella:RITA:risultatiTestEffettuati:attaccanteVittima} 
\end{longtable} 
\end{center}  




