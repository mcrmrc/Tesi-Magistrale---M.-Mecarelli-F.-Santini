Le entita coinvolte sono: 
\begin{itemize}
    \item \textbf{Attaccante}: \newline
    %Il primo avrà bisogno di 
    Carica un file di configurazione nel quale viene definito l'indirizzo IP della vittima, 
    il metodo di attacco e i proxy utilizzabili per farlo. 
    %e il comando da eseguire. 
    Inoltre inserirà il comando che la vittima dovrà eseguire o il dato che gli si vuole mandare.  
    %dovo averlo inoltrato, tramite i proxy, aspetta i dati relativi ad esso. 
    Nel caso l'attaccante utilizzasse dei proxy, aspettera da essi i dati che la vittima ha restituito. 
    \item \textbf{Proxy}: \newline 
    Ha bisongo di sapere qual'è l'indirizzo IP dell'attaccante. 
    %Questo perchè 
    Deve potersi connettere ad esso ed ottenere l'indirizzo IP della vittima e il comando da inoltrare. 
    Una volta stabilita una connesisone sia con l'attaccante che con la vittima; 
    si occuperà di inoltrare i dati che le due entità si inviano. 
    \item \textbf{Vittima}: \newline 
    Aspetta le connessioni o dall'attaccante o dai proxy se vengonoutilizzati. 
    Dopodichè aspetterà un comando (o un dato). 
    Nel primo caso lo eseguirà e invierà i dati ricavati dall'esecuzione. 
\end{itemize}
\vspace{2ex}  
%Il come avviene 
La comunicazione fra le entità è definita dall'immagine seguente [Figura \ref{fig:covertchannel:struttura:flow}]. 
\begin{center} 
    \includegraphics[width=\textwidth]{../img/implementazione/diagramma_struttura_entita.jpg} 
    \captionof{figure}{Flusso di comunicazione fra le entita}
    \label{fig:covertchannel:struttura:flow} 
\end{center}  
Il canale di comunicazione che il proxy e l'attaccante potranno avere dipende dall'affidabilità del proxy. 
%Nel caso si reputi il proxy insicuro, 
Si potrà utilizare una comunicazione tramite ICMP (ovvero la stessa che il proxy avrà con la vittima) 
oppure si potrà usare il protocollo TCP per avere una comunicazione maggiormente stabile e affidabile. 
\vspace{1ex} \newline
Inoltre l'attaccante potrà usare uno o più proxy per comunicare con la vittima [Figura \ref{fig:covertchannel:struttura:schema}]. 
Il caso standard sarà quello in cui l'attaccante usa un singolo proxy. 
Tuttavià per l'attaccante sarà possibile comunicare direttamente con la vittima o tramite ulteriori proxy. 
\vspace{1ex} \newline 
Nel primo caso alcune funzioni presenti nell'entità proxy verranno eseguite dall'attaccante (e.g connettersi 
alla vittima). Nell'ultimo invece, l'attaccante dovrà prevedere un metodo per riunire in modo ordinato i messaggi ricevuti dai proxy. 
%\begin{center} 
%    \includegraphics[width=0.7\textwidth]{../img/implementazione/struttura attacker.proxy.victim 2.jpg} 
%    \captionof{figure}{Struttura delle entità presenti e come dialogano}
%    \label{fig:covertchannel:struttura:schema} 
%\end{center} 
\begin{center} 
    \includegraphics[width=0.7\textwidth]{../img/implementazione/struttura attacker.proxy.victim 3.jpg} 
    \captionof{figure}{Struttura delle entità presenti e come dialogano}
    \label{fig:covertchannel:struttura:schema} 
\end{center} 

\subsubsection{Struttura dell'attaccante} 
%Quando si esegue il programma, tramite linea di comando, devono essere passati degli argomenti; 
%in particolare verrà utilizzato 
Tramite un parser rileve se nella linea di comando sono state inserite le opzioni necessarie 
all'inizializzazione [Tabella \ref{table:comunicazione:attacccante:opzioni}]. 
\begin{longtable}{|p{0.3\textwidth}|p{0.6\textwidth}|} 
    \hline 
    \textbf{Opzione} & \textbf{Utilizzo}  \\
    \hline
    Path file & 
    Percorso per il file di configuraizone da caricare. In questo file JSON viene specificato 
    l'indirizzo IP della vittima, la metodologia di attacco e la lista dei proxy che si vogliono usare. 
    \\ 
    \hline 
\end{longtable}
\captionof{table}{Opzioni richieste nel comando dell'attaccante} 
\label{table:comunicazione:attacccante:opzioni} 
\vspace{2ex} 
Per ogni proxy specificato nel file di configurazione, si verifica quali sono disponibili; 
ovvero quali proxy sono connessi sia all'attaccante che alla vititma. 
%Quindi si creeerà un socket per ogni proxy connettendosi all'indirizzo e alla porta in cui si sono messi 
%in ascolto e tramite esso, si invia un messaggio di conferma. 
Per farlo si crea un canale di comunicazione con ciascun proxy, o tramite ICMP o TCP/IP, 
e si invia un messaggio di connessione. 
In questo messaggio si specifica sia l'indirizzo IP della vittima che la metodologia di attacco scelta.
Successivamente si rimarrà in attesa che il proxy confermi la connessione con la vittima. 
%WAIT PROXY UPDATE: %Eseguito su ogni thread. %Tra il proxy e l'attaccante ci può essere un canale TCP/IP oppure ICMP. 
%Aspetta di ricevere dal proxy un messaggio di conferma; se la conferma aspettata non combacia con quella ricevuta 
%il proxy viene scartato e il socket chiuso. Nel messaggio di conferma il proxy indica se è connesso alla vititma. 
\vspace{2ex} \newline 
Nel caso non si usassero dei sockets TCP/IP ma il protocollo ICMP, bisognerà definire un thread 
che si occupi di monitorare il traffico di rete per poter catturare i messaggi ICMP contenenti i dati 
inviati dai proxy. 
Quindi prima di inviare alcun comando (o dato), bisogna assicurarsi che il thread sia già partito altrimenti si 
potrebbero perdere delle informazioni. 
\vspace{1ex} \newline
Una volta ricevuti tutti i dati inoltrati dai proxy, l'attaccante dovrà riordinarli per ricavare il messaggio. 
Inoltre se si volesse mandare un altro comando, si dovranno reimpostare le variabili necessarie per farlo.
Altrimenti si può decidere di interrompere la comunicazione e in quel caso i proxy ne verranno aggiornati. 


\subsubsection{Struttura del Proxy} 
Come per l'attaccante, il proxy richiede degli argomenti [Tabella \ref{table:comunicazione:proxy:opzioni}]. 
\begin{longtable}{|p{0.3\textwidth}|p{0.6\textwidth}|} 
    \hline 
    \textbf{Opzione} & \textbf{Utilizzo}  \\
    \hline
    Indirizzo IP & 
    Rappresenta l'indirizzo IP dell'attaccante. Quando il proxy crea il socket e rimane in ascolto delle connessioni, 
    accetterà solo quelle che combaciano con questo indirizzo; qualunque altra connessione verrà rifiutata. 
    Questo viene fatto anche nel caso di un canale tramite ICMP. 
    \\ 
    \hline 
\end{longtable}
\captionof{table}{Opzioni richieste nel comando del proxy} 
\label{table:comunicazione:proxy:opzioni} 
\vspace{2ex} 
%Il proxy ricaverà il proprio indirizzo IP. 
Per poter comunicare con l'attaccante, il proxy definisce un socket in cui rimane in ascolto. 
Nel caso si utilizzi il protoccollo ICMP, monitorerà il flusso di rete per catturare i messaggi inviati dall'attaccante.  
%Confronta l'indirizzo di chi si è connesso con quello passato 
%Controlla se il messaggio ricevuto successivamente corrisponde ad un messaggio di conferma. 
%Nel caso non fosse il socket viene chiuso e si ritorna in ascolto; 
%altrimenti dal messaggio si ricava l'indirizzo Ip della vittima e la metodologia di attacco. 
Stabilita una comunicazione con l'attaccante, il proxy aspetterà un messaggio indicante l'indirizzo IP della 
vittima oltre alla metodologia di esfiltrazione scelta. 
\vspace{1ex} \newline 
I messaggi che un proxy può ricevere dall'attacante sono: 
\begin{enumerate}
    \item il messaggio indicante il comando, che il proxy dovrà inoltrare alla vittima, o se invece deve 
    aspettare solo i dati, perchè non ha ricevuto il comando. 
    \item quello che indica la volonta, da parte dell'attaccante, di terminare la comunicazione. 
    In questo caso il proxy aggiornerà la vittima della cosa. 
\end{enumerate} 
%Le principali variabili utilizzate sono: 
%ip\_attaccante: che indica l'indirzzo IP dell'attaccante
%ip\_host: che indica l'indirzzo IP del host
%ip\_vittima: che indica l'indirzzo IP della vittima.  
%attack\_function: indica la tipologia di attacco 
%Nel caso il proxy e l'attaccante combaciassero, in questo caso non verrà definito alcun socket ma, come per l'attaccante, si ricaveranno i dati necessari dal file di configurazione. 
Per la connessione con la vittima, il proxy comunicherà tramite pacchetti ICMP. 
Prima di inviare alcun dato, si imposta un thread con lo scopo di analizzare il traffico e catturare i dati che 
la vittima ritornerà. %per confermare la connesisone. 
Se questo non viene fatto dopo i pacchetti potrebbe andare persi. 
%Dopo aver fatto partire il thread, si procederà codificando l'attacco scelta nel campo identifier del messaggio 
%ICMP Echo Request mentre nel payload verrà inserito il messaggio che richiede la connessione.  
%Infine si aspetta che il thread termini e ritorni lo stato della connessione con la vititma. 
%Dopodichè si procede ad aggiornare l'attaccante sul risultato della connesisone che si ha con la vittima e, 
%nel caso il risultato sia negativo il programma viene terminato (siccome non potrà essere utilizzato per l'inoltro delle informazioni).  
%Se invece si è stabilita una connessione con la vittima, si rimarra in attesa di messaggi da aprte dell'attaccante. 
Una volta ricevuti tutti i dati dalla vittima, il proxy procederà ad inoltrarli all'attaccante. 
%Siccome lo scambio di messaggi avviene fra l'attaccante e il proxy, l'invio e la ricezione avverrà tramite 
%il socket definito all'inizio della comunicazione. 
%Tuttavia questo non varrà per l'inoltro del comando; infatti la comunicazione avverrà fra la vittima e 
%il proxy e per questo si invieranno (e riceveranno) i dati in base alla tipologia di attacco definita.  
%Inoltre non essendo presente alcun tipo di socket fra le due entità, 
%il thread che si occuperà di ricevere i dati dovrà partire prima di inviare il comando.  
%Nel caso il proxy e l'attaccante combaciassero, in questo caso non si aspetterà alcun comando lo si richiederà 
%in input. Inoltre i dati non verranno inoltrati. 



\subsubsection{Struttura Vittima} 
Come per l'attaccante e il proxy, la vittima richiede degli argomenti [Tabella \ref{table:comunicazione:vittima:opzioni}]. 
\begin{longtable}{|p{0.3\textwidth}|p{0.6\textwidth}|} 
    \hline 
    \textbf{Opzione} & \textbf{Utilizzo}  \\
    \hline
    Numero di Proxy & 
    Quando si eseguirà il programma, dovranno essere definitiil numero di proxy necessarri. 
    Ciò indicherà il numero minimo di proxy necessari che serviranno per l'esecuzione dell'attacco. 
    Una volta raggiunto questo numero, la vittima procederà con l'esecuzione del comando. 
    Altrimenti, se il numero non viene raggiunto, allo scadere del timer si chiederà se si vuole procedere comunque.
    %attessa dei proxy che si vorranno connettere
    \\ 
    \hline 
\end{longtable}
\captionof{table}{Opzioni richieste nel comando della vittima} 
\label{table:comunicazione:vittima:opzioni} 
%Successivamente si ricaverà il prorpio indirizzo IP e si andrà in attessa dei proxy che si vorranno connettere. 
La vittima rimane in attesa di connessioni da parte dei proxy.
Ciò viene fatto monitornado il flusso di rete e filtrando i messaggi ICMP, destinati alla vittima, che rappresentano 
una richiesta di connessione. 
Se il messaggio catturato è valido, si risponde al mittente con un messaggio di conferma e il suo l'indirizzo IP 
viene inserito nella lista indicante i proxy connessi.
Da questo messaggio la vittima ricaverà anche la metodologia di attacco scelta. 
%\vspace{1ex} \newline 
%La vititma smetterà di monitorare il traffico finche o non verrà raggiunto il nuemro minimo 
%di proxy o finche il tmier non scade. 
%In ogni caso la lista dei proxy connessi verrà controllata per verificare se ci sia alemo un proxy connesso. 
%Se così non fosse (e il numero risulti zero) termina immediatamente la connessione altrimenti 
%se il numero è inferiore a quello richiesto, chiese se si vuole procedere ciò nonostante.
\vspace{1ex} \newline 
Definiti i proxy con cui si comunicherà, si attenderà che inoltrino il comando dell'attaccante. 
Una volta ricevuto, verrà eseguito e i risultati ricavati (sia quelli legati all'output che quelli legati ad 
eventuali errori) vengono inviati all'attaccante tramite i proxy disponibili. %che possono essere accaduti durante l'esecuzione. 
%Questi dati saranno poi inoltrati ai proxy. 
Nel caso siano presenti molteplici proxy connessi, si definirà una lista indicante i dati che ciscuno di essi dovrà ricevere. 
%i dati verranno suddivisi in più porzioni e  
%Per ogni porxy connesso, si definirà una lista indicante i dati che dovrà ricevere. 
%Il modo in cui verranno distribuiti è sequenziale. 
%I dati che un proxy deve mandare, non verrnno mandati immediatamente; 
%ma si ciclerà in modo circolare fra tutti i proxy vinchè tutti hanno mandato i dati. 
%Dopodichè si invierà un messaggio a tutti i proxy per indicare che tutti i dati sono stati mandati.  
L'ultimo messaggio che verrà inviato a ciascuno dei proxy sarà quello che indica che tutti i dati sono stati mandati. 
%

%Il proxy invece, al momento dell'inizializzazione, imposta un server su cui l'attaccante si connetterà. 
%Si potrebbe impostare una comunicazione tramite ICMP (come nel caso con la vittima) ma questo renderebbe 
%la comunicaizone instabile. 
%Invece, siccome il proxy può ritenersi sicuro, un canale TCP/IP può essere utilizzato. 

%Infine, per esfiltare i dati, invierà a ciascuno dei proxy connessi una porzione delle informazioni ricavate. 
%Terminato di farlo, si metterà in ascolto del prossimo comando dell'attaccante, che determinerà la prossima 
%operazione o la terminazione della comunicazione. 

