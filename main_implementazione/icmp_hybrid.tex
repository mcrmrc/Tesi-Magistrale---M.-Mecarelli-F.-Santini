
Un Covert Channel ibrido è stato implementato combinando le tre tipologie di Covert Channel usate 
precedentemente: 
\begin{itemize}
    \item \textit{Timing Channel}: un byte indicherà il delay fra un pacchetto e l'altro. 
    \item \textit{Behavioural Channel}: da un byte verranno ricavate le due tipologie di messaggi da mandare. 
    \item \textit{Storage Channel}: in ciascun pacchetto verranno inseriti tanti dati quanto la capacità di trasmissione disponibile. 
\end{itemize} 
%\captionof{lstlisting}{Pseudocodice per l'invio dei dati con Hybrid Covert Channel} 
%\label{lstlisting:codice:hybridchannel:mittente}
%\begin{center}
%    \begin{tabular}{|c|c|} 
%        \hline
%        \textbf{Messaggio ICMP} & \textbf{Dati trasportabili} \\ 
%        \hline
%        Information Request (15-16) & 2 byte \\ 
%        \hline
%        Timestamp Request (13-14) & 5 byte  \\ 
%        \hline 
%        Echo Request (8-0) & 2 byte  \\ 
%        \hline
%        Redirect (5) & 4 byte  \\ 
%        \hline
%        Source Quench (4) & 8 byte  \\ 
%        \hline
%        Parameter Problem (12) & 7 byte  \\ 
%        \hline 
%        Time Exceeded (11) & 6 byte  \\ 
%        \hline
%        Destination Unreachable (3) & 8 byte  \\ 
%        \hline 
%    \end{tabular} 
%    \captionof{table}{Quantità di byte trasportabili da un messaggio} 
%    \label{table:messaggiICMP:byte}
%\end{center}
\vspace{2ex} 
%Dalla sequenza di dati da inviare, il mittente ricaverà: il byte per l'intervallo di tempo da aspettare, 
%il byte i cui primi ed ultimi 4 bit indicheranno la tipologia di messaggio ICMP da inviare. 
%In ciascun messaggio (indicato da questi 4 bit), inserirà tanti dati quanta la loro capacità di trasmissione. 
%Ovvero quanti byte sono esfiltrabili dal messaggio ICMP in questione. 
La comunicazione viene iniziata tramite un messaggio \textit{Information Request} e chiusa sempre da un messaggio dello stesso tipo. 
%\vspace{1ex} \newline 
%\begin{minipage}{\textwidth}
%    \centering
%    \includegraphics[width=0.6\textwidth]{./img/hybrid_channel/sender_hybridChannel_1.png}
%    \captionof{figure}{Invio dei dait tramite Hybrid Covert Channel} 
%    \label{fig:hybridCC:invio}
%\end{minipage} 
\vspace{2ex}  \newline
Il destinatario monitora il flusso dei dati nella rete per rilevare l'intervallo di tempo con cui arrivano le 
coppie di messaggi. Da questo si ricaverà il byte relativo al tempo. 
Dalla coppia di messaggi ricava i quattro bit associati a ciascun messaggio. 
Dall'unione dei bit ricaverà il byte relativo alla tipologia di messaggi inviati. 
Ora non resta che ricavare i dati che sono stati nasconsti nei campi dei messaggi ICMP ricevuti.
%\vspace{1ex} \newline
%\begin{minipage}{\textwidth}
%    \centering
%    \includegraphics[width=0.6\textwidth]{./img/hybrid_channel/receiver_hybridChannel.jpg}
%    \captionof{figure}{Ricezione dei dait tramite Hybrid Covert Channel} 
%    \label{fig:hybridCC:ricezione}
%\end{minipage} 









