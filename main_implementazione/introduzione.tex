Dopo aver analizzato gli strumenti che già hanno provato a sviluppare un covert channel tramite ICMP e 
dopo aver analizzato metodologie sfruttabili per poter nascondere i dati; 
procediamo nel svilupparne uno. 
%Prima di sviluppare un covert Channel Channel che potesse esfiltrare i dati dalla macchina vittima; 
%si sono analizzati strumenti già presenti per studiarne il comportamento. 
\vspace{2ex} \newline 
Per la definizione della struttura, si è preso spunto da \textbf{icmp tunnel} \cite{icmp-tunnel}. 
Lo schema generale di funzionamento è illustrato nella figura seguente [Figura \ref{fig:icmpTunnel:architettura:generale}].
\begin{center} 
    \includegraphics[width=0.7\textwidth]{../img/ICMPtunnel/architettura_generale.png} 
    \captionof{figure}{Architettura generale in \textbf{icmp tunnel} \cite{icmp-tunnel}}
    \label{fig:icmpTunnel:architettura:generale} 
\end{center} 
\textit{icmp tunnel} reindirizza il traffico IP verso un interfaccia virtuale per poi reinoltrarlo al proxy 
tramite il protocollo ICMP [Figura \ref{fig:icmpTunnel:architettura:vittima}]. Il proxy invece rimane in ascolto 
di questi pacchetti ICMP e li invia verso la destinazione finale. I pacchetti poi in entrata nel proxy, che devono essere inoltrati alla macchina vittima, 
sono poi incapsulati tramite il protocollo IP e poi inviati [Figura \ref{fig:icmpTunnel:architettura:proxy}]. 
\begin{center} 
    \includegraphics[width=0.65\textwidth]{../img/ICMPtunnel/architettura_vittima.png} 
    \captionof{figure}{Architettura vitttima-proxy in \textbf{icmp tunnel} \cite{icmp-tunnel}}
    \label{fig:icmpTunnel:architettura:vittima} 
\end{center} 
\begin{center} 
    \includegraphics[width=0.65\textwidth]{../img/ICMPtunnel/architettura_proxy.png} 
    \captionof{figure}{Architettura proxy-internet in \textbf{icmp tunnel} \cite{icmp-tunnel}}
    \label{fig:icmpTunnel:architettura:proxy} 
\end{center} 
\vspace{2ex} 
Alcuni aspetti da considerare per l'implementazione della comunicazione sono: 
\begin{enumerate}
    \item La presenza di più di un proxy intermediario fra l'attaccante e la vittima. 
    %Si è realizzata la necessita di \textbf{proxy intermediari} fra la vitima e l'attaccante così da poter nascondere l'identità di quest'ultimo. 
    Sebbene sia possibile usare un singolo proxy, l'utilizzo di più macchine proxy permette di nascondere 
    meglio l'attacco. Con un singolo proxy, un sistema di difesa potrebbe notare che tutto il traffico ICMP viene 
    inviato verso un'unica sorgente. Utilizzando più proxy, il traffico verrà distribuito fra più sorgenti 
    diverse. Tuttavia molteplici proxy comportano un aumento nella complessità della comunicazione. 
    %Quindi sebbene sia possibile anche nessun utilizzo di questi proxy, e quindi si utilizzerà solo la 
    %machcina dell'attaccante;  nel caso li si volesse utilizzare, si deve tenere conto di una 
    %\textbf{distribuzione omogenea del traffico} così da non dirigere il throughput creato solo verso una singola macchina. 
%Se il traffico generato, venisse instradato attraverso un singolo proxy alla volta, il numero di messaggi scambiati genererebbe anomalie che potrebbero essere notato dai meccanismi di difesa. 
%del traffico generato e non generare un throughput elevato (dato il numero di messaggi scambiati)
%l'attaccante si può connettere ai proxy. 
%Si può quindi utilizzare una tipologia di comunicazione meno nascosta e più diretta.
%Si possono quindi creare dei Socket
    \item La quantità di dati che si vuole inviare. %quindi il possibile \textbf{limite} della loro dimensione. 
    Se mai si volesse esfiltrare un file contenente una grande quantità di dati; questo potrebbe generare rumore e destare sospetti. 
    Quindi si deve tenere conto di un \textbf{limite massimo} di dati da inviare in un intervallo di tempo ed 
    in caso implementare un \textbf{periodo di riposo} prima di inviarne altri.
    %Maggiore è il numero continuativo di richieste, maggiore sarà la possibilità di far notare la presenza del Covert Channel (e quindi di essere scoperti). 
    %Questo soglia può non essere direttamente legata a un fattore temporale, ma può variare anche in base a quanti messaggi sono stati scambiati fra le due parti. 
    \item La trasmissione dei dati in chiaro permettera a chiunque di leggerne il contenuto. 
    Un sistema di difesa che fà un ispezione approfondita dei pacchetti, potrebbe rilevare la comunicazione. 
    Tuttavia anche mandarli cifrati potrebbe destare sospetti. 
    Si devono qundi poter mandare i dati cifrati ma che sembrino in chiaro. 
    Al momento non si è trovato un metodo efficace a parte l'inserimento di dati in forma numerica; 
    anche se potrebbe essere trovato un pattern anche in questo caso. 
    %Per esempio nel campo \textit{identifier}, che richiede  interi, l'inserimento di numeri non risulterebbe strano. 
    %Tuttavia questo non varrà per il campo \textit{data}. 
    %Quindi un metodo per poter inserire dati codificati ma che risultino normali è necessario. 
    %Trasmettere i dati in chiaro potrebbe essere semplice da implementare, ma potrebbe essere facilmente rilevabile. 
    %Trasmettere i dati codificati potrebbe essere più difficile, ma potrebbe risultare più sicuro. 
    \item Un ulteriore fattore sono le tipologie di messaggi che richiedano una risposta. 
    Siccome ad ogni richiesta viene mandata una risposta (avente gli stessi dati ricevuti), si avranno due 
    messaggi identici. %tranne per il campo che indica il tipo di messaggio. 
    Ciò potrebbe non risultare un problema se la dimensione del campo dei dati non sia eccessiva. 
    Una possibile soluzione a questo problema è l'uso solamente dei messaggi di risposta. 
    Tuttavia sarà anomalo che il numero delle risposte non combaci con quello delle richieste. 
    Un ulteriore possibilità è quella, se possibile, di disabilitare l'invio da parte del sistema delle risposte 
    e mandare una risposta che non ripeterà il contenuto della richiesta. 
    In questo caso ad una richiesta combacierà una richiesta sebbene non conforme agli standard. 
    %Così quando al sistema arriverà una richiesta, non invieerà una risposta ma la nostra che 
    %non ripeterà il contenuto mandato dall'attaccante ma un'altro (possibilmente di dimensione minore). 
    %\item Evitare un numero importante di pacchetti tutti verso la stessa deastinazione. 
    %Per esempio non mandare più di 5 pacchetti, in un determinato istante, verso la stessa destinazione. 
    %\uppercase{è} preferibile mandare meno e poi aspettare che mandarli tutti e subit. 
    %In questo modo si evita di essere scoperti facilmente.
\end{enumerate} 

