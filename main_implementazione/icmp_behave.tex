In questo caso l'evento che si osserverà, e che indicherà il dato trasmesso, è la tipologia ed il 
codice del messaggio ICMP arrivato. 
%Siccome oltre alla tipologia si potrà variare il codice inviato; 
\vspace{1ex} \newline 
Seguendo questo approccio, la quantità totale di eventi possibili risulta 17 [Figura \ref{fig:behaveCC:eventi:mappatura}]. 
Quindi ogni messaggio permetterà di codificare 4 bit alla volta (siccome $\log_216=4$). 
Siccome un evento (tipologia,codice) rimane inutilizzato; verrà sfruttato per indicare l'inizio 
e la terminazione dell'esfiltrazione dei dati. 
\vspace{1ex} \newline
\begin{minipage}{\textwidth}
    \centering
    \includegraphics[width=\textwidth]{./img/ICMP_behaveCC/behavioural_channel_1.png}
    \captionof{figure}{Schema Hybrid Channel \cite{francesco_santini}} 
    \label{fig:behaveCC:eventi:mappatura}
\end{minipage}  
\vspace{1ex} \newline 
Siccome un singolo evento può trasmettere solo 4 bit, ogni byte che si vuole inviare dovrà essere diviso in due parti. 
La divisione viene fatta mascherando il byte [Figura \ref{fig:behaveCC:eventi:estrazionebit}]. 
Questi  primi quattro bit ricavati, così come gli ultimi quattro bit ricavati, determineranno la 
tipologia (oltre al codice) del messaggio che verrà inviato.   
\begin{minipage}{\textwidth}
    \centering
    \includegraphics[width=\textwidth]{./img/ICMP_behaveCC/estrazione_bit_orizzontale.png}
    \captionof{figure}{Divisione del byte} 
    \label{fig:behaveCC:eventi:estrazionebit}
\end{minipage}   
\vspace{2ex} \newline
Il destinatario, per ricavare i dati, monitora il traffico di rete e controllerà le tipologie di 
messaggi ICMP ricevuti; per ognuno di essi, prende la coppia (Tipologia, Codice) e da questa 
ricaverà i quatttro bit associati.  
%invece si metterà in ascolto nel traffico di rete di un messaggio \textit{Information Request} 
%proveniente dall'attaccante. 
%Una volta ricevuto, memorizzerà le tipologie di messaggi ICMP provenienti dalla sorgente e 
%ogni volta ricava il byte associato alla coppia di messaggi ottenuta. 
%Infine quando rileverà ancora un pacchetto \textit{Information Request} smetterà di monitorare la rete. 

\subsubsection{Tipologie di messaggi deprecati \cite{icmp_deprecated} \cite{rfc6918_deprecated_types} \cite{rfc6633_sourcequench}} 
Un problema di questa possibile implementazione, sono le tipologie deprecate dei messaggi. 
Questi messaggi potranno comunque essere inviati ma le linee guide indicano ai router di ignorarli e all'host 
utente di non mandarli. %Un effetto collaterale potrà quindi essere la non ricezione dei messaggi 
%senza che la cosa venga scoperta una volta che si cerca di ricostruire il dato esfiltrato. 
%This document does not modify the security properties of the ICMPv4 message types being deprecated.  
%However, formally deprecating these message types serves as a basis for, e.g., filtering these packets. \cite{rfc6633_sourcequench}
\vspace{1ex} \newline 
L'uso di tipologie deprecate renderà il canale meno legittimo e più facile da individuare. 
Traffico di questo tipo inoltre sarà visto con sospetto e non è assicurata la ricezione dei messaggi. 
%Deprecated types = less likely to be used legitimately, so unusual traffic with those types is suspicious 
%(which is relevant for covert channels).
%
%Network devices (routers, firewalls) may not expect or properly process deprecated types, increasing 
%risk/residual side-effects for covert use. 
%
%Filtering or blocking deprecated types reduces attack surface but you must weigh this with legitimate traffic 
%needs (especially for older hosts/networks).
%
%Because some deprecated types are still recognized in code or by older devices, they might leak information 
%(fingerprinting) or behave differently — a potential covert channel vector or detection point.
%
%Just because a type is deprecated doesn’t guarantee it’s never used — there may be legacy systems still 
%sending/responding. So filtering must be tested carefully.
%
%Overblocking (e.g., all ICMP) may break diagnostic/troubleshooting functions — an important risk if you operate 
%in an environment where ICMP matters.
%
%Covert channel detection based purely on unusual ICMP types may generate false positives (or false negatives) — 
%so you’ll want behavioural/contextual checks (source, destination, rate, payload size).
%
%Router behaviour for deprecated types may vary by vendor; undocumented implementation-specific behaviours may exist.


\subsubsection*{Source Quench} 
Il messaggio ICMP Source Quench (tipo 4, codice 0) era concepito come meccanismo per il controllo della congestione.
%The Host Requirements RFC [RFC1122] stated that hosts MUST react to ICMP Source Quench messages by slowing 
%transmission on the connection, and further added that the RECOMMENDED procedure was to put the corresponding 
%connection in the slow-start phase of TCP's congestion control algorithm [RFC5681].
Tuttavia, data la sua inefficacia per la congestione, la generazione di questi messaggi da parte dei router è 
stata formalmente deprecata dal 1995. %\cite{rfc1812}. 
%[RFC1812] noted that research suggested that ICMP Source Quench was an ineffective (and unfair) antidote for 
%congestion, and formally deprecated the generation of ICMP Source Quench messages by routers,stating that 
%routers SHOULD NOT send ICMP Source Quench messages in response to congestion.
Per questo la maggior parte delle implementazioni TCP ignorano silenziosamente i messaggi ICMP Source Quench. 
Infatti TCP implementa i propri meccanismi di controllo della congestione (che non dipendono dai messaggi ICMP Source Quench).
%But the reaction to ICMP Source Quench messages in transport protocols has never been formally deprecated. 
%The rationale for these specification updates is as follows: 
%\begin{itemize}
%    \item Processing of ICMP Source Quench messages by routers has been deprecated for nearly 17 years [RFC1812]. 
%    \item Virtually all popular host implementations have removed support for ICMP Source Quench messages since (at least) 2005 [RFC5927]. 
%    \item Widespread deployment of ICMP filtering makes it impossible to rely on ICMP Source Quench messages for congestion control. 
%    \item The IETF has moved away from ICMP Source Quench messages for congestion control (e.g., note the development of Explicit 
%    Congestion Notification (ECN) [RFC3168] and the fact that ICMPv6 [RFC4443] does not even specify a Source Quench message). 
%    ICMP Source Quench messages are not normally seen in the deployed Internet and were considered rare 
%    at least as far back as 1994 [Floyd1994]. 
%\end{itemize}  
%
%[RFC5927] discussed the use of ICMP Source Quench messages for performing "blind throughput-reduction" attacks, 
%and noted that most TCP implementations silently ignore ICMP Source Quench messages.
%
\vspace{1ex} \newline 
Quindi, siccome la reazione a questi messaggi nei protocolli di trasporto non è mai stata formalmente deprecata, 
la IETF (Internet Engineering Task Force) ha aggiornato le linee guida riguardo il messaggio Source Quench in questo modo: 
%RFC 1122 \cite{rfc1122} \footnote{riguarda i protocolli a livello di comunicazione (link layer, IP layer, transport layer)}, 
%RFC 1812 \cite{} \footnote{rfc1812} \footnote{definisce i requisiti per i dispositivi che eseguono la funzione di network layer forwarding} 
%This memo defines and discusses requirements for devices that perform the network layer forwarding function of the Internet protocol suite.
\begin{itemize}
    \item Un host \textbf{NON DEVE inviare} dei messaggi di questo tipo. 
    Se viene ricevuto un messaggio di questo tipo, \textbf{PUÒ ignorarlo} (silenziosamente). 
    \item Un router \textbf{DEVE ignorare} tutti i messaggi ICMP Source Quench ricevuti. 
    Inoltre \textbf{NON DOVR\uppercase{à} inviare} messaggi ICMP Source Quench in risposta a una congestione. 
    \item Se un protocollo di trasporto riceve un messaggio Source Quench, \textbf{DEVE essere ignorato/scartato} silenziosamente. 
    \item Host, gateway e firewall DEVONO scartare silenziosamente i pacchetti ICMP Source Quench 
    ricevuti e \textbf{DOVREBBERO registrare} la cosa come un errore di sicurezza con almeno i seguenti dettagli 
    (indirizzo IP sorgente, indirizzo IP di destinazione, tipo di messaggio ICMP, data/ora in cui il pacchetto 
    è stato visualizzato). 
\end{itemize}  
%The consensus of the TSV WG was that 
%There are no valid reasons for a host to generate or react to an ICMP Source Quench message in the current Internet.  
Nell'Internet attuale, non ci sono ragioni per cui un host debba generare o reagire a un messaggio ICMP Source 
Quench %nell'attuale Internet. 
(o perchè un router debba reagire). %ai messaggi ICMP Source Quench.
%The recommendation that a sender "MUST NOT" send an ICMP Source Quench message is because there is no known valid reason for a host to generate this message. 
%The only known impact of a sender ignoring this requirement is that it may necessarily consume network and endpoint resources. 
L'unico motivo per cui un utente ignori questo requisito, è per poter consumare le risorse di rete. %e di endpoint.
I messagg ICMP Source Quench potrebbero essere infatti sfruttati per eseguire attacchi "blind throughput-reduction". 
Le linee guida indicano di ignorare silenziosamente questi messaggi; questo eliminerà il vettore di attacco. 
\vspace{1ex} \newline
%The TSV WG is not aware of any mechanism that requires processing of these messages and therefore expects other 
%transports to follow the recommendations in Section 3.  
%Note that since generation of ICMP Source Quench messages has been deprecated for many years, and since this 
%document additionally deprecates reaction to ICMP Source Quench messages by IETF-specified transports, future applications cannot expect to receive these messages. 
Inoltre, siccome la generazione e la reazione ai messaggi ICMP Source Quench è stata deprecata, le applicazioni 
non dovranno aspettarsi di ricevere questi messaggi. 

\subsubsection*{Information Request/Reply} 
\uppercase{è} stato deprecato siccome meccanismi come il DHCP \cite{rfc2131_DHCP}, lo hanno sostituito per la configurazione dell'host
